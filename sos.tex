\documentclass[12pt, a4paper, oneside]{book}
\usepackage[utf8]{inputenc}
\usepackage[francais]{babel}
\usepackage[T1]{fontenc}
\usepackage{amsmath}
\usepackage{amsfonts}
\usepackage{arial}
\usepackage{setspace}
\usepackage{amsthm}
\usepackage{hyperref}
\usepackage{mathrsfs}
\usepackage{amssymb}
\usepackage{makeidx}
\usepackage{graphicx}
\usepackage{lmodern}
\usepackage[left=2.8cm,right=2.8cm,top=3.5cm,bottom=3.5cm]{geometry}
\usepackage{tocloft}
\usepackage{stmaryrd}
\hypersetup{colorlinks=true, urlcolor=blue,linkcolor=blue, breaklinks=true}
\usepackage{tikz}
\newtheorem{maremarque}{Remarque}[chapter]
\newtheorem{montheoreme}{Théorème}[chapter]
\newtheorem{madefinition}{Définition}[chapter]
\newtheorem{maproposition}{Proposition}[chapter]
\newtheorem{moncorollaire}{Corollaire}[chapter]
\newtheorem{monlemme}{Lemme}[chapter]
\newtheorem{monexemple}{Exemple}[chapter]
\newtheorem{maconsequence}{Consequence}[chapter]
\begin{document}
\begin{moncorollaire}
	Soient $f=(I_n), g=(J_n) \in \mathbb{F}(A)$, tel que $f \leqslant g$ et $A$ est noethérien.\\ Si $f$ ou $g$ est noethérien alors $g$ est $f-bonne$ $\Longleftrightarrow$ $g$ est fortement entière sur $f$.
\end{moncorollaire}
\begin{maproposition}
	Soit $A$ un anneau noethérien. Soient $f,g \in \mathbb{F}(A)$.\\ Si $f$ est noethérienne alors $g$ est fortement entière sur $f$ $\Longleftrightarrow$ il existe un entier naturel $N \geqslant 1$ tel que $t_{N}g \leqslant f \leqslant g$.
\end{maproposition}
\begin{maproposition}
	Soient $f,g \in \mathbb{F}(A)$. Alors:
	\[ g \text{ est entière sur } f \Longleftrightarrow \forall k \in \mathbb{N}^{*}, g^{(k)} \text{est entière sur } f^{(k)} \Longleftrightarrow \exists k \in \mathbb{N}^{*}, g^{(k)} \text{est entière sur } f^{(k)} \]
\end{maproposition}

\begin{moncorollaire}
	Soient $f,g \in \mathbb{F}(A)$, tel que $f \leqslant g$. Si $A$ est noethérien et $g$ est fortement entière sur $f$. Alors $f$ est fortement $A.P$ $\Longleftrightarrow$ $g$ est fortement $A.P.$
\end{moncorollaire}
\begin{maproposition}
	Si $f=f_I$ alors:\\
	$f$ est fortement A.P $\Longleftrightarrow $ $f$ est A.P $\Longleftrightarrow $ $f$ est fortement noethérienne $\Longleftrightarrow $ $f$ est noethérienne $\Longleftrightarrow $ $f$ est E.P
\end{maproposition}

\begin{moncorollaire}
	Soient $f=(I_n)_{n \in \mathbb{N}}$ et $g=(J_n)_{n \in \mathbb{N}}$ deux filtrations de $A$,\\ tel que $f \leqslant  g$.
	Si $f$ ou $g$ est noethérienne. Alors: \\
	$g$ est $f-bonne$ $\Longleftrightarrow $ $g$ est fortement intégral sur $f$
\end{moncorollaire}
\begin{maproposition}
	Soient $f=(I_n)_{n \in \mathbb{N}}$ et $g=(J_n)_{n \in \mathbb{N}}$ deux filtrations de $A$ tel que $f$ est une réduction de $g$ alors:	$g$ est E.P et $g$ est $f-bonne$.
\end{maproposition}

\begin{maproposition}
	Lorsque $f$ est une filtration fortement noethérienne et $g$ est une
	filtration noethérienne de l'anneau noethérien A vérifiant $
	f=(I_{n})\leq g=(J_{n})$, on montre que les assertions suivantes sont équivalentes:
	
	(i) $f$ est une réduction de $g$
	
	(ii) $J_{n}^{2}=I_{n}J_{n},\forall n>>0$
	
	(iii) L'idéal $I_{n}$ est une réduction de l'idéal $J_{n}$ pour
	tout $n>>0$
	
	(iv) Il existe un entier $k\geq 1$ tel que $g^{k}$ soit $I_{k}-bonne$
	
	(v) $\forall m\geq 1$, $f^{(m)}$ est une réduction de $g^{(m)}$
	
	(vi) $\exists m\geq 1,$ tel que $f^{(m)}$ soit une réduction de $g^{(m)}$
	
	(vii) $g$ est entière sur $f$
	
	(viii) $g$ est fortement entière sur $f$
	
	(ix) $g$ est $f-fine$
	
	(xi) $g$ est faiblement $f-bonne$
	
	(x) $g$ est $f-bonne$
	
	(xii) $\exists m\geq 1,$ tel que $t_{m}f\leq f\leq g$
	
	(iii) $(P_{k}(f))=(P_{k}(g))$ pour tout $k\in\mathbb{N}$
	
	En particulier, il résulte des équivalences ci-dessus que si $f$ est une filtration $I-adique$ de l'anneau noethérien $A$ et si $g$ est
	une filtration noethérienne dominée par $g$, les notions suivantes
	sont équivalentes :
	
	(1) $f_{I}$ est une réduction de $g$
	
	(2) $g$ est entière sur $f_{I}$.
	
	(3) $g$ est fortement entière sur $f_{I}$
	
	(4) $g$ est $I-bonne$
	
\end{maproposition}

\begin{montheoreme}
	Soient $f=(I_{n})_{_{n\in \mathbb{N}}}\leq $ $g=(J_{n})_{_{n\in \mathbb{N}}}$ des filtrations sur l'anneau $A.$ Nous
	considérons les assertions suivantes:
	
	$i)$ $f$ est une réduction de $g.$
	
	$ii)$ $J_{n}^{2}=I_{n}J_{n}$ pour tout $n$ assez grand.
	
	$iii)$ $I_{n}$ est une réduction de $J_{n}$ pour tout $n$ assez grand.
	
	$iv)$ Il existe un entier $s\geq 1$ tel que pour tout $n\geq s,$ $J_{s+n}=J_{s}J_{n},$
	
	$I_{s+n}=I_{s}I_{n},$ $J_{s}^{2}=I_{s}J_{s},$ $J_{s+p}I_{s}=I_{s+p}J_{s}$ pour tout $p=1,2,...,s-1$
	
	$v)$ Il existe un entier $k\geq 1$ tel que $g^{(k)}$ est $I_{k}-bonne$
	
	$vi)$ Il existe un entier $r\geq 1$ tel que $f^{(r)}$ est une réduction de $g^{(r)}.$
	
	$vii)$ Pour tout entier $m\geq 1$ tel que $f^{(m)}$ est une réduction de $g^{(m)}.$
	
	$viii)$ $g$ est $enti\grave{e}re$ sur $f.$
	
	$ix)$ $g$ est $\ fortement$ $enti\grave{e}re$ sur $f.$
	
	$x)$ $g$ est $f-fine.$
	
	$xi)$ $g$ est $f-bonne.$
	
	$xii)$ $g$ est $faiblement$ $f-bonne.$
	
	$xiii)$ Il existe un entier $N\geq 1$ tel que $t_{N}g\leq f\leq g$
	
	$xiv)$ Il existe un entier $N\geq 1$ tel que $t_{N}g^{\prime }\leq
	t_{N}f^{\prime \text{ }}$ o\`{u} $f^{\prime }$ est la clôture intégrale de $f.$
	
	$xv)$ $P(f)=P(g)$, o\`{u} $P(f)$ est la clôture prüférien de $f.$
	
	1) On a:
	
	$(i)$ $\Longleftrightarrow (vii)$ ; $(v)$ $\Longleftrightarrow (vi)$ ; $(viii)$ $\Longleftrightarrow (xv)$ ; $(ii)$ $\Longrightarrow (iii)$ ; $(iv)$ 
	$\Longrightarrow (i)\Longrightarrow (v)$ ; $(ix)\Longrightarrow (vii),(xii)$
	et $(xiii)$ ;
	
	$(i)\Longrightarrow (x)\Longrightarrow (xi)\Longrightarrow
	(xii)\Longrightarrow (xiii)$
	
	2) Si de plus on suppose $A$ noethérien, alors:
	
	$(i)\Longleftrightarrow (xiv)$ ; $(i)\Longrightarrow (ix)\Longleftrightarrow
	(xii)$ ; $(i)\Longrightarrow (ii)$
	
	3) Par ailleurs, si $f$ est $noeth\acute{e}rienne,$ alors $A$ est noethérien et les assertions suivantes sont équivalentes:
	
	$(ix)\Longleftrightarrow (x)\Longleftrightarrow (xi)\Longleftrightarrow
	(xii)\Longleftrightarrow (xiii)$
	
	4) Si $f$ et $g$ sont $noeth\acute{e}riennes$ alors nous avons:
	
	$(iii)\Longrightarrow (viii)\Longleftrightarrow (ix)$ ; $(vi)\Longrightarrow (ix)$
	
	5) Si $f$ est fortement $noeth\acute{e}rienne$ et $g$ est $noeth\acute{e}rien
	$ alors les quinze (15) assertions sont équivalentes et dans ce cas $g$
	est fortement $noeth\acute{e}rienne.$
	
	$(i)\Longleftrightarrow (ii)\Longleftrightarrow (iii)\Longleftrightarrow
	(iv)\Longleftrightarrow (v)\Longleftrightarrow (vi)\Longleftrightarrow
	(vii)\Longleftrightarrow (viii)\Longleftrightarrow (ix)\Longleftrightarrow
	(x)\Longleftrightarrow (xi)\Longleftrightarrow (xii)\Longleftrightarrow
	(xiii)\Longleftrightarrow (xiv)\Longleftrightarrow (xv).$
\end{montheoreme}
\begin{proof}
	
	1)
	
	$(i)\Longleftrightarrow (vii).$
	
	Supposons $(i)$ et choisissons $k$ comme dans \ref{maprop4} (i) alors pour tout
	entiers $m\geq 1$ et $n\geq k,$ $J_{m(k+n)}=J_{mk}I_{mn}$, ce qui entraîne $(vii).$
	
	La réciproque est évidente.
	
	$(v)\Longrightarrow (vi).$
	
	Posons $f^{(k)}=(H_{n});$ $g^{(k)}=(K_{n});$ $H_{n}=I_{nk};$ $K_{n}=J_{nk};$ 
	$H_{1}=I_{k};$
	
	Par hypothèse, $H_{1}K_{n}\subseteq K_{n+1}$ pour tout entier $n$ et il
	existe un entier $n_{0}\geq 1$ tel que $H_{1}K_{n}=K_{n+1}$ pour tout $n\geq
	n_{0}.$
	
	Pour tout entier $m\geq 0,$ $K_{n_{0}+m}=H_{1}^{m}K_{n_{0}}\subseteq
	H_{m}K_{n_{0}}\subseteq K_{n_{0}+m}.$
	
	Donc $K_{n_{0}+m}=K_{n_{0}}H_{m}$ pour tout entier $m.$ Et donc $f^{(k)}$
	est une réduction de $g^{(k)}.$
	
	$(vi)\Longrightarrow (v).$
	
	Il suffit de montrer que si $f$ est une réduction de $g$ alors il existe 
	$k\geq 1$ tel que $g^{(k)}$ est $I_{k}-bonne.$
	
	Posons $k$ comme dans \ref{maprop4} (i), alors pour tout entiers $m\geq 1$ et  $J_{k(m+1)}=J_{mk}I_{k}$, donc $g^{(k)}$ est $I_{k}-bonne.$
	
	Donc $(vi)\Longrightarrow (v).$
	
	$(viii)\Longleftrightarrow (xv).$
	
	Si $g$ est entière sur $f$ alors $f\leq g\leq P(f),$ ainsi $P(f)\leq
	P(g)\leq P(P(f))=P(f),$ donc $P(g)=P(f).$
	
	Réciproquement si $P(f)=P(g)$ alors $g\leq P(g)=P(f)$ et donc $g$ est entière sur $f.$
	
	$(ii)\Longrightarrow (iii).$
	
	Évident.
	
	$(iv)\Longrightarrow (i).$
	
	Posons $n\geq 2s$ et $n=qs+p$ avec $0\leq p<s.$
	
	Alors $J_{s+n}=J_{(q-2)s+2s+(s+p)}=J_{s}^{q-2}J_{2s+(s+p)}=J_{s}^{q-2}J_{s}^{2}J_{s+p}=J_{s}^{q-1}I_{s}J_{s+p}=J_{s}^{q-1}J_{s}I_{s+p}=J_{s}^{q}I_{s+p}=J_{s}I_{s}^{q-1}I_{s+p}\subseteq J_{s}I_{n}\subseteq J_{s+n}.
	$
	
	Par suite $J_{s+n}=J_{s}I_{n}$ pour tout $n\geq 2s.$ Donc $%
	J_{2s+n}=J_{2s}I_{n}$ pour tout $n\geq 2s.$ D'où $(i).$
	
	
	$(i)\Longrightarrow (v)$
	
	Évident car $(vi)\Longrightarrow (v).$
	
	$(ix)\Longrightarrow (viii)$
	
	Évident
	
	$(ix)\Longrightarrow (xii)\Longrightarrow (xiii)$ en utilisant \ref{maprop6} (5)
	
	$(i)\Longrightarrow (x).$
	
	Pour tout entier $n\geq N=2k-1,$ posons $n=qk+r,$ avec $0\leq r<k$ où $k$
	est comme dans $(4.3)$ $(i).$
	
	Alors $J_{n}=J_{k(q-1)}I_{k+r}.$
	
	Ainsi $1\leq k+r<2k-1,$ $J_{n}\subseteq
	\sum\limits_{p=1}^{N}I_{p}J_{n-p}\subseteq J_{n}$, d'où $J_{n}=\sum\limits_{p=1}^{N}I_{p}J_{n-p}$ pour tout $n\geq N=2k-1$.
	
	Ce qui prouve que $g$ est $f-fine.$
	
	$(x)\Longrightarrow (xi)$ par \ref{maprop3}
	
	$(xi)\Longrightarrow (xii)$ par \ref{maprop6} (1).
	
	
	2) 
	
	On suppose maintenant que $A$ est noethérien.
	
	Alors $(i)\Longrightarrow (ix)$ en utilisant \ref{maprop7}.
	
	$(i)\Longrightarrow (ii)$
	
	$f$ est noethérienne par $\ref{maprop7}$ donc il existe un entier $k^{\prime }$ tel que $I_{n+k^{\prime }}=I_{n}I_{k^{\prime }},$ pour tout $n\geq k^{\prime }$.
	
	Choisissons $k$ comme dans \ref{maprop4} (i) nous pouvons supposons que $k=k^{\prime }$ et même prendre $kk^{\prime }$ \`{a} la place de $k$ ou $k^{\prime }$ si nécessaire.
	
	Pour tout $n\geq 3k,$ posons $n=qk+r,$ avec $0\leq r<k.$ Alors $q=E(\frac{n}{k})\geq 3.$
	
	$J_{n}=J_{k}I_{(q-1)k+r}=J_{k}I_{k}^{q-2}I_{k+r}.$
	
	$J_{n}^{2}=J_{k}^{2}I_{k}^{q-3}(I_{k}^{q-1}I_{k+r})I_{k+r}\subseteq J_{2k}I_{(q-3)k}I_{n}I_{k+r}.$
	
	D'où $J_{n}^{2}\subseteq J_{n}I_{n}$
	
	Donc $J_{n}^{2}=J_{n}I_{n}$ pour tout $n\geq 3k.$
	
	$(i)\Longleftrightarrow (iv).$ 
	
	D'après 1) il suffit de montrer que $(i)\Longrightarrow (iv).$
	
	Nous avons vu que $(i)\Longrightarrow (ii).$ Alors il existe un entier $%
	k^{\prime }\geq 1$ tel que $J_{n}^{2}=I_{n}J_{n}$ pour tout $n\geq k^{\prime}.$
	
	Dans la preuve de la même implication, nous avons aussi montrer qu'il existe un entier $k\geq 1$ tel que $J_{k+n}=J_{k}I_{n}=J_{k}J_{n}$ et que $I_{k+n}=I_{k}I_{n}$ pour tout $n\geq k.$
	
	Posons $n\geq 2kk^{\prime }=s,$ $k"=kk^{\prime }$ et $n=qk"+r$ avec $0\leq r<k".$ Alors $q\geq 2$ et:
	
	$J_{s+n}=J_{(q+2)k"+r}=J_{k"}^{3}J_{(q-1)k"+r}=I_{k"}^{2}J_{k"}J_{(q-1)k"+r}=I_{s}J_{n}.$
	
	$J_{s+n}=J_{s}I_{n}=J_{s}J_{n}$
	
	$I_{s+n}=I_{s}I_{n}$
	
	$J_{s}^{2}=I_{s}J_{s}$
	
	D'où $(iv).$
	
	$(ix)\Longleftrightarrow (xii)$ d'après \ref{maprop8}
	
	$(iii)\Longleftrightarrow (xiv)$
	
	Nous savons que pour tout idéal $I\subseteq J$ d'un anneau noethérien, $I$ est une réduction de $J$ si et seulement si $I^{\prime }=J^{\prime },$ où $I^{\prime }$est la clôture intégrale de $I.$ D'où l'équivalence. 
	
	3) Supposons que $f$ est noethérienne.
	
	Alors d'après \ref{maprop10}, $(ix)\Longleftrightarrow (xiii)$ et d'après $\ref{maprop9},$ 
	
	$(ix)\Longleftrightarrow (x)\Longleftrightarrow (xi)\Longleftrightarrow (xii)\Longleftrightarrow (xiii)$
	
	4) Supposons que $f$ et $g$ sont noethériens. Alors $(viii)\Longleftrightarrow (ix)$ d'après (\cite{Di1}, \ref{maprop11},(b)).
	
	$(iii)\Longrightarrow (viii).$
	
	Supposons que $I_{n}$ est une réduction de $J_{n}$ pour tout $n\geq n_{0}.$
	
	$f$ et $g$ sont noethérien d'où fortement $A.P.$ \`{a} partir d'un rang
	commun $k.$
	
	L'idéal $J_{n_{0}k}$ est entière sur l'idéal $I_{n_{0}k}.$ D'où $g$ est entière sur $f$ d'après (\cite{Di1}, 4.5).
	
	$(vi)\Longrightarrow (ix).$
	
	Si $f^{(r)}$ est un réduction de $g^{(r)}$ alors $g^{(r)}$ est fortement entière sur $f^{(r)}$ d'après \ref{maprop7} (iii) et $g$ est fortement entière sur $f$ d'après \ref{maprop12}
	
	5) Supposons que $f$ est fortement noethérienne. D'après
	l'implication précédente il est facile de montrer que $(xii)\Longrightarrow (i).$
	
	Supposons que $(xii),$ alors il existe un entier $N\geq 1$ tel que pour tout $n>N,$ $J_{n}=\sum\limits_{p=0}^{N}I_{n-p}J_{p}.$
	
	$f$ étant fortement noethérienne, il existe un entier $N^{\prime}\geq 1$ tel que $I_{m}I_{n}=I_{m+n}$ pour tout $m,n\geq N^{\prime }.$
	
	Posons $n\geq k=N+N^{\prime }.$
	
	Si $0\leq p\leq N,$ alors $N^{\prime }=k-N\leq k-p\leq n-p,$ $J_{n+k}=\sum\limits_{p=0}^{N}I_{n+k-p}J_{p}=\sum\limits_{p=0}^{N}I_{n+}I_{k-p}J_{p}=I_{n}J_{k}$ , et $f$ est une réduction de $g.$
	
	Pour compléter la preuve, nous avons montrer par exemple que si $f$ est une réduction de $g$ et si $f$ est fortement noethérienne alors $g$ est fortement noethérien.
	
	Soient $k,$ $k^{\prime }$ des entiers $\geq 1$ tel que $J_{k+n}=J_{k}I_{n}$
	pour tout $n\geq k$ et $I_{m+n}=I_{m}I_{n}$ pour tout $m,n\geq k^{\prime }.$
	
	Posons $m,n\geq k^{\prime }.$ Alors $J_{m}J_{n}\subseteq
	J_{m+n}=J_{k}I_{m+(n-k)}=J_{k}I_{m}I_{n-k}\subseteq J_{m}J_{n},$ d'où $J_{m+n}=J_{m}J_{n}$ pour tout $m,n\geq k+k^{\prime }$ et $g$ est fortement noethérienne.
\end{proof}

\end{document}
