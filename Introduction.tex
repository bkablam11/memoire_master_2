\newpage
\tableofcontents
\renewcommand{\contentsname}{Table des matières}
\thispagestyle{empty}

\newpage

\setcounter{page}{0} 
\pagenumbering{arabic}
\thispagestyle{empty}
\addcontentsline{toc}{chapter}{\large{Introduction}}
\begin{center}
	\LARGE{\textbf{Introduction}}
\end{center}
\vspace{1cm}

Les notions de dépendance intégrale, de réduction et de filtrations bonnes constituent des axes cruciaux de l'Algèbre Commutative dans le domaine des mathématiques. Leur exploration minutieuse offre une clarté conceptuelle quant aux propriétés structurales des anneaux commutatifs. Retraçons brièvement l'évolution historique de ces notions fondamentales avant de rappeler les mathématiciens qui ont contribué à leur compréhension et à leur développement.

La notion de dépendance intégrale remonte aux travaux pionniers de Joseph Fourier au début du XIXe siècle, en lien avec l'étude des équations différentielles. Cependant, c'est surtout grâce aux développements ultérieurs de Évariste Galois dans le domaine des équations polynomiales que la dépendance intégrale a commencé à cristalliser dans le contexte algébrique. Elle a ensuite été explorée de manière systématique par des mathématiciens tels que David Hilbert et Emmy Noether au cours du XXe siècle, ouvrant ainsi la voie à des résultats fondamentaux dans la théorie des corps et des anneaux.

Les travaux de Krull et de Nagata ont considérablement enrichi la compréhension de la réduction et des filtrations I-adiques au milieu du XXe siècle. Ces avancées ont été particulièrement marquantes dans le contexte des anneaux où la réduction se révèle être un outil essentiel pour analyser les propriétés des idéaux premiers. La filtration I-adique, quant à elle, a émergé comme un concept clé dans le cadre des anneaux complets, offrant une perspective novatrice sur la convergence des suites d'éléments.

Le concept de filtrations bonnes a pris son essor dans les travaux de Jean-Pierre Serre, influent mathématicien du XXe siècle, notamment dans son ouvrage "Faisceaux Algébriques Cohérents". Les filtrations bonnes se sont révélées être des structures particulièrement adaptées à l'analyse des propriétés des anneaux locaux, ouvrant ainsi des horizons nouveaux dans la compréhension des singularités et des propriétés locales des structures algébriques.

En embrassant ce panorama historique et en soulignant les contributions majeures des éminents mathématiciens, cette étude aspire à expliciter les résultats de H. Dichi dans \cite{2} qui constitue une contribution significative à l'édifice des connaissances, en poursuivant le développement et la généralisation des notions de dépendance intégrale, de réduction, et de filtrations bonnes.

Malgré les avancées significatives réalisées dans l'étude des notions de dépendance intégrale, de réduction, et de filtrations bonnes, des questionnements demeurent quant à leur généralisation et à leur interconnexion dans des cadres mathématiques diversifiés. Comment étendre de manière rigoureuse les résultats obtenus dans le contexte restreint de la filtration I-adique à des filtrations de nature plus générale, telles que les filtrations bonnes ? Comment ces notions interagissent-elles dans des environnements mathématiques variés, et quelles applications peuvent en découler dans des domaines tels que la géométrie algébrique ou encore la théorie des nombres ? 

Afin de répondre à ces interrogations, notre démarche s'articulera autour de trois axes majeurs. Dans un premier temps, nous plongerons dans l'étude approfondie de la dépendance intégrale, en explorant ses origines historiques et en analysant ses implications dans le contexte des anneaux commutatifs. Nous poursuivrons ensuite notre exploration en examinant la réduction au sein de la filtration I-adique, avant d'élargir notre perspective à des filtrations bonnes, mettant en lumière les liens conceptuels et les différences inhérentes à ces contextes variés. En embrassant cette démarche, nous aspirons à contribuer à l'enrichissement des connaissances dans le domaine de l'algèbre commutative, tout en offrant des perspectives nouvelles sur des concepts fondamentaux de la discipline.


