\newpage
\tableofcontents
\renewcommand{\contentsname}{Table des matières}
\thispagestyle{empty}

\newpage

\setcounter{page}{0} 
\pagenumbering{arabic}
\thispagestyle{empty}
\addcontentsline{toc}{chapter}{\large{Introduction}}
\begin{center}
	\LARGE{\textbf{Introduction}}
\end{center}
\vspace{1cm}

Les notions de dépendance intégrale, de réduction et de filtrations bonnes constituent des axes cruciaux de l'Algèbre Commutative dans le domaine des mathématiques. Leur exploration minutieuse offre une clarté conceptuelle quant aux propriétés structurales des anneaux commutatifs.

La notion d'élément entier sur un idéal $I$ d'un anneau noethérien $A$ a été introduit par Prüfer \cite{2} dans les années 1930. Un élément $x \in A $ est dit entier sur l'idéal $I$ de $A$ s'il vérifie une équation de la forme $x^n + a_1 x^{n-1} +a_2 x^{n-2}+ \cdots + a_n = 0$ où $a_j$ appartient à $I^j$ pour tout entier $j$. La clôture intégrale de l'idéal $I$ est l'idéal $I'$ des éléments $x \in A$ qui sont entiers sur $I$. L'idéal $J$ de $A$ est dit entier sur l'idéal $I$ si $J \subseteq I'$.\\ Northcott D.G. et Rees D. dans \cite{4} ont défini dans un anneau local noethérien la notion de réduction d'un idéal sur un autre qui est voisine de la notion d'élément entier sur un idéal introduit par Prüfer. L'idéal $I$ est une réduction de l'idéal $J$ si $I$ est contenu dans $J$ et s'il existe un entier $N \geqslant 1$ tel que $J^N = IJ^{N-1}$. Les réductions de l'idéal $J$ sont exactement les idéaux de $I$ contenus dans $J$ et ayant la même clôture intégrale que $J$.\\
Une filtration de l'anneau $A$ est une suite $f=(I_n)_{n \in \mathbb{Z}}$ d'idéaux de $A$, décroissante pour l'inclusion et vérifiant $I_0 = A$ et $I_n I_m \subseteq I_{n+m}$. On note $\mathbb{F}(A)$ l'ensemble des filtrations de l'anneau A. Soit $I$ un idéal de $A$, une filtration $f=(I_n)_{n \in \mathbb{Z}}$ est dite $I-bonne$ si pour tout $n \in \mathbb{N}, \quad II_n \subseteq I_{n+1}$ et s'il existe $k$ un entier tel que pour tout $n \geqslant k$, $II_n = I_{n+1}$. Les filtrations les plus courantes sont les filtrations $adiques$. En particulier, toute filtrations $I-adiques$ est $I-bonne$ pour tout $I$ idéal de $A$.

Les filtrations bonnes se sont révélées être des structures particulièrement adaptées à l'analyse des propriétés des anneaux locaux, ouvrant ainsi des horizons nouveaux dans la compréhension des propriétés locales des structures algébriques.

En embrassant ce panorama historique et en soulignant les contributions majeures des éminents mathématiciens, cette étude aspire à expliciter les résultats de H. Dichi dans \cite{2} qui constitue une contribution significative à l'édifice des connaissances, en poursuivant le développement et la généralisation des notions de dépendance intégrale, de réduction, et de filtrations bonnes.

Malgré les avancées significatives réalisées dans l'étude des notions de dépendance intégrale, de réduction, et de filtrations bonnes, des questionnements demeurent quant à leur généralisation et à leur interconnexion dans des cadres mathématiques diversifiés. Comment étendre de manière rigoureuse les résultats obtenus dans le contexte restreint de la filtration I-adique à des filtrations de nature plus générale, telles que les filtrations bonnes ? Comment ces notions interagissent-elles dans des environnements mathématiques variés, et quelles applications peuvent en découler dans des domaines tels que la géométrie algébrique ou encore la théorie des nombres ? 

Afin de répondre à ces interrogations, notre démarche s'articulera autour de trois axes majeurs. Dans un premier temps, nous plongerons dans l'étude approfondie de la dépendance intégrale, en explorant ses origines historiques et en analysant ses implications dans le contexte des anneaux commutatifs. Nous poursuivrons ensuite notre exploration en examinant la réduction au sein de la filtration I-adique, avant d'élargir notre perspective à des filtrations bonnes, mettant en lumière les liens conceptuels et les différences inhérentes à ces contextes variés. En embrassant cette démarche, nous aspirons à contribuer à l'enrichissement des connaissances dans le domaine de l'algèbre commutative, tout en offrant des perspectives nouvelles sur des concepts fondamentaux de la discipline.


