\newpage
\tableofcontents
\renewcommand{\contentsname}{Table des matières}
\thispagestyle{empty}

\newpage

\setcounter{page}{0} 
\pagenumbering{arabic}
\thispagestyle{empty}
\addcontentsline{toc}{chapter}{\large{Introduction}}
\begin{center}
	\LARGE{\textbf{Introduction}}
\end{center}
\vspace{1cm}

La notion d'élément entier sur un idéal $I$ d'un anneau noethérien $A$ a été introduit dans \cite{Pr} par Prüfer dans les années 1930. Un élément $x$ de $A $ est dit entier sur l'idéal $I$ de $A$ s'il vérifie une équation de la forme $x^n + a_1 x^{n-1} +a_2 x^{n-2}+ \cdots + a_n $ qui est nulle, où les $a_j$ appartiennent à $I^j$ pour tout entier $j$ allant de 1 à $n$.\\ 
La clôture intégrale de l'idéal $I$ est l'idéal $I'$ formés des éléments $x$ de $A$ qui sont entiers sur $I$. L'idéal $J$ de $A$ est dit entier sur l'idéal $I$ si $J$ contenu ou égal à $I'$.\\ Northcott D.G. et Rees D. dans \cite{No} ont défini dans un anneau local noethérien la notion de réduction d'un idéal sur un autre qui est voisine de la notion d'élément entier sur un idéal introduit par Prüfer. L'idéal $I$ est une réduction de l'idéal $J$ si $I$ est contenu dans $J$ et s'il existe un entier $N$ supérieur ou égal à $1$ tel que $J^N$ est égal à $IJ^{N-1}$.\\
Une filtration de l'anneau $A$ est une famille $f=(I_n)_{n \in \mathbb{Z}}$ d'idéaux de $A$, décroissante pour l'inclusion et vérifiant $I_0$ est égal à $A$ et $I_n I_m$ contenu dans $I_{n+m}$. On note $\mathbb{F}(A)$ l'ensemble des filtrations de l'anneau A. Soit $I$ un idéal de $A$, une filtration $f=(I_n)_{n \in \mathbb{Z}}$ est dite $I-bonne$ si pour tout entier $n$ de $\mathbb{N}$, $II_n$ est contenu ou égal à $ I_{n+1}$ et s'il existe $k$ un entier tel que pour tout $n$ supérieur ou égal à $k$, $II_n$ est égal à $I_{n+1}$. Les filtrations les plus connues sont les filtrations $adiques$. En particulier, toute filtration $I-adique$ est $I-bonne$ pour tout $I$ idéal de $A$.

Les filtrations bonnes se sont révélées être des structures particulièrement adaptées à l'analyse des propriétés des anneaux locaux, ouvrant ainsi des horizons nouveaux dans la compréhension des propriétés locales des structures algébriques.

En embrassant ce panorama historique et en soulignant les contributions majeures des éminents mathématiciens, cette étude aspire à expliciter les résultats de Dichi H. dans \cite{Di2} qui constitue une contribution significative à l'édifice des connaissances, en poursuivant le développement et la généralisation des notions de dépendance intégrale, de réduction, et de filtrations bonnes.

Malgré les avancées significatives réalisées dans l'étude des notions de dépendance intégrale, de réduction et de filtrations bonnes des questionnements demeurent quant à leur généralisation et à leur interconnexion dans des cadres mathématiques diversifiés. Comment étendre de manière rigoureuse les résultats obtenus dans le contexte restreint de la filtration I-adique à des filtrations de nature plus générale, telles que les filtrations bonnes ? Comment ces notions interagissent-elles ? 

Afin de répondre à ces interrogations, notre démarche s'articule autour de trois axes majeurs. Dans un premier temps, nous plongeons dans l'étude approfondie de la dépendance intégrale, en explorant ses origines historiques et en analysant ses implications dans le contexte des anneaux commutatifs. Nous poursuivons ensuite, notre exploration en examinant la réduction au sein de la filtration I-adique, avant d'élargir notre perspective à des filtrations bonnes, mettant en lumière les liens conceptuels et les différences inhérentes à ces contextes variés. En embrassant cette démarche, nous aspirons à contribuer à l'enrichissement des connaissances dans le domaine de l'Algèbre Commutative, tout en offrant des perspectives nouvelles sur des concepts fondamentaux de cette discipline.


