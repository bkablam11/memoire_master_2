\documentclass[12pt, a4paper, oneside]{book}
\usepackage[utf8]{inputenc}
\usepackage[francais]{babel}
\usepackage[T1]{fontenc}
\usepackage{amsmath}
\usepackage{amsfonts}
\usepackage{arial}
\usepackage{setspace}
\usepackage{amsthm}
\usepackage{hyperref}
\usepackage{mathrsfs}
\usepackage{amssymb}
\usepackage{makeidx}
\usepackage{graphicx}
\usepackage{lmodern}
\usepackage[left=2.8cm,right=2.8cm,top=3.5cm,bottom=3.5cm]{geometry}
\usepackage{tocloft}
\usepackage{stmaryrd}
\hypersetup{colorlinks=true, urlcolor=blue,linkcolor=blue, breaklinks=true}
\usepackage{tikz}
\usepackage[none]{hyphenat} % Désactive la césure des mots

%\newtheoremstyle pour définir un nouveau style d'environnement appelé break. Ce style inclut \newline avant chaque nouvelle définition. 
%\newtheoremstyle{break}
%{\topsep}{\topsep}%
%{\itshape}{}%
%{\bfseries}{}%
%{\newline}{}%
%\theoremstyle{break}
%Redéfinir le format du numéro de chapitre 
\renewcommand{\cftchappresnum}{Chapitre } \renewcommand{\cftchapaftersnum}{:}
% Ajuster l’espace entre le numéro et le titre 
\newlength{\mylen} \settowidth{\mylen}{\cftchappresnum\cftchapaftersnum} \addtolength{\cftchapnumwidth}{\mylen}

\newtheorem{maremarque}{Remarque}[chapter]
\newtheorem{montheoreme}{Théorème}[chapter]
\newtheorem{madefinition}{Définition}[chapter]
\newtheorem{maproposition}{Proposition}[chapter]
\newtheorem{moncorollaire}{Corollaire}[chapter]
\newtheorem{monlemme}{Lemme}[chapter]
\newtheorem{monexemple}{Exemple}[chapter]
\newtheorem{maconsequence}{Conséquence}[chapter]
\newtheorem{moncontreexemple}{Contre exemple}[chapter]


\begin{document}
	\setcounter{tocdepth}{2}
	\renewcommand\thepage{\roman{page}}
	\begin{center}
	\LARGE{\textbf{Dédicace}}
\end{center}

\begin{center}
	Je dédie ce travail\\
	\uppercase{à} la mémoire de mon père KABLAM Oi Kablam Simon,\\ que ce travail soit un témoignage de ma reconnaissance pour tout ce qu'il a fait pour moi et de mon attachement indéfectible à ses principes et règles. \\
	\uppercase{à} ma mère KASSI Ettien Cécile,\\ qui m'a inculqué des valeurs de persévérance, d'honnêteté et de bonne moralité. \\
	\uppercase{à} mes frères et sœurs KABLAM Koua Marc Leroy, KABLAM Ettien Jean-Michel, KABLAM Aser Daniel, KABLAM Marie Andrée Chantale, KABLAM Adjoua Hortense Radija et à mon fils KABLAM Brou Régis Emmanuel Mael,\\ pour leur soutien tout au long de mon parcours.\\
	\uppercase{à} mes amis ASSOVIE Boka Gédéon Ruben, KOUAME Kouadio Timothé, KOUADIO Kouadio Fabrice, AMADOU Hari et N'GORAN Mouroua Sophia,\\ pour leurs encouragements et leurs conseils qui m'ont été d'une immense utilité.\\
	\uppercase{à} tous ceux que j'aime.\\
	\uppercase{à} tous ceux qui m'aiment.
	
	
\end{center} 

\newpage
\addcontentsline{toc}{chapter}{\large{Remerciements}}
\begin{center}
	\LARGE{\textbf{Remerciements}}
\end{center}
C'est avec un immense plaisir et une vive émotion que je m'apprête à soutenir mon projet de mémoire de MASTER. Je me donne, à cet effet, au délicat exercice des remerciements.

De prime abord, je rends gloire et louange à Dieu, Seigneur de l'univers, Maître de la destinée, pour la grâce qu'il m'a accordée à pouvoir garder le moral haut le long de mon parcours en dépit de nombreuses péripéties.

La réalisation de ce mémoire est le fruit de plusieurs années de formation et de dévouement. 
Par ailleurs, ce travail a été possible grâce au soutien et à la contribution de nombreuses personnes que je remercie.


Je souhaite adresser une note spéciale d'appréciation à mon Directeur de mémoire, Monsieur ASSANE Abdoulaye, Maître de Conférences à l'Université Nangui Abrogoua. Il n'a ménagé aucun effort pour donner forme à ce travail. Vos conseils éclairés et vos orientations ont été d'une grande valeur tout au long de la rédaction de ce mémoire.\\
Mes remerciements et ma profonde gratitude s'étendent également aux responsables du Laboratoire de Mathématiques et Informatiques qui, par leurs enseignements et leurs discussions, ont influencé ma progression académique.\\
J'exprime une reconnaissance infinie et mes sincères remerciements à mon encadrant scientifique et enseignant, Monsieur BROU Kouadjo Pierre, Maître Assistant à l'Université Nangui Abrogoua. Ses encouragements, ses conseils, sa disponibilité et ses critiques objectives ont été d'une aide précieuse durant de la réalisation de ce travail. Je souhaite également remercier le Professeur DIAGANA Youssouf, Directeur du Laboratoire de Mathématiques et Informatique de l'UFR-SFA (Unité de Formation et de Recherche Sciences Fondamentales et Appliquées) de l'Université Nangui Abrogoua. Il a été une source d'inspiration pour nous et nous a transmis un véritable amour pour les Mathématiques.\\
Je tiens à exprimer ma gratitude envers Monsieur BOMISSO Jean-Marc, Maître Assistant à l'Université Nangui Abrogoua, pour ses conseils et ses critiques lors de l'élaboration de ce travail, ainsi que pour son soutien et son assistance envers nous, étudiants de Master, tout au long de notre parcours.\\
Je souhaite également remercier les membres du jury, notamment Monsieur KPATA Akon Berenger, Maître de Conférences à l'Université Nangui Abrogoua, pour l'honneur qu'il nous fait en faisant partie de notre jury, ainsi que le professeur DIAGANA Youssouf pour avoir accepté de présider le jury de ce mémoire.\\
Mes remerciements s'adressent également au Directeur de l'Unité de Formation et de Recherche Sciences Fondamentales et Appliquées (UFR-SFA), Monsieur KRE N'Guessan Raymond, Professeur Titulaire, au Vice Doyen de l'UFR-SFA, le Professeur BENIE Anoubilé, ainsi qu'à la secrétaire principale, Madame KOUASSI Bienvenue, qui œuvrent inlassablement pour le bien-être des étudiants de l'UFR-SFA.\\
Enfin, je souhaite exprimer ma gratitude envers mes amis et condisciples de parcours depuis mes débuts à l'Université Nangui Abrogoua.

	\newpage
\addcontentsline{toc}{chapter}{\large{Résumé}}
\begin{center}
	\LARGE{\textbf{Résumé}}
\end{center}

Dans le cadre de ce mémoire, nous nous penchons sur une étude approfondie de la dépendance intégrale et de la réduction dans le contexte des filtrations bonnes en reprenant les travaux déjà publiés par divers chercheurs dans le domaine de la théorie des filtrations.

Notre approche consiste à étudier les concepts de dépendance et de réduction au sein des filtrations I-adiques, afin d'explorer les propriétés qui y sont associées. Nous remarquons que les filtrations I-adiques sous certaines conditions présentent toujours une réduction. Et donc sous certaines hypothèses nous tentons d'élargir ces propriétés aux filtrations f-bonnes de manière générale. 
\\
\\
\\
\\
\\
\textbf{ Mots-clés: Dépendance intégrale, Réduction, Filtration bonne.} 



\newpage
\addcontentsline{toc}{chapter}{\large{Abstract}}
\begin{center}
	\LARGE{\textbf{Abstract}}
\end{center}

As part of this dissertation, we focus on an in-depth study of integral dependence and reduction in the context of good filtrations by taking up work already published by various researchers in the field of filtration theory.

Our approach consists of studying the concepts of dependence and reduction within I-adic filtrations, in order to explore the properties associated with them. We notice that I-adic filtrations under certain conditions always present a reduction. And so under certain hypotheses we try to extend these properties to f-good filtrations in general.
\\
\\
\\
\\
\\
\textbf{Keywords: Integral dependence, Reduction, Good filtration.} 
	\newpage

\addcontentsline{toc}{chapter}{\large{Notations}}
\begin{center}
	\LARGE{\textbf{Notations}}
\end{center}


\begin{eqnarray*}
	\mathbb{K} & & \quad \text{Le  corps} \ \ \mathbb{R} \ \ \text{ou} \ \ \mathbb{C}\\
	A[X] & & \quad \text{Anneau des polynômes à coefficients dans A d'indéterminée X}\\
	\mathbb{F}(A) & & \quad \text{Ensemble des filtrations sur le module A}\\
	R(A,f) & & \quad \text{Anneau de Rees droit par rapport à la filtration f de A}\\
	\mathcal{R}(A,f) & & \quad \text{Anneau de Rees généralisé par rapport à la filtration f de A}\\
\end{eqnarray*}
	\newpage
\tableofcontents
\renewcommand{\contentsname}{Table des matières}
\thispagestyle{empty}

\newpage

\setcounter{page}{0} 
\pagenumbering{arabic}
\thispagestyle{empty}
\addcontentsline{toc}{chapter}{\large{Introduction}}
\begin{center}
	\LARGE{\textbf{Introduction}}
\end{center}
\vspace{1cm}

Les notions de dépendance intégrale, de réduction et de filtrations bonnes constituent des axes cruciaux de l'Algèbre Commutative dans le domaine des mathématiques. Leur exploration minutieuse offre une clarté conceptuelle quant aux propriétés structurales des anneaux commutatifs.

La notion d'élément entier sur un idéal $I$ d'un anneau noethérien $A$ a été introduit par Prüfer \cite{Pr} dans les années 1930. Un élément $x \in A $ est dit entier sur l'idéal $I$ de $A$ s'il vérifie une équation de la forme $x^n + a_1 x^{n-1} +a_2 x^{n-2}+ \cdots + a_n = 0$ où $a_j$ appartient à $I^j$ pour tout entier $j$. La clôture intégrale de l'idéal $I$ est l'idéal $I'$ des éléments $x \in A$ qui sont entiers sur $I$. L'idéal $J$ de $A$ est dit entier sur l'idéal $I$ si $J \subseteq I'$.\\ Northcott D.G. et Rees D. dans \cite{No} ont défini dans un anneau local noethérien la notion de réduction d'un idéal sur un autre qui est voisine de la notion d'élément entier sur un idéal introduit par Prüfer. L'idéal $I$ est une réduction de l'idéal $J$ si $I$ est contenu dans $J$ et s'il existe un entier $N \geqslant 1$ tel que $J^N = IJ^{N-1}$. Les réductions de l'idéal $J$ sont exactement les idéaux de $I$ contenus dans $J$ et ayant la même clôture intégrale que $J$.\\
Une filtration de l'anneau $A$ est une suite $f=(I_n)_{n \in \mathbb{Z}}$ d'idéaux de $A$, décroissante pour l'inclusion et vérifiant $I_0 = A$ et $I_n I_m \subseteq I_{n+m}$. On note $\mathbb{F}(A)$ l'ensemble des filtrations de l'anneau A. Soit $I$ un idéal de $A$, une filtration $f=(I_n)_{n \in \mathbb{Z}}$ est dite $I-bonne$ si pour tout $n \in \mathbb{N}, \quad II_n \subseteq I_{n+1}$ et s'il existe $k$ un entier tel que pour tout $n \geqslant k$, $II_n = I_{n+1}$. Les filtrations les plus courantes sont les filtrations $adiques$. En particulier, toute filtration $I-adique$ est $I-bonne$ pour tout $I$ idéal de $A$.

Les filtrations bonnes se sont révélées être des structures particulièrement adaptées à l'analyse des propriétés des anneaux locaux, ouvrant ainsi des horizons nouveaux dans la compréhension des propriétés locales des structures algébriques.

En embrassant ce panorama historique et en soulignant les contributions majeures des éminents mathématiciens, cette étude aspire à expliciter les résultats de Dichi H. dans \cite{Di2} qui constitue une contribution significative à l'édifice des connaissances, en poursuivant le développement et la généralisation des notions de dépendance intégrale, de réduction, et de filtrations bonnes.

Malgré les avancées significatives réalisées dans l'étude des notions de dépendance intégrale, de réduction, et de filtrations bonnes, des questionnements demeurent quant à leur généralisation et à leur interconnexion dans des cadres mathématiques diversifiés. Comment étendre de manière rigoureuse les résultats obtenus dans le contexte restreint de la filtration I-adique à des filtrations de nature plus générale, telles que les filtrations bonnes ? Comment ces notions interagissent-elles dans des environnements mathématiques variés, et quelles applications peuvent en découler dans des domaines tels que la géométrie algébrique ou encore la théorie des nombres ? 

Afin de répondre à ces interrogations, notre démarche s'articulera autour de trois axes majeurs. Dans un premier temps, nous plongerons dans l'étude approfondie de la dépendance intégrale, en explorant ses origines historiques et en analysant ses implications dans le contexte des anneaux commutatifs. Nous poursuivrons ensuite notre exploration en examinant la réduction au sein de la filtration I-adique, avant d'élargir notre perspective à des filtrations bonnes, mettant en lumière les liens conceptuels et les différences inhérentes à ces contextes variés. En embrassant cette démarche, nous aspirons à contribuer à l'enrichissement des connaissances dans le domaine de l'algèbre commutative, tout en offrant des perspectives nouvelles sur des concepts fondamentaux de la discipline.



	\chapter{GÉNÉRALITÉS}
\chaptermark{GÉNÉRALITÉS}
Ce chapitre revêt la forme d'une rétrospective, visant à revisiter et consolider diverses définitions dans \cite{5} et propositions évoquées au sein des Unités d'Enseignement (U.E.) consacrées à l'Algèbre Commutative et à la Théorie des Filtrations. L'ensemble des résultats présentés dans ce chapitre s'enracine dans \cite{6}. Leur maîtrise s'avère incontournable pour une appréhension éclairée des sections à venir. Cependant, il convient de préciser que les démonstrations des propositions énoncées dans ce chapitre ne seront exposées que lorsqu'elles s'avéreront nécessaires. \\\\ Dans l'ensemble de ce mémoire, sauf mention contraire, les anneaux considérés sont supposés \textbf{commutatifs et unitaires}.
\section{Relation d'équivalence, groupe et sous-groupe}
\subsection{Relation d'équivalence}
\begin{madefinition}
	On appelle relation d’équivalence sur un ensemble non vide E , une relation
	$\mathcal{R}$ sur E vérifiant les conditions suivantes :
	\begin{enumerate}
		\item[a) ] $\forall x \in E, x\mathcal{R}x$ (réflexivité)
		\item[b) ] $\forall x,y \in E, x\mathcal{R}y \Longleftrightarrow y\mathcal{R}x $ (symétrie)
		\item[c) ] $\forall x,y,z \in E, (x\mathcal{R}y)$ et $(y\mathcal{R}z)$ $\Longrightarrow (x\mathcal{R}z)$  (transitivité)
	\end{enumerate}
	La partie $C_x = \{y \in E,\quad x\mathcal{R}y\}  $ de E est appelée la classe d’équivalence modulo $\mathcal{R}$ de $x \in E$. Les	classes d’équivalence constituent une partition de E . Notons $\mathcal{P}(E)$ l’ensemble des parties de E. La famille $(C_x)_{x \in E}$ constitue un sous-ensemble de $\mathcal{P}(E)$ appelé l’ensemble quotient de E par R. On le note $E/\mathcal{R}$. 
	Dans la suite, la classe de $x \in E$ sera vue essentiellement comme élément de ce nouvel ensemble $E/\mathcal{R}$ et sera notée $\bar{x}$
\end{madefinition}
\subsection{Relation d'ordre}
\begin{madefinition}
	On appelle relation d’ordre sur un ensemble non vide E , une relation
	$\mathcal{R}$ sur E vérifiant les conditions suivantes :
	\begin{enumerate}
		\item[a) ] $\forall x \in E, x\mathcal{R}x$ (réflexivité)
		\item[b) ] $\forall x,y \in E, x\mathcal{R}y $ et $y\mathcal{R}x$ $ \Longleftrightarrow y = x $ (anti-symétrie)
		\item[c) ] $\forall x,y,z \in E, (x\mathcal{R}y)$ et $(y\mathcal{R}z)$ $\Longrightarrow (x\mathcal{R}z)$  (transitivité)
	\end{enumerate}
	L'ensemble $(E, \mathcal{R})$ s'appelle alors ensemble ordonné. L'ordre est dit total si deux éléments sont toujours comparables (on dit aussi que l'ensemble est totalement ordonné). Dans le cas contraire, il est dit partiel.
\end{madefinition}

\subsection{Groupe}
\begin{madefinition}
	%\label{madef1} [\ref{madef1}]
	Soit G un ensemble non vide.\\
	On dit que $(G, \star)$ est un groupe si $\star$ est une loi de composition qui a tout élément de $G$ associe un élément dans $G$ vérifiant:
	\begin{enumerate}
		\item[(i)] La loi $\star$ est associative 
		\[ a\star(b\star c) = (a\star b) \star a, \forall (a,b,c) \in G^3,\]
		\item[(ii)] G possède un élément neutre $e$
		\[ \exists!e \in G, e\star a = e =  a\star e, \forall a \in G,\]
		\item[(iii)] Tout élément $a$ de $G$ admet un symétrique
		\[ a\star b = e = b \star a, \forall (a,b) \in G^2.\]
		\item [(vi)] Si de plus la loi $\star$ est commutative, 
		\[ a\star b = b \star a, \forall (a,b) \in G^2,\]
		On dit que le groupe $(G,\star)$ est abélien ou commutatif.
	\end{enumerate}
\end{madefinition}
\subsection{Sous-groupe}
\begin{madefinition}
	Soit $(G, \star)$ un groupe. \\
	On dit que $H \subset G$ est un sous-groupe de $G$ si et seulement si:
	\begin{enumerate}
		\item[(i)] e $\in$ H, où $e$ est l'élément neutre de $G$
		\item[(ii)] La partie H est stable par la loi :
		\[\forall (a,b) \in H^2, a \star b \in H\]
		\item[(iii)] $\forall a \in H, a^{-1} \in H$.
	\end{enumerate}
\end{madefinition}

\section{Anneau, Corps, Espace vectoriel et Morphisme d'anneaux}
\subsection{Anneau}
\begin{madefinition}
	Soit $A$ un ensemble non vide muni de deux lois de compositions interne, $(A, +, \times)$. On dit que $(A, +, \times)$ est un anneau si :
	\begin{enumerate}
		\item[(i)] $(A,+)$ est un groupe abélien
		\item[(ii)] La loi $\times$ est distributive par rapport à la loi $+$
		\item[(iii)] La loi $\times$ est associative.
	\end{enumerate}
	L'anneau est dit commutatif si la loi $\times$ est commutatif.
\end{madefinition}
\begin{monexemple}
	$(\mathbb{Z}, +, .), (\mathbb{R}, +, .)$ et $(\dfrac{\mathbb{Z}}{n\mathbb{Z}}, +,.)$ sont des anneaux commutatifs unitaires.
\end{monexemple}
\begin{maremarque}
	L'élément neutre de la loi $+$ dans $A$ est noté $0_A$. Si de plus il existe un élément neutre pour la loi $\times$ dans $A$, cet élément est appelé l'élément unité et est noté $1_A$. On dit alors que l'anneau $A$ est unitaire.
\end{maremarque}
\begin{madefinition}
	Soient $(A, + , \times)$ un anneau et $B$ un sous-ensemble de $A$.\\ On dit que $B$ est un sous-anneau de $A$ si :
	\begin{enumerate}
		\item[(i)] $1_A \in B$,
		\item[(ii)] Pour tous $x, y \in B, x-y \in B$,
		\item[(iii)] Pour tous $x, y \in B, xy \in B$.
	\end{enumerate}
\end{madefinition}
\begin{maproposition}
	(Anneau noethérien)\\
	Soient $A$ une anneau commutatif unitaire. On dit que $A$ est un anneau noethérien si l'une des trois assertions équivalentes sont vérifiée:
	\begin{enumerate}
		\item[(i)] Tout idéal de $A$ est de type fini,
		\item[(ii)] Toute suite croissante d'idéaux de $A$ est stationnaire,
		\item[(iii)] Toute famille non vide d'idéaux de $A$ admet un élément maximal pour l'inclusion.
	\end{enumerate} 
\end{maproposition}
\subsection{Corps}
\begin{madefinition}
	On appelle \textbf{corps} (commutatif), la donnée d'un ensemble $\mathbb{K}$ et de deux lois de composition interne l'une notée $+$ et l'autre multiplicative notée $\times$ sur $\mathbb{K}$ telles que:
	\begin{enumerate}
		\item $+$ est un groupe
		\item $\times $ vérifie (i), (ii), (iii) pour tout élément \textbf{non nul} et distributive (à droite et à gauche) par rapport à la loi $+$ c'est-a-à-dire:
		\[ (a+b)\times c = a \times c + b \times a \text{ et } c \times (a+b) = c \times a + c \times b \quad \forall (a,b,c) \in \mathbb{K}^3.\]
		\item $\times $ est commutative.
	\end{enumerate}
\end{madefinition}
\begin{monexemple}
	$\mathbb{Q}, \mathbb{R}$ et $\mathbb{C}$ sont des corps. 
\end{monexemple}
\subsection{Espace vectoriel}
\begin{madefinition}
	Soit $\mathbb{K}$ un corps et soit $E$ un ensemble (non vide). On appelle \textbf{loi externe} sur $E$ la donnée d'une application définie sur le produit cartésien $\mathbb{K} \times E$ et à valeur dans $E$.
\end{madefinition}
\begin{madefinition}
	Soit $\mathbb{K}$ un corps. On dit qu’un ensemble $E$ est un $\mathbb{K}$-espace vectoriel
	lorsqu’il est muni d’une loi de composition interne commutative, notée + et d’une
	loi externe de $\mathbb{K}$, notée $\times$, telle que:
	\begin{enumerate}
		\item (E, $+$) est un groupe additif d'élément neutre noté $0_E$ ou $0$.
		\item Les lois $+$ et $\times$ sont compatibles entre elles et avec les lois de la structure de corps de $\mathbb{K}$. C'est-à-dire qu'elle vérifient:
		\begin{enumerate}
			\item (a) $\forall \lambda \in \mathbb{K}, \forall u,v \in E, \lambda \times (u+v) = \lambda \times u + \lambda \times v$
			\item (b) $\forall \lambda , \mu \in \mathbb{K}, \forall u \in E, (\lambda \times \mu)\times u 
			= \lambda \times(\mu \times u) $
			\item (c) $\forall \lambda, \mu \in \mathbb{K}, \forall u \in E, (\lambda + \mu)\times u = (\lambda \times u) + (\mu \times v)$
			\item (d) $\forall u \in E, 1_{\mathbb{K}}.u=u$
		\end{enumerate}
	\end{enumerate}	
\end{madefinition}
\begin{monexemple}
	Voici des exemples d'espaces vectoriels:
	\begin{enumerate}
		\item $\{0_E\}$ l'espace vectoriel réduit à $0$, sur n’importe quel corps $\mathbb{K}$. 
		\item Le corps $\mathbb{K}$ lui-même est un espace vectoriel sur est $\mathbb{K}$.
		\item  L'ensemble des polynômes $\mathbb{K}[X]$ en une indéterminée $X$ et à coefficients dans $\mathbb{K}$ est un $\mathbb{K}$-espace vectoriel.
	\end{enumerate}
\end{monexemple}
\begin{maremarque}
	La notion d’espace vectoriel : pour en donner un aperçu intuitif ce sont des généralisations de l’ensemble $\mathbb{R}^n$ (produit cartésien de $\mathbb{R}$ avec lui-même $n$ fois), et des opérations linéaires sur cet ensemble. Plus précisément un espace vectoriel est défini à partir d’un ensemble $E$. Les éléments de $E$ sont appelés \textbf{vecteurs}. On associe implicitement à $E$ un \textbf{corps} de base noté $\mathbb{K}$ dont les éléments sont appelés les \textbf{scalaires} (pour $\mathbb{R}^n$ le corps $\mathbb{K}$ est égal à $\mathbb{R}$). On dit alors que $E$ est un $\mathbb{K}$-espace vectoriel si on peut effectuer dans $E$ des opérations $\mathbb{K}$-linéaires, c’est-à-dire si on sait définir le vecteur $\lambda.u+\mu.v$ à partir des données de deux vecteurs $u$ et $v$ et de deux scalaires $\lambda$ et $\mu$.
\end{maremarque}

\begin{madefinition}
	Soient $E$ et $F$ deux $\mathbb{K}$-espaces vectoriels. Une application f :$E \longrightarrow F$
	est dite linéaire si elle est compatible avec les structures de $\mathbb{K}$-espaces sur $E$ et
	sur $F$. Autrement dit si $f$ vérifie : 
	\begin{enumerate}
		\item $\forall \lambda \in \mathbb{K}, \forall u \in E, f(\lambda \times u) = \lambda \times f(u)$
		\item $\forall u,v \in E, f(u+v) = f(u)+f(v)$
	\end{enumerate}
\end{madefinition}

\begin{maremarque}
	La notion d’applications linéaires entre espaces vectoriels. Ce sont des applications entre deux espaces vectoriels, disons $E$ et $F$, sur un même corps de base, disons $\mathbb{K}$. Pour être appelées linéaires, ces applications doivent être compatibles avec les opérations $\mathbb{K}$-linéaires. La définition de cette notion est très simple. Mais son utilisation est vraiment au cœur de l’algèbre linéaire. On peut même dire que toute l’algèbre consiste à manipuler des applications compatibles avec les structures algébriques (on parle de “morphismes” dans d’autres contextes)
\end{maremarque}
\subsection{Morphisme d'anneaux}
\begin{madefinition}
	Soient deux anneaux $(A,\star, \circ )$ et $(B, \ast, \diamond)$.\\
	Une application $f:A \longrightarrow B$ est un morphisme d'anneaux  ou homomorphisme si et seulement si:
	\[ f(x \star y) = f(x) \ast f(y) \quad et \quad f(x \circ y) = f(x) \diamond f(y), \forall (x,y) \in A^2.\]
	On dit de plus que f est un :\\
	- Endomorphisme lorsque A = B \\
	- Isomorphisme lorsque $f$ admet une fonction réciproque $g$ telle que $f \circ g = Id_B$ et $g \circ f = Id_A$ 
\end{madefinition}
\begin{maproposition}
	(Théorème d'isomorphisme d'anneaux)\\
	Soit $\varphi:A \longrightarrow B$ un morphisme d'anneaux.\\
	Son noyau Ker($\varphi$) est un idéal de $A$ et son image Im($\varphi$) est un sous-anneau de B.\\ De plus, on a que:
	\[ \dfrac{A}{Ker(\varphi)} \simeq  Im(\varphi).\]
	Avec Ker($\varphi$) = $\left\{a \in A : \varphi(a) = 0 \right\}$ 
	et Im($\varphi$) = $\left\{\varphi(a) : a \in A \right\}$.
\end{maproposition}
\subsection{Anneau gradué}
\begin{madefinition}
	Soit $A$ un anneau.\\
	Une \textbf{graduation sur A} est une famille $(A_n)_{n \in \mathbb{Z}}$ de  sous-groupes stables de $(A,+)$ vérifiant:
	\begin{enumerate}
		\item[i)] $ A_p A_q \subset A_{p+q} $
		\item[ii)] $ A =\displaystyle \bigoplus_{n \in \mathbb{Z}}{A_n} $
	\end{enumerate}
	Si $\forall n < 0, A_n = \left\{0_A\right\}$, on dit que \textbf{A est positivement gradué} ou que \textbf{A est gradué de type n}. Ainsi $ A =\displaystyle \bigoplus_{n \in \mathbb{N}}{A_n} $. Les éléments de $A_n$, pour tout $n \in \mathbb{N} $ sont dits de degré $n$.\\
	De plus $ A =\displaystyle \bigoplus_{n \in \mathbb{N}}{A_n} $ si et seulement si:
	\[ \forall x \in A, \exists x_i \in A, \text{tel que } x = \sum_{finie}^{} x_i \text{ (Existence et Unicité)}. \]
\end{madefinition} 
\begin{maremarque}
Si $ A =\displaystyle \bigoplus_{n \in \mathbb{N}}{A_n} $ alors:
$$  
\begin{cases}
	1_A \in A_0,\\
	A_0 \text{ est un sous-anneau de }A. 
\end{cases}
$$
\newline
\begin{monexemple}
	(Exemple d'anneau gradué)\\
	\item[1)] $A[X] =  A =\displaystyle \bigoplus_{n \in \mathbb{N}}{A_n} =\displaystyle \bigoplus_{n \in \mathbb{N}}{A}X^n $,
	avec $A_n=AX^n = \left\{\alpha X^n, \alpha \in A \right\}$.
	\item[2)] $A[\dfrac{1}{X}, X] = A[X^{-1}, X] = \displaystyle \bigoplus_{n \in \mathbb{Z}}{A_n}$, avec $A_n=AX^n, n \in \mathbb{Z}$
\end{monexemple}
\end{maremarque}
\section{Module}
\begin{madefinition}
	Soit $A$ un anneau commutatif unitaire.\\
	Un $A$-module ou module sur $A$ est un groupe abélien $(M,+)$ muni d'une multiplication externe $A \times M \rightarrow M, (a,x) \mapsto ax$ vérifiant les propriétés suivantes:
	\begin{enumerate}
		\item[(i)]$ \forall a, b \in A, \forall x \in M,(a+b)x = ax+bx$;
		\item[(ii)] $ \forall a \in A, \forall x, y \in M,a(x+y) = ax+ay$;
		\item[(iii)] $ \forall x \in M, \,1_A x = x$;
		\item[(iv)] $\forall a, b \in A, \forall x \in M, \, a(bx)=(ab)x$.
	\end{enumerate}
\end{madefinition}
\begin{maremarque}
	\begin{enumerate}
		\item[(i)] Si $A$ est un anneau, alors $A$ est un $A-module$.
		\item[(ii)] Un espace vectoriel sur $A$ est un module sur $A$, si $A$ est un corps.\\ Cette propriété n'est pas vrai en générale.
	\end{enumerate}
\end{maremarque}
\begin{madefinition}
	Soit $M$ un $A$-module. Un sous-module de $M$ est un sous-ensemble $N$ de $M$ tel que :
	\begin{enumerate}
		\item[(i)]$ \, 0_M \in N$;
		\item[(ii)]$ \forall x, y \in N \, \, x+y \in N$;
		\item[(iii)] $\forall x \in N, \forall a \in A \, \, ax \in N$.
	\end{enumerate}
\end{madefinition}
\begin{madefinition}
	Soit $M$ un $A$-module. $M$ est dit de type fini s'il admet un système générateur fini, $\exists \, \, x_1, \cdots ,x_r \in M$ tel que $\forall \,  x \in M, x = \displaystyle \sum_{i=1}^{r}{a_i x_i}$, $a_i \in A$.
\end{madefinition}
\begin{maproposition}
	(Module noethérien)\\
	Soient $A$ une anneau et $M$ un $A-module$. On dit que $M$ est un module noethérien si l'une des trois assertions équivalentes sont vérifiée:
	\begin{enumerate}
		\item[(i)] Tout sous-module de $M$ est de type fini,
		\item[(ii)] Toute suite croissante de sous-modules de $M$ est stationnaire,
		\item[(iii)] Tout ensemble non vide de sous-modules de $M$ admet un élément maximal pour l'inclusion.
	\end{enumerate} 
\end{maproposition}
\begin{madefinition}(A-algèbre de type fini)\\
	Soit $A$ un anneau. \\
	On dit que $B$ est une $A-algèbre$ de type fini si:
	\begin{enumerate}
		\item[(i)] $B$ est un $A-module$
		\item[(ii)] $B$ est un anneau
		\item[(iii)] $B = A [b_1, \cdots, b_r] \simeq \dfrac{A[X_1, \cdots, X_r]}{J}, \quad b_i \in B , \quad J$ idéal de $A[X_1, \cdots, X_r]$
	\end{enumerate}
\end{madefinition}
\section{Idéal}
\begin{madefinition}
	Soit $A$ un anneau.\\
	Un idéal de $A$ est une partie $I$ de $A$ vérifiant les propriétés suivantes : \\
	\begin{enumerate}
		\item[(i)] $0_A \in I$,
		\item[(ii)]$ pour \ tout \ x, y \in I, x+y \in I$
		\item[(iii)]$pour \ tout \ a \in A$ et $x \in I , ax \in I$.
	\end{enumerate}
\end{madefinition}
\begin{monexemple}
	\item[(i)] Les idéaux de $\mathbb{Z}$ sont de la forme $n\mathbb{Z}$, avec $n \in \mathbb{N}$.
	\item[(ii)] Si $A$ est un corps (un anneau dans lequel tout élément non nul est inversible) alors $I=(0)$ ou $I=A$
\end{monexemple}
\subsection{Idéal premier}
\begin{madefinition}
	Soit $A$ un anneau. Un idéal $P$ de $A$ est dit \textbf{premier} s'il est strict et si pour tout $x,y$ deux éléments de $A$ tels que $xy \in P$ alors: $$x \in P \text{ \quad ou \quad} y\in P$$
	On note $Spec(A)$  l'ensemble des idéaux premiers de l'anneau A. 
\end{madefinition}
\subsection{Idéal primaire}
\begin{madefinition}
	Soit $A$ un anneau. Un idéal $Q$ de $A$ est dit \textbf{primaire} s'il est strict et si pour tout $x,y$ deux éléments de $A$ tels que $xy \in Q$ alors: $$x \in Q \text{ \quad ou \quad} \exists n \in \mathbb{N}, y^{n} \in Q$$.
\end{madefinition}
\subsection{Idéal maximal}
\begin{madefinition}
	Soit $A$ un anneau. Un idéal $I$ de $A$ est dit \textbf{maximal} s'il est strict et s'il n'est contenu dans aucun autre idéal strict de $A$.
\end{madefinition}
\begin{maremarque}
	Un anneau qui ne possède qu'un seul idéal maximal est appelé \textbf{anneau local}.
\end{maremarque}
\subsection{Radical d'un idéal}
\begin{madefinition}
	Soient $A$ un anneau, $I$ un idéal de $A$. On appelle radical de $I$
	\[\sqrt[]{I} = \{ a \in A, \exists n \in \mathbb{N}, a^n \in I \} \]
\end{madefinition}
\begin{maproposition}
	Soient $A$ un anneau et $I$ un idéal de $A$.
	\begin{enumerate}
		\item[(i)] $\sqrt{\sqrt{I}}=\sqrt{I}$
		
		\item[(ii)] $\sqrt{IJ}=\sqrt{I\cap J}=\sqrt{I}\cap \sqrt{J}$
		
		\item[(iii)] $\sqrt{I+J}=\sqrt{\sqrt{I}+\sqrt{J}}$
		
		\item[(iv)] Si $P$ $\in Spec(A),\sqrt{P}=P$
	\end{enumerate}
\end{maproposition}
\begin{monexemple}
	$A=\mathbb{Z},$ $I=6\mathbb{Z}$
	
	$\sqrt{I}=\sqrt{6\mathbb{Z}}=6\mathbb{Z}$
	
	$\sqrt{24\mathbb{Z}}=\sqrt{(2^{3}\times 3)\mathbb{Z}}=\sqrt{2^{3}\mathbb{Z}}\cap \sqrt{3\mathbb{Z}}=2\mathbb{Z}\cap 3\mathbb{Z}=ppcm(2,3)\mathbb{Z}=6\mathbb{Z}$
\end{monexemple}
\section{Filtrations}
\subsection{Filtration sur un anneau}
\begin{madefinition}
	Soit $A$ un anneau. On appelle filtration de $A$ toute famille\\ $f = (I_n)_{n \in \mathbb{Z}}$ d'idéaux de $A$ telle que:\\
	\begin{enumerate}
		\item[(i)] $I_0 = A$ \\
		\item[(ii)] Pour tout $n \in \mathbb{Z}, I_{n+1} \subseteq I_n$.\\
		\item[(iii)] Pour tout $p,q \in \mathbb{Z}, I_pI_q \subseteq I_{p+q}$.\\
	\end{enumerate}
	L'ensemble des filtrations de l'anneau A est noté $\mathbb{F}(A)$. Pour tout $f,g \in \mathbb{F}(A)$, cet ensemble est ordonné par:
	\[\forall n \in \mathbb{Z}, f = (I_n) \leqslant g = (J_n) \implies  I_n \subseteq J_n \]
\end{madefinition}

\begin{maremarque}
	Dans la définition précédente, il est facile de remarquer que\\ $\forall n \leq 0, I^n = A$.\\ En effet, en utilisant la décroissance des idéaux (ii) et que $I_0 = A$ (i), il vient d'une part que:\\ $I_0 \subseteq I_n , \forall n \leq 0$ (ii). Ainsi $A \subseteq I_n$.\\
	D'autre part, comme $\forall n \in \mathbb{Z}$, les $I_n$ sont des idéaux de $A$, alors $I_n \subseteq A$.\\
	Donc $I_n = A, n \leq 0$.
\end{maremarque}
Ainsi au lieu d'étudier la famille $f = (I_n)_{n \in \mathbb{Z}}$ nous pouvons nous ramener à étudier la famille $f = (I_n)_{n \in \mathbb{N}}$.

\begin{monexemple}
	\begin{enumerate}
		\item 	Soit $I$ un idéal de $A$ et $f=(I_n)_{n \in \mathbb{Z}}$ telle que pour tout $n \in \mathbb{Z}$:
		$$I_{3n} = I_{3n-1} = I_{3n-2} =I^{n} $$
		\item 	Soit $A = \dfrac{\mathbb{Z}}{4 \mathbb{Z}} $ et $f=(I_n)_{n \in \mathbb{Z}}$ telle que pour tout $n \in \mathbb{Z}$:
		$$I_1 = I_2 = (\bar{2})$$
		$$I_n = (\bar{0})  \text{ pour tout} \quad n \geqslant 3 $$
	\end{enumerate}
\end{monexemple}
\begin{madefinition}
	On définit pour toutes filtrations $f=(I_n)$ et $g=(J_n)$ de $A$ les trois opérations suivantes: 
	\begin{enumerate}
		\item[(1)] le produit $fg=(I_nJ_n)$
		\item[(2)] l'intersection $f \cap g = (I_n \cap J_n)$
		\item[(3)] la somme $f+g=(K_n)$ où pour tout entier $n$, $K_n =\displaystyle  \sum_{k=0}^{n}I_{k}J_{n-k} $
	\end{enumerate}
	On vérifie que $fg, f \cap g , f + g$ sont des filtrations de $A$ et pour toutes filtrations \\ $f,g \in \mathbb{F}(A):$
		\begin{enumerate}
		\item[(4)] $\inf (f,g) = f \cap g$
		\item[(5)] $\sup (f,g) = f+g $
		\item[(6)] $fg \leqslant f \cap g \leqslant f \leqslant f+g$
	\end{enumerate}
\end{madefinition}

\subsection{Filtration sur un module}
\begin{madefinition}
	Soit $M$ un $A$-module. On appelle filtration de $M$ toute famille $\varphi = (M_n)_{n \in \mathbb{Z}}$ de sous-modules de $M$ telle que:\\
	\begin{enumerate}
		\item[i)] $M_0 = M$ \\
		\item[ii)] pour tout $n \in \mathbb{Z}, M_{n+1} \subseteq M_n$.\\
	\end{enumerate}
	
	La filtration $f = (I_n)_{n \in \mathbb{Z}}$ de $A$ et la filtration $\varphi = (M_n)_{n \in \mathbb{Z}}$ du $A$-module $M$ sont dites compatibles si:
	\[ I_p M_q \subseteq M_{p+q} ,\, \forall \, p, q \in \mathbb{Z}. \]
\end{madefinition}
\section{Anneaux gradués associés à une filtration}
\subsection{Anneau de Rees d'une filtration}
\begin{madefinition}
	Soit $f=(I_n)_{n \in \mathbb{Z}}$ une filtration de l'anneau A.\\ X est une indéterminée.\\
	On appelle \textbf{anneau de Rees} de f, l'anneau gradué noté R(A,f) tel que: 
	\[ R(A,f)  =\displaystyle \bigoplus_{n \in \mathbb{N}}{I_n X^n}  \]
	
	On appelle \textbf{anneau de Rees généralisé} de f, l'anneau gradué noté $\mathcal{R}(A,f)$ tel que: 
	\[ \mathcal{R}(A,f) = \displaystyle \bigoplus_{n \in \mathbb{Z}}{I_n X^n}  \]
	On munit $R(A,f)$ et $\mathcal{R}(A,f)$ d'une structure de sous-anneau gradué de $A[X]$, l'ensemble des polynômes d'indéterminée $X$ à coefficients dans A dont les opérations sont définies par:
	
	\begin{enumerate}
		\item la multiplication:	$(\sum\limits_{n=0}^{r}a_{n}X^{n})(\sum\limits_{k=0}^{s}b_{k}X^{k})=\sum%
		\limits_{p=0}^{r+s}c_{p}X^{p}=$ $\sum\limits_{p=0}^{r+s}\sum%
		\limits_{q=0}^{p}a_{p-q}b_{q}X^{p}$
		\item l'addition:   $\sum\limits_{n=0}^{r}a_{n}X^{n}+\sum\limits_{k=0}^{s}b_{k}X^{k}=\sum%
		\limits_{j=0}^{r}(a_{j}+b_{j})X^{j}$
	\end{enumerate}
\end{madefinition}
\subsection{Anneau gradué d'une filtration}
\begin{madefinition}
	Soient A un anneau, $M$ un A-module
	
	$f=(I_{n})_{n\in \mathbb{Z}}\in \mathbb{F}(A),$ $\phi =(M_{n})_{n\in \mathbb{Z}}\in \mathbb{F}(M)$ telle que $\phi $ est $f-compatible$. On pose:
	
	$G(A,f)= G_{f}(A)=\displaystyle \bigoplus_{n \in \mathbb{Z}}{\frac{I_{n}}{I_{n+1}}} = \displaystyle \bigoplus_{n \in \mathbb{N}}{\frac{I_{n}}{I_{n+1}}} $
	
	$G(A, \phi)=G_{\phi }(M)=\displaystyle \bigoplus_{n \in \mathbb{Z}}{\frac{M_{n}}{M_{n+1}}}$
	
	$G_{f}(A)$ est muni d'une structure d'anneau gradué dont la
	multiplication est définie par:
	
	Pour tout $a_{n}+I_{n+1},b_{p}+I_{p+1},$ deux éléments homogènes de degré $n$
	de $G_{f}(A),$ On pose:
	
	$(a_{n}+I_{n+1})(b_{p}+I_{p+1})=a_{n}b_{p}+I_{n+p+1}$
\end{madefinition}
\begin{maproposition}
	Soient A un anneau, I un idéal de A. Soient $f = (I_n)_{n \in \mathbb{Z}} \in \mathbb{F}(A)$ et $X$ une indéterminée.
	\begin{enumerate}
		\item[(i)] $A \subseteq R(A,f) \subseteq A[X]$
		\item[(ii)] $R(A,f) \subseteq \mathcal{R}(A,f) \subseteq A[u,X]$ où $u = \dfrac{1}{X} = X^{-1}$
		\item[(iii)] $\mathcal{R}(A,f) = R(A,f)[u] $
		\item[(iv)] $u^n\mathcal{R}(A,f) \cap  A = I_n, \forall n \in \mathbb{Z}$
		\item[(v)] $R(A,f) = A[I_1X,I_2X^2, \cdots, I_nX^n, \cdots] $
		\item[(vi)] $u=\dfrac{1}{X} = X ^{-1}$ est régulier de degré -1 dans $\mathcal{R}(A,f)$ et on a: \[ \mathcal{R}(A,f) = A[u,I_1X,I_2X^2, \cdots, I_nX^n, \cdots ] \]
		\item[(vii)] $ \dfrac{R(A,I)}{IR(A,I)} \simeq G_I(A) $ avec $R(A,I) = R(A, f_I) $ et $G_I(A) = G_{f_{I}}(A) = \displaystyle \bigoplus_{n \in \mathbb{N}}{\frac{I^{n}}{I^{n+1}}} $
		\item[(viii)] $ \dfrac{\mathcal{R}(A,I)}{u\mathcal{R}(A,I)} \simeq G_I(A) $
		\item[(ix)] $ \dfrac{\mathcal{R}(A,I)}{(u-1)\mathcal{R}(A,I)} \simeq A $
	\end{enumerate} 
\end{maproposition}
\subsection{Quelques exemples de filtration}
\subsubsection{a) \underline{Filtration I-adique}}
\begin{madefinition}
	Soient $A$ un anneau et $I$ un idéal de A.\\
	La famille $f_I = (I^n)_{n\in \mathbb{Z}}$ telle que pour tout $n \leqslant 0, I^n = A$ est une filtration de A appelé \textbf{filtration I-adique} et noté $f_I$
\end{madefinition}
Soit I un idéal de l'anneau $A$ si $f_I$ est la filtration $I-adique$ alors l'anneau de Rees de $f_I$ sera simplement noté $R(A,I)$.
\begin{maproposition}
	Soient I un idéal de l'anneau $A$ et $f_I$ est la filtration $I-adique$.
	Si $I$ est de type fini avec $I = (a_1, a_2, \cdots, a_r)$ alors $R(A,I) = A[a_1X,a_2X, \cdots, a_rX] $
\end{maproposition}
\begin{maconsequence}
	Si $A$ est un anneau \textbf{noethérien} alors $R(A,I)$ est aussi \textbf{noethérien}. \\
	Cependant, il est important de noter que si $A$ n'est pas noethérien alors $R(A,f)$ n'est pas \textbf{nécessairement} noethérien.
\end{maconsequence}
\begin{monexemple}
	En effet, supposons que $A$ n'est pas noethérien.
	Posons $I = (0)$, l'idéal nul. Alors $R(A,I) = A$ et donc $R(A,I)$ n'est pas noethérien
\end{monexemple}
%\subsubsection{b) \underline{Filtration extraite d'ordre k, $k \in \mathbb{N}$}}
%\begin{madefinition}
%	Soient $A$ un anneau et $I$ un idéal de A.\\
%	$f = (I_n)_{n\in \mathbb{Z}}$ une filtration de l'anneau A.\\
%	On pose $f^{(k)} = (I_{nk})_{n\in \mathbb{Z}}$.\\
%	$f^{(k)} $ est une filtration de A appelé \textbf{filtration extraite d'ordre k}.
%\end{madefinition}

\subsubsection{b) \underline{Filtration tronqué d'ordre k de $f$}}
\begin{madefinition}
	Soient $A$ un anneau et $I$ un idéal de A.\\
	$f = (I_n)_{n\in \mathbb{Z}}$ une filtration de l'anneau A.\\
	Soit $k \in \mathbb{N}^{*}$, on pose $f^{(k)} = (I_{nk})_{n\in \mathbb{N}}$ et $t_{k}f=(K_n)$ avec $K_n = I_{n+k}$ si $n \geqslant 1 $ et $K_n = A$ si $n \leqslant 0$.\\
	Ainsi $t_{k}f$ est une filtration de $A$ appelé \textbf{filtration tronqué d'ordre k de $f$}.
\end{madefinition}

\subsubsection{c) \underline{Filtration définie par une graduation}}
\begin{madefinition}
	Soit $A = \displaystyle \bigoplus_{n \in \mathbb{N}}{A_n}$ un anneau gradué de type $n$.\\ Posons $J_n = \displaystyle \bigoplus_{n \geqslant p }{A_p}$.\\
	$f=(J_n)_{n \in \mathbb{N}}$ est une filtration de A. 
\end{madefinition}
\subsubsection{d) \underline{Filtration de type fini}}
\begin{madefinition}
	Soient $A$ un anneau et $I$ un idéal de A.\\
	$f = (I_n)_{n\in \mathbb{Z}}$ une filtration de l'anneau A.\\
	On dit que $f$ est de type fini si $I_n$ est de type fini pour tout $n \in \mathbb{N}$ assez grand.
\end{madefinition}

\subsection{Classification des filtrations sur un anneau}
\subsubsection{a) \underline{Filtrations I-bonnes}} 
\begin{madefinition}
	Soient $A$ un anneau et $I$ un idéal de l'anneau $A$.\\
	$f = (I_n)_{n \in \mathbb{Z}}$ de $A$. $f$ est dite $I$-bonne si : \\
	\begin{enumerate}
		\item[i)]$ II_n \subseteq I_{n+1} \, \forall n \in \mathbb{Z}$.\\
		\item[ii)]$\exists \, n_0 \in \mathbb{N}$ tel que $II_n = I_{n+1}, \forall n \geqslant n_0$
	\end{enumerate}
\end{madefinition}
\begin{maconsequence}
	$II_{n_0} = I_{n_{0}+1}$, en multipliant par $I$, on a:\\ $I^{2}I_{n_0} = II_{n_{0}+1}$ et $II_{n_0+1} = I_{n_{0}+2}$. \\
	Ainsi par récurrence, on obtient $I^{n}I_{n_0} = I_{n_{0}+n},$ pour tout $n \geqslant 1$ 
\end{maconsequence}
\subsubsection{b) \underline{Filtrations A.P.}}
\begin{madefinition}
	La filtration $f = (I_n)_{n \in \mathbb{Z}}$ de l'anneau $A$ est dite \textbf{Approximable par Puissances d'idéaux} (en abrégé \textbf{A.P.}) s'il existe un entier $(k_{n})_{n \in \mathbb{N}}$ une suite d'entiers naturels telle que :
	\begin{enumerate}
		\item[(i)] $\forall$ n,m $\in \mathbb{N}$, $I_{mk_n} \subset I_n^{m}$
		\item[(ii)] $\underset{n\longrightarrow +\infty }{\lim }\dfrac{k_{n}}{n}=1$
	\end{enumerate}
\end{madefinition}
\subsubsection{c) \underline{Filtrations fortement A.P.}}
\begin{madefinition}
	La filtration $f = (I_n)_{n \in \mathbb{Z}}$ de l'anneau $A$ est dite \textbf{fortement Approximable par Puissances d'idéaux} (en abrégé \textbf{fortement A.P.}) s'il existe un entier $k \geqslant 1$ tel que :
	\[ \forall \, n \in \mathbb{N}, \ I_{nk} = I_k^n \]
\end{madefinition}
\subsubsection{d) \underline{Filtrations E.P}}
\begin{madefinition}
	La filtration $f = (I_n)_{n \in \mathbb{Z}}$ de l'anneau $A$ est dite \textbf{Essentiellement par Puissances d'idéaux} (en abrégé \textbf{E.P}) s'il existe un entier $N \geqslant 1$ tel que :
	\[ \forall \, n \leqslant N, \ I_n =\sum_{p=1}^{N} I_{n-p}I_p. \]
\end{madefinition}
\subsubsection{e) \underline{Filtrations noethériennes}}
\begin{madefinition}
	La filtration $f = (I_n)_{n \in \mathbb{Z}}$ de l'anneau $A$ est dite \textbf{noethérienne} si son anneau de Rees ${R}(A,f)$ est noethérien.
\end{madefinition}
\subsubsection{f) \underline{Filtrations fortement noethériennes}}
\begin{madefinition}
	La filtration $f = (I_n)_{n \in \mathbb{Z}}$ de l'anneau $A$ est dite \textbf{fortement noethérienne} s'il existe un entier $k \geqslant 1$ tel que:
	\[ \forall \, m, n \in \mathbb{Z}, \ m, n \geqslant k, I_m I_n = I_{m+n} \]
\end{madefinition}
\begin{maremarque}
	Il résulte des références \cite{4} , \cite{5}, \cite{6} que si $A$ est un anneau noethérien et $f=(I_n)_{n \in \mathbb{Z}} \in \mathbb{F}(A)$, les assertions suivantes sont équivalentes : 
		\begin{enumerate}
				\item[(a)] $f$ est une $E.P.$ filtration
				\item[(b)] $\exists k \in \mathbb{N}^{*}$ tel que $f$ soit la plus petite filtration ayant pour $k+1$ premiers $I_0, I_1,\dots, I_k $
				\item[(c)] $f$ est noethérienne
				\item[(d)] $\exists m \in \mathbb{N}^{*}$ tel que $\forall j \geqslant m, I_{j+m} = I_{j}I_{m}$
				\item[(e)] L'anneau de Rees $R(A,f)$ de $f$ est une $A-algèbre$ de type fini
				\item[(f)] L'anneau de Rees généralisé $\mathcal{R}(A,f)$ de $f$ est une $A-algèbre$ de type fini
		\end{enumerate}
\end{maremarque}
\begin{maremarque}(\cite{2})\\
	$\mathcal{R}(A,f)$ est noethérien si et seulement si $R(A,f)$ est noethérien.
\end{maremarque}

\begin{maremarque}(\cite{2})\\
%	\begin{enumerate}
%		\item[(1)] Si $f$ est $I-adique$ alors $f$ est $I-bonne$
%		\item[(2)] $f$ est $I-bonne$ si et seulement si f est $f_I - bonne$
%		\item[(3)] Si $f$ est $I-bonne$ alors $f$ est fortement A.P. 
%		\item[(4)] Si $f$ est $I-bonne$ alors $f$ est A.P.
%		\item[(5)] Si $f$ est $I-bonne$ alors $f$ est fortement noethérienne.
%		\item[(5)] Si $f$ est $I-bonne$ alors $f$ est E.P.
%	\end{enumerate}
Dans un anneau noethérien, nous avons la classification suivantes des filtrations \\ classiques $f$ où $I$ est un idéal de l'anneau $A$:
\begin{center}
	\begin{tikzpicture}
		% Création des nœuds
		\node (A) at (0,0) {f I-adique};
		\node (B) at (4,0) {f I-bonne};
		\node (C) at (4,-2) {f fortement noethérienne};
		\node (D) at (10,-2) {f noethérienne};
		\node (E) at (14,0) {f A.P};
		\node (F) at (10,0) {f fortement A.P};
		
		% Dessin des flèches avec des modifications pour les rendre plus visibles
		\draw[->, ultra thick, >=stealth] (A) -- (B);
		\draw[->, ultra thick, >=stealth] (B) -- (C);
		\draw[->, ultra thick, >=stealth] (B) -- (F);
		\draw[->, ultra thick, >=stealth] (C) -- (D);
		\draw[->, ultra thick, >=stealth] (F) -- (E);
		\draw[<->, ultra thick, >=stealth] (F) -- (D);
	\end{tikzpicture}
\end{center}


\end{maremarque}



	\chapter{DÉPENDANCE INTÉGRALE, RÉDUCTION ET FILTRATION I-BONNE}
\chaptermark{DÉPENDANCE, RÉDUCTION ET FILTRATION I-BONNE}

Dans ce second chapitre, notre étude se concentre sur les filtrations I-adiques, également connues sous le terme de filtrations I-bonnes, lesquelles représentent un cas spécifique de filtrations bonnes. Ensuite, dans le troisième chapitre, nous élargirons la portée de ces résultats en les généralisant à toute filtration g-bonne.

\section{Dépendance intégrale}
\subsection{Dépendance intégrale sur les anneaux}
\begin{madefinition}\textbf{(Élément Entier)}\cite{Di2} \\
	Soient $B$ un anneau et $A$ un sous-anneau de $B$ ($A \subseteq B$).\\
	Un élément $x$ de $B$ est dit \textbf{entier} sur $A$ s'il est \textbf{racine d'un polynôme unitaire à coefficient dans} $A$.\\
	En d'autres termes, s'ils existent $a_1, a_2, \cdots , a_n$ éléments de $A$ tels que :\\
	\begin{equation}
		x^n + a_1 x^{n-1} +\cdots+a_i x^{n-i} +\cdots + a_n = x^n + \sum_{i=1}^{n} a_i x^{n-i} = 0, \; n \in \mathbb{N^*}
	\end{equation}
	
	Cette relation (2.1) est appelée \textbf{équation de dépendance intégrale} de $x$ sur $A$.\\
	On dit que $B$ est entier sur $A$ lorsque tous les éléments de $B$ sont entiers sur $A$.
\end{madefinition}

\begin{monexemple}
	Posons $B = \mathbb{R}$ et $A = \mathbb{Z}$. $A$ est un sous-anneau de $B$.\\
	$\bullet$ $x = \sqrt{5}$ est solution de l'équation $x^2 - 5$. Donc $\sqrt{5}$ est entier sur $\mathbb{Z}$.\\
	$\bullet$ $x = \sqrt{2}+1$ est solution de l'équation $x^2 - 2x -1$. Donc $x = \sqrt{2}+1$ est entier sur $\mathbb{Z}$.\\
	$\bullet$ $\dfrac{1}{2}$ n'est pas entier sur $\mathbb{Z}$.\\
	En effet, si $\dfrac{1}{2}$ est entier sur $\mathbb{Z}$ alors $\dfrac{1}{2}$ est solution d'une équation de dépendance intégrale. Alors: \\
	$\exists$ n $\geq 1 , \left(\dfrac{1}{2} \right)^n + \sum_{i=1}^{n} a_i \left(\dfrac{1}{2} \right)^{n-i} = 0$ , $a_i \in\mathbb{Z}$\\
	D'où $\left(\dfrac{1}{2} \right)^n \left[1+\sum_{i=1}^{n} a_i \left(\dfrac{1}{2} \right)^{-i}\right] = 0$\\
	Comme $\left(\dfrac{1}{2} \right)^n \neq 0$ alors $1+\sum_{i=1}^{n} a_i2^{i} = 0\Longrightarrow$ $1+\sum_{i=1}^{n} a_i2^{i} = 0 \Longrightarrow 1 = 2\left(\sum_{i=1}^{n} -a_i 2^{i-1}\right)\\ \Longrightarrow 1\in 2\mathbb{Z}.$ Ce qui est \textbf{Absurde}. 
\end{monexemple}
\begin{maproposition}
	Soient $B$ un anneau et $A$ un sous-anneau de $B$ ($A \subseteq B$).\\
	Soit $x \in B$. \\
	Les assertions suivantes sont équivalentes:
	\begin{enumerate}
		\item[i)]$x$ est entier sur $A$;
		\item[ii)]$A\left[ x\right]$ est un $A$-module de type fini;
		\item[iii)]Il existe $C$ un sous-anneau de $B$ contenant $A\left[ x\right]$ tel que $C$ soit un $A$-module de type fini.
	\end{enumerate}
\end{maproposition}
\begin{proof}
	$i=>ii)$
	
	Supposons que $x$ est entier sur $A.$
	
	Alors il existe $n$ $\in \mathbb{N}^{\ast },$ tel que $x^{n}+\sum\limits_{i=1}^{n}a_{i}x^{n-i}=0,$ $\forall i\in \llbracket 1, n \rrbracket ,a_{i}\in A.$
	
	D'où il existe $n$ $\in \mathbb{N}^{\ast },$ tel que $x^{n}=\sum\limits_{i=1}^{n}(-a_{i})x^{n-i},$ $\forall i\in \llbracket 1, n \rrbracket ,a_{i}\in A.$
	
	Donc il existe $n$ $\in \mathbb{N}^{\ast },$ tel que $x^{n}\in A(1_A,x,x^{2},...,x^{n-1})=A[x].$\\
	$\bullet$ Montrons que $A(1_A,x,x^{2},...,x^{n-1})=A[x]$\\
	a) Par construction, $A(1_A,x,x^{2},...,x^{n-1})\subset A[x]$\\
	b) Réciproquement, montrons par récurrence sur $m$\\ que pour tout $m	\in \mathbb{N},x^{m}\in A(1,x,x^{2},...,x^{n-1})$
	
	* si $m\in \llbracket 0, n-1 \rrbracket ,$ alors $x^{m}\in
	A(1_A,x,x^{2},...,x^{n-1})$
	
	* si $m\geq n$, alors $m=n+p,$ $p\geq 0.$
	
	\underline{Initialisation}
	
	si $p=0$ alors  $x^{n}+\sum\limits_{i=1}^{n}a_{i}x^{n-i}=0\Rightarrow
	x^{n}=\sum\limits_{i=1}^{n}(-a_{i})x^{n-i}$
	
	Comme $1\leq i\leq n,$ alors $0\leq n-i\leq n-1$
	
	D'où, $x^{n}=\sum\limits_{i=1}^{n}(-a_{i})x^{n-i}\in
	A(1_A,x,x^{2},...,x^{n-1})$
	
	Donc la propriété est vraie pour $p=0$
	
	\underline{Hérédité}
	
	Soit $p\geq 0.$ Supposons que $\forall k\in \llbracket 0, p \rrbracket
	,x^{n+k}\in A(1,x,x^{2},...,x^{n-1})$.\\ Montrons que $x^{n+p+1}\in
	A(1_A,x,x^{2},...,x^{n-1}).$
	
	$x^{n+p+1}=x^{p+1}x^{n}=x^{p+1}\times
	\sum\limits_{i=1}^{n}(-a_{i})x^{n-i}=\sum\limits_{i=1}^{n}(-a_{i})x^{n+p+1-i}$
	
	Comme $1\leq i\leq n,$ alors $p+1\leq n+p+1-i\leq n+p$
	
	Ainsi par hypothèse de récurrence, $x^{n+p+1-i}\in
	A(1,x,x^{2},...,x^{n-1})$ et par stabilité, il vient $x^{n+p+1}\in
	A(1,x,x^{2},...,x^{n-1}).$
	
	Par suite, pour tout $m$ $\in \mathbb{N},x^{m}\in A(1,x,x^{2},...,x^{n-1}).$
	
	Donc $A(1,x,x^{2},...,x^{n-1})=A[x]$
	
	$A[x]$ est donc un $A-$module de type fini de générateur $
	(1,x,x^{2},...,x^{n-1}).$
	
	$ii)=>iii)$ Il suffit de poser $A[x]=C$
	
	$iii)=>i)$ Supposons qu'il existe $C$ un sous module de $B$ contenant $A[x]$ qui soit un $A-$module de type fini.
	
	Soit $x\in A.$
	
	$A[x]\subset C=A(y_{1},y_{2},...,y_{r})$ de type fini.
	
	Ainsi, pour tout $i\in \llbracket 1, r \rrbracket ,$ $xy_{i}\in C.$ On
	a:
	
	$xy_{i}=\sum\limits_{j=1}^{r}a_{ij}y_{j}\Rightarrow
	\sum\limits_{j=1}^{r}a_{ij}y_{j}-xy_{i}=0\Rightarrow
	\sum\limits_{j=1}^{r}(a_{ij}-\delta _{ij}x)y_{j}=0$ où $\delta_{ij}=\left\{ 
	\begin{array}{ccc}
		1 & { si } & i=j \\ 
		0 & { sinon}
	\end{array}
	\right. $
	
	D'où $\left( 
	\begin{array}{cccc}
		a_{11}-x & a_{12} & \cdots & a_{1r} \\ 
		a_{22} & a_{11}-x & \cdots& \vdots\\ 
		\vdots & \vdots& \ddots  & \vdots\\ 
		a_{r1} & \cdots & \cdots & a_{rr}-x
	\end{array}
	\right) \left( 
	\begin{array}{c}
		y_{1} \\ 
		y_{2} \\ 
		\vdots \\ 
		y_{r}
	\end{array}
	\right) =\left( 
	\begin{array}{c}
		0 \\ 
		0 \\ 
		\vdots\\ 
		0
	\end{array}
	\right) $
	
	Posons $T=(a_{ij})_{1\leq i,j\leq r}$ $,$ $Y=(y_{j})_{1\leq j\leq r}$ 
	
	Ainsi $(T-xI_{r})\times Y=0$ où $I_{r}$ est la matrice identité d'ordre $r$ et $0$ est la matrice nulle de dimension $(r,1).$
	
	$(T-xI_{r})\times Y=0 \text{ ainsi } {t_{com}}(T-xI_{r})\times Y[(T-xI_{r})\times Y]=0 \\ (\text{ car } t_{com}(A) \times A = det(A) \times I_r \text{ avec } I_r \text{ la matrice identité }),	\text{ alors } \det [(T-xI_{r})Y]=0 \\
	\text{ D'où } \det(T-xI_{r})y_{i}=0,$ $\forall i\in \llbracket 1, r \rrbracket $$, \quad P_{T}(x)y_{i}=0,$ $\forall i\in \llbracket 1, r \rrbracket
	$ où $P_{T}$ est le polynôme caractéristique associé à $T.$
	
	De plus, $1\in C,$ on peut donc supposer que $\exists i\in \llbracket 1, r \rrbracket ,$ tel que $y_{i}=1$
	D'où $P_{T}(x)=0.$
	Or $P_{T}$ est un polynôme unitaire qui s'écrit\\ $P_{T}(x)=x^{n}-tr(T)x^{n-1}+...+(-1)^{n}\det
	(T)=x^{n}+\sum\limits_{i=1}^{n}\alpha _{i}x^{n-i},\alpha _{i}\in A.$
	
	Donc $P_{T}(x)=0\Rightarrow x^{n}+\sum\limits_{i=1}^{n}\alpha
	_{i}x^{n-i}=0,\alpha _{i}\in A\Rightarrow x$ est entier sur $A.$
	
\end{proof}
\begin{moncorollaire}
	Soient $A$ et $B$ deux anneaux tels que $A \subseteq  B$ et $x_1, x_2, \cdots, x_n \in B$.\\
	Alors les assertions suivantes sont équivalentes: \\
	\begin{enumerate}
		\item[i)]$\forall \ 1 \leqslant i \leqslant n, x_i$ est entier sur $A$,
		\item[ii)]$A[x_1, x_2, \cdots, x_n]$ est un $A$-module de type fini,
		\item[iii)]Il existe un $A-module$ de type fini $C \subseteq  B$ tel que $x_i C \subseteq  C, \forall \ 1 \leqslant i \leqslant n$.
	\end{enumerate}
\end{moncorollaire}

\begin{madefinition}
	Soit $A \subseteq B$ une inclusion d'anneaux.\\
	On qualifie $B$ d'une \textbf{A-algèbre} lorsque $B$ peut être appréhendé simultanément en tant que module sur A et en tant qu'anneau.\\
	De plus, B est dit \textbf{de type fini} si $B$ est un $A-module$ de type fini:
\end{madefinition}

\begin{maproposition}
	Soit $A \subseteq B$ une inclusion d'anneaux.\\
	Si B est une \textbf{A-algèbre de type fini} alors:
	\[ B \simeq \dfrac{A[X_1, X_2, \cdots, X_r]}{J} \]
	Où $J$ est un idéal de $A[X_1, X_2, \cdots, X_r]$
\end{maproposition}
\begin{moncorollaire}\textbf{(Clôture intégrale d'anneaux)}\cite{Di2}\\
	Soit $A \subseteq B$ une inclusion d'anneaux.\\
	L'ensemble des éléments de B entiers sur A est un sous anneau de B contenant A appelé \textbf{clôture intégrale} de $B$ dans $A$ notée $A'$.
\end{moncorollaire}
\begin{proof}
	Il s'agit de montrer que $A^{\prime }$ est un sous anneau de $B.$
	
	$i)$ $A\subset A^{\prime }\subset B$
	
	$ii)$ Soient $x,y\in A^{\prime }.$
	
	$x\in A^{\prime }$ alors $A[x]$ est un $A-$module de type fini.
	
	$y\in A^{\prime }$ alors $A[y]$ est un $A-$module de type fini.
	
	Posons $C=A[x,y]=A[x][y].$
	
	Ainsi $x+y\in C$ et $xy\in C$.
	
	Alors pour tout $z\in C,z=\sum\limits_{i=1}^{r}\alpha _{i}y^{i},\alpha
	_{i}\in A[x].$ Ainsi $\alpha _{i}=\sum\limits_{j=1}^{s}\alpha _{ij}x^{j}$
	
	D'où $z=\sum\limits_{i=1}^{r}\sum\limits_{j=1}^{s}\alpha _{ij}x^{j}y^{i}$
	
	Donc $C=A(x^{j}y^{i})_{1\leq i\leq r,1\leq j\leq s,}$ est un système de générateur fini de $C.$
	
	Par suite, $C$ est un $A-$module de type fini.
\end{proof}

\begin{moncorollaire}
	Soient $A \subseteq B \subseteq C $ deux inclusions d'anneaux.\\
	Si $C$ est entier sur $B$ et $B$ est entier sur $A$ alors $C$ est entier sur $A$.
\end{moncorollaire}
\begin{proof}
	Soit $x\in C.$
	
	Comme $x$ est entier sur $B$ alors il existe $n\in \mathbb{N}^{\ast },x^{n}=\sum\limits_{i=1}^{n}(-b_{i})x^{n-i},$ pour tout $b_{i}\in
	B,i\in \llbracket 1, n \rrbracket.$
	
	Soit $i\in \llbracket 1, n \rrbracket,$ $A[b_{i}]$ est un $A-$module de type fini
	(car $B$ entier sur $A)$
	
	Donc $A[b_{1},b_{2},...,b_{n}]$ est un $A-$module de type fini.
	
	Ainsi $A[b_{1}x^{n-1},b_{2}x^{n-2},...,b_{n}]$ est un $A-$module de type
	fini.
	
	Soit $p\geq 0.$ Montrons que $x^{n+p}\in
	A[b_{1}x^{n-1},b_{2}x^{n-2},...,b_{n}]$
	
	$\underline{Initialisation}$ $(p=0)$
	
	$x^{n}=\sum\limits_{i=1}^{n}(-b_{i})x^{n-i}\in
	A[b_{1}x^{n-1},b_{2}x^{n-2},...,b_{n}]$
	
	\underline{Hérédité} $(p\geq 0)$
	
	Supposons que pour tout $k\in \lbrack \lbrack 0,p]],x^{n+k}\in
	A[b_{1}x^{n-1},b_{2}x^{n-2},...,b_{n}]$
	
	$x^{n+p+1}=x^{n}\times x^{p+1}=\sum\limits_{i=1}^{n}(-b_{i})x^{n+p+1-i}$
	
	Comme $1\leq i\leq r$ alors $n+p+1-r\leq n+p+1-i\leq n+p$
	
	Donc par hypothèse de récurrence, $x^{n+p+1-i}\in
	A[b_{1}x^{n-1},b_{2}x^{n-2},...,b_{n}]$
	
	Par stabilité, il vient $x^{n+p+1}\in
	A[b_{1}x^{n-1},b_{2}x^{n-2},...,b_{n}]$
	
	Donc $x^{n+p}\in A[b_{1}x^{n-1},b_{2}x^{n-2},...,b_{n}]$
	
	Par suite, $A[b_{1}x^{n-1},b_{2}x^{n-2},...,b_{n}]$ est un $A-$module de
	type fini, c'est à dire 
	
	$A[b_{1}x^{n-1},b_{2}x^{n-2},...,b_{n}]=A(z_{1},z_{2},...,z_{s})$
	
	Posons $H=A[b_{1}x^{n-1},b_{2}x^{n-2},...,b_{n}]$ et $H^{\prime
	}=A(1,x,...,x^{n-1},z_{1},z_{2},...,z_{s})$
	
	$H^{\prime }$ est un $A-$module de type fini tel que $A[x]\subset H^{\prime
	}\subset C.$
	
	Donc $x$ est entier sur $A.$
	
	Par suite $C$ entier sur $A.$
\end{proof}
\begin{maproposition}
	Soit $A \subseteq B$ une inclusion d'anneaux tel que $B$ entier sur $A$. Alors:\\
	\begin{enumerate}
		\item[i)] Si $J$ est un idéal de $B$ alors:
		\[ \dfrac{B}{J} \text{ est entier sur } \dfrac{A}{J \cap A}\]
		\item[ii)] Pour toute partie multiplicative $S$ de $A$
		\[ S^{-1}B \text{ est entier sur } S^{-1}A.\]
	\end{enumerate}
\end{maproposition}

\begin{proof}
	$i)$ $B$ entier sur $A$ 
	
	Soit $x\in B,$ alors il existe $n\in \mathbb{N}^{\ast },x^{n}+\sum\limits_{i=1}^{n}a_{i}x^{n-i}=0,$ pour tout $a_{i}\in
	A,i\in \llbracket 1, n \rrbracket.$
	
	Ainsi $(x+J)^{n}+\sum\limits_{i=1}^{n}(a_{i}+J\cap
	A)(x+J)^{n-i}=x^{n}+J+\sum\limits_{i=1}^{n}a_{i}x^{n-i}+J=(x^{n}+\sum\limits_{i=1}^{n}a_{i}x^{n-i})+J=0+J=0_{J\cap A}$
	
	Donc $x+J$ est entier sur $\frac{A}{J\cap A}$
	
	Donc $\frac{B}{J}$est entier sur $\frac{A}{J\cap A}.$
	
	
	
	$ii)$ Supposons $B$ entier sur $A$
	
	Soit $\frac{x}{s}\in S^{-1}B,$ $x\in B$ et $s\in S$
	
	Comme $x$ est entier sur $A$ alors il existe $n\in \mathbb{N}^{\ast },x^{n}+\sum\limits_{i=1}^{n}a_{i}x^{n-i}=0,$ pour tout $a_{i}\in
	A,i\in \llbracket 1, n \rrbracket.$
	
	D'où, $(\frac{x}{s})^{n}+\sum\limits_{i=1}^{n}(\frac{a_{i}}{s_{i}})(\frac{x}{s})^{n-i}=\frac{0}{1}$
	
	Donc $\frac{x}{s}$ est entier sur $S^{-1}A.$
	
	Par suite $S^{-1}B$ est entier sur $S^{-1}A.$
\end{proof}
\begin{maproposition}
	Soit $A \subseteq B$ une inclusion d'anneaux tel que $B$ entier sur $A$.
	\[ B \text{ corps} \Longleftrightarrow  A \text{ corps} \]
\end{maproposition}
\begin{proof}
	$i)\implies ii)$ Supposons $B$ corps.
	
	Soit $x\in A\backslash \{0\}.$
	
	Comme $A\subset B$ alors $x\in B\backslash \{0\}$. Donc $x$ est inversible d'inverse $x^{-1}\in B.$
	
	De plus $B$ entier sur $A,$alors il existe $n\in \mathbb{N}^{\ast },(x^{-1})^{n}+\sum\limits_{i=1}^{n}a_{i}(x^{-1})^{n-i}=0,$ pour tout 
	$a_{i}\in A,i\in \llbracket 1, n \rrbracket.$
	
	Ainsi $(x^{-1})^{n}=\sum\limits_{i=1}^{n}(-a_{i})(x^{-1})^{n-i},$ pour tout $
	a_{i}\in A,i\in \llbracket 1, n \rrbracket.$
	
	$(x^{-1})\times (x^{-1})^{n-1}=\sum\limits_{i=1}^{n}(-a_{i})(x^{-1})^{n-i},$
	pour tout $a_{i}\in A,i\in \llbracket 1, n \rrbracket.$
	
	$(x^{-1})=\sum\limits_{i=1}^{n}(-a_{i})(x^{-1})^{n-i}(x^{-1})^{-n+1},$ pour
	tout $a_{i}\in A,i\in \llbracket 1, n \rrbracket.$
	
	$x^{-1}=\sum\limits_{i=1}^{n}(-a_{i})x^{i-1},$ pour tout $a_{i}\in A,i\in
	\llbracket 1, n \rrbracket.$
	
	Comme $1\leq i\leq n$ alors $0\leq i-1\leq n-1$
	
	Donc $x^{i-1}\in A$.
	
	Par stabilité $x^{-1}\in A$
	
	Donc $A$ corps.
	
	$ii)\implies i)$ Supposons que $A$ corps.
	
	Soit $x\in B\backslash \{0\}.$
	
	Comme $B$ entier sur $A,$alors il existe $n\in \mathbb{N}^{\ast },x^{n}+\sum\limits_{i=1}^{n}a_{i}x^{n-i}=0,$\\ pour tout $a_{i}\in
	A,i\in \llbracket 1, n \rrbracket.$

	Par récurrence sur $n$, montrons que $x\in A.$
	
	\underline{Initialisation} $(n=1)$
	
	l'équation de dépendance intégrale devient, $x+a_{1}=0
	\Rightarrow x=-a_{1}\in A\backslash \{0\}$
	
	Donc $x$ est inversible dans $A\subset B.$ Donc $x$ est inversible dans $B.$
	
	\underline{Hérédité} $(n\geq 1)$
	
	Supposons que la propriété est vraie jusqu'à l'ordre $n.$
	
	Alors $x^{n}+\sum\limits_{i=1}^{n}a_{i}x^{n-i}=0\Rightarrow x\in A.$
	
	Ainsi $x^{n+1}+\sum\limits_{i=1}^{n+1}a_{i}x^{n+1-i}=0\Rightarrow 
	x(x^{n}+\sum\limits_{i=1}^{n}a_{i}x^{n-i})+a_{n+1}=0\Rightarrow \\
	x(x^{n}+\sum\limits_{i=1}^{n}a_{i}x^{n-i})=-a_{n+1}$
	
	* si $-a_{n+1}\neq 0$ alors $x$ est inversible dans $A\subset B.$ Donc $x$
	est inversible dans $B.$
	
	* si $-a_{n+1}=0,$ alors $x(x^{n}+\sum\limits_{i=1}^{n}a_{i}x^{n-i})=0 \Rightarrow x^{n}+\sum\limits_{i=1}^{n}a_{i}x^{n-i}=0$\\ (car $x\neq 0$ et $B$ intègre)
	
	Ainsi $x^{n}+\sum\limits_{i=1}^{n}a_{i}x^{n-i}=0\Rightarrow
	x(x^{n-1}+\sum\limits_{i=1}^{n-1}a_{i}x^{n-1-i})=-a_{n+2}$
	
	* si $-a_{n+2}\neq 0$ alors $x$ est inversible dans $A\subset B.$ Donc $x$ est inversible dans $B.$
	
	* sinon de proche en proche, il vient que $x+a_{1}=0\Rightarrow x=-a_{1}\in
	A\backslash \{0\}$
	
	Donc $x$ est inversible dans $A\subset B.$ Donc $x$ est inversible dans $B$
	
	Dans tous les cas, il vient $B$ corps.
\end{proof}
\subsection{Dépendance intégrale sur un idéal}
\begin{madefinition}
	Soient $A$ un anneau commutatif unitaire et $I$ un idéal de $A$.\\ Un élément $x$ de $A$ est dit entier sur $I$ s'il existe un entier $m \in \mathbb{N}$ tel que : 
	\[ 	x^m + a_1 x^{m-1} + \cdots + a_m = 0\text{, avec} \ a_i \in I^i,\, \forall i=1, \cdots ,m. \]	
\end{madefinition}
\begin{maproposition}
	Soient $A$ un anneau et $I$ un idéal de $A$.
	\[ x \ entier \ sur \ I \Longleftrightarrow Le \ monôme \ xX \ \in A[x] \ est \ entier \ sur \ R(A, I) = \displaystyle \bigoplus_{n \in \mathbb{N}}{I^nX^n} \]
\end{maproposition}
\begin{proof}
	$i) \implies ii)$
	
	Supposons $x$ entier sur $I.$ Alors il existe $n\in \mathbb{N}^{\ast },x^{n}+\sum\limits_{i=1}^{n}a_{i}x^{n-i}=0,$ $a_{i}\in I^{i}.$
	
	Ainsi il existe $n\in \mathbb{N}^{\ast },(xX)^{n}+\sum\limits_{i=1}^{n}a_{i}X^{i}(xX)^{n-i}=0,$ $
	a_{i}X^{i}\in I^{i}X^{i}\in R(A,I).$
	
	D’où $xX$ est entier sur $R(A,I).$
	
	$ii)\implies i)$
	
	Supposons que  $xX\in A[X]$ est entier sur $R(A,I).$
	
	Ainsi il existe $n\in \mathbb{N}^{\ast },(xX)^{n}+\sum\limits_{i=1}^{n}a_{i}(xX)^{n-i}=0,$ $a_{i}\in R(A,I).
	$
	
	Alors il existe $n\in \mathbb{N}^{\ast },(xX)^{n}=\sum\limits_{i=1}^{n}(-a_{i})(xX)^{n-i},$ $a_{i}\in
	R(A,I).$
	
	Comme $(xX)^{n}$ est homogène de degré $n,$ alors pour tout $i\in \llbracket 1, n \rrbracket,$
	
	$\deg [a_{i}(xX)^{n-i}]=n\Rightarrow \deg (a_{i})+n-i=n\Rightarrow \deg
	(a_{i})=i.$
	
	Donc $a_{i}\in I^{i}X^{i}\Rightarrow a_{i}=\alpha _{i}X^{i},$ avec $\alpha
	_{i}\in I^{i}.$
	
	D’où, il existe $n\in \mathbb{N}^{\ast },(xX)^{n}+\sum\limits_{i=1}^{n}\alpha _{i}X^{i}(xX)^{n-i}=0,$
	
	il existe $n\in \mathbb{N}^{\ast },X^{n}[x^{n}+\sum\limits_{i=1}^{n}\alpha _{i}x^{n-i}]=0,$avec $\alpha _{i}\in I^{i}.$
	
	Par identification des polynômes, il existe $n\in \mathbb{N}^{\ast },x^{n}+\sum\limits_{i=1}^{n}\alpha _{i}x^{n-i}=0,$avec $\alpha_{i}\in I^{i}.$
	
	Donc $x$ est entier sur $I.$
\end{proof}
\begin{moncorollaire}\textbf{(Clôture intégrale d'idéaux)}\cite{Di2}\\
	Soient $A$ un anneau et $I$ un idéal de $A$.
	L'ensemble noté: 
	\[ I'=\{x \in A, x \; entier \; sur \; A \} = \bar{I} \]
	Est un idéal de $A$ appelé \textbf{clôture intégrale de I}.
\end{moncorollaire}
\begin{proof}
	$i)$ Par construction, $I^{\prime }\subset A.$
	
	$ii)$ $0^{1}+0=0$, donc $0\in I^{\prime }.$
	
	$iii)$ Soient $b\in A,x\in I^{\prime }.$
	
	Alors il existe $n\in \mathbb{N}^{\ast },x^{n}+\sum\limits_{i=1}^{n}a_{i}x^{n-i}=0,$avec $a_{i}\in I^{i}.$
	
	Ainsi, il existe $n\in \mathbb{N}^{\ast },(bx)^{n}+\sum\limits_{i=1}^{n}b^{i}a_{i}(bx)^{n-i}=0,$avec $
	b^{i}a_{i}\in I^{i}.$
	
	Donc $bx$ est entier sur $I.$ D'où $bx\in I^{\prime }.$
	
	$iv)$ Soient $x,y\in I^{\prime }.$
	
	Ainsi $xX,yX$ sont entiers sur $R(A,I).$
	
	Alors $xX+yX=(x+y)X\in A[X]$ est aussi entier sur $R(A,I)$
	
	Donc $x+y\in I^{\prime }.$
	
	Par suite, $I^{\prime }$ est un idéal de $A.$
\end{proof}
\begin{maremarque}\textbf{(Clôture intégrale d'idéaux et radical)}\cite{Di2} \\
	Soient $A$ un anneau et $I$ et $J$ des idéaux de $A$.
	\begin{enumerate}
		\item[1] ) $I \subseteq J \implies I' \subseteq J' $.
		\item[2] ) $I \subseteq I' \subseteq \sqrt[]{I} $.
		\item[3] ) $\sqrt[]{I} = \; \sqrt[]{I'} $.
	\end{enumerate}
\end{maremarque}
\begin{proof}
	$i)$ Soit $x\in I.$
	
	Alors $x^{1}-x=0$ ,où $a_{1}=-x^{1}\in I^{1}.$
	
	D'où $x\in I^{\prime }.$ Donc $I\subset I^{\prime }$
	
	$ii)$ Soit $x\in I^{\prime }.$
	
	Alors il existe $n\in \mathbb{N}^{\ast },x^{n}+\sum\limits_{i=1}^{n}a_{i}x^{n-i}=0,$avec $a_{i}\in I^{i}.$
	
	Ainsi pour tout $i\in \llbracket 1, n \rrbracket,$ $a_{i}\in I^{i}\Rightarrow
	a_{i}x^{n-i}\in I^{i}\subset I$
	
	Par stabilité, il vient, qu'il existe $n\in \mathbb{N}^{\ast },x^{n}=\sum\limits_{i=1}^{n}(-a_{i})x^{n-i}\in I$
	
	Donc $x\in \sqrt{I}.$
	
	D'où, $I^{\prime }\subset \sqrt{I}.$
	
	Par suite, $I\subset I^{\prime }\subset \sqrt{I}.$
	
	$iii)$ D'après ce qui précède
	
	$I\subset I^{\prime }\subset \sqrt{I}\Rightarrow \sqrt{I}\subset \sqrt{I^{\prime }}\subset \sqrt{\sqrt{I}}\Rightarrow \sqrt{I}\subset \sqrt{I^{\prime }}\subset \sqrt{I}\Rightarrow \sqrt{I}=\sqrt{I^{\prime }}.$
\end{proof}
\begin{maconsequence}
	Soit $x \in A$.
	Si $x \in I'$ alors il existe $m \in \mathbb{N^*}$ tel que $x^m \in I$.
\end{maconsequence}
\begin{madefinition}
	Un idéal $I$ de $A$ est dit \textbf{intégralement fermé} si $I = I'$.
\end{madefinition}

\section{Réduction d'un idéal}
\subsection{Définitions et propriétés}
\begin{madefinition}
	Soient $A$ un anneau commutatif unitaire, $I$ et $J$ deux id\'eaux de $A$.
	On dit que $I$ est une réduction de $J$ si :\\
	\begin{enumerate}
		\item[i)] I $\subseteq$ J,
		\item[ii)] $\exists \, r\in \mathbb{N}^{*} \text{, tel que } J^{r+1} = IJ^{r}$.
	\end{enumerate}
\end{madefinition}
\begin{monexemple}
	\begin{enumerate}
		\item[1)] $I$ est une réduction de $I$ lui-même.
		\item[2)] $A =\mathbb{K}[X,Y]$ avec $\mathbb{K}$ corps.\\
		$I = (X^2, Y^2)$\\
		$J = (X,Y)^2 = (X^2, Y^2, XY) $ D'où $I \subseteq J$.
		\begin{align*}
			IJ&= (X^{2},Y^{2})(X,Y)^{2}\\
			&= (X^{2},Y^{2})(X^{2},XY,Y^{2})\\
			&= (X^{4},Y^{4},X^{3}Y,XY^{3},X^{2}Y^{2})\\
		\end{align*}
		Et \\ 
		\begin{align*}
			J^2 &= (X^2X^2, X^2Y^2, X^3Y, Y^2Y^2, Y^3X)\\
			&= (X^4, Y^4, X^2Y^2, X^3Y, XY^3)
		\end{align*}
		Donc $J^{1+1} = J^2 = IJ $ avec $r=1$.\\
		Par suite $I$ est une réduction de $J$.
	\end{enumerate}
\end{monexemple}
\begin{maremarque}
	$\forall \, r\geq n$, on a $J^{r+1} = IJ^{r}$.\\
	D'une manière générale, $I^{m}J^{n}=J^{n+m}, \, \, \forall \, m\in \mathbb{N}$.	
\end{maremarque}

\begin{maproposition}
	Soient $A$ un anneau, $I$, $J$ et $K$ trois id\'eaux de $A$ tels que 
	$I \subseteq J \subseteq K $.\\ Si $I$ est une réduction de $J$ et $J$ est une réduction de $K$ alors $ I \text{est une réduction de }  K $.
\end{maproposition}
\begin{proof}
	Supposons que $I$ est une réduction de $J$ et $J$ est une réduction de $K$.\\
	$I$ est une réduction de $J$ alors $I \subseteq J$ et $\exists \, n\in \mathbb{N}^{*}$ tel que $J^{m+1} = IJ^{m}$, de même $J$ est une réduction de $K$ alors $J \subseteq K$ et $\exists \, n\in \mathbb{N}^{*}$ tel que $K^{n+1} = JK^{n}$\\
	Posons $r = mn+n+m \in \mathbb{N}^{*}$
	\begin{align*}
		K^{r+1} = K^{mn+n+m+1}& = (K^{n+1})^{m+1}\\
		& = (K^{n+1})^{m}(K^{n+1})\\
		& = (JK^{n})^{m}(JK^{n})\\
		& = J^{m}K^{nm}JK^{n}\\
		& = J^{m+1}K^{nm+n}\\
		& = J^{m+1}K^{n}K^{nm}\\
		& = IJ^{m}K^{n}K^{nm}\\
		& = I(JK^{n})^{m}K^{n}\\
		& = I(K^{n+1})^{m}K^{n}\\
		& = IK^{mn+n+m}  \\
		& = IK^r.          
	\end{align*}
	Il existe donc $r\in \mathbb{N}^{*}$ tel que $K^{r+1} = IK^{r}$ ainsi $I$ est une réduction de $K$.
\end{proof}
\begin{monlemme}
	Soit $I_1, I_2, J_1$ et $J_2$ des idéaux de $A$ alors, \\ si $I_1$ est  une réduction de $J_1$ et $I_2$ est une réduction de $J_2$ alors $I_1+I_2$ est une réduction de $J_1+J_2$.
\end{monlemme}
\begin{proof}
	Supposons que $I_1$ est une réduction de $J_1$ et $I_2$ est une réduction de $J_2$.\\
	$I_1$ est une réduction de $J_1$ alors $I_1 \subseteq J_1$ et $\exists \, m \in \mathbb{N^*}$ tel que $J_1^{m+1} = I_1 J_1^m$.\\
	$I_2$ est une réduction de $J_2$ alors $I_2 \subseteq J_2$ et $\exists \, n \in \mathbb{N^*}$ tel que $J_2^{n+1} = I_2 J_2^n$.\\
	Posons $r=m+n \in \mathbb{N^*} $\\
	\begin{align*}
		I_1(J_1+J_2)^{m+n}& = \displaystyle \sum_{k=0}^{m+n}{I_1 J_1^k J_2^{m+n-k}}\\
		& = \displaystyle \sum_{k=0}^{m-1}{I_1 J_1^k J_2^{m+n-k}} + \displaystyle \sum_{k=m}^{m+n}{I_1 J_1^k J_2^{m+n-k}}
	\end{align*}
	Or $I_1$ est une réduction de $J_1$ donc $\forall \, k\geq m$ on a $J_1^{k+1} = I_1 J_1^{k}$
	\begin{align*}
		I_1(J_1+J_2)^{m+n}& = \displaystyle \sum_{k=0}^{m+n}{I_1 J_1^k J_2^{m+n-k}}\\
		& = \displaystyle \sum_{k=0}^{m-1}{I_1 J_1^k J_2^{m+n-k}} + \displaystyle \sum_{k=m}^{m+n}{J_1^{k+1} J_2^{m+n-k}}
	\end{align*}
	Ainsi donc on a $\displaystyle \sum_{k=m}^{m+n}{J_1^{k+1} J_2^{m+n-k}} \subseteq  I_1(J_1+J_2)^{m+n}$.\\
	De façon similaire on montre que $\displaystyle \sum_{k=0}^{m}{J_1^{k} J_2^{m+n-k+1}} \subseteq  I_2(J_1+J_2)^{m+n}$\\
	D'où $\displaystyle \sum_{k=m}^{m+n}{J_1^{k+1} J_2^{m+n-k}} + \displaystyle \sum_{k=0}^{m}{J_1^{k} J_2^{m+n-k+1}} \subseteq I_1(J_1+J_2)^{m+n} + I_2(J_1+J_2)^{m+n}$.\\ Alors $\displaystyle \sum_{k=0}^{m+n+1}{J_1^{k} J_2^{m+n+1-k}} = (J_1+J_2)^{m+n+1} \subseteq (I_1+I_2)(J_1+J_2)^{m+n}$\\
	Par hypothèse on a $I_1 \subseteq J_1$ et $I_2 \subseteq J_2 \Rightarrow I_1+I_2 \subseteq J_1+J_2$ ,\\
	Par suite on a $(I_1+I_2)(J_1+J_2)^{m+n} \subseteq (J_1+J_2)^{m+n+1}$\\ Par conséquent $(J_1+J_2)^{m+n+1} = (I_1+I_2)(J_1+J_2)^{m+n}$ , on a donc trouver $r$ tel que $(J_1+J_2)^{r+1} = (I_1+I_2)(J_1+J_2)^{r}$ ce qui fait de $I_1+I_2$  une réduction de $J_1+J_2$.
\end{proof}
\begin{maproposition}
	Soient $A$ un anneau, $I$ un idéal de $A$ et $x \in A$.\\
	$x$ est entier sur $I$ si et seulement si $I$ est une réduction de $I + (x) = I +xA $.
\end{maproposition}
\begin{proof}
	Supposons que $x$ est entier sur $I$. Alors il existe $n \in \mathbb{N^*}$ tel que $x^n = \displaystyle \sum_{i=1}^{n}{a_i x^{n-i}}$, avec $a_i \in I^i, i=1, \cdots ,n$.\\
	Montrons que $I$ est une réduction de $I + (x)$.\\
	$(I+(x))^n = (I+(x))(I+(x))^{n-1}= I(I+(x))^{n-1} + (x)(I+(x))^{n-1}$\\
	En prouvant que $(x)(I+(x))^{n-1} \subseteq I(I+(x))^{n-1}$ on aura,
	\begin{center}
		$I(I+(x))^{n-1} + (x)(I+(x))^{n-1} = I(I+(x))^{n-1}$.
	\end{center}
	\begin{align*}
		(x)(I+(x))^{n-1} &= (x)\displaystyle \sum_{i=0}^{n-1}{I^i (x)^{n-1-i}}\\
		&= \displaystyle \sum_{i=0}^{n-1}{I^i (x)^{n-i}}\\
		&= (x)^n + \displaystyle \sum_{i=1}^{n-1}{I^i (x)^{n-i}}\\
		&= (x)^n + I\displaystyle \sum_{i=1}^{n-1}{I^{i-1} (x)^{n-i}}\\
		&= (x)^n + I\displaystyle \sum_{i=0}^{n-2}{I^i (x)^{n-1-i}}
	\end{align*}
	Donc $(x)(I+(x))^{n-1} = (x)^n + \displaystyle \sum_{i=0}^{n-2}{I^i (x)^{n-1-i}} \subseteq (x)^n + \displaystyle \sum_{i=0}^{n-1}{I^i (x)^{n-1-i}}$\\
	d'où $(x)(I+(x))^{n-1} \subseteq (x)^n + I(I+(x))^{n-1}$\\ et comme $x^n = \displaystyle \sum_{i=1}^{n}{a_i x^{n-i}} \in \displaystyle \sum_{i=1}^{n}{I^i x^{n-i}} \Rightarrow x^n \in I\displaystyle \sum_{i=1}^{n}{I^{i-1} x^{n-i}} = I\displaystyle \sum_{i=0}^{n}{I^i x^{n-1-i}}$\\
	alors $(x)^n \in I(I+(x))^{n-1} \Rightarrow (x)^n + I(I+(x))^{n-1} = I(I+(x))^{n-1}$.\\
	En somme $(x)(I+(x))^{n-1} \subseteq I(I+(x))^{n-1} \Rightarrow (I+(x))^{n} = I(I+(x))^{n-1}$.\\
	Par conséquent $I$ est une réduction de $I + (x)$.\\
	Supposons que $I$ est une réduction de $I + (x)$.\\
	Alors $\exists \, n \in \mathbb{N^*}$ tel que $(I + (x))^{n+1} = I(I + (x))^{n}$\\
	$x^{n+1} \in (I + (x))^{n+1} = I(I + (x))^{n} \Rightarrow x^{n+1} \in I\displaystyle \sum_{i=0}^{n}{I^i (x)^{n-i}} = \displaystyle \sum_{i=0}^{n}{I^{i+1} (x)^{n-i}}$.\\
	D'où $x^{n+1} \in \displaystyle \sum_{i=1}^{n+1}{I^i (x)^{n+1-i}} \Rightarrow x^{n+1} =  \displaystyle \sum_{i=1}^{n+1}{a_i x^{n+1-i}}$, avec $a_i \in I^i$. Ainsi $x$ est donc entier sur $I$.\\
\end{proof}
\begin{maproposition}
	Soit $A$ un anneau noethérien, $I$ et $J$ deux idéaux de $A$ tels que $I \subseteq J$. Les assertions suivantes sont équivalentes: 
	\begin{enumerate}
		\item[i] ) $I$ est une réduction de $J$.
		\item[ii] ) $R(A,J)$ est un $R(A,I)$-module de type fini.
		\item[iii] ) $R(A,J)$ est entier sur $R(A,I)$
		\item[iv] ) $\forall n \in \mathbb{N^*}$, $J^n$ est entier sur $I^n$.
		\item[v] ) $J$ est entier sur $I$
	\end{enumerate}
\end{maproposition}
\begin{proof}
	$i) \Rightarrow ii)$ \\
	Supposons que $I$ est une réduction de $J$ alors $\exists \, n \in \mathbb{N}^{*}$ tel que $J^{n+1} = IJ^{n}$.\\
	$J^{n+1} = IJ^{n} \Rightarrow \forall r \in \mathbb{N} \, ; J^{n+r} = I^rJ^{n}$.\\
	Ainsi $J^{n+r} X^{n+r} = I^r X^rJ^{n} X^n \Rightarrow R(A,J) = R(A,I)(JX, \cdots ,J^rX^r)$.\\
	$R(A,J)$ est donc un $R(A,I)$-module de type fini.\\
	$ii) \Rightarrow iii)$ \\
	Supposons que $R(A,J)$ est un $R(A,I)$-module de type fini.\\
	Soit $z \in R(A,J)$,\\
	$z \in R(A,J) \Rightarrow (R(A,J)[z])$ est un sous-module de $R(A,J)$.\\
	$A$ est noethérien alors $R(A,I)$ est noethérien, $R(A,J)$ est un $R(A,I)$-module de type fini alors, $R(A,J)$ est un module noethérien.\\
	$(R(A,J)[z])$ étant un sous-module de $R(A,J)$ qui est noethérien alors $(R(A,J)[z])$ est de type fini. Par suite $z$ est entier sur $R(A,I)$.\\
	$iii) \Rightarrow iv)$ \\
	Supposons que $R(A,J)$ est entier sur $R(A,I)$.\\
	Soit $a \in J^n \Rightarrow aX^n \in R(A,J)$, donc $aX^n$ est entier sur $R(A,I)$.\\
	Ainsi il existe $m \in \mathbb{N^*}$  tel que  $\, (aX^n)^m = \displaystyle \sum_{i=1}^{m}{b_i (aX^n)^{m-i}}$,\\ où $b_i \in I^{ni} X^{ni} \Rightarrow b_i = c_i X^{ni} , c_i \in I^{ni}$.\\
	$a^m X^{nm} = \displaystyle \sum_{i=1}^{m}{c_i X^{ni} (aX^n)^{m-i}} = \displaystyle \sum_{i=1}^{m}{c_i X^{ni} a^{m-i} X^{mn-ni}} = \displaystyle \sum_{i=1}^{m}{c_i a^{m-i} X^{mn}}$.\\
	Et donc $a^m X^{nm} = \displaystyle \sum_{i=1}^{m}{c_i a^{m-i} X^{mn}} \Rightarrow a^m = \displaystyle \sum_{i=1}^{m}{c_i a^{m-i}}$ , $c_i \in I^{ni} = (I^n)^i$.\\
	$J^n$ est ainsi entier sur $I^n$.\\
	$iv) \Rightarrow v)$\\
	En prenant $n = 1$, alors pour les mêmes raisons que la preuve $iii) \Rightarrow iv)$ on a le résultat souhaité à savoir $J$ entier sur $I$.\\
	$v) \Rightarrow i)$\\
	Supposons que $J$ est entier sur $I$.\\
	L'anneau $A$ étant noethérien alors l'idéal $J$ est de type fini.\\ Posons $J = (x_1, \cdots ,x_n)$\\
	$x_1 \in J$ qui est entier sur $I$ donc, $x$ est entier sur $I$ ce qui entraine que l'idéal $I$ soit une réduction de $I + (x_1)$. De même $x_2$ est entier sur $I$ donc sur $I + (x_1)$ ainsi, $I + (x_1)$ devient une réduction de $I + (x_1) + (x_2)$.\\ En répétant le même raisonnement à chaque élément de $J$, on obtient $x_r$ entier sur $I$ donc nécessairement sur $I + (x_1) + \cdots +(x_{r-1})$ ce qui implique que $I + (x_1) + \cdots +(x_{r-1})$ est une réduction de $I + (x_1) + \cdots +(x_{r}) = I+J$.\\
	On déduit donc que $I$ est une réduction de $J$. 
\end{proof}

\begin{maproposition}
	Soient A un anneau, I un idéal de A et x un élément de A. Alors:\\
	$x$ est entier sur $I$ si et seulement si il existe $J$ un idéal de A de type fini tel que: xJ est contenu dans IJ et pour tout x' élément de A, x'J est égal au module nul entraîne qu'il existe m supérieur ou égal à zéro, tel que $x'x^m = 0$
\end{maproposition}
\begin{moncorollaire}
	Soient I, J deux idéaux de A. Alors:
	\[ I'J' \subseteq (IJ)' \]
\end{moncorollaire}
\begin{montheoreme}
	Soit $A$ un anneau noethérien.
	Pour tout idéal $I$ de $A$, il existe un idéal $\hat{I}$ (nécessairement unique) ayant les propriétés suivantes : \\
	$(i)$ $I$ est une réduction de $\hat{I}$;\\
	$(ii)$ Tout idéal ayant $I$ comme réduction est contenu dans $\hat{I}$.
\end{montheoreme}
\begin{proof}
	Notons $\displaystyle \sum$ l'ensemble de tous les idéaux qui ont $I$ comme réduction. $I$ étant une réduction de lui-même alors, $I \in \displaystyle \sum$ donc $\displaystyle \sum \neq \emptyset$. Puisque la condition de maximalité des idéaux est valable dans notre anneau, on peut trouver un idéal $\hat{I} \in \displaystyle \sum$ et qui est maximal dans $\displaystyle \sum$.\\
	Soit $I_1 \in \displaystyle \sum$, alors $I$ est une réduction de $I_1$ et $I$ est une réduction de $\hat{I}$. Ainsi d'après le Lemme précédent $I+I=I$ est une réduction de $I_1+\hat{I}$. Cela montre que $I_1+\hat{I} \in \displaystyle \sum$. Mais $\hat{I} \subseteq I_1+\hat{I}$ d'où $\hat{I} = I_1+\hat{I}$, par choix de $\hat{I}$ donc $I_1 \subseteq \hat{I}$. Ce qui achève la preuve.
\end{proof} 
\begin{moncorollaire}
	Soient $A$ un anneau noethérien et $I$, $J$ deux idéaux de $A$. Si $I$ est une réduction de $J$, alors $\hat{I} = \hat{J}$.
\end{moncorollaire}
\begin{proof}
	D'une part $I$ est une réduction de $J$ et $J$ est une réduction de $\hat{J}$ alors, d'après (1.1.2 Proposition) $I$ est une réduction de $\hat{J}$ par conséquent $\hat{J} \subseteq \hat{I}$, par le Théorème précédent. D'autre part $I \subseteq J \subseteq \hat{J} \subseteq \hat{I}$ et comme $I$ est une réduction de $\hat{I}$, alors il existe $n \in \mathbb{N^*}$ tel que $\hat{I}^{n+1} = I \hat{I}^n$.\\
	$I \subseteq J \Rightarrow I \hat{I}^n \subseteq J \hat{I}^n$ or $\hat{I}^{n+1} = I \hat{I}^n$ donc $\hat{I}^{n+1} \subseteq J \hat{I}^n$. De plus $J \subseteq \hat{I} \Rightarrow  J \hat{I}^n \subseteq \hat{I}^{n+1}$, ainsi $\hat{I}^{n+1} = J \hat{I}^n$ par conséquent $J$ devient également une réduction de $\hat{I}$. Il s'ensuit que $\hat{I} \subseteq \hat{J}$, on obtient donc le résultat souhaité.
\end{proof}
\subsection{Réduction minimale d'un idéal}
La notion d'idéal basique a été introduite et étudiée par Northcott et Rees \cite{No}.
\begin{madefinition}
	Un idéal $I$ de l'anneau local noethérien $(A,m)$ est basique si la seule réduction de $I$ est $I$ lui-m\^{e}me. 
	Northcott et Rees ont aussi défini la notion de réduction minimale
	d'un idéal $J$:
	
	Un idéal $I$ est une réduction minimale de $J$ si $I$ est une réduction de $J$ et si I est $minimal$ au sens de l'inclusion ( $\subseteq$ ) parmi l'ensemble des réductions de J. 
\end{madefinition}
\begin{maremarque}
	On prouve dans \cite{Di2} que la réduction minimale des filtrations $I$-bonne existe toujours dans un anneau local noethérien. Ce qui n'est pas le cas en générale pour une filtration quelconque.
\end{maremarque}
Ces notions s'étendent sans difficulté aux filtrations noethériennes d'un anneau noethérien $A$, non nécessairement local \cite{Di2}. 

Les questions naturelles qui se posent sont alors les suivantes :
\begin{enumerate}
	\item[(a)] Quelles sont les filtrations basiques de $A$ si elles existent ?
	\item[(b)] Toute filtration de $A$ admet-elle une réduction minimale?
\end{enumerate}

Nous avons obtenu une caractérisation des filtrations basiques et avons montré que la réponse à la question (b) est négative.

Pour (a) nous avons obtenu le résultat suivant:
\begin{maproposition}
	Une filtration noethérienne $f $ de l'anneau noethérien A est basique si et seulement si $f$ est $I-adique$ avec $I$ idempotent ($I^{2}=I$).
\end{maproposition}
\begin{proof}
	Pour le voir, considérons une filtration $f=(I_{n})$ noethérienne et basique. 
	
	Il existe alors un entier $k\geq 1$ tel que pour tout entier $n\geq k,$ on
	ait $I_{n+k}=I_{k}I_{n}.$ Considérons la filtration $h=(H_{n})$ de $A$ définie par:
	
	$H_{n}=\left\{ 
	\begin{array}{c}
		I_{k-1}\text{ si }1\leq n\leq k-1 \\ 
		I_{n}\text{ pour }n\geq k
	\end{array}
	\right. $ 
	
	On a alors pour tout $n\geq k$, $I_{n+k}=I_{k}H_{n},$ et par conséquent $h$ est une réduction de $f.$
	
	Comme $f$ est basique, on en déduit que $I_{n}=I_{k}$ pour tout $n.$
	
	D'autre part, $I_{2k}=I_{k}^{2}$ donc $f=f_{I}$ avec $I=I_{k}$ et $I^{2}=I.$
	
	Réciproquement, si $f$ est $I-adique$ avec $I=I^{2}$ et si $h=(H_{n})$ est une réduction de $f,$ on a pour tout $n\geq 1,$ $I=I^{n}$ et $
	H_{n}\subseteq I.$
	
	De plus $h$ étant une réduction de $f,$ il existe un entier $k\geq 1$
	tel que $I=I^{k+n}=I^{k}H_{n}\subseteq H_{n}\subseteq I$ pour tout $n\geq k.$
	
	D'où $\forall n\geq k$ ,$I=H_{n}.$ Or pour tout entier $n$ tel que $1\leq n\leq k-1,$ on a $H_{k}=I\subseteq H_{n}\subseteq I.$
	
	D'où $H_{n}=I,$ $f_{I}=h$ et $f_{I}$ est basique.
\end{proof}

Concernant la question (b), nous avons obtenu le résultat suivant:

\begin{maproposition}
	Une filtration noethérienne $f=(I_{n})$ de l'anneau noethérien $A$
	admet une réduction minimale si et seulement s'il existe un entier $r\geq 1$ tel que $I_{r}$ soit un idéal idempotent.
\end{maproposition}
\begin{proof}
	Il est facile de voir que la condition est nécessaire. Supposons en effet que $h=(H_{n})$ soit une réduction minimale de $f$. Alors $f $ est noethérienne et basique et d'après le résultat précédent énonce plus haut, il existe un idéal idempotent $I$ tel que $h=f_{I}$. 
	
	Or l'une des caractérisations de réduction d'une filtration sur une
	autre assure l'existence d'un entier $N\geq 1$ tel que $I_{n}^{2}=I_{n}H_{n}=I_{n}I$ pour tout $n\geq N$. 
	
	De plus, $f$ étant noethérienne, elle est fortement $A.P.$ et il
	existe un entier $k\geq N$ tel que $I_{nk}=I_{k}^{n}$ pour tout entier $n$. 
	
	Ainsi $I_{2k}=I_{k}^{2}=I_{k}I$ et \ $I_{2k}^{2}=I_{2k}I=I_{k}I^{2}=I_{k}I=I_{2k}$ donc $I_{2k}$ est idempotent.
	
	La réciproque est plus technique et nous renvoyons aux références citées plus haut. La preuve de la réciproque demande l'utilisation de critères assurant que l'anneau de Rees d'une filtration $g$ soit entier (respectivement soit une algèbre de type fini) sur l'anneau de Rees d'une filtration $f.$ Or, suivant les références, les auteurs ont utilisé soit l'anneau de Rees classique soit l'anneau de Rees généralisé associé à une filtration pour définir et caractériser la dépendance intégrale ou la dépendance intégrale forte d'une filtration sur une autre. 
	
	Pour montrer que les propriétés utilisées, portant tantôt sur l'un ou l'autre des anneaux de
	Rees sont équivalentes, nous avons montré les résultats suivants:
	
	Étant données deux filtrations $f,g$ de A vérifiant $f\leq g$, on a:
	\begin{enumerate}
		\item[(i)]  $\mathcal{R}(A,g)$ est entier sur $\mathcal{R}(A,f)\Longleftrightarrow $ R(A,g) est entier
		sur R(A,f)
		\item[(ii)] $\mathcal{R}(A,g)$ est entier sur $\mathcal{R}(A,f)-$ algèbre de type fini $\Longleftrightarrow $ R(A,g) est entier sur $R(A,f)-algèbre$ de type fini
	\end{enumerate}
	Concernant la noethérianité des anneaux de Rees, une preuve astucieuse de l'équivalence entre la noethérianité de l'anneau $\mathcal{R}(A,f)$ et celle de l'anneau $R(A,f)$ pour toute filtration $f$ de l'anneau noethérien $A$ a été donnée dans \cite{Ra2}. Nous l'exposons brièvement ici. Il est clair que si l'anneau de Rees de $f$ est noethérien, il en est de m\^{e}me de son anneau de Rees généralisé. La réciproque est donc triviale. Par conséquent, l'anneau de Rees $R(A,f)$ de $f$ est noethérien.
	
\end{proof}



	\chapter{DÉPENDANCE, RÉDUCTION ET FILTRATIONS BONNES}
\chaptermark{DÉPENDANCE, RÉDUCTION ET FILTRATIONS BONNES}

Après avoir étudier les propriétés des filtrations I-bonnes, nous verrons dans ce chapitre les conditions pour étendre ces propriétés aux filtrations bonnes en générale.
\section{Dépendance intégrale de filtration}
\begin{madefinition}
	Soient $A$ un anneau commutatif unitaire, $I$ un idéal de $A$\\
	et $f=(I_n)_{n\in \mathbb{N}} $ appartenant à $ \mathbb{F}(A)$. Un élément $x$ de $A$ est dit entier sur $f$ s'il existe un entier $m \in \mathbb{N}$ tel que 
	\[ 	x^m + a_1 x^{m-1} + \cdots + a_m = x^m + \sum_{i=1}^{m} a_i x^{m-i} = 0, \; m \in \mathbb{N^*} \ , \ a_i \in I_i,\, \forall i=1, \cdots ,m. \]	
\end{madefinition}
\begin{maproposition}
	\label{maprop1}
	Soit $f=(I_n)_{n \in \mathbb{N}} $ appartenant à $ \mathbb{F}(A), x $ appartenant à $ A $ et $n $ appartenant à $ \mathbb{N}$.\\
	Les assertions suivantes sont équivalentes: 
	\begin{enumerate}
		\item[i)] $x$ est entier sur $f^{(n)}=((I_{nk})_{k \in \mathbb{N}})$;
		\item[ii)] $xX^n(\in A[X])$ est entier sur $R(A,f)$;
		\item[iii)] $xX^n(\in A[X])$ est entier sur $ \mathcal{R}(A,f)$.
	\end{enumerate}
	\begin{proof}
		i)$\implies$ ii)
		
		Supposons $x$ entier sur $f^{(n)}$.
		
		Alors \\ 
		$\exists$  $r\geq 1,$ $x^{r}+\sum\limits_{i=1}^{r}a_{i}x^{r-i}=0,$ $a_{i}\in I_{ni}$.
		
		Ainsi \\
		$\exists$  $r\geq 1,$ $(xX^{n})^{r}+\sum
		\limits_{i=1}^{r}a_{i}X^{ni}(xX^{n})^{r-i}=0,$ $a_{i}\in I_{ni} X^{ni} \subset R(A,f)$.
		
		Donc $xX^{n}$ est entier sur $R(A,f).$ \\
		
		ii)$\implies$ iii) Évident car $R(A,f)$ est contenu dans $  \mathcal{R}(A,f).$ \\
		iii)$\implies$ i) Supposons $xX^{n}$ est entier sur $\mathcal{R}(A,f).$
		
		Alors \\ $\exists$  $m\geq 1,$ $(xX^{n})^{m}+\sum\limits_{i=1}^{m}a_{i}(xX^{n})^{m-i}=0,$ $a_{i}\in \mathcal{R}(A,f)$.
		
		D'où \\ $\deg ((xX^{n})^{m})=nm$ alors $\deg (a_{i})=ni.$
		
		Ainsi \\ $a_{i}\in I_{ni}X^{ni},$ alors $a_{i}=\alpha _{i}X^{ni}$ ,$\alpha _{i}\in I_{ni}$.
		
		Donc \\ $\exists$  $m\geq 1,$ $(xX^{n})^{m}+\sum\limits_{i=1}^{m}(\alpha
		_{i}X^{ni})(xX^{n})^{m-i}=0,$ $\alpha _{i}\in I_{ni}$.
		
		Alors\\ $\exists$  $m\geq 1,$ $X^{nm}[x^{m}+\sum\limits_{i=1}^{m}\alpha
		_{i}x^{m-i}]=0X^{nm},$ $\alpha _{i}\in I_{ni}$.
		
		Par identification\\ $\exists$  $m\geq 1,$ $ x^{m}+\sum\limits_{i=1}^{m}\alpha _{i}x^{m-i}=0,\alpha _{i}\in I_{ni}$.
		
		Par suite, $x$ est entier sur $f^{(n)}$.
	\end{proof}
\end{maproposition}
\section{Clôture intégrale d'une filtration}
\begin{maproposition}
	Soit $f=(I_n)_{n\in \mathbb{N}} $ appartenant à $ \mathbb{F}(A).$ Alors:\\
	$f'=(I'_n)_{n\in \mathbb{N}} $ appartenant à $ \mathbb{F}(A)$ appelée \textbf{clôture intégrale de f}.
\end{maproposition}
\begin{proof}
	Supposons que $f=(I_{n})_{n\in \mathbb{N}}$ appartenant à $ F(A).$
	
	i) $I_{0}^{\prime }=\{x\in A,x$ entier sur $I_{0}\}=A$, car $I_{0}=A$ $.$
	
	ii) Soit $n$ appartenant à $ \mathbb{N}.$
	Comme $I_{n+1}$ est contenu dans $  I_{n}$ alors par croissance, $I_{n+1}^{\prime }$ est contenu dans $ I_{n}^{\prime }.$
	
	iii) Soient $p,q$ appartenant à $ \mathbb{N}.$ \\
	$I_{q}^{\prime }I_{p}^{\prime }\subset (I_{q}I_{p})^{\prime }\subset (I_{p+q})^{\prime }=I_{p+q}^{\prime }$ car $I^{\prime }J^{\prime }\subset (IJ)^{\prime }$ pour tout idéaux de $A.$
	
	Par suite $f^{\prime }=(I_{n}^{\prime })_{n\in \mathbb{N}}$ appartenant à $ \mathbb{F}(A).$
\end{proof}
\begin{moncorollaire}
	Soit $f=(I_n)_{n \in \mathbb{N}} $ appartenant à $ \mathbb{F}(A)$. Alors:\\
	$\text{ pour tout } k $ appartenant à $ \mathbb{N}, \text{ on pose: } P_k(f)=\left\{x \in A, x \text{ entier sur } f^{(k)}\right\}$ est un idéal de $A$ et la famille \\ $P(f)=(P_k(f))_{k \in \mathbb{N}}$ est une filtration de $A$ appelé \textbf{clôture prüférienne} de $f$.
\end{moncorollaire}
\begin{proof}
	Soit $k$ appartenant à $ \mathbb{N}.$ \\
	A) Montons que $P_{k}(f)$ est un idéal de $A$.
	
	i) Par définition, $P_{k}(f)$ est contenu dans $ A.$
	
	ii) $0_{A}$ est entier sur $A,$ donc $0_{A}$ appartient à $ P_{k}(f)$.
	
	iii) Soient $x,y$ appartenant à $ P_{k}(f).$
	
	Comme $x,y$ appartiennent à $ P_{k}(f)$ alors $xX^{k},yX^{k}$ sont entiers sur $R(A,f)$.
	
	D'où $xX^{k}+yX^{k}$ est entier sur $R(A,f)$ car $R(A,f)^{\prime }$ est
	un anneau.
	
	Donc $x+y$ est entier sur $f^{(k)}$.
	
	Par suite $x+y$ appartiennent à $ P_{k}(f).$
	
	iv) Soit $a$ appartenant à $ A,$ soit $x$ appartenant à $ P_{k}(f).$
	
	$x$ appartient à $ P_{k}(f)$ alors $xX^{k}$ est entier sur $R(A,f).$
	
	$(ax)X^{k}$ est entier sur $R(A,f)$.
	
	Donc $ax$ est entier sur $f^{(k)}$.
	
	D'où $ax$ appartient à $ P_{k}(f)$.
	
	Par suite $P_{k}(f)$ est un idéal de $A.$ \\
	B) Montrons que $P(f)=(P_{k}(f))_{k\in \mathbb{N}}$ appartient à $ \mathbb{F}(A)$.
	
	i) $P_{0}(f)=\{x\in A,$ $x$ entier sur $f^{(0)}=(A,...,A)=A\}$.
	
	D'où $P_{0}(f)=A$.
	
	ii) Soit $x$ appartenant à $ P_{k+1}(f).$
	
	Ainsi $x$ est entier sur $f^{(k+1)}.$
	
	Alors \\ $\exists$  $n\geq 1,$ $x^{n}+\sum\limits_{i=1}^{n}a_{i}x^{n-i}=0,$ $
	a_{i}\in I_{(k+1)i}\subset I_{ki}$.
	
	Donc $x$ est entier sur $f^{(k)}$.
	
	Par suite $x$ appartient à $ P_{k}(f).$
	
	D'où $P_{k+1}(f)$ est contenu dans $ P_{k}(f).$
	
	iii) Soient $x$ appartenant à $ P_{k_{1}}(f)$ et $y$ appartenant à $ P_{k_{2}}(f)$.
	
	$xX^{k_{1}}$ appartient à $ R(A,f)^{\prime }$ et $yX^{k_{2}}$ appartient à $ R(A,f)^{\prime }.$
	
	D'où $xyX^{k_{1}+k_{2}}$ appartient à $ R(A,f)^{\prime }$ car $R(A,f)^{\prime }$ est un anneau.
	
	Donc $xy$ appartient à $ P_{k_{1}+k_{2}}(f)$.
	
	Par suite, $P_{k_{1}}(f)$ $P_{k_{2}}(f)$ est contenu dans $ P_{k_{1}+k_{2}}(f).$
	
	On en déduit que $P(f)=(P_{k}(f))_{k\in \mathbb{N}}$ appartenant à $ F(A).$
\end{proof}
\begin{maremarque}
	La clôture intégrale d'un idéal $I$ de $A$ est : $I'=P_1(f_I)$.
\end{maremarque}
\begin{maproposition}
	Soit $f=(I_n)_{n \in \mathbb{N}} $ appartenant à $ \mathbb{F}(A)$. Alors:
	\begin{enumerate}
		\item[(i)] $ f \leqslant f' \leqslant P(f)$;
		\item[(ii)] $ P(P(f)) = P(f)$;
		\item[(iii)] $P(f) = P(f')$, avec $f'=(I'_n)_{n \in \mathbb{N}}$ est la clôture intégrale de $f$.
	\end{enumerate}
\end{maproposition}
\begin{proof}
	i) Il s'agit de montrer que pour tout $n$ appartenant à $ \mathbb{N},$ $I_{n}\subset I_{n}^{\prime }\subset P_{n}(f)$.
	
	Soit $n$ appartenant à $ \mathbb{N}.$ \\
	a) Soit $x$ appartenant à $ I_{n}.$
	
	Posons $r=1$.\\
	Ainsi $x+a_{1}=x+(-x)=0,$ $a_{1}=-x$.
	
	Alors $x$ appartient à $ I_{n}^{\prime }.$ D'où $I_{n}$ est contenu dans $ I_{n}^{\prime }.$
	
	b) Soit $x$ appartenant à $ I_{n}^{\prime }.$
	
	Soit $n$ appartenant à $ \mathbb{N}.$
	$x$ appartient à $ I_{n}^{\prime }$ alors \\ $\exists$  $r\geq 1,$ $x^{r}+\sum\limits_{i=1}^{r}a_{i}x^{r-i}=0,\alpha _{i}\in I_{n}^{i}$.
	
	Or $I_{n}^{i}$ est contenu dans $ $ $I_{ni}$ , d'où \\ $\exists$  $m=r\geq 1,$ $x^{m}+\sum\limits_{i=1}^{m}\alpha _{i}x^{m-i}=0,\alpha _{i}=a_{i}\in I_{ni}$.\\
	D'où $x$ appartenant à $ P_{n}(f)$.
	
	Donc $I_{n}^{\prime }$ est contenu dans $ P_{n}(f)$.
	
	Par conséquent, pour tout $n$ appartenant à $ \mathbb{N},$ $I_{n}\subset I_{n}^{\prime }\subset P_{n}(f).$
	
	c'est-\`{a}-dire que $f\leq f^{\prime }\leq P(f)$.
	
	ii) Montrons que $P(P(f))=P(f).$
	
	A) D'après i), $f\leq P(f)\text{ alors } P(f)\subset P(P(f))$.
	
	B) Posons $g=P(f)=(J_{n})_{n\in \mathbb{N}}$ avec $J_{n}=P_{n}(f)$.
	
	Donc $P(P(f))=P(g)$.
	
	Soient $n$ appartenant à $ \mathbb{N},x$ appartenant à $ P_{n}(g).$
	
	$x$ appartient à $ P_{n}(g)$ alors $x$ est entier sur $g^{(n)}$.
	
	$x\in P_{n}(g)$ alors \\ $\exists$  $s\geq 1,$ $x^{s}+\sum\limits_{i=1}^{s}a_{i}x^{s-i}=0,a_{i}\in J_{ni}$.
	
	Or $J_{ni}=P_{ni}(f)$ , ainsi $a_{i}\in P_{ni}(f)$ pour tout $i\in \llbracket 1, s \rrbracket.$
	
	Ainsi les $a_{i}$ sont entiers sur $f^{(ni)}.$ D'où $a_{i}X^{ni}\in R(A,f)^{\prime }$.
	
	Alors $(xX^{n})^{s}+\sum\limits_{i=1}^{s}a_{i}X^{ni}(xX^{n})^{s-i}=0,$ avec 
	$a_{i}X^{ni}\in R(A,f)^{\prime }$.
	
	Donc $xX^{n}$ est entier sur $R(A,f)^{\prime }$.
	
	D'où $xX^{n}\in \lbrack R(A,f)^{\prime }]^{\prime }=R(A,f)^{\prime }$.
	
	Par suite $xX^{n}$ est entier sur $R(A,f)$.
	
	Donc $x$ est entier sur $f^{(n)}$.
	
	Par suite $x\in P_{n}(f).$
	
	D'où $P_{n}(g)$ est contenu dans $ P_{n}(f).$
	
	c'est-\`{a}-dire $P(P(f))$ est contenu dans $ P_{n}(f).$ Donc $P(P(f))=P_{n}(f).$
	
	iii) Montrons que $P(f)=P(f^{\prime })$.
	
	D'après i) on a $f\leq f^{\prime }\text{ alors } P(f)\leq P(f^{\prime })$.
	
	Réciproquement, $f^{\prime }\leq P(f)\text{ alors } P(f^{\prime })\leq
	P(P(f))=P(f)$.
	
	D'où $P(f)\leq P(f^{\prime })\leq P(f)$.
	
	Par suite $P(f)=P(f^{\prime })$.
\end{proof}

%\begin{maproposition}
%	Soit $f=(I_n)_{n \in \mathbb{N}} \in \mathbb{F}(A)$. Alors:\\
%	\begin{enumerate}
%		\item[(i)] $x \in P_k(f)$ si et seulement si $xX^k$ est entier sur $R(A,f)$;
%		\item[(ii)] $ I_k \subset P_k(f) \subset \sqrt[]{f}$ et en particulier $\ \sqrt[]{(P(f))} = \sqrt[]{f}$.
%	\end{enumerate}
%\end{maproposition}
\begin{maproposition}
	Soient $A $ contenu dans $ B$ deux anneaux.\\
	$A' =\left\{x \in B, x\text{ entier sur } A\right\}$. On a:
	\begin{enumerate}
		\item[i)] $A \subset A' \subset B $;
		\item[ii)] $A'' = A'$.
	\end{enumerate}
\end{maproposition}
\begin{proof}
	i) Montrons que $A\subset A^{\prime }\subset B$.
	
	a) Soit $x\in A.$
	
	On a: $x^{1}+a_{1}=x^{1}+(-x)=0\text{ alors } x\in A^{\prime }.$
	
	Donc $A$ est contenu dans $ A^{\prime }.$
	
	b) Soit $x^{\prime }\in A^{\prime }.$ 
	
	Par construction $x\in B.$
	
	D'où $A^{\prime }$ est contenu dans $ B.$
	
	Par suite $A \subset A^{\prime }\subset B$.
	
	ii) D'après i) $A\subset A^{\prime }\text{ alors } A^{\prime }\subset
	A^{\prime \prime }$ (Par croissance).
	
	Réciproquement soit $x\in A^{\prime \prime }\text{ alors } x$ entier sur $
	A^{\prime }$.
	
	Ainsi \\ $\exists$  $n\geq 1,$ tel que  $x^{n}+\sum
	\limits_{i=1}^{n}a_{i}x^{n-i}=0,a_{i}\in A^{\prime }.$
	
	Alors $a_{i}$ entier sur $A,$ pour tout $i\in \llbracket 1, n \rrbracket.$
	
	D'où $A[a_{i}]$ est un $A$-module de type fini, ainsi $A[a_{1},a_{2},...,a_{n}]$ est un A-module de type fini.
	
	En effet, si,
	
	$\left\{ 
	\begin{array}{c}
		A[a_{1}]=A(1_{A},a_{1},a_{1}^{2},...a_{1}^{n-1}) \\ 
		A[a_{2}]=A(1_{A},a_{2},a_{2}^{2},...a_{2}^{m-1})
	\end{array}
	\right. $
	
	des A-modules de type fini, alors 
	
	$A[a_{1},a_{2}]=A(a_{1}^{i}a_{2}^{j})_{\substack{ 0\leq i\leq n-1 \\ 0\leq
			j\leq m-1}}$.
	
	En procédant de proche en proche, il vient $A[a_{1},a_{2},...,a_{n}]$
	est un A-module de type fini.
	
	Par suite, $K=A[a_{1}x^{n-1},a_{2}x^{n-2},...,a_{n}]$ est un A-module de
	type fini.
	
	Soient $(y_{1},...,y_{n})$ les générateurs de $K.$
	
	Donc $K=A(y_{1},...,y_{n},x^{n-1},x^{n-2},...,x^{2},x,1)$.
	
	D'où $A[x]$ est contenu dans $ H.$
	
	Par suite $x$ est entier sur $A$.
	
	Donc $x\in A^{\prime }$.
	
	Par suite $A^{\prime \prime }$ est contenu dans $ A$.
	
	Donc  $A^{\prime \prime }=A$.
	
	iii) De la même manière, on montre que $I^{\prime \prime }=I^{\prime}.$
\end{proof}
\begin{madefinition}
	Soit $I$ un idéal de l'anneau $A$ et $f $ appartenant à $ \mathbb{F}(A)$. \\
	On dit que $I$ est entier sur $f$ si tout élément de $I$ est entier sur $f$. \\
	Ce qui signifie que I est entier sur $f$ si $I $ est contenu dans $ P_1(f)$.
\end{madefinition}
\begin{maconsequence}
	Soit $I$ un idéal de l'anneau $A$ et $f \in \mathbb{F}(A)$. \\
	$I$ est entier sur $f$ si et seulement si $f_I \leqslant P(f)$.
\end{maconsequence}
\begin{proof}
	$(ii)\Longrightarrow (i)$.
	
	Supposons que $f_{I}\leq P(f).$ Alors $I^{n}$ est contenu dans $ P_{n}(f),$ pour tout $n\geq 1.$
	
	En particulier pour $n=1,$ on a $I$ est contenu dans $ P_{1}(f).$ Donc $I$ est entier sur $f.$
	
	$(i)\Longrightarrow (ii)$.
	
	Supposons que $I$ est entier sur $f.$ Alors $I$ est contenu dans $ P_{1}(f).$ Montrons que $I^{n}$ est contenu dans $ P_{n}(f),$ pour tout $n\geq 1.$
	
	\textbf{Initialisation}
	
	Comme $I$ est contenu dans $ P_{1}(f)$ alors la propriété est vrai à l'ordre 1.
	
	\textbf{Hérédité}
	
	Soit $n\geq 1.$ Supposons que la propriété est vraie jusqu'à l'ordre $n$, c'est à dire $I^{n}$ est contenu dans $ P_{n}(f).$
	
	Montrons que la propriété est vraie jusqu'à l'ordre $n+1$, c'est à dire $I^{n+1}$ est contenu dans $ P_{n+1}(f).$
	
	On a: $I^{n}\subset P_{n}(f)\Longrightarrow I^{n+1}\subset
	IP_{n}(f)\subset P_{1}(f)P_{n}(f)\subset P_{n+1}(f)$ (car $
	(P_{n}(f))_{n\in \mathbb{N}}\in \mathbb{F}(A)$).
	
	D'où $I^{n+1}\subset P_{n+1}(f).$
	
	Donc la propriété est vraie jusqu'à l'ordre $n+1.$
	
	Par suite $I^{n}\subset P_{n}(f),$ pour tout $n\geq 1.$ Par conséquent $f_{I}\leq P(f)$.
\end{proof}

\begin{madefinition}
	Soit $f=(I_n)_{n \in \mathbb{N}} , g = (J_n)_{n \in \mathbb{N}}$ appartenant à $ \mathbb{F}(A)$.  Alors:\\
	\begin{enumerate}
		\item[(a)]$g$ est \textbf{entière sur} $f$ si $g \leqslant P(f)$. C'est à dire:
		\[\forall n \geqslant 1, J_n \subset P_{n}(f). \]
		\item[(b)]$g$ est \textbf{fortement entière sur} $f$ si $f \leqslant g$ et si $R(A,g)$ est un $R(A,f)-module$ de type fini.
	\end{enumerate}
\end{madefinition}
\begin{maproposition}
	\label{maprop2}
	Soient $f$ et $g$ deux filtrations de l'anneau $A$ telles que $f \leqslant g$.\\ Alors $R(A,g)$ est entier sur $R(A,f)$ si et seulement si $\mathcal{R}(A,g)$ est entier sur $\mathcal{R}(A,f)$. 
\end{maproposition}
\begin{proof}
	Soient $f=(I_n)_{n \in \mathbb{N}},g=(J_n)_{n \in \mathbb{N}}$ deux filtrations de $A$ telles que $f \leqslant g$. \\ Supposons que $\mathcal{R}(A,g)$ est entier sur $\mathcal{R}(A,f)$. \\
	Soit $b \in J_n$ alors $bX^n \in \mathcal{R}(A,g)$ est entier sur $\mathcal{R}(A,f)$ d'après \ref{maprop1}, on a $bX^n$ entier sur $R(A,f)$. Par conséquent $R(A,g)$ est entier sur $R(A,f)$.
	Réciproquement, il suffit de remarquer que $\mathcal{R}(A,f)=R(A,f)[u]$ qui est une algèbre de type fini sur $R(A,f)$.
\end{proof}
\begin{maproposition}
	Soient $f,g$ deux filtrations de $A$ telles que $f \leqslant g$. Alors les assertions suivantes sont équivalentes:
	\begin{enumerate}
		\item[(i)] $g$ est entière sur $f$;
		\item[(ii)] $R(A,g)$ est entier sur $R(A,f)$;
		\item[(iii)] $\mathcal{R}(A,g)$ est entier sur $\mathcal{R}(A,f)$.
	\end{enumerate}
\end{maproposition}
\begin{proof}
	(i) $\Longleftrightarrow$ (ii), immédiat en utilisant la proposition \ref{maprop1}. \\
	(ii) $\Longleftrightarrow$ (iii), immédiat en utilisant la proposition \ref{maprop2}.
\end{proof}
\begin{maproposition} \cite{Di2} \\
	Soient $f,g$ deux filtrations de $A$. Alors les assertions suivantes sont équivalentes:
	\begin{enumerate}
		\item[(i)] $g$ est entière sur $f$;
		\item[(ii)] $\forall k \geqslant 1$, $g^{k}$ est entière sur $f^{(k)}$;
		\item[(iii)] $\exists k \geqslant 1$, $g^{k}$ est entière sur $f^{(k)}$.
	\end{enumerate}
\end{maproposition}
\begin{proof}
	Soient $f=(I_n)_{n \in \mathbb{N}},g=(J_n)_{n \in \mathbb{N}}$ deux filtrations de $A$. \\
	$(i)\Longrightarrow (ii)$.
	
	Supposons que $g$ est entière sur $f.$ Montrons que pour tout $k\geq 1,$ $g^{(k)}$ est entière sur $f^{(k)}.$
	
	Soit $k\geq 1,$ on a: $f^{(k)}=(I_{nk})_{n\in \mathbb{N}}$ et $g^{(k)}=(J_{nk})_{n\in \mathbb{N}}$.
	
	Comme $g$ est entière sur $f$ alors l'idéal $J_{nk\text{ }}$est entier sur $f^{(nk)}.$
	
	Or $f^{(nk)}=(f^{(k)})^{(n)}$ donc $J_{nk}$est entier sur $(f^{(k)})^{(n)}.$
	
	Par conséquent $g^{(k)}$ est entière sur $f^{(k)}$ pour tout $k\geq 1.$
	
	$(ii)\Longrightarrow (iii)$ immédiat.
	
	$(iii)\Longrightarrow (i)$.
	
	Supposons qu'il existe $k$ supérieur ou égal à 1 tel que $g^{(k)}$ est entière sur $f^{(k)}$. Montrons que $g$ est entière sur $f.$
	
	Pour cela il suffit de montrer que pour tout $p\geq 1,$ l'idéal $J_{p}$ est entier sur $f^{(p)}.$
	
	Soit $x\in $ $J_{p}.$ Alors $x^{k}$ $\in $ $J_{p}^{k}\subset J_{pk}.$
	
	Or $J_{pk}$ est entier sur $f^{(pk)}=(f^{(k)})^{(p)}$. D'où il existe $n\in \mathbb{N}^{\ast }$ tel que:
	
	$(x^{k})^{n}+a_{1}(x^{k})^{n-1}+\cdots +a_{j}(x^{k})^{n-j}+\cdots +a_{n}=0,$ avec $a_{j}\in I_{pkj}.$
	
	On a pour tout $j=1,\cdots ,n$ , $a_{j}(x^{k})^{n-j}=a_{j}x^{kn-kj},$ ainsi $x^{kn}+a_{1}x^{kn-k}+\cdots +a_{j}x^{kn-kj}+\cdots +a_{n}=0,$ avec $a_{j}\in I_{pkj}.$
	
	Posons $m=kn.$ On obtient $x^{m}+a_{1}x^{m-k}+\cdots +a_{j}x^{m-kj}+\cdots +a_{n}=0,$ avec $a_{j}\in I_{p(kj)}.$
	
	Par suite $x$ est entier sur $f^{(p)}$pour tout $p\geq 1.$ Par conséquent pour tout $p\geq 1$ $J_{p}$ est entier sur $f^{(p)}.$
	
	Donc $g$ est entière sur $f.$
	
\end{proof}

\section{Réduction d'une filtration}
\subsection{Réduction au sens de Okon-Ratliff}
\begin{madefinition}
	($\alpha$-réduction ou réduction au sens de Okon-Ratliff\cite{Ok} )\\
	Soient $f=(I_n)_{n \in \mathbb{N}}$, $g=(J_n)_{n \in \mathbb{N}}$ deux filtrations de $A$.\\
	$f$ est une $\alpha$-réduction de $g$ si :
	\begin{enumerate}
		\item[i)] $f \leq g$;
		\item[ii)] $\exists \, N \geq 1$ , $\forall n \geq N , J_n = \displaystyle \sum_{p=0}^{N}{I_{n-p} J_p}$.
	\end{enumerate}
\end{madefinition}
\begin{maremarque}
	Soit $\Re$ la relation $f \Re g \Leftrightarrow$ $f$ est une $\alpha$-réduction de $g$. Notons que cette relation est une relation d'ordre sur l'ensemble des filtrations.	
\end{maremarque}
\begin{proof}
	(i) \textbf{Réflexivité}. \\
	Posons $N=1$. \\
	$\displaystyle \sum_{p=0}^{1}I_{n-p}I_{p}=I_{n}I_{0}+I_{n-1}I_{1}=I_{n-1}I_{1}=I_{n}$ (car $I_{0}\subset I_{1}$).
	
	D'où $f \Re f$. \\
	(ii) \textbf{Transitivité}. \\
	Soient $f=(I_{n})_{n\in \mathbb{N}}$ $,g=(J_{n})_{n\in \mathbb{N}}$ $,h=(H_{n})_{n\in \mathbb{N}}$ \ des filtrations de $A.$ \\
	Supposons que $f \Re g$.
	alors $f\leq g$ et il existe $N_{1}\geq 1,\text{ pour tout } m\geq N_{1},$ $J_{m}=\displaystyle  \sum_{p=0}^{N_{1}}I_{m-p}J_{p}$.
	Supposons que $g \Re h$ alors $g\leq h$ et il existe $N_{2}\geq 1,\text{ pour tout } k\geq N_{2},$ $H_{k}=\displaystyle  \sum_{p=0}^{N_{2}}J_{k-p}H_{p}$.\\
	D'où $f\leq g\leq h.$ Donc $f\leq h$. \\
	De plus posons $N=N_{1}+N_{2}$. \\
	$H_{k}=\displaystyle  \sum_{p=0}^{N_{2}}J_{k-p}H_{p}+\displaystyle  \sum_{p=N_{2}+1}^{N}J_{k-p}H_{p}=\displaystyle  \sum_{p=0}^{N_{2}}J_{k-p}H_{p}+\displaystyle  \sum_{p=N_{2}+1}^{N_{1}+N_{2}}J_{k-p}H_{p}$ pour tout $k\geq N_{1}+N_{2}$. \\
	Comme $k\geq N_{1}+N_{2}$ alors $k-N_{2}\geq N_{1}$ et $p\leq N_{2}$ alors $k-p\geq N_{1}$. \\
	D'où $J_{k-p}=\displaystyle  \sum_{i=0}^{N_{1}}I_{k-p-i}J_{i}$. \\
	Ainsi $H_{k}=\displaystyle  \sum_{p=0}^{N_{2}}(\displaystyle  \sum_{i=0}^{N_{1}}I_{k-p-i}J_{i})H_{p}+\displaystyle  \sum_{p=N_{2}+1}^{N_{1}+N_{2}}J_{k-p}H_{p}$. \\
	Donc $H_{k}\subset \displaystyle  \sum_{p=0}^{N_{2}}\displaystyle  \sum_{i=0}^{N_{1}}I_{k-p-i}J_{i}H_{p+i}+\displaystyle  \sum_{p=N_{2}+1}^{N_{1}+N_{2}}J_{k-p}H_{p}$  car $g\leq h$.\\ D'où $ J_{i}\subset H_{i}\subset H_{i}H_{p}\subset H_{i+p}$. \\
	Posons $l=p+i,$ ainsi $0\leq l\leq N_{1}+N_{2}$ car $0\leq p\leq N_{2}$ et $ 0\leq i\leq N_{1}$. \\
	D'où $H_{k}\subset \displaystyle  \sum_{l=0}^{N_{1}+N_{2}}I_{k-l}H_{l}+\displaystyle  \sum_{p=N_{2}+1}^{N_{1}+N_{2}}J_{k-p}H_{p}$. \\
	Posons $K=\displaystyle  \sum_{p=N_{2}+1}^{N_{1}+N_{2}}J_{k-p}H_{p}$. \\
	Comme $p\geq N_{2}$ alors $H_{p}=\displaystyle  \sum_{p=0}^{N_{2}}J_{p-i}H_{i}$. \\
	D'où $K=\displaystyle  \sum_{p=N_{2}+1}^{N_{1}+N_{2}}J_{k-p}(\displaystyle  \sum_{p=0}^{N_{2}}J_{p-i}H_{i})=\displaystyle  \sum_{p=N_{2}+1}^{N_{1}+N_{2}}\displaystyle  \sum_{p=0}^{N_{2}}J_{k-p}J_{p-i}H_{i}$. \\
	Donc $K\subset\displaystyle  \sum_{p=N_{2}+1}^{N_{1}+N_{2}}\displaystyle  \sum_{p=0}^{N_{2}}J_{k-i}H_{i}=\displaystyle  \sum_{p=0}^{N_{2}}J_{k-i}H_{i}$
	$K\subset \displaystyle  \sum_{p=0}^{N_{2}}J_{k-i}H_{i}$
	or $0\leq i\leq N_{2}$ et $N_{1}+N_{2}\leq k$. \\
	D'où $k-i\geq N_{1}$. \\
	Ainsi $J_{k-i}=\displaystyle  \sum_{l=0}^{N_{1}}I_{k-i-l}J_{l}$. \\
	D'où $K\subset\displaystyle  \sum_{p=0}^{N_{2}}(\displaystyle  \sum_{l=0}^{N_{1}}I_{k-i-l}J_{l})H_{i}\subset\displaystyle  \sum_{p=0}^{N_{2}}\displaystyle  \sum_{l=0}^{N_{1}}I_{k-i-l}J_{l}H_{i}$.\\
	Posons $p=i+l$. \\
	D'où $K\subset \displaystyle \sum_{p=0}^{N_{1}+N_{2}}I_{k-p}H_{p}$.\\
	Or $H_{k}\subset\displaystyle \sum_{l=0}^{N_{1}+N_{2}}I_{k-l}H_{l}+\displaystyle  \sum_{p=N_{2}+1}^{N_{1}+N_{2}}J_{k-p}H_{p}=\displaystyle  \sum_{l=0}^{N_{1}+N_{2}}I_{k-l}H_{l}+K\subset \displaystyle  \displaystyle  \sum_{l=0}^{N_{1}+N_{2}}I_{k-l}H_{l}$.\\ Car $K\subset \displaystyle  \displaystyle  \sum_{l=0}^{N_{1}+N_{2}}I_{k-l}H_{l}$.\\
	Donc $H_{k}\subset H_{k-p}H_{p}\subset H_{k}$.\\
	Finalement $H_{k}=\displaystyle  \displaystyle  \sum_{l=0}^{N_{1}+N_{2}}I_{k-l}H_{l}$
	Par suite $f \Re h$.\\
	
	(iii) \textbf{Anti-symétrie}. \\
	Soient $f=(I_{n})_{n\in \mathbb{N}}$ $,g=(J_{n})_{n\in \mathbb{N} }$\ des filtrations de $A$ telles que: \\
	$f \Re g$ alors $f\leq g$ et $g \Re f$ alors $g\leq f.$ \\
	D'où $f=g$. \\
	On en déduit que $\Re$ est une relation d'ordre. 
\end{proof}
\begin{maproposition}
	Soient $f=(I_n)_{n \in \mathbb{N}}$ et $g=(J_n)_{n \in \mathbb{N}}$ deux filtrations de $A$. $f\leqslant g$ et chaque $J_n$ est de type fini, alors $f$ est une $\alpha$-réduction de $g$ si et seulement si $R(A,g)$ est un $R(A,f)$-module de type fini.
\end{maproposition}
\begin{proof}
	Supposons que $f$ est une $\alpha$-réduction de $g$.\\
	Alors $f \leq g$ et $\text{ il existe } \, N \geq 1$ tel que $\text{ pour tout } n \geq N ; J_n = \displaystyle \sum_{p=0}^{N}{I_{n-p} J_p}$.\\
	Prouvons que $R(A,g) = R(A,f)(J_0, J_1X_1, \cdots , J_NX_N) = M$.\\
	$J_nX_n \subset R(A,g), \text{ pour tout } n \in \mathbb{N} \text{ alors } R(A,f) \subset R(A,g)$ car $f \leq g$,
	Donc $M \subset R(A,g)$.\\
	Montrons par récurrence que $\text{ pour tout } n \in \mathbb{N}, J_nX^n \subset M$
	Si $n \leq N, J_nX^n \subset M$, par construction.\\
	Soit $n \geq N$, supposons la propriété vraie et montrons que $J_{n+1}X^{n+1} \subset M$.\\
	$J_{n+1} = \displaystyle \sum_{p=0}^{N}{I_{n+1-p} J_p} \text{ alors } J_{n+1}X^{n+1} = \displaystyle \sum_{p=0}^{N}{I_{n+1-p} J_pX^{n+1}}$, \\
	D’où $ J_{n+1}X^{n+1} = \displaystyle \sum_{p=0}^{N}{I_{n+1-p}X^{n+1-p} J_pX^{p}} \subset M$.\\
	Ainsi $R(A,g)$ est un $R(A,f)$-module de type fini.\\
	Supposons que $R(A,g)$ est un $R(A,f)$-module de type fini.\\
	Montrons que $f$ est une $\alpha$-réduction de $g$.\\
	Par hypothèse on a : $f \leq g$.\\
	Trouvons $N \in \mathbb{N}$ tel que $\text{ pour tout } n \geq N ; J_n = \displaystyle \sum_{p=0}^{N}{I_{n-p} J_p}$.\\
	$R(A,g)$ étant un $R(A,f)$-module de type fini alors \\
	$R(A,g) = R(A,f)(1, Z_1, \cdots , Z_r), Z_i \in R(A,g)$ et les $Z_i$ sont homogènes de degré $i$.\\
	Posons $N=r$.\\
	Soit $z \in J_n$.\\
	$z \in J_n \text{ alors } zX^n \in J_nX^n \subset R(A,g)$, d'où $zX^n = \displaystyle \sum_{p=0}^{r}{h_p Z_p}$ , $h_p \in R(A,f)$ et $Z_0 = 1$.\\
	$h_p$ homogène de degré $n-p \text{ alors } h_p \in I_{n-p}X^{n-p}$ , on déduit de cela que:\\ $h_p = a_{n-p}X^{n-p} \in I_{n-p}X^{n-p}$.\\
	Aussi $Z_p$ est homogène de degré $p \text{ alors } Z_p = b_{p}X^{p} \in J_pX^p$.\\
	Ainsi $zX^n = \displaystyle \sum_{p=0}^{r}{a_{n-p}X^{n-p} b_{p}X^{p}} \text{ alors } zX^n = \displaystyle \sum_{p=0}^{r}{a_{n-p}X^{n} b_{p}}$, ce qui implique que \\ $z = \displaystyle \sum_{p=0}^{r}{a_{n-p} b_{p}} \in \displaystyle \sum_{p=0}^{r}{I_{n-p} J_{p}}$, par conséquent $J_n \subset \displaystyle \sum_{p=0}^{r}{I_{n-p} J_{p}} \subset J_n$.\\
	En somme, $J_n = \displaystyle \sum_{p=0}^{r}{I_{n-p} J_{p}}$, donc $f$ est une $\alpha$-réduction de $g$.
\end{proof}
\subsection{Réduction au sens de Dichi-Sangaré}
\begin{madefinition}
	($\beta$-réduction ou réduction au sens de Dichi-Sangaré \cite{Di4})\\
	Soient $f = (I_n)_{n \in \mathbb{N}}$, $g = (J_n)_{n \in \mathbb{N}}$ deux filtrations de $A$.\\
	$f$ est une $\beta$-réduction de $g$ si :
	\begin{enumerate}
		\item[i)] $f \leq g$;
		\item[ii)]  $\exists \, k \geq 1$ , $J_{n+k} = I_n J_k , \forall n \geq k$.
	\end{enumerate}
\end{madefinition}
\begin{maremarque}
	Si $f$ est une $\beta$-réduction de $g$ alors $f$ est une $\alpha$-réduction de $g$.	
\end{maremarque}
\begin{proof}
	Supposons que $f$ est une $\beta$-réduction de $g$.\\
	Alors $f \leq g$ et $\text{ il existe } \, k \geq 1$ tel que $I_{n+k} = I_n J_k , \text{ pour tout } n \geq k$.\\
	Posons $N = 2k$.\\
	Soit $n \geq N= 2k$.\\
	$\displaystyle \sum_{p=0}^{2k}{I_{n-p} J_{p}} = \displaystyle \sum_{p=0}^{k-1}{I_{n-p} J_{p}} + I_{n-k} J_k + \displaystyle \sum_{p=0}^{k+1}{I_{n-p} J_{p}}$, or $n \geq  2k \text{ alors } n-k \geq k$ et comme $f$ est une $\beta$-réduction de $g$ alors, $I_{n-k} J_k = J_n$.\\
	Donc $\displaystyle \sum_{p=0}^{2k}{I_{n-p} J_{p}} = \displaystyle \sum_{p=0}^{k-1}{I_{n-p} J_{p}} + J_k + \displaystyle \sum_{p=0}^{k+1}{I_{n-p} J_{p}} \text{ alors } J_n \subset \displaystyle \sum_{p=0}^{2k}{I_{n-p} J_{p}}$.\\
	De plus on a : $\displaystyle \sum_{p=0}^{2k}{I_{n-p} J_{p}} \subset J_n$. 
	Par conséquent $J_n = \displaystyle \sum_{p=0}^{2k}{I_{n-p} J_{p}}$.\\
	On déduit donc de tout ce qui précède que $f$ est une $\alpha$-réduction de $g$.
\end{proof}
\begin{maproposition}
	Soient $A$ un anneau et $I$ , $J$ deux idéaux de $A$.\\
	Alors les assertions suivantes sont équivalentes.\\
	$i)$ $I$ est une réduction de $J$.\\
	$ii)$ $f_I$ est une $\alpha$-réduction de $f_J$.\\
	$iii)$ $f_I$ est une $\beta$-réduction de $f_J$.
\end{maproposition}
\begin{proof}
	$i) \implies ii)$.\\
	Supposons que $I$ est une réduction de $J$.\\
	Alors $\text{ il existe } N \in \mathbb{N^*}$ tel que $J^{N+1} = IJ^N$.\\ $I \subset J \text{ alors } I^n \subset J^n , \text{ pour tout } n \in \mathbb{N}$, d'où $f_I \leq f_J$.\\
	Posons $N_0 = N+1$.\\
	Soit $n \geq N_0$.\\
	$\displaystyle \sum_{p=0}^{N+1}{I^{n-p} J^{p}} = \displaystyle \sum_{p=0}^{N}{I^{n-p} J^{p}} + I^{n-N-1} J^{N+1}$, comme $I$ est une réduction de $J$ alors\\ $I^{n-N-1} J^{N+1} = J^{n-N-1+N+1} = J^n$. Donc $\displaystyle \sum_{p=0}^{N+1}{I^{n-p} J^{p}} = \displaystyle \sum_{p=0}^{N}{I^{n-p} J^{p}} + J^{n}$.\\ Ainsi $J^n \subset \displaystyle \sum_{p=0}^{N+1}{I^{n-p} J^{p}} \subset J^n$.\\
	$f_I$ est donc une $\alpha$-réduction de $f_J$.\\
	$ii) \implies iii)$.\\
	Supposons que $f_I$ est une $\alpha$-réduction de $f_J$.\\
	Alors $f_I \leq f_J$ et $\text{ il existe } \, N_0 \in \mathbb{N^*} , \text{ pour tout } \, n \geq N_0, \, J^n = \displaystyle \sum_{p=0}^{N_0}{I^{n-p} J^{p}}$.\\
	Posons $N = N_0$.\\
	Soit $n \geq N$. \\
	$I^n J^{N_0} = \displaystyle \sum_{p=0}^{N_0}{I^n I^{N_0-p} J^{p}} = \displaystyle \sum_{p=0}^{N_0}{I^{n+N_0-p} J^{p}} = J^{N_0+n} \text{ alors } I^n J^{N_0} = J^{N_0+n}$.\\D'où $f_I$ est une $\beta$-réduction de $f_J$.\\
	$iii) \implies i)$
	Supposons que $f_I$ est une $\beta$-réduction de $f_J$.\\
	$f_I \leq f_J$ et $\text{ il existe } N_0 \in \mathbb{N^*}$ tel que $\text{ pour tout } n \geq N_0 , J^{n+N_0} = I^n J^{N_0}$.\\
	$f_I \leq f_J \text{ alors } I \subset J$.\\
	Posons $N = 2N_0$\\
	$J^{N+1} = J^{2N_0+1} = J^{N_0+N_0+1} = I^{N_0+1} J^N_0 = I I^{N_0} J^{N_0} = IJ^{2N_0}$.\\ Donc $J^{N+1}= IJ^{N}$ ce qui fait que $I$ est une réduction de $J$.
\end{proof}
\begin{maproposition}
	Soient $A$ un anneau commutatif unitaire, $I$ un idéal de $A$ et \\ $g = (J_n)_{n \in \mathbb{N}}$ une filtration de $A$. $f_I$ est une $\beta$-réduction de $g$ si et seulement si $g$ est $I$-bonne.
\end{maproposition}
\begin{proof}
	Supposons que $f_I$ est une $\beta$-réduction de $g$.\\
	Cela implique que $f_I \leq g$ et $\text{ il existe } \, N_0 \in \mathbb{N^*} , \text{ tel que pour tout } n \geq N_0 \\ J_{n+N_0} = I^n J_{N_0}$.\\
	$f_I \leq g \text{ alors } \text{ pour tout } n \in \mathbb{N} , IJ_n \subset J_{n+1}$.\\
	Posons $N = 2N_0$\\
	Soit $n \geq N$. 
	\begin{align*}
		J_{n+1} &= J_{n-N_0-N_0+1}.\\
		J_{n+1} &= J_{n-N_0+1N_0}.\\
		J_{n+1} &= I^{n-N_0+1} J_{N_0}.\\
		J_{n+1} &= II^{n-N_0} J_{N_0}.\\
		J_{n+1} &= IJ_n.
	\end{align*}
	D'où $\text{ pour tout } n \geq 2N_0$, $IJ_n = J_{n+1}$, $g$ est donc $I$-bonne. 
\end{proof}
\begin{maremarque}
	La $\beta$-réduction étant plus générale que la $\alpha$-réduction, nous parlerons de $\beta$-réduction lorsque nous emploierons le terme réduction.
\end{maremarque}

\begin{maproposition}
	Soit $f,g \in \mathbb{F}(A),$ telles que $f \leqslant g$.
	\begin{enumerate}
		\label{maprop4}
		\item[(i)] $f$ est un réduction de $g$ si et seulement s'il existe un entier naturel $k \geqslant 1$ tel que $J_{k+n}  = J_{k}I_n$ pour tout $n \geqslant k$.\\ Pour un tel entier $k$ et pour tout $m \geqslant 1$, $J_{mk}=J_{mk+pk-pk}=J_{pk+(m-p)k}=J_{pk}I_{(m-p)k}=J_{pk}J_{(m-p)k}=J_{k}^{p}J_{(m-p)k}=J_{k}^{p}I_{(m-p)k}=J_{k}^{p}J_{(m-p)k},$\\ pour tout $p=1,2,\cdots,m$;
		\item[(ii)] Si $f$ est une réduction de $g$ et que $g$ est une réduction de $h \in \mathbb{F}(A)$, alors $f$ est une réduction de $h$;
		\item[(iii)] Si $f$ est une réduction de $g$ et si $h$ est une filtration $A$ telle que $f \leqslant h \leqslant g$ alors $h$ est une réduction de $g$.
	\end{enumerate}
\end{maproposition}
\begin{proof}
	i) Supposons que $f$ soit une réduction de $g$. Alors:
	\begin{enumerate}
		\item[(a)] $f \leqslant g$;
		\item[(b)] $\exists r \geqslant 1,n_o \geqslant 0 ,\quad \forall n \geqslant n_0,\quad J_{r+n}= J_r I_n $.
	\end{enumerate}
	Soit $m_{0}\in \mathbb{N},$ tel que $m_{0}r\geq n_{0}$.
	
	Posons $k=m_{0}r$.
	
	Alors $J_{k+n}=J_{m_{0}r+n}=J_{m_{0}r}I_{n}=J_{k}I_{n}$ car $k\geq n_{0}.$
	
	La réciproque est évidente.
	
	ii) Supposons que $f$ est une réduction de $g$ et $g$ une réduction
	de $h.$
	
	* $f\leq g\leq h\text{ alors } f\leq h$.
	
	* Comme $\ g$ est une réduction de $h$ alors, il existe $k^{\prime }\geq
	1,$ $H_{k^{\prime }+n}=H_{k^{\prime }}J_{n},$ pour tout $n\geq k^{\prime }.$
	
	Posons $k^{\prime \prime }=k^{\prime }(k^{\prime }+1)$ comme dans (i).
	
	Ainsi en utilisant (i) car $f$ est une réduction de $g$, il vient  $H_{k^{^{\prime \prime }}+n}=H_{k^{^{\prime \prime }}}I_{n},$ pour tout $n\geq k^{^{\prime \prime }}.$
	
	Par suite $f$ est une réduction de $h$. \\
	
	iii) Supposons que $f$ réduction de $g$ et que $f\leq h\leq g.$
	
	Soit $k$ comme dans (i).
	
	Comme $h\leq g$ alors pour tout $n\geq k,$ $J_{k}H_{n}\subset
	J_{k}J_{n}=J_{k+n}\subset J_{k}H_{n}$ car $f\leq h.$
	
	Donc $J_{k+n}=J_{k}H_{n}$ $,$ pour tout $n\geq k.$
	
	Par suite $h$ est réduction de $g.$
\end{proof}

\begin{maremarque}
	Cependant, le fait que $g$ soit fortement entière sur $f$ n'implique pas
	nécessairement que $f$ soit une réduction de $g$, même si $f$ et 
	$g$ sont noethériennes. On peut le voir sur l'exemple suivant :\\ 
	Soit $A=k[X]$ l'anneau des polynômes à une indéterminée sur le
	corps $k$. \\ Soit $I=XA$.\\
	On considère les filtrations $f=(I_{n})_{_{n\in \mathbb{N}}}$ et $g=(J_{n})_{_{n\in \mathbb{N}}}$ définies par:\\
	$I_{n}=\left\{ 
	\begin{array}{c}
		I^{\frac{3n}{2}}\text{ si }n\text{ pair} \\ 
		I^{\frac{3n+3}{2}}\text{ si }n\text{ impair}
	\end{array}
	\right. $
	
	$J_{n}=\left\{ 
	\begin{array}{c}
		I^{\frac{3n}{2}}\text{ si }n\text{ pair} \\ 
		I^{\frac{3n+1}{2}}\text{ si }n\text{ impair}
	\end{array}
	\right. $
	
	On vérifie que $g$ est noethérienne et que $f\leq g$. \\ De plus, la filtration $g$ est entière sur $f$.\\ En effet pour tout élément $b\in J_{n}$, $b^{2}\in J_{2n}$ et on a $(bY^{n})^{2}=b^{2}Y^{2n}\in R(A,f)$. 
	
	L'anneau $R(A,g)$ est donc entier sur $R(A,f)$. De plus, comme $g$ est noethérienne, il résulte que $g$ est fortement entière sur $f$ et que $f$ est noethérienne.\\
	Néanmoins, $f$ n'est pas une réduction de $g$ puisqu'on n'a pas $J_{2p+1}^{2}=I_{2p+1}J_{2p+1}$, pour $p$ suffisamment grand, condition nécessaire pour qu'une filtration $f$ soit une réduction de $g$ quand
	l'anneau A est noethérien. 
\end{maremarque}
%\begin{maremarque}
%	Soit $f \in \mathbb{F}(A)$. On suppose que A est noethérien. Alors $f$ est une réduction de $f$ $\Longleftrightarrow$ f est noethérienne.
%\end{maremarque}
\section{Filtrations f-bonnes}
\begin{madefinition}
	\label{maprop11}
	Soient $A$ un anneau et $M$ un $A-module$.\\
	On suppose que $\varphi=(M_n)_{n \in \mathbb{N}}$ est $f-compatible$, avec $f $ appartenant à $ \mathbb{F}(A)$. Alors:
	\begin{itemize}
		\item[(a)] $\varphi$ est \text{faiblement $f-$ bonne} s'il existe un entier naturel N supérieur ou égal à 1 tel que:
		\[\forall n > N, M_{n}=\sum_{p=0}^{N}I_{n-p}M_{p}; \]
		\item[(b)] $\varphi$ est \text{$f-$ bonne} s'il existe un entier naturel N supérieur ou égal à 1 tel que:
		\[\forall n > N, M_{n}=\sum_{p=1}^{N}I_{n-p}M_{p}; \]
		\item[(c)] $\varphi$ est \text{$f-$ fine} s'il existe un entier naturel N supérieur ou égal à 1 tel que:
		\[\forall n > N, M_{n}=\sum_{p=1}^{N}I_{p}M_{n-p}. \]
	\end{itemize} 
\end{madefinition}
\begin{maremarque}
	\label{maprop6}
	\begin{enumerate}
		\item[(1)] Toute filtration $f-$bonne est faiblement $f-$ bonne.
		\item[(2)] Soit $f \in \mathbb{F}(A)$. Alors f est faiblement $f-$bonne.
		\item[(3)] Soit $f \in \mathbb{F}(A)$. Alors f est $f-$bonne si et seulement si f est $E.P.$
		\item[(4)] Soient $\varphi \in \mathbb{F}(M)$ et $I$ un idéal de $A$. Alors $\varphi$ est $I-bonne$ $\text{ si et seulement si }$ $\varphi$ est $f_{I}-bonne$, où $f_{I}$ est la filtration $I-adique$.
		\item[(5)] Soit $g \in \mathbb{F}(A)$ telle que $f \leqslant g$. Alors si $g$ est fortement entière alors g est faiblement $f-bonne$.
	\end{enumerate}
\end{maremarque}
\begin{proof}
	(1) Soit $f=(I_n)$ une filtration $f-bonne$. Alors il existe $N \geqslant 1$ tel que :
	\[ \forall n > N, I_n = \sum\limits_{p=1}^{N} I_{n-p}I_p. \]
	Ainsi pour $n> N$, $\sum\limits_{p=0}^{N} I_{n-p}I_p =  I_n + \sum\limits_{p=1}^{N} I_{n-p}I_p = I_n + I_n = I_n$. \\
	Donc $f$ est $faiblement$ $f-bonne$. \\
	(2) Soit $f=(I_n) \in \mathbb{F}(A)$. Alors:\\
	$I_n \subset \sum\limits_{p=0}^{N} I_{n-p}I_p$ et $ I_{n-p}I_p \subset I_n$
	Donc $\sum\limits_{p=0}^{N}I_{n-p}I_p \subset I_n$. \\ Par suite, $I_n = \sum\limits_{p=0}^{N} I_{n-p}I_p.$ Et donc $f$ est $faiblement$ $f-bonne$.\\
	(3) Par définition toute filtration $f-bonne$ est $E.P$.\\
	(4) Supposons que $\varphi \in \mathbb{F}(M)$ est $I-bonne$. Alors: \\
	Pour tout $n \in \mathbb{N}$, $IM_n \subset M_{n+1}$ et il existe $n_0 \in \mathbb{N}^{*}$ tel que pour tout $n \geqslant n_0$, $IM_n = M_{n+1}$.\\
	On a: $IM_n=M{n+1}$ alors $I^{2}M_n=M_{n+2}$, d'où $I^{n-p}M_p=M_n$ pour tout $n \geqslant n_0$.\\ Par suite $\sum\limits_{p=0}^{n_0}I^{n-p}M_p=M_n$ et donc $\varphi$ est $f_{I}-bonne$.\\
	Réciproquement supposons que $\varphi$ est $f_{I}-bonne$ alors il existe $N \geqslant 1$ tel que pour tout $n> N$, $M_{n+1} = \sum\limits_{p=0}^{N}I^{n+1-p}M_p =I(\sum\limits_{p=0}^{N}I^{n-p}M_p ) = IM_n$. Ainsi $\varphi$ est $I-bonne$.	
	
\end{proof}
\begin{maproposition}
	\label{maprop7}
	Si $f=(I_n)$ est une réduction de $g=(J_n)$ alors:
	\begin{enumerate}
		\item[(i)] $f$ est $A.P.$ et $g$ est fortement $A.P$;
		\item[(ii)] $g$ est $E.P$ et $f-bonne$;
		\item[(iii)] En plus, si $A$ est noethérien alors $f$ et $g$ sont noethériennes et g est fortement entière sur $f$.
	\end{enumerate}
\end{maproposition}
\begin{proof}
	Supposons que $f=(I_{n})_{_{n\in \mathbb{N}}}$ est une réduction de $g=(J_{n})_{_{n\in \mathbb{N}}}.$
	
	Alors $f\leq g$ et il existe $k\geq 1,$ tel que pour tout $n\geq k,J_{n+k}=I_{n}J_{k}=J_{k}J_{n}$.\\
	i) Nous avons $J_{nk}=J_{k}^{n}$ pour tout $n.$ Donc $g$ est $fortement$ $A.P.$\\
	De plus la division euclidienne de $n$ par $k$ donne $n=kq_{n}+r_{n}$, avec $0\leq r_{n}<k.$\\
	Posons $k_{n}=k(q_{n}+1).$\\
	Alors $\underset{n\longrightarrow +\infty }{\lim }\frac{k_{n}}{n}=\underset{n\longrightarrow +\infty }{\lim }\frac{kq_{n}+r_{n}+k-r_{n}}{n}=\underset{n\longrightarrow +\infty }{\lim}1+\frac{k-r_{n}}{n}=1.$\\
	Par ailleurs, $J_{k_{n}m}=J_{k_{n}}^{m}=J_{k(q_{n}+1)}^{m}\subset J_{n}^{m}.$\\
	Posons $k_{n}^{\prime }=k_{2k+n}.$\\
	Alors $I_{k_{n}^{\prime }m}\subset J_{k_{n}^{\prime }m}\subset J_{(k_{2k+n})m}=J_{k_{2k+n}}^{m}=J_{k}^{m}I_{k+n}^{m}\subset I_{n}^{m}.$\\
	Par suite $\underset{n\longrightarrow +\infty }{\lim }\frac{k_{n}^{\prime }}{n}=1$\\
	Donc $f$ est $A.P.$\\
	ii) Posons $N=2k.$ Alors si $n\geq N,$ $n-k\geq k$ et $J_{n}=I_{n-k}J_{k}\subset \sum\limits_{p=1}^{2k}I_{n-p}J_{p}$\\
	Donc $J_{n}=\sum\limits_{p=1}^{2k}I_{n-p}J_{p}$ et $g$ est $f-bonne$ et donc $g$ est $E.P.$\\
	iii) Si $A$ est noethérien alors d'après ii) $g$ et $f$ sont noethérienne. Et donc $g$ est $fortement$ $entière$ sur $f.$
\end{proof}
\begin{maproposition}
	\label{maprop3}
	Toute filtration $\varphi$ de $M$ $f-fine$ est $f-bonne$.
\end{maproposition}
\begin{proof}
	Supposons que $\varphi =(M_{n})$ est une filtration de $M$ qui est $f-fine$,
	où $f=(I_{n})_{_{n\in \mathbb{N}}}$ une filtration de $A.$
	
	Alors il existe $N\geq 1$ tel que pour tout $n>N,M_{n}=$ $
	\sum\limits_{p=1}^{N}I_{p}M_{n-p}.$
	
	Comme $n>N,$ posons $n=N+1$, ainsi\\
	
	$M_{N+1}=$ $\sum\limits_{p=1}^{N}I_{p}M_{N+1-p}=$ $\sum\limits_{q=1}^{N}I_{N+1-q}M_{q},$ avec $q=N+1-p.$
	
	Ainsi, il vient de proche en proche que $M_{N+j}=$  $\sum\limits_{p=1}^{N}I_{N+j-p}M_{p}\,\ ,$ pour tout $j$ avec $1\leq j\leq m.$
	
	Alors $M_{N+m}=$ $\sum\limits_{p=1}^{N}I_{p}M_{N+m-p}\,=\sum\limits_{q=m}^{N+m-1}I_{N+m-q}M_{q}\,=\sum\limits_{q=m}^{N}I_{N+m-q}M_{q}\,+\sum\limits_{q=N+1}^{N+m-1}I_{N+m-q}M_{q}=\sum\limits_{q=m}^{N}I_{N+m-q}M_{q}\,+\sum\limits_{q=N+1}^{N+m-1}I_{N+m-q}(\sum\limits_{p=1}^{N}I_{q-p}M_{p}).$
	
	Or $\sum\limits_{q=m}^{N}I_{N+m-q}M_{q}\,\subset
	\sum\limits_{p=1}^{N}I_{N+m-p}M_{p}\,$\ et $\sum\limits_{q=N+1}^{N+m-1}I_{N+m-q}(\sum\limits_{p=1}^{N}I_{q-p}M_{p})=\sum\limits_{p=1}^{N}(\sum\limits_{q=N+1}^{N+m-1}I_{N+m-p})M_{p}=\sum\limits_{p=1}^{N}I_{N+m-p}M_{p}\subset M_{N+m}$
	
	Par suite $\varphi $ est $f-bonne,$ l'inclusion inverse étant évidente.
\end{proof}
\begin{moncorollaire}
	\label{maprop8}
	Soient $f,g \in \mathbb{F}(A)$ avec $f \leqslant g$. Si $A$ est noethérien alors:\\ 
	$g$ faiblement $f-bonne$ $\text{ si et seulement si }$ $g$ est fortement entière sur $f$.
\end{moncorollaire}
\begin{maproposition}
	Soient $f=(I_n)$ une filtration $E.P.$ de $A$ et $\varphi=(M_n) \in \mathbb{F}(M)$. Nous avons les assertions suivantes:
	\[ \varphi \text{ est } f-fine \text{ si et seulement si } \varphi \text{ est } f-bonne \text{ si et seulement si } \varphi \text{ est faiblement } f-bonne. \]
\end{maproposition}
\begin{proof}
	Il suffit de montrer que $\varphi$ est faiblement $f-bonne$ si $\varphi$ est $f-fine$.\\ Supposons que $\varphi$ est faiblement $f-bonne$.\\
	Soient $N, N' \geqslant 1$ des entiers tels que pour tout $n \geqslant N,$
	$M_{n}=$ $\sum\limits_{p=0}^{N}I_{n-p}M_{p}$ et pour tout $n\geq
	1,I_{n}=\sum\limits_{p=1}^{N^{\prime }}I_{n-p}I_{p}.$ Alors pour $n>N^{\prime \prime }=N+N^{\prime},$
	
	$M_{n}=$ $\sum\limits_{p=0}^{N}I_{n-p}M_{p}=\sum\limits_{p=0}^{N}(
	\sum\limits_{q=1}^{N^{\prime}}I_{n-p-q}I_{p})M_{p}=\sum\limits_{q=1}^{N^{\prime}}I_{q}(\sum\limits_{p=0}^{N}I_{n-p-q}M_{p})=\sum\limits_{q=1}^{N^{\prime}}I_{q}M_{n-q}\subset\sum\limits_{q=1}^{N^{^{\prime \prime }}}I_{q}M_{n-q}.$
	
	Donc $M_{n}=\sum\limits_{q=1}^{N^{^{\prime \prime }}}I_{q}M_{n-q}$ ,
	l'inclusion inverse étant triviale. 
\end{proof}
\begin{moncorollaire}
	\label{maprop9}
	Soient $f,g \in \mathbb{F}(A)$. Si $A$ est noethérien, $f \leqslant g$ et $f$ noethérien. Alors nous avons les assertions suivantes:\\
	g est $f-fine$ $\text{ si et seulement si }$  g est $f-bonne$ $\text{ si et seulement si }$  g est faiblement $f-bonne$ $\text{ si et seulement si }$  g est fortement entière sur f.
\end{moncorollaire}
\begin{maproposition}
	Soient $f=(I_n), g=(J_n) \in \mathbb{F}(A)$, tel que $f \leqslant g$.\\ Si $g$ est $f-bonne$, $E.P.$ et $A$ est noethérien alors $f$ et $g$ sont noethériennes.
\end{maproposition}
\begin{proof}
	Il existe un entier $N\geq 1$ tel que pour tout $n>N,J_{n}=\sum
	\limits_{p=1}^{N}I_{n-p}J_{p}\subset
	\sum\limits_{p=1}^{N}J_{n-p}J_{p}\subset J_{n}$
	
	Donc $J_{n}=\sum\limits_{p=1}^{N}J_{n-p}J_{p}$ pour tout $n>N.$
	
	Cette égalité est valable si $1\leq n\leq N.$
	
	Comme $g$ est $E.P$ et $A$ noethérien alors $g$ est fortement entière sur $f.$ 
	
	Par suite $g$ est noethérien et d'après \cite{Eak}, $f$ est noethérien. 
\end{proof}
\begin{maproposition}
	Soient $f=(I_n), g=(J_n) \in \mathbb{F}(A)$, tel que $f \leqslant g$.\\Si $g$ est faiblement $f-bonne$ alors:\\ $f$ est $A.P$ $\text{ si et seulement si }$ $g$ est $A.P$.
\end{maproposition}
\begin{proof}
	Soient $f=(I_{n})_{_{n\in \mathbb{N}}},g=(J_{n})_{_{n\in\mathbb{N}}}\in \mathbb{F}(A).$
	
	Alors il existe un entier $N\geq 1$ tel que $I_{n}\subset J_{n}\subset
	I_{n-N}\subset J_{n-N}$ pour tout $n> N.$
	
	Si $f$ est $A.P.$ alors il existe une suite d'entiers $(k_{n})_{n\in \mathbb{N}}$ telle que $\underset{n\longrightarrow \infty }{\lim }\frac{k_{n}}{n}=1$ et $I_{k_{n}m}\subset I_{n}^{m}$,
	pour tout $m,n\in \mathbb{N}.$
	Par suite, $J_{(k_{n}+N)m}\subset J_{k_{n}m+Nm}\subset J_{k_{n}m+N}\subset I_{k_{n}m+N}\subset I_{k_{n}m}\subset I_{n}^{m}\subset J_{n}^{m}.$
	
	D'où $\underset{n\longrightarrow \infty }{\lim }\frac{k_{n}+N}{n}=1,$ $g$ est $A.P.$\\
	Réciproquement si $g$ est $A.P.$ alors il existe une suite d'entiers $(k_{n}^{^{\prime }})_{n\in \mathbb{N}}$ associée à $g.$
	
	Alors $I_{k_{n}^{\prime }+N.m}\subset J_{k_{n}^{\prime }+N.m}\subset J_{n+N}^{m}\subset I_{n}^{m}$ pour tout $m,n\in \mathbb{N}.$
	
	Et $\underset{n\longrightarrow \infty }{\lim }\frac{k_{n}^{\prime }+N}{n}=1,f$ est $A.P.$
\end{proof}
\begin{maproposition}
	\label{maprop10}
	Soient $A$ un anneau noethérien, $f=(I_n) , g=(J_n)$ deux filtrations de $A$. Si $f$ est noethérienne alors les assertions suivantes sont équivalentes:
	\begin{enumerate}
		\item[(i)] $g$ est fortement entière sur $f$;
		\item[(ii)] Il existe un entier $N \geqslant 1$, tel que $t_Ng \leqslant f \leqslant g$.
	\end{enumerate}
\end{maproposition}
\begin{proof}
	D'après \ref{maprop6} (3), il suffit de montrer que $(ii) \text{ alors } (i)$. Ce qui est une conséquence de \ref{maprop8} et de (\cite{Ok},2.9)
\end{proof}
%\begin{maremarque}
%	\label{maprop12}
%	Sous certaines conditions (voir \cite{Di1}, 3.1 (2), (b)) certaines propositions sur les filtrations entières sont équivalentes avec des filtrations fortement entières. 
%\end{maremarque}
%\begin{maproposition}
%	Soit A un anneau noethérien et $g$ une filtration de $A$ fortement entière sur $f$ ou $g$ est fortement $A.P.$ de rang $r$, alors pour tout entier naturel $m \geqslant 1$, $g^{(rm)}$ est fortement entière sur $f^{(rm)}$.  
%\end{maproposition}
%\begin{proof}
%	Si $g$ est fortement $A.P.$ alors $f$ est fortement $A.P.$
%	Alors Supposons que seulement $f$ est fortement $A.P.$ de rang $r.$
%	
%	Il existe un entier $N\geq 0$ tel que $t_{N}g\leq f\leq g.$\\
%	Posons $f=(I_{n})_{_{n\in \mathbb{N}}},g=(J_{n})_{_{n\in \mathbb{N}}},f_{1}=f^{(rm)},g_{1}=g^{(rm)}.$
%	
%	Nous avons $J_{N+n}\subset I_{n}$ pour tout $n\in \mathbb{N}.$
%	
%	De même $J_{rm(N+n)}\subset J_{N+rmn}$ $\subset I_{rmn}$  pour tout $n\in \mathbb{N}.$
%	
%	D'où $t_{N}g_{1}\leq f_{1}\leq g_{1}.$
%	
%	Or par hypothèse, $f^{(r)}=f_{I_{r}\text{ }}$alors $f^{(rm)}=f_{I_{rm}}$ qui est noethérienne.
%	
%	Donc $g_{1}$ est fortement entière sur $f_{1}.$
%\end{proof}
%
Voyons à présent le théorème principal de ce mémoire.
\begin{montheoreme}
	Soient $f=(I_{n})_{_{n\in \mathbb{N}}}\leq $ $g=(J_{n})_{_{n\in \mathbb{N}}}$ des filtrations sur l'anneau $A.$ \\
	Nous considérons les assertions suivantes:
	
	$i)$ $f$ est une réduction de $g.$
	
	$ii)$ $J_{n}^{2}=I_{n}J_{n}$ pour tout $n$ assez grand.
	
	$iii)$ $I_{n}$ est une réduction de $J_{n}$ pour tout $n$ assez grand.
	
	$iv)$ Il existe un entier $s$ supérieur ou égal à $ 1$ tel que pour tout $n$ supérieur ou égal à $ s,$ $J_{s+n}=J_{s}J_{n},$
	
	$I_{s+n}=I_{s}I_{n},$ $J_{s}^{2}=I_{s}J_{s},$ $J_{s+p}I_{s}=I_{s+p}J_{s}$ pour tout $p=1,2,...,s-1$
	
	$v)$ Il existe un entier $k$ supérieur ou égal à $ 1$ tel que $g^{(k)}$ est $I_{k}-bonne$
	
	$vi)$ Il existe un entier $r$ supérieur ou égal à $ 1$ tel que $f^{(r)}$ est une réduction de $g^{(r)}.$
	
	$vii)$ Pour tout entier $m$ supérieur ou égal à $ 1$ tel que $f^{(m)}$ est une réduction de $g^{(m)}.$
	
	$viii)$ $g$ est entière sur $f.$
	
	$ix)$ $g$ est fortement entière sur $f.$
	
	$x)$ $g$ est $f-fine.$
	
	$xi)$ $g$ est $f-bonne.$
	
	$xii)$ $g$ est $faiblement$ $f-bonne.$
	
	$xiii)$ Il existe un entier $N$ supérieur ou égal à $ 1$ tel que $t_{N}g\leq f\leq g$
	
	$xiv)$ Il existe un entier $N$ supérieur ou égal à $ 1$ tel que $t_{N}g^{\prime }\leq
	t_{N}f^{\prime}$.
	
	$xv)$ $P(f)=P(g)$.
	
	1) On a:
	
	$(i)$ $\Longleftrightarrow (vii)$ ; $(v)$ $\Longleftrightarrow (vi)$ ; $(viii)$ $\Longleftrightarrow (xv)$ ; $(ii)$ $\Longrightarrow (iii)$ ; $(iv)$ 
	$\Longrightarrow (i)\Longrightarrow (v)$ ; $(ix)\Longrightarrow (vii),(xii)$
	et $(xiii)$ ;
	
	$(i)\Longrightarrow (x)\Longrightarrow (xi)\Longrightarrow
	(xii)\Longrightarrow (xiii)$
	
	2) Si de plus on suppose $A$ noethérien, alors:
	
	$(i)\Longleftrightarrow (xiv)$ ; $(i)\Longrightarrow (ix)\Longleftrightarrow
	(xii)$ ; $(i)\Longrightarrow (ii)$
	
	3) Par ailleurs, si $f$ est $noeth\acute{e}rienne,$ alors $A$ est noethérien et les assertions suivantes sont équivalentes:
	
	$(ix)\Longleftrightarrow (x)\Longleftrightarrow (xi)\Longleftrightarrow
	(xii)\Longleftrightarrow (xiii)$
	
	4) Si $f$ et $g$ sont noethériennes alors nous avons:
	
	$(iii)\Longrightarrow (viii)\Longleftrightarrow (ix)$ ; $(vi)\Longrightarrow (ix)$
	
	5) Si $f$ est fortement noethérienne et $g$ est noethérienne alors les quinze (15) assertions sont équivalentes et dans ce cas $g$ est fortement noethérienne
	
	$(i)\Longleftrightarrow (ii)\Longleftrightarrow (iii)\Longleftrightarrow
	(iv)\Longleftrightarrow (v)\Longleftrightarrow (vi)\Longleftrightarrow
	(vii)\Longleftrightarrow (viii)\Longleftrightarrow (ix)\Longleftrightarrow
	(x)\Longleftrightarrow (xi)\Longleftrightarrow (xii)\Longleftrightarrow
	(xiii)\Longleftrightarrow (xiv)\Longleftrightarrow (xv).$
\end{montheoreme}
\begin{proof}
	1)
	
	$(i)\Longleftrightarrow (vii).$
	
	Supposons $(i)$ et choisissons $k$ comme dans \ref{maprop4} (i) alors pour tout
	entiers $m\geq 1$ et $n\geq k,$ $J_{m(k+n)}=J_{mk}I_{mn}$, ce qui entraîne $(vii).$
	
	La réciproque est évidente.
	
	$(v)\Longrightarrow (vi).$
	
	Posons $f^{(k)}=(H_{n});$ $g^{(k)}=(K_{n});$ $H_{n}=I_{nk};$ $K_{n}=J_{nk};$ 
	$H_{1}=I_{k};$
	
	Par hypothèse, $H_{1}K_{n}\subset K_{n+1}$ pour tout entier $n$ et il
	existe un entier $n_{0}\geq 1$ tel que $H_{1}K_{n}=K_{n+1}$ pour tout $n\geq
	n_{0}.$
	
	Pour tout entier $m\geq 0,$ $K_{n_{0}+m}=H_{1}^{m}K_{n_{0}}\subset
	H_{m}K_{n_{0}}\subset K_{n_{0}+m}.$
	
	Donc $K_{n_{0}+m}=K_{n_{0}}H_{m}$ pour tout entier $m.$ Et donc $f^{(k)}$
	est une réduction de $g^{(k)}.$
	
	$(vi)\Longrightarrow (v).$
	
	Il suffit de montrer que si $f$ est une réduction de $g$ alors il existe 
	$k\geq 1$ tel que $g^{(k)}$ est $I_{k}-bonne.$
	
	Posons $k$ comme dans \ref{maprop4} (i), alors pour tout entiers $m\geq 1$ et  $J_{k(m+1)}=J_{mk}I_{k}$, donc $g^{(k)}$ est $I_{k}-bonne.$
	
	Donc $(vi)\Longrightarrow (v).$
	
	$(viii)\Longleftrightarrow (xv).$
	
	Si $g$ est entière sur $f$ alors $f\leq g\leq P(f),$ ainsi $P(f)\leq
	P(g)\leq P(P(f))=P(f),$ donc $P(g)=P(f).$
	
	Réciproquement si $P(f)=P(g)$ alors $g\leq P(g)=P(f)$ et donc $g$ est entière sur $f.$
	
	$(ii)\Longrightarrow (iii).$
	
	Évident.
	
	$(iv)\Longrightarrow (i).$
	
	Posons $n\geq 2s$ et $n=qs+p$ avec $0\leq p<s.$
	
	Alors $J_{s+n}=J_{(q-2)s+2s+(s+p)}=J_{s}^{q-2}J_{2s+(s+p)}=J_{s}^{q-2}J_{s}^{2}J_{s+p}=J_{s}^{q-1}I_{s}J_{s+p}=J_{s}^{q-1}J_{s}I_{s+p}=J_{s}^{q}I_{s+p}=J_{s}I_{s}^{q-1}I_{s+p}\subset J_{s}I_{n}\subset J_{s+n}.
	$
	
	Par suite $J_{s+n}=J_{s}I_{n}$ pour tout $n\geq 2s.$ Donc $
	J_{2s+n}=J_{2s}I_{n}$ pour tout $n\geq 2s.$ D'où $(i).$
	
	
	$(i)\Longrightarrow (v)$
	
	Évident car $(vi)\Longrightarrow (v).$
	
	$(ix)\Longrightarrow (viii)$
	
	Évident
	
	$(ix)\Longrightarrow (xii)\Longrightarrow (xiii)$ en utilisant \ref{maprop6} (5)
	
	$(i)\Longrightarrow (x).$
	
	Pour tout entier $n\geq N=2k-1,$ posons $n=qk+r,$ avec $0\leq r<k$ où $k$
	est comme dans $(4.3)$ $(i).$
	
	Alors $J_{n}=J_{k(q-1)}I_{k+r}.$
	
	Ainsi $1\leq k+r<2k-1,$ $J_{n}\subset
	\sum\limits_{p=1}^{N}I_{p}J_{n-p}\subset J_{n}$, d'où $J_{n}=\sum\limits_{p=1}^{N}I_{p}J_{n-p}$ pour tout $n\geq N=2k-1$.
	
	Ce qui prouve que $g$ est $f-fine.$
	
	$(x)\Longrightarrow (xi)$ par \ref{maprop3}
	
	$(xi)\Longrightarrow (xii)$ par \ref{maprop6} (1).
	
	
	2) 
	
	On suppose maintenant que $A$ est noethérien.
	
	Alors $(i)\Longrightarrow (ix)$ en utilisant \ref{maprop7}.
	
	$(i)\Longrightarrow (ii)$
	
	$f$ est noethérienne par $\ref{maprop7}$ donc il existe un entier $k^{\prime }$ tel que $I_{n+k^{\prime }}=I_{n}I_{k^{\prime }},$ pour tout $n\geq k^{\prime }$.
	
	Choisissons $k$ comme dans \ref{maprop4} (i) nous pouvons supposons que $k=k^{\prime }$ et même prendre $kk^{\prime }$ \`{a} la place de $k$ ou $k^{\prime }$ si nécessaire.
	
	Pour tout $n\geq 3k,$ posons $n=qk+r,$ avec $0\leq r<k.$ Alors $q=E(\frac{n}{k})\geq 3.$
	
	$J_{n}=J_{k}I_{(q-1)k+r}=J_{k}I_{k}^{q-2}I_{k+r}.$
	
	$J_{n}^{2}=J_{k}^{2}I_{k}^{q-3}(I_{k}^{q-1}I_{k+r})I_{k+r}\subset J_{2k}I_{(q-3)k}I_{n}I_{k+r}.$
	
	D'où $J_{n}^{2}\subset J_{n}I_{n}$
	
	Donc $J_{n}^{2}=J_{n}I_{n}$ pour tout $n\geq 3k.$
	
	$(i)\Longleftrightarrow (iv).$ 
	
	D'après 1) il suffit de montrer que $(i)\Longrightarrow (iv).$
	
	Nous avons vu que $(i)\Longrightarrow (ii).$ Alors il existe un entier $
	k^{\prime }\geq 1$ tel que $J_{n}^{2}=I_{n}J_{n}$ pour tout $n\geq k^{\prime}.$
	
	Dans la preuve de la même implication, nous avons aussi montrer qu'il existe un entier $k\geq 1$ tel que $J_{k+n}=J_{k}I_{n}=J_{k}J_{n}$ et que $I_{k+n}=I_{k}I_{n}$ pour tout $n\geq k.$
	
	Posons $n\geq 2kk^{\prime }=s,$ $k"=kk^{\prime }$ et $n=qk"+r$ avec $0\leq r<k".$ Alors $q\geq 2$ et:
	
	$J_{s+n}=J_{(q+2)k"+r}=J_{k"}^{3}J_{(q-1)k"+r}=I_{k"}^{2}J_{k"}J_{(q-1)k"+r}=I_{s}J_{n}.$
	
	$J_{s+n}=J_{s}I_{n}=J_{s}J_{n}$
	
	$I_{s+n}=I_{s}I_{n}$
	
	$J_{s}^{2}=I_{s}J_{s}$
	
	D'où $(iv).$
	
	$(ix)\Longleftrightarrow (xii)$ d'après \ref{maprop8}
	
	$(iii)\Longleftrightarrow (xiv)$
	
	Nous savons que pour tout idéal $I\subset J$ d'un anneau noethérien, $I$ est une réduction de $J$ si et seulement si $I^{\prime }=J^{\prime },$ où $I^{\prime }$est la clôture intégrale de $I.$ D'où l'équivalence. 
	
	3) Supposons que $f$ est noethérienne.
	
	Alors d'après \ref{maprop10}, $(ix)\Longleftrightarrow (xiii)$ et d'après $\ref{maprop9},$ 
	
	$(ix)\Longleftrightarrow (x)\Longleftrightarrow (xi)\Longleftrightarrow (xii)\Longleftrightarrow (xiii)$
	
	4) Supposons que $f$ et $g$ sont noethériens. Alors $(viii)\Longleftrightarrow (ix)$ d'après (\cite{Di1}, \ref{maprop11},(b)).
	
	$(iii)\Longrightarrow (viii).$
	
	Supposons que $I_{n}$ est une réduction de $J_{n}$ pour tout $n\geq n_{0}.$
	
	$f$ et $g$ sont noethérien d'où fortement $A.P.$ \`{a} partir d'un rang
	commun $k.$
	
	L'idéal $J_{n_{0}k}$ est entière sur l'idéal $I_{n_{0}k}.$ D'où $g$ est entière sur $f$ d'après (\cite{Di1}, 4.5).
	
	$(vi)\Longrightarrow (ix).$
	
	Si $f^{(r)}$ est une réduction de $g^{(r)}$ alors $g^{(r)}$ est fortement entière sur $f^{(r)}$ d'après \ref{maprop7} (iii) et $g$ est fortement entière sur $f$ d'après \ref{maprop12}
	
	5) Supposons que $f$ est fortement noethérienne. D'après
	l'implication précédente il est facile de montrer que $(xii)\Longrightarrow (i).$
	
	Supposons que $(xii),$ alors il existe un entier $N\geq 1$ tel que pour tout $n>N,$ $J_{n}=\sum\limits_{p=0}^{N}I_{n-p}J_{p}.$
	
	$f$ étant fortement noethérienne, il existe un entier $N^{\prime}\geq 1$ tel que $I_{m}I_{n}=I_{m+n}$ pour tout $m,n\geq N^{\prime }.$
	
	Posons $n\geq k=N+N^{\prime }.$
	
	Si $0\leq p\leq N,$ alors $N^{\prime }=k-N\leq k-p\leq n-p,$ $J_{n+k}=\sum\limits_{p=0}^{N}I_{n+k-p}J_{p}=\sum\limits_{p=0}^{N}I_{n}I_{k-p}J_{p}=I_{n}J_{k}$ , et $f$ est une réduction de $g.$
	
	Pour compléter la preuve, nous avons montrer par exemple que si $f$ est une réduction de $g$ et si $f$ est fortement noethérienne alors $g$ est fortement noethérien.
	
	Soient $k,$ $k^{\prime }$ des entiers $\geq 1$ tel que $J_{k+n}=J_{k}I_{n}$
	pour tout $n\geq k$ et $I_{m+n}=I_{m}I_{n}$ pour tout $m,n\geq k^{\prime }.$
	
	Posons $m,n\geq k^{\prime }.$ Alors $J_{m}J_{n}\subset
	J_{m+n}=J_{k}I_{m+(n-k)}=J_{k}I_{m}I_{n-k}\subset J_{m}J_{n},$ d'où $J_{m+n}=J_{m}J_{n}$ pour tout $m,n\geq k+k^{\prime }$ et $g$ est fortement noethérienne.
\end{proof}
	\addcontentsline{toc}{chapter}{\large{Conclusion}}
\begin{center}
	\chapter*{Conclusion}
\end{center}

$ \quad $ L'objectif de ce travail était l'étude de la dépendance intégrale et de la réduction par rapport aux idéaux à travers la filtration I-adique qui est une filtration I-bonne. \\
Ensuite, nous avons montré la dépendance intégrale et la réduction des filtrations bonnes en générale. Nous pouvons donc retenir que toutes les filtrations bonnes admettent une réduction minimale. De plus, nous pouvons si certaines conditions sont vérifiées établir des propositions équivalentes entre les notions de réduction, dépendance intégrale et filtrations bonne.

Comme perspective, nous projetons d'étudier sous quelles hypothèses nous pourrons étendre ces résultats aux autres classes de filtrations notamment les filtrations noethériennes.



	\begin{thebibliography}{99}
	
	\addcontentsline{toc}{chapter}{\large{Bibliographie}}
	
	\bibitem{Bi} Bishop W., Petro J.W., Ratliff Jr L. J. et Rush D.,\textit{ Note on noetherian filtrations}, Communications in Algebra, Vol.17, No. 2, 1988, pp. 471-485.
	
	\bibitem{Di1} Dichi H.,\textit{ Integral dependence over a filtration}, Journal of Pure and Applied Algebra, Vol. 58, No. 1, 1989, pp. 7-18. 
	
	\bibitem{Di2} Dichi H., \textit{Travaux de recherches, en vue de l'habilitation à diriger une recherche}, 1999.
	
	\bibitem{Di3} Dichi H., Sangare D., et Soumare M.,\textit{ Filtrations, integral dependence, reduction, f-good filtrations}, Communications in Algebra, Vol. 20, No. 8, 1992, pp. 2393-2418.
	
	\bibitem{Di4} Dichi H. et Sangaré D.,\textit{ Filtrations, asymptotic ans prüferian closures, cancellation laws}, Proceedings of the American Mathematical Society, Vol. 113, No. 3, 1991, pp. 617-624. 
	
	\bibitem{Eak} Eakin P., \textit{The converse to a well known theorem on noetherian rings}, Math. Ann., Vol. 177, 1968, pp. 278-282.
	
	\bibitem{No} Northcott D. G., et Rees D.,\textit{ Reductions of ideals in local rings}, in Mathematical Proceedings of the Cambridge Philosophical Society, Vol. 50, No. 2, 1954, pp. 145-158.
	
	\bibitem{Ok} Okon J. S., et Ratliff L,\textit{ Reductions of filtrations}, Pacific Journal of Mathematics, Vol. 144, No. 1, 1990, pp. 137-154.
	
	\bibitem{Pr} Prüfer H.,\textit{ Untersuchungen über Teilbarkeitseigenschaften in Körpern}, Journal für die reine und angewandte Mathematik, Vol. 168, 1932, pp. 1-36.
	
	\bibitem{Ra1} Ratliff Jr L. J.,\textit{ Notes on essentially powers filtrations}, Michigan Mathematical Journal, Vol.26, No. 3, 1979, pp. 313-324.
	
	\bibitem{Ra2} Ratliff Jr L. J. et Rush D.,\textit{ Note on I-good filtrations and noetherian Rees rings}, Communications in Algebra, Vol.16, No. 5, 1988, pp. 955-975.

\end{thebibliography}

\end{document}