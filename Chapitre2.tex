\chapter{DÉPENDANCE INTÉGRALE, RÉDUCTION ET FILTRATION I-ADIQUE}
\chaptermark{DÉPENDANCE, RÉDUCTION ET FILTRATION I-ADIQUE}

Dans ce second chapitre, notre étude se concentrera sur les filtrations I-adiques, lesquelles représentent un cas spécifique de filtrations bonnes.

\section{Dépendance intégrale}
\subsection{Dépendance intégrale sur les anneaux}
\begin{madefinition}\textbf{(Élément Entier)} \\
	Soient $B$ un anneau et $A$ un sous-anneau de $B$ ($A $ est contenu dans $ B$).\\
	Un élément $x$ de $B$ est dit \textbf{entier} sur $A$ s'il est \textbf{racine d'un polynôme unitaire à coefficient dans} $A$.\\
	En d'autres termes, s'ils existent $a_1, a_2, \cdots , a_n$ éléments de $A$ tels que :\\
	\begin{equation}
		\label{eq1}
		x^n + a_1 x^{n-1} +\cdots+a_i x^{n-i} +\cdots + a_n = x^n + \sum_{i=1}^{n} a_i x^{n-i} = 0, \; n \in \mathbb{N^*}.
	\end{equation}
	
	Cette relation (\ref{eq1}) est appelée \textbf{équation de dépendance intégrale} de $x$ sur $A$.\\
	On dit que $B$ est entier sur $A$ lorsque tous les éléments de $B$ sont entiers sur $A$.
\end{madefinition}

\begin{monexemple}
	Posons $B = \mathbb{R}$ et $A = \mathbb{Z}$. $A$ est un sous-anneau de $B$.\\
	$\bullet$ $x = \sqrt{5}$ est solution de l'équation $x^2 - 5=0$. Donc $\sqrt{5}$ est entier sur $\mathbb{Z}$.\\
	$\bullet$ $x = \sqrt{2}+1$ est solution de l'équation $x^2 - 2x -1=0$.\\Donc $x = \sqrt{2}+1$ est entier sur $\mathbb{Z}$.\\
\end{monexemple}
\begin{moncontreexemple}
	Cependant, $\dfrac{1}{2}$ n'est pas entier sur $\mathbb{Z}$.\\
	En effet, si $\dfrac{1}{2}$ est entier sur $\mathbb{Z}$, alors $\dfrac{1}{2}$ est solution d'une équation de dépendance intégrale. Alors: \\
	$\exists$ n $\geq 1 , \left(\dfrac{1}{2} \right)^n + \sum\limits_{i=1}^{n} a_i \left(\dfrac{1}{2} \right)^{n-i} = 0$ , $a_i \in\mathbb{Z}$.\\
	D'où $\left(\dfrac{1}{2} \right)^n \left[1+\sum\limits_{i=1}^{n} a_i \left(\dfrac{1}{2} \right)^{-i}\right] = 0$.\\
	Comme $\left(\dfrac{1}{2} \right)^n \neq 0$ alors $1+\sum\limits_{i=1}^{n} a_i2^{i} = 0.$\\D'où $1+\sum\limits_{i=1}^{n} a_i2^{i} = 0.$\\ Donc $1 = 2\left(\sum\limits_{i=1}^{n} -a_i 2^{i-1}\right).$\\Par suite, $ 1$ appartient à $ 2\mathbb{Z}.$ Ce qui est \textbf{Absurde}.
\end{moncontreexemple}
\begin{maproposition}
	\label{maprop12}
	Soient $B$ un anneau et $A$ un sous-anneau de $B$ ($A $ est contenu dans $ B$).\\
	Soit $x $ appartenant à $ B$. \\
	Les assertions suivantes sont équivalentes:
	\begin{enumerate}
		\item[i)]$x$ est entier sur $A$;
		\item[ii)]$A\left[ x\right]$ est un $A$-module de type fini;
		\item[iii)]Il existe $C$ un sous-anneau de $B$ contenant $A\left[ x\right]$ tel que $C$ soit un $A$-module de type fini.
	\end{enumerate}
\end{maproposition}
\begin{proof}
	$i \implies ii)$
	
	Supposons que $x$ est entier sur $A.$
	
	Alors il existe $n$ $\in \mathbb{N}^{\ast },$ tel que $x^{n}+\sum\limits_{i=1}^{n}a_{i}x^{n-i}=0,$ $\text{ pour tout } i\in \llbracket 1; n \rrbracket ,a_{i}\in A.$
	
	D'où il existe $n$ $\in \mathbb{N}^{\ast },$ tel que $x^{n}=\sum\limits_{i=1}^{n}(-a_{i})x^{n-i},$ $\text{ pour tout } i\in \llbracket 1; n \rrbracket ,a_{i}\in A.$
	
	Donc il existe $n$ $\in \mathbb{N}^{\ast },$ tel que $x^{n}\in A(1_A,x,x^{2},...,x^{n-1})=A[x].$\\
	$\bullet$ Montrons que $A(1_A,x,x^{2},...,x^{n-1})=A[x].$\\
	a) Par construction, $A(1_A,x,x^{2},...,x^{n-1})$ est contenu dans $ A[x].$\\
	b) Réciproquement, montrons par récurrence sur $m$ que pour tout \\ $m	\in \mathbb{N},x^{m}\in A(1_A,x,x^{2},...,x^{n-1}).$
	
	* Si $m\in \llbracket 0; n-1 \rrbracket ,$ alors $x^{m}\in A(1_A,x,x^{2},...,x^{n-1}).$
	
	* Si $m\geq n$, alors $m=n+p,$ $p\geq 0.$
	
	\underline{Initialisation}
	
	Si $p=0$ alors  $x^{n}+\sum\limits_{i=1}^{n}a_{i}x^{n-i}=0\text{ alors }
	x^{n}=\sum\limits_{i=1}^{n}(-a_{i})x^{n-i}.$
	
	Comme $1\leq i\leq n,$ alors $0\leq n-i\leq n-1$.
	
	D'où, $x^{n}=\sum\limits_{i=1}^{n}(-a_{i})x^{n-i}\in
	A(1_A,x,x^{2},...,x^{n-1}).$
	
	Donc la propriété est vraie pour $p=0$.
	
	\underline{Hérédité}
	
	Soit $p\geq 0.$ Supposons que $\text{ pour tout } k\in \llbracket 0, p \rrbracket
	,x^{n+k}\in A(1,x,x^{2},...,x^{n-1})$.\\ Montrons que $x^{n+p+1}\in
	A(1_A,x,x^{2},...,x^{n-1}).$
	
	$x^{n+p+1}=x^{p+1}x^{n}=x^{p+1}\times
	\sum\limits_{i=1}^{n}(-a_{i})x^{n-i}=\sum\limits_{i=1}^{n}(-a_{i})x^{n+p+1-i}$.
	
	Comme $1\leq i\leq n,$ alors $p+1\leq n+p+1-i\leq n+p$.
	
	Ainsi par hypothèse de récurrence, $x^{n+p+1-i}\in
	A(1_A,x,x^{2},...,x^{n-1})$ et par stabilité, il vient $x^{n+p+1}\in
	A(1_A,x,x^{2},...,x^{n-1}).$
	
	Par suite, pour tout $m$ $\in \mathbb{N},x^{m}\in A(1_A,x,x^{2},...,x^{n-1}).$
	
	Donc $A(1_A,x,x^{2},...,x^{n-1})=A[x]$.
	
	$A[x]$ est donc un $A-$module de type fini de générateur $
	(1_A,x,x^{2},...,x^{n-1}).$
	
	$ii)\implies iii)$ Il suffit de poser $A[x]=C$.
	
	$iii)\implies i)$ Supposons qu'il existe $C$ un sous module de $B$ contenant $A[x]$ qui soit un $A-$module de type fini.
	
	Soit $x\in A.$
	
	$A[x]$ est contenu dans $ C=A(y_{1},y_{2},...,y_{r})$ de type fini.
	
	Ainsi, pour tout $i\in \llbracket 1; r \rrbracket ,$ $xy_{i}\in C.$ On
	a:
	
	$xy_{i}=\sum\limits_{j=1}^{r}a_{ij}y_{j}\text{ alors }
	\sum\limits_{j=1}^{r}a_{ij}y_{j}-xy_{i}=0\text{ alors }
	\sum\limits_{j=1}^{r}(a_{ij}-\delta _{ij}x)y_{j}=0$. \\Où $\delta_{ij}=\left\{ 
	\begin{array}{ccc}
		1 & { si } & i=j; \\ 
		0 & { sinon}.
	\end{array}
	\right. $
	
	D'où $\left( 
	\begin{array}{cccc}
		a_{11}-x & a_{12} & \cdots & a_{1r} \\ 
		a_{22} & a_{11}-x & \cdots& \vdots\\ 
		\vdots & \vdots& \ddots  & \vdots\\ 
		a_{r1} & \cdots & \cdots & a_{rr}-x
	\end{array}
	\right) \left( 
	\begin{array}{c}
		y_{1} \\ 
		y_{2} \\ 
		\vdots \\ 
		y_{r}
	\end{array}
	\right) =\left( 
	\begin{array}{c}
		0 \\ 
		0 \\ 
		\vdots\\ 
		0
	\end{array}
	\right) $
	
	Posons $T=(a_{ij})_{1\leq i,j\leq r}$ $,$ $Y=(y_{j})_{1\leq j\leq r}$. 
	
	Ainsi $(T-xI_{r})\times Y=0_{r}.$
	
	$(T-xI_{r})\times Y=0 \text{ ainsi } {^{t}com}(T-xI_{r})\times Y[(T-xI_{r})\times Y]=0. \\ 
	\text{ Alors } \det [(T-xI_{r})Y]=0. \\
	\text{ D'où } \det(T-xI_{r})y_{i}=0,$ $\text{ pour tout } i\in \llbracket 1; r \rrbracket $. 
	\\$P_{T}(x)y_{i}=0,$ $\text{ pour tout } i\in \llbracket 1; r \rrbracket.$
	
	De plus, $1_A\in C,$ on peut donc supposer qu' $\text{il existe } i\in \llbracket 1; r \rrbracket ,$ tel que $y_{i}=1_A.$\\
	D'où $P_{T}(x)=0.$\\
	Or $P_{T}$ est un polynôme unitaire qui s'écrit:\\ $P_{T}(x)=x^{n}-tr(T)x^{n-1}+...+(-1)^{n}\det
	(T)=x^{n}+\sum\limits_{i=1}^{n}\alpha _{i}x^{n-i},\alpha _{i}\in A.$
	
	Donc $P_{T}(x)=0\text{ alors } x^{n}+\sum\limits_{i=1}^{n}\alpha
	_{i}x^{n-i}=0,\alpha _{i}\in A\text{ alors } x$ est entier sur $A.$
	
\end{proof}
\begin{moncorollaire}
	Soient $A$ et $B$ deux anneaux tels que $A $ est contenu dans $  B$ et $x_1, x_2, \cdots, x_n $ appartenant à $ B$.\\
	Alors les assertions suivantes sont équivalentes:
	\begin{enumerate}
		\item[i)]$\text{ pour tout } i $ appartenant à $ \llbracket 1; n \rrbracket, x_i$ est entier sur $A$;
		\item[ii)]$A[x_1, x_2, \cdots, x_n]$ est un $A$-module de type fini;
		\item[iii)]Il existe un $A-module$ de type fini $C $ est contenu dans $  B$ tel que $x_i C $ est contenu dans $  C, \text{ pour tout } i $ appartenant à $ \llbracket 1; n \rrbracket$.
	\end{enumerate}
\end{moncorollaire}
\begin{proof}
	En faisant comme démonstration de la proposition (\ref{maprop12}) et en procédant de proche en proche. 
\end{proof}

\begin{madefinition}
	Soit $A $ contenu dans $ B$ une inclusion d'anneaux.\\
	On qualifie $B$ d'une \textbf{A-algèbre} lorsque $B$ peut être appréhendé simultanément en tant que module sur A et en tant qu'anneau.\\
	De plus, B est dit \textbf{de type fini} si $B$ est un $A-module$ de type fini.
\end{madefinition}

%\begin{maproposition}
%	Soit $A $ contenu dans $ B$ une inclusion d'anneaux.\\
%	Si B est une \textbf{A-algèbre de type fini} alors:
%	\[ B \simeq \dfrac{A[X_1, X_2, \cdots, X_r]}{J} \]
%	Où $J$ est un idéal de $A[X_1, X_2, \cdots, X_r]$
%\end{maproposition}
\begin{moncorollaire}\textbf{(Clôture intégrale d'anneaux)}\cite{Di2}\\
	Soit $A $ contenu dans $ B$ une inclusion d'anneaux.\\
	L'ensemble des éléments de B entiers sur A est un sous-anneau de B contenant A appelé \textbf{clôture intégrale} de $B$ dans $A$ notée $A'$.
\end{moncorollaire}
\begin{proof}
	Il s'agit de montrer que $A^{\prime }$ est un sous anneau de $B.$
	
	$i)$ $1_B $ appartient à $ A \text{ et } x-1_B $ est nul $ \text{ donc } 1_B$ appartient à $ A'.$ \\
	$\text{ D'où } A\subset A^{\prime }\subset B.$
	
	$ii)$ Soient $x,y$ appartenant à $ A^{\prime }.$
	
	$x$ appartient à $ A^{\prime }$ alors $A[x]$ est un $A-$module de type fini.
	
	$y$ appartient à $ A^{\prime }$ alors $A[y]$ est un $A-$module de type fini.
	
	Posons $C=A[x,y]=A[x][y].$
	
	Ainsi pour tout $z\in C,z=\sum\limits_{i=1}^{r}\alpha _{i}y^{i},\alpha
	_{i}\in A[x].$ Ainsi $\alpha _{i}=\sum\limits_{j=1}^{s}\alpha _{ij}x^{j}.$
	
	D'où $z=\sum\limits_{i=1}^{r}\sum\limits_{j=1}^{s}\alpha _{ij}x^{j}y^{i}.$
	
	Donc $C=A(x^{j}y^{i})_{1\leq i\leq r,1\leq j\leq s,}$ est un système de générateur fini de $C.$
	
	Par suite, $C$ est un $A-$module de type fini.\\
	De plus $A[x+y] $ est contenu dans $ C$ ainsi $A[x+y]$ est de type fini et donc $x+y$ est entier sur $A$.\\ Par suite $x+y \in A'$. De même $A[xy]$ est de type fini et donc $xy$ est entier sur $A$.\\ Par suite $xy \in A'$. \\ On en déduit que $A'$ est un sous anneau de $B$.
\end{proof}

\begin{moncorollaire}
	Soient $A $ contenu dans $ B$ et $B$ contenu dans $ C $ deux inclusions d'anneaux.\\
	Si $C$ est entier sur $B$ et $B$ est entier sur $A$ alors $C$ est entier sur $A$.
\end{moncorollaire}
\begin{proof}
	Soit $x\in C.$
	
	Comme $x$ est entier sur $B$ alors il existe $n\in \mathbb{N}^{\ast },x^{n}=\sum\limits_{i=1}^{n}(-b_{i})x^{n-i},$ pour tout $b_{i}\in
	B,i\in \llbracket 1; n \rrbracket.$
	
	Soit $i\in \llbracket 1; n \rrbracket,$ $A[b_{i}]$ est un $A-$module de type fini
	(car $B$ entier sur $A).$
	
	Donc $A[b_{1},b_{2},...,b_{n}]$ est un $A-$module de type fini.
	
	Ainsi $A[b_{1}x^{n-1},b_{2}x^{n-2},...,b_{n}]$ est un $A-$module de type
	fini.
	
	Soit $p\geq 0.$ Montrons que $x^{n+p}\in
	A[b_{1}x^{n-1},b_{2}x^{n-2},...,b_{n}].$
	
	$\underline{Initialisation}$ $(p=0).$
	
	$x^{n}=\sum\limits_{i=1}^{n}(-b_{i})x^{n-i}\in
	A[b_{1}x^{n-1},b_{2}x^{n-2},...,b_{n}].$
	
	$\underline{Hérédité}$ $(p\geq 0)$.
	
	Supposons que pour tout $k\in \llbracket 0; p \rrbracket,x^{n+k}\in
	A[b_{1}x^{n-1},b_{2}x^{n-2},...,b_{n}].$
	
	$x^{n+p+1}=x^{n}\times x^{p+1}=\sum\limits_{i=1}^{n}(-b_{i})x^{n+p+1-i}.$
	
	Comme $1\leq i\leq r$ alors $n+p+1-r\leq n+p+1-i\leq n+p.$
	
	Donc par hypothèse de récurrence, $x^{n+p+1-i}\in
	A[b_{1}x^{n-1},b_{2}x^{n-2},...,b_{n}].$
	
	Par stabilité, il vient $x^{n+p+1}\in
	A[b_{1}x^{n-1},b_{2}x^{n-2},...,b_{n}].$
	
	Donc $x^{n+p}\in A[b_{1}x^{n-1},b_{2}x^{n-2},...,b_{n}].$
	
	Par suite, $A[b_{1}x^{n-1},b_{2}x^{n-2},...,b_{n}]$ est un $A-$module de
	type fini, c'est à dire 
	
	$A[b_{1}x^{n-1},b_{2}x^{n-2},...,b_{n}]=A(z_{1},z_{2},...,z_{s}).$
	
	Posons $H=A[b_{1}x^{n-1},b_{2}x^{n-2},...,b_{n}]$ et $H^{\prime
	}=A(1,x,...,x^{n-1},z_{1},z_{2},...,z_{s})$.
	
	$H^{\prime }$ est un $A-$module de type fini tel que $A[x]$ est contenu dans $ H^{\prime
	}$ et $H'$ est contenu dans $ C.$
	
	Donc $x$ est entier sur $A.$
	
	Par suite $C$ entier sur $A.$
\end{proof}
\begin{maproposition}
	Soit $A $ contenu dans $ B$ une inclusion d'anneaux tel que $B$ entier sur $A$.\\
		Si $J$ est un idéal de $B$ alors:
		\[ \dfrac{B}{J} \text{  est entier sur } \dfrac{A}{J \cap A};\]
%		\item[ii)] Pour toute partie multiplicative $S$ de $A$
%		\[ S^{-1}B \text{ est entier sur } S^{-1}A.\]

\end{maproposition}

\begin{proof}
	$i)$ $B$ entier sur $A.$ 
	
	Soit $x$ appartenant à $ B,$ alors il existe $n\in \mathbb{N}^{\ast },x^{n}+\sum\limits_{i=1}^{n}a_{i}x^{n-i}=0,$ pour tout $a_{i}\in	A,i\in \llbracket 1; n \rrbracket.$
	
	Ainsi $(x+J)^{n}+\sum\limits_{i=1}^{n}(a_{i}+J\cap
	A)(x+J)^{n-i}=x^{n}+J+\sum\limits_{i=1}^{n}a_{i}x^{n-i}+J=\\
	(x^{n}+\sum\limits_{i=1}^{n}a_{i}x^{n-i})+J=0+J=0_{J\cap A}.$
	
	Donc $x+J$ est entier sur $\dfrac{A}{J\cap A}.$
	
	Ainsi $\dfrac{B}{J}$ est entier sur $\dfrac{A}{J\cap A}.$
	
	
	
%	$ii)$ Supposons $B$ entier sur $A$
%	
%	Soit $\frac{x}{s}\in S^{-1}B,$ $x\in B$ et $s\in S$
%	
%	Comme $x$ est entier sur $A$ alors il existe $n\in \mathbb{N}^{\ast },x^{n}+\sum\limits_{i=1}^{n}a_{i}x^{n-i}=0,$ pour tout $a_{i}\in
%	A,i\in \llbracket 1; n \rrbracket.$
%	
%	D'où, $(\frac{x}{s})^{n}+\sum\limits_{i=1}^{n}(\frac{a_{i}}{s_{i}})(\frac{x}{s})^{n-i}=\frac{0}{1}$
%	
%	Donc $\frac{x}{s}$ est entier sur $S^{-1}A.$
%	
%	Par suite $S^{-1}B$ est entier sur $S^{-1}A.$
\end{proof}
\begin{maproposition}
	Soit $A $ contenu dans $ B$ une inclusion d'anneaux tel que $B$ soit entier sur $A$.
	\[ B \text{ est un corps} \text{ si et seulement si }  A \text{ est un corps}.\]
\end{maproposition}
\begin{proof}
	$i)\implies ii)$ Supposons $B$ corps.
	
	Soit $x$ appartenant à $ A\backslash \{0\}.$
	
	Comme $A$ est contenu dans $ B$ alors $x$ appartenant à $ B\backslash \{0\}$. Donc $x$ est inversible d'inverse $x^{-1}$ appartenant à $ B.$
	
	De plus $B$ entier sur $A,$ alors il existe $n$ appartenant à $ \mathbb{N}^{\ast }, \\ (x^{-1})^{n}+\sum\limits_{i=1}^{n}a_{i}(x^{-1})^{n-i}=0,$ pour tout 
	$a_{i}$ appartenant à $ A,i$ appartenant à $ \llbracket 1; n \rrbracket.$
	
	Ainsi $(x^{-1})^{n}=\sum\limits_{i=1}^{n}(-a_{i})(x^{-1})^{n-i},$ pour tout $
	a_{i}$ appartenant à $ A,i$ appartenant à $ \llbracket 1; n \rrbracket.$
	
	$(x^{-1})\times (x^{-1})^{n-1}=\sum\limits_{i=1}^{n}(-a_{i})(x^{-1})^{n-i},$
	pour tout $a_{i}$ appartenant à $ A,i$ appartenant à $ \llbracket 1; n \rrbracket.$
	
	$(x^{-1})=\sum\limits_{i=1}^{n}(-a_{i})(x^{-1})^{n-i}(x^{-1})^{-n+1},$ pour
	tout $a_{i}$ appartenant à $ A,i$ appartenant à $ \llbracket 1; n \rrbracket.$
	
	$x^{-1}=\sum\limits_{i=1}^{n}(-a_{i})x^{i-1},$ pour tout $a_{i}$ appartenant à $ A,i$ appartenant à $
	\llbracket 1; n \rrbracket.$
	
	Comme $1\leq i\leq n$ alors $0\leq i-1\leq n-1$.
	
	Donc $x^{i-1}$ appartient à $ A$.
	
	Par stabilité $x^{-1}$ appartient à $ A$.
	
	Donc $A$ est un corps.
	
	$ii)\implies i)$ Supposons que $A$ soit un corps.
	
	Soit $x$ appartenant à $ B\backslash \{0\}.$
	
	Comme $B$ entier sur $A,$ alors il existe $n$ appartenant à $ \mathbb{N}^{\ast },x^{n}+\sum\limits_{i=1}^{n}a_{i}x^{n-i}=0,$\\ pour tout $a_{i}$ appartenant à $
	A,i$ appartenant à $ \llbracket 1; n \rrbracket.$

	Par récurrence sur $n$, montrons que $x$ appartient à $ A.$
	
	\underline{Initialisation} $(n=1)$.
	
	L'équation de dépendance intégrale devient, $x+a_{1}=0
	\text{ alors } x=-a_{1}$ appartenant à $ A\backslash \{0\}.$
	
	Donc $x$ est inversible dans $A$ contenu dans $ B.$ Donc $x$ est inversible dans $B.$
	
	\underline{Hérédité} $(n\geq 1)$.
	
	Supposons que la propriété est vraie jusqu'à l'ordre $n.$
	
	Alors $x^{n}+\sum\limits_{i=1}^{n}a_{i}x^{n-i}=0\text{ alors } x$ appartient à $ A.$
	
	Ainsi $x^{n+1}+\sum\limits_{i=1}^{n+1}a_{i}x^{n+1-i}=0\text{ alors } 
	x(x^{n}+\sum\limits_{i=1}^{n}a_{i}x^{n-i})+a_{n+1}=0\text{ alors } \\
	x(x^{n}+\sum\limits_{i=1}^{n}a_{i}x^{n-i})=-a_{n+1}.$
	
	* Si $-a_{n+1}\neq 0$ alors $x$ est inversible dans $A$ contenu dans $ B.$ Donc $x$
	est inversible dans $B.$
	
	* si $-a_{n+1}=0,$ alors $x(x^{n}+\sum\limits_{i=1}^{n}a_{i}x^{n-i})=0 \text{ alors } x^{n}+\sum\limits_{i=1}^{n}a_{i}x^{n-i}=0$\\ (car $x\neq 0$ et $B$ intègre).
	
	Ainsi $x^{n}+\sum\limits_{i=1}^{n}a_{i}x^{n-i}=0\text{ alors }
	x(x^{n-1}+\sum\limits_{i=1}^{n-1}a_{i}x^{n-1-i})=-a_{n+2}.$
	
	* si $-a_{n+2}\neq 0$ alors $x$ est inversible dans $A$ contenu dans $ B.$ Donc $x$ est inversible dans $B.$
	
	* sinon de proche en proche, il vient que $x+a_{1}=0\text{ alors } x=-a_{1}$ appartient à $
	A\backslash \{0\}$
	
	Donc $x$ est inversible dans $A$ contenu dans $ B.$ Donc $x$ est inversible dans $B$
	
	Dans tous les cas, il vient $B$ corps.
\end{proof}
\subsection{Dépendance intégrale sur un idéal}
\begin{madefinition}
	Soient $A$ un anneau commutatif unitaire et $I$ un idéal de $A$.\\ Un élément $x$ de $A$ est dit entier sur $I$ s'il existe un entier $m \in \mathbb{N}$ tel que  
	\[ 	x^m + a_1 x^{m-1} + \cdots + a_m = 0\text{, } \ a_i \in I^i,\, \forall i=1, \cdots ,m. \]	
\end{madefinition}
\begin{maproposition}
	Soient $A$ un anneau et $I$ un idéal de $A$.
	\[ x \ entier \ sur \ I \text{ si et seulement si } le \ monôme \ xX  \ est \ entier \ sur \ R(A, I) \]
\end{maproposition}
\begin{proof}
	$i) \implies ii)$
	
	Supposons $x$ entier sur $I.$ Alors il existe $n\in \mathbb{N}^{\ast },x^{n}+\sum\limits_{i=1}^{n}a_{i}x^{n-i}=0,$ $a_{i}\in I^{i}.$
	
	Ainsi il existe $n\in \mathbb{N}^{\ast },(xX)^{n}+\sum\limits_{i=1}^{n}a_{i}X^{i}(xX)^{n-i}=0,$ $
	a_{i}X^{i}\in I^{i}X^{i}\in R(A,I).$
	
	D’où $xX$ est entier sur $R(A,I).$
	
	$ii)\implies i)$
	
	Supposons que  $xX\in A[X]$ est entier sur $R(A,I).$
	
	Ainsi il existe $n\in \mathbb{N}^{\ast },(xX)^{n}+\sum\limits_{i=1}^{n}a_{i}(xX)^{n-i}=0,$ $a_{i}\in R(A,I).
	$
	
	Alors il existe $n\in \mathbb{N}^{\ast },(xX)^{n}=\sum\limits_{i=1}^{n}(-a_{i})(xX)^{n-i},$ $a_{i}\in
	R(A,I).$
	
	Comme $(xX)^{n}$ est homogène de degré $n,$ alors pour tout $i\in \llbracket 1; n \rrbracket,$
	
	$\deg [a_{i}(xX)^{n-i}]=n\text{ alors } \deg (a_{i})+n-i=n\text{ ainsi } \deg
	(a_{i})=i.$
	
	Donc $a_{i}\in I^{i}X^{i}\text{ alors } a_{i}=\alpha _{i}X^{i}$, avec $\alpha
	_{i}\in I^{i}.$
	
	D’où, il existe $n\in \mathbb{N}^{\ast },(xX)^{n}+\sum\limits_{i=1}^{n}\alpha _{i}X^{i}(xX)^{n-i}=0,$
	
	il existe $n\in \mathbb{N}^{\ast },X^{n}[x^{n}+\sum\limits_{i=1}^{n}\alpha _{i}x^{n-i}]=0$, avec $\alpha _{i}\in I^{i}.$
	
	Par identification des polynômes, il existe $n\in \mathbb{N}^{\ast },x^{n}+\sum\limits_{i=1}^{n}\alpha _{i}x^{n-i}=0,$avec $\alpha_{i}\in I^{i}.$
	
	Donc $x$ est entier sur $I.$
\end{proof}
\begin{moncorollaire}\textbf{(Clôture intégrale d'idéaux)}\cite{Di2}\\
	Soient $A$ un anneau et $I$ un idéal de $A$.
	L'ensemble noté: 
	\[ I'=\{x \in A, x \; entier \; sur \; I \} \]
	est un idéal de $A$ appelé \textbf{clôture intégrale de I}.
\end{moncorollaire}
\begin{proof}
	
	$i)$ Par construction, $I^{\prime }$ est contenu dans $ A.$
	
	$ii)$ $0^{1}+0=0$, donc $0\in I^{\prime }.$
	
	$iii)$ Soient $b\in A,x\in I^{\prime }.$
	
	Alors il existe $n\in \mathbb{N}^{\ast },x^{n}+\sum\limits_{i=1}^{n}a_{i}x^{n-i}=0,$avec $a_{i}\in I^{i}.$
	
	Ainsi, il existe $n\in \mathbb{N}^{\ast },(bx)^{n}+\sum\limits_{i=1}^{n}b^{i}a_{i}(bx)^{n-i}=0,$avec $
	b^{i}a_{i}\in I^{i}.$
	
	Donc $bx$ est entier sur $I.$ D'où $bx\in I^{\prime }.$
	
	$iv)$ Soient $x,y\in I^{\prime }.$
	
	Ainsi $xX,yX$ sont entiers sur $R(A,I).$
	
	Alors $xX+yX=(x+y)X\in A[X]$ est aussi entier sur $R(A,I)$
	
	Donc $x+y\in I^{\prime }.$
	
	Par suite, $I^{\prime }$ est un idéal de $A.$
\end{proof}
\begin{maremarque}\textbf{(Clôture intégrale d'idéaux et radical)}\cite{Di2} \\
	Soient $A$ un anneau et $I$ et $J$ des idéaux de $A$.
	\begin{enumerate}
		\item[(i)]$I \subset J \implies I' \subset J' $;
		\item[(ii)]$I \subset I' \subset \sqrt[]{I} $;
		\item[(iii)]$\sqrt[]{I} = \; \sqrt[]{I'} $.
	\end{enumerate}
\end{maremarque}
\begin{proof}
	$i)$ Soit $x \in I'$ alors $x$ est entier sur $I$ ainsi pour tout $n \in \mathbb{N}^*,\\ x^n + \sum\limits_{k=1}^{n} a_k x^{n-k}=0$, avec $a_k \in I^k$, pour tout $k \in \llbracket 1; n \rrbracket.$\\ 
	Or $I $ est contenu dans $ J$ d'où $I^k $ est contenu dans $ J^k$, pour tout $k \in \llbracket 1; n \rrbracket$.\\
	Par suite, pour tout $n \in \mathbb{N}^*,\\ x^n + \sum\limits_{k=1}^{n} a_k x^{n-k}=0$, avec $a_k \in J^k$.\\
	Donc $x$ est entier sur $J$.\\ Par suite, $I' $ est contenu dans $ J'$.\\
	$ii)$ Soit $x\in I.$
	
	Alors $x^{1}-x=0$ ,où $a_{1}=-x^{1}\in I^{1}.$
	
	D'où $x\in I^{\prime }.$ Donc $I$ est contenu dans $ I^{\prime }.$
	
	$iii)$ Soit $x\in I^{\prime }.$
	
	Alors il existe $n\in \mathbb{N}^{\ast },x^{n}+\sum\limits_{i=1}^{n}a_{i}x^{n-i}=0,$avec $a_{i}\in I^{i}.$
	
	Ainsi pour tout $i\in \llbracket 1; n \rrbracket,$ $a_{i}\in I^{i}\text{ alors }
	a_{i}x^{n-i}\in I^{i}$ est contenu dans $ I.$
	
	Par stabilité, il vient, qu'il existe $n\in \mathbb{N}^{\ast },x^{n}=\sum\limits_{i=1}^{n}(-a_{i})x^{n-i}\in I.$
	
	Donc $x\in \sqrt{I}.$
	
	D'où, $I^{\prime }\subset \sqrt{I}.$
	
	Par suite, $I\subset I^{\prime }\subset \sqrt{I}.$
	
	$iii)$ D'après ce qui précède
	
	$I\subset I^{\prime }\subset \sqrt{I}\text{ alors } \sqrt{I}\subset \sqrt{I^{\prime }}\subset \sqrt{\sqrt{I}}\text{ ainsi } \sqrt{I}\subset \sqrt{I^{\prime }}\subset \sqrt{I}\text{ d'où } \sqrt{I}=\sqrt{I^{\prime }}.$
\end{proof}
\begin{maconsequence}
	Soit $x \in A$.
	Si $x \in I'$ alors il existe $m \in \mathbb{N^*}$ tel que $x^m \in I$.
\end{maconsequence}
\begin{madefinition}
	Un idéal $I$ de $A$ est dit \textbf{intégralement fermé} si $I = I'$.
\end{madefinition}

\section{Réduction d'un idéal}
\subsection{Définitions et propriétés}
\begin{madefinition}
	Soient $A$ un anneau commutatif unitaire, $I$ et $J$ deux id\'eaux de $A$.
	On dit que $I$ est une réduction de $J$ si :
	\begin{enumerate}
		\item[i)] I $\subset$ J;
		\item[ii)] $\exists\, r\in \mathbb{N}^{*} \text{, } J^{r+1} = IJ^{r}$.
	\end{enumerate}
\end{madefinition}
\begin{monexemple}
		%\\
		\item[1)] $I$ est une réduction de $I$ lui-même;
		\item[2)] $A =\mathbb{K}[X,Y]$ avec $\mathbb{K}$ corps.\\
		$I = (X^2, Y^2)$.\\
		$J = (X,Y)^2 = (X^2, Y^2, XY) $.\\
		D'où $I \subset J$.
		\begin{align*}
			IJ&= (X^{2},Y^{2})(X,Y)^{2}.\\
			IJ&= (X^{2},Y^{2})(X^{2},XY,Y^{2}).\\
			IJ&= (X^{4},Y^{4},X^{3}Y,XY^{3},X^{2}Y^{2}).\\
		\end{align*}
		Et \\ 
		\begin{align*}
			J^2 &= (X^2X^2, X^2Y^2, X^3Y, Y^2Y^2, Y^3X).\\
			J^2&= (X^4, Y^4, X^2Y^2, X^3Y, XY^3).
		\end{align*}
		Donc $J^{1+1} = J^2 = IJ $ avec $r=1$.\\
		Par suite $I$ est une réduction de $J$.
\end{monexemple}
\begin{maremarque}
	$\text{ Pour tout } \, r\geq n$, on a $J^{r+1} = IJ^{r}$.\\
	D'une manière générale, $I^{m}J^{n}=J^{n+m}, \, \, \text{ pour tout } \, m\in \mathbb{N}$.	
\end{maremarque}

\begin{maproposition}
	Soient $A$ un anneau, $I$, $J$ et $K$ trois id\'eaux de $A$ \\ tels que 
	$I $ est contenu dans $ J $ et $J$ est contenu dans $ K $.\\ Si $I$ est une réduction de $J$ et $J$ est une réduction de $K$ alors $ I \text{est une réduction de }  K $.
\end{maproposition}
\begin{proof}
	Supposons que $I$ est une réduction de $J$ et $J$ est une réduction de $K$.\\
	$I$ est une réduction de $J$ alors $I $ est contenu dans $ J$ et $\text{il existe } \, n$ appartenant à $ \mathbb{N}^{*}$ tel que $J^{m+1} = IJ^{m}$, de même $J$ est une réduction de $K$ alors $J \subset K$ et $\text{ il existe } \, n$ appartenant à $ \mathbb{N}^{*}$ tel que $K^{n+1} = JK^{n}$.\\
	Posons $r = mn+n+m \in \mathbb{N}^{*}$.
	\begin{align*}
		K^{r+1} = K^{mn+n+m+1}& = (K^{n+1})^{m+1}.\\
		K^{r+1} & = (K^{n+1})^{m}(K^{n+1}).\\
		K^{r+1} & = (JK^{n})^{m}(JK^{n}).\\
		K^{r+1} & = J^{m}K^{nm}JK^{n}.\\
		K^{r+1} & = J^{m+1}K^{nm+n}.\\
		K^{r+1} & = J^{m+1}K^{n}K^{nm}.\\
		K^{r+1} & = IJ^{m}K^{n}K^{nm}.\\
		K^{r+1} & = I(JK^{n})^{m}K^{n}.\\
		K^{r+1} & = I(K^{n+1})^{m}K^{n}.\\
		K^{r+1} & = IK^{mn+n+m}.\\
		K^{r+1} & = IK^r.          
	\end{align*}
	Il existe donc $r $ appartenant à $ \mathbb{N}^{*}$ tel que $K^{r+1} = IK^{r}$ ainsi $I$ est une réduction de $K$.
\end{proof}
\begin{monlemme}
	Soit $I_1, I_2, J_1$ et $J_2$ des idéaux de $A$ alors, \\ si $I_1$ est  une réduction de $J_1$ et $I_2$ est une réduction de $J_2$ alors $I_1+I_2$ est une réduction de $J_1+J_2$.
\end{monlemme}
\begin{proof}
	Supposons que $I_1$ est une réduction de $J_1$ et $I_2$ est une réduction de $J_2$.\\
	$I_1$ est une réduction de $J_1$ alors $I_1 $ est contenu dans $ J_1$ et $\text{ il existe } \, m \in \mathbb{N^*}$ tel que $J_1^{m+1} = I_1 J_1^m$.\\
	$I_2$ est une réduction de $J_2$ alors $I_2 $ est contenu dans $ J_2$ et $\text{ il existe } \, n \in \mathbb{N^*}$ tel que $J_2^{n+1} = I_2 J_2^n$.\\
	Posons $r=m+n \in \mathbb{N^*} $.\\
	\begin{align*}
		I_1(J_1+J_2)^{m+n}& = \displaystyle \sum_{k=0}^{m+n}{I_1 J_1^k J_2^{m+n-k}}.\\
		I_1(J_1+J_2)^{m+n}& = \displaystyle \sum_{k=0}^{m-1}{I_1 J_1^k J_2^{m+n-k}} + \displaystyle \sum_{k=m}^{m+n}{I_1 J_1^k J_2^{m+n-k}}.
	\end{align*}
	Or $I_1$ est une réduction de $J_1$ donc $\text{ pour tout } \, k$ supérieur ou égal à $ m$, on a \\ $J_1^{k+1} = I_1 J_1^{k}.$
	\begin{align*}
		I_1(J_1+J_2)^{m+n}& = \displaystyle \sum_{k=0}^{m+n}{I_1 J_1^k J_2^{m+n-k}}.\\
		I_1(J_1+J_2)^{m+n}& = \displaystyle \sum_{k=0}^{m-1}{I_1 J_1^k J_2^{m+n-k}} + \displaystyle \sum_{k=m}^{m+n}{J_1^{k+1} J_2^{m+n-k}}.
	\end{align*}
	Ainsi donc on a $\displaystyle \sum_{k=m}^{m+n}{J_1^{k+1} J_2^{m+n-k}} $ est contenu dans $  I_1(J_1+J_2)^{m+n}$.\\
	De façon similaire on montre que $\displaystyle \sum_{k=0}^{m}{J_1^{k} J_2^{m+n-k+1}} $ est contenu dans $  I_2(J_1+J_2)^{m+n}.$\\
	D'où $\displaystyle \sum_{k=m}^{m+n}{J_1^{k+1} J_2^{m+n-k}} + \displaystyle \sum_{k=0}^{m}{J_1^{k} J_2^{m+n-k+1}} \subset I_1(J_1+J_2)^{m+n} + I_2(J_1+J_2)^{m+n}$.\\ Alors $\displaystyle \sum_{k=0}^{m+n+1}{J_1^{k} J_2^{m+n+1-k}} = (J_1+J_2)^{m+n+1} \subset (I_1+I_2)(J_1+J_2)^{m+n}$.\\
	Par hypothèse on a $I_1 \subset J_1$ et $I_2 \subset J_2 \text{ alors } I_1+I_2 \subset J_1+J_2$.\\
	Par suite on a $(I_1+I_2)(J_1+J_2)^{m+n} \subset (J_1+J_2)^{m+n+1}$.\\ Par conséquent $(J_1+J_2)^{m+n+1} = (I_1+I_2)(J_1+J_2)^{m+n}$ , on a donc trouver $r$ tel que $(J_1+J_2)^{r+1} = (I_1+I_2)(J_1+J_2)^{r}$ ce qui fait de $I_1+I_2$  une réduction de $J_1+J_2$.
\end{proof}
\begin{maproposition}
	Soient $A$ un anneau, $I$ un idéal de $A$ et $x \in A$.\\
	$x$ est entier sur $I$ si et seulement si $I$ est une réduction de $I + (x) = I +xA $.
\end{maproposition}
\begin{proof}
	$(i)$ Supposons que $x$ est entier sur $I$. Alors il existe $n \in \mathbb{N^*}$ tel que $x^n = \displaystyle \sum_{i=1}^{n}{a_i x^{n-i}}$, avec $a_i \in I^i, i=1, \cdots ,n$.\\
	Montrons que $I$ est une réduction de $I + (x)$.\\
	On a : $(I+(x))^n = (I+(x))(I+(x))^{n-1}= I(I+(x))^{n-1} + (x)(I+(x))^{n-1}$.\\
	En prouvant que $(x)(I+(x))^{n-1} \subset I(I+(x))^{n-1}$ on aura,
	\begin{center}
		$I(I+(x))^{n} = I(I+(x))^{n-1}$.
	\end{center}
	\begin{align*}
		(x)(I+(x))^{n-1} &= (x)\displaystyle \sum_{i=0}^{n-1}{I^i (x)^{n-1-i}}.\\
		(x)(I+(x))^{n-1} &= \displaystyle \sum_{i=0}^{n-1}{I^i (x)^{n-i}}.\\
		(x)(I+(x))^{n-1} &= (x)^n + \displaystyle \sum_{i=1}^{n-1}{I^i (x)^{n-i}}.\\
		(x)(I+(x))^{n-1} &= (x)^n + I\displaystyle \sum_{i=1}^{n-1}{I^{i-1} (x)^{n-i}}.\\
		(x)(I+(x))^{n-1} &= (x)^n + I\displaystyle \sum_{i=0}^{n-2}{I^i (x)^{n-1-i}}.
	\end{align*}
	Donc $(x)(I+(x))^{n-1} = (x)^n + \displaystyle \sum_{i=0}^{n-2}{I^i (x)^{n-1-i}} \subset (x)^n + \displaystyle \sum_{i=0}^{n-1}{I^i (x)^{n-1-i}}$.\\
	d'où $(x)(I+(x))^{n-1} \subset (x)^n + I(I+(x))^{n-1}$.\\ Et comme, \\$x^n = \displaystyle \sum_{i=1}^{n}{a_i x^{n-i}} \in \displaystyle \sum_{i=1}^{n}{I^i x^{n-i}} \text{ alors } x^n \in I\displaystyle \sum_{i=1}^{n}{I^{i-1} x^{n-i}} = I\displaystyle \sum_{i=0}^{n-1}{I^i x^{n-1-i}} =I(I+(x))^{n-1}.$\\ Alors $(x)^n \in I(I+(x))^{n-1} \text{ alors } (x)^n + I(I+(x))^{n-1} = I(I+(x))^{n-1}$.\\
	En somme $(x)(I+(x))^{n-1} \subset I(I+(x))^{n-1} \text{ alors } (I+(x))^{n} = I(I+(x))^{n-1}$.\\
	Par conséquent $I$ est une réduction de $I + (x)$.\\
	$(ii)$ Supposons que $I$ est une réduction de $I + (x)$.\\
	Alors $\text{ il existe } \, n \in \mathbb{N^*}$ tel que $(I + (x))^{n+1} = I(I + (x))^{n}$.\\
	$x^{n+1} \in (I + (x))^{n+1} = I(I + (x))^{n} \text{ alors } x^{n+1} \in I\displaystyle \sum_{i=0}^{n}{I^i (x)^{n-i}} = \displaystyle \sum_{i=0}^{n}{I^{i+1} (x)^{n-i}}$.\\
	D'où $x^{n+1} \in \displaystyle \sum_{i=1}^{n+1}{I^i (x)^{n+1-i}} \text{ alors } x^{n+1} =  \displaystyle \sum_{i=1}^{n+1}{a_i x^{n+1-i}}$, avec $a_i \in I^i$. Ainsi $x$ est donc entier sur $I$.\\
\end{proof}
\begin{maproposition}
	Soit $A$ un anneau noethérien, $I$ et $J$ deux idéaux de $A$ tels que $I $ est contenu dans $ J$. Les assertions suivantes sont équivalentes: 
	\begin{enumerate}
		\item[i] ) $I$ est une réduction de $J$;
		\item[ii] ) $R(A,J)$ est un $R(A,I)$-module de type fini;
		\item[iii] ) $R(A,J)$ est entier sur $R(A,I)$;
		\item[iv] ) $\text{ pour tout } n \in \mathbb{N^*}$, $J^n$ est entier sur $I^n$;
		\item[v] ) $J$ est entier sur $I$.
	\end{enumerate}
\end{maproposition}
\begin{proof}
	$i) \implies ii)$ .\\
	Supposons que $I$ est une réduction de $J$ alors $\text{ il existe } \, n \in \mathbb{N}^{*}$ tel que $J^{n+1} = IJ^{n}$.\\
	$J^{n+1} = IJ^{n} \text{ alors } \text{ pour tout } r \in \mathbb{N} \, ; J^{n+r} = I^rJ^{n}$.\\
	Ainsi $J^{n+r} X^{n+r} = I^r X^rJ^{n} X^n \text{ alors } R(A,J) = R(A,I)(JX, \cdots ,J^rX^r)$.\\
	$R(A,J)$ est donc un $R(A,I)$-module de type fini.\\
	$ii) \implies iii)$.\\
	Supposons que $R(A,J)$ est un $R(A,I)$-module de type fini.\\
	Soit $z \in R(A,J)$,\\
	$z \in R(A,J) \text{ alors } (R(A,J)[z])$ est un sous-module de $R(A,J)$.\\
	$A$ est noethérien alors $R(A,I)$ est noethérien, $R(A,J)$ est un $R(A,I)$-module de type fini alors, $R(A,J)$ est un module noethérien.\\
	$(R(A,J)[z])$ étant un sous-module de $R(A,J)$ qui est noethérien alors $(R(A,J)[z])$ est de type fini. Par suite $z$ est entier sur $R(A,I)$.\\
	$iii) \implies iv)$.\\
	Supposons que $R(A,J)$ est entier sur $R(A,I)$.\\
	Soit $a \in J^n \text{ alors } aX^n \in R(A,J)$, donc $aX^n$ est entier sur $R(A,I)$.\\
	Ainsi il existe $m \in \mathbb{N^*}$  tel que  $\, (aX^n)^m = \displaystyle \sum_{i=1}^{m}{b_i (aX^n)^{m-i}}$,\\ où $b_i \in I^{ni} X^{ni} \text{ alors } b_i = c_i X^{ni} , c_i \in I^{ni}$.\\
	$a^m X^{nm} = \displaystyle \sum_{i=1}^{m}{c_i X^{ni} (aX^n)^{m-i}} = \displaystyle \sum_{i=1}^{m}{c_i X^{ni} a^{m-i} X^{mn-ni}} = \displaystyle \sum_{i=1}^{m}{c_i a^{m-i} X^{mn}}$.\\
	Et donc $a^m X^{nm} = \displaystyle \sum_{i=1}^{m}{c_i a^{m-i} X^{mn}} \text{ alors } a^m = \displaystyle \sum_{i=1}^{m}{c_i a^{m-i}}$ , $c_i \in I^{ni} = (I^n)^i$.\\
	$J^n$ est ainsi entier sur $I^n$.\\
	$iv) \implies v)$\\
	En prenant $n = 1$, alors pour les mêmes raisons que la preuve $iii) \implies iv)$ on a le résultat souhaité à savoir $J$ entier sur $I$.\\
	$v) \implies i)$\\
	Supposons que $J$ est entier sur $I$.\\
	L'anneau $A$ étant noethérien alors l'idéal $J$ est de type fini.\\ Posons $J = (x_1, \cdots ,x_n)$\\
	$x_1 \in J$ qui est entier sur $I$ donc, $x$ est entier sur $I$ ce qui entraine que l'idéal $I$ soit une réduction de $I + (x_1)$. De même $x_2$ est entier sur $I$ donc sur $I + (x_1)$ ainsi, $I + (x_1)$ devient une réduction de $I + (x_1) + (x_2)$.\\ En répétant le même raisonnement à chaque élément de $J$, on obtient $x_r$ entier sur $I$ donc nécessairement sur $I + (x_1) + \cdots +(x_{r-1})$ ce qui implique que $I + (x_1) + \cdots +(x_{r-1})$ est une réduction de $I + (x_1) + \cdots +(x_{r}) = I+J$.\\
	On déduit donc que $I$ est une réduction de $J$. 
\end{proof}

\begin{maproposition}
	Soient A un anneau, I un idéal de A et x un élément de A. Alors:\\
	$x$ est entier sur $I$ si et seulement si il existe $J$ un idéal de A de type fini tel que: xJ est contenu dans IJ et pour tout x' élément de A, x'J est égal au module nul entraîne qu'il existe m supérieur ou égal à zéro, tel que $x'x^m = 0$.
\end{maproposition}
\begin{moncorollaire}
	Soient I, J deux idéaux de A. Alors:
	\[ I'J' \subset (IJ)'.\]
\end{moncorollaire}
%\begin{montheoreme}
%	Soit $A$ un anneau noethérien.
%	Pour tout idéal $I$ de $A$, il existe un idéal $\hat{I}$ (nécessairement unique) ayant les propriétés suivantes : \\
%	$(i)$ $I$ est une réduction de $\hat{I}$;\\
%	$(ii)$ Tout idéal ayant $I$ comme réduction est contenu dans $\hat{I}$.
%\end{montheoreme}
%\begin{proof}
%	Notons $\displaystyle \sum$ l'ensemble de tous les idéaux qui ont $I$ comme réduction. $I$ étant une réduction de lui-même alors, $I \in \displaystyle \sum$ donc $\displaystyle \sum \neq \emptyset$. Puisque la condition de maximalité des idéaux est valable dans notre anneau, on peut trouver un idéal $\hat{I} \in \displaystyle \sum$ et qui est maximal dans $\displaystyle \sum$.\\
%	Soit $I_1 \in \displaystyle \sum$, alors $I$ est une réduction de $I_1$ et $I$ est une réduction de $\hat{I}$. Ainsi comme $I+I=I$ est une réduction de $I_1+\hat{I}$. Cela montre que $I_1+\hat{I} \in \displaystyle \sum$. Mais $\hat{I} \subset I_1+\hat{I}$ d'où $\hat{I} = I_1+\hat{I}$, par choix de $\hat{I}$ donc $I_1 \subset \hat{I}$. Ce qui achève la preuve.
%\end{proof} 
%\begin{moncorollaire}
%	Soient $A$ un anneau noethérien et $I$, $J$ deux idéaux de $A$. Si $I$ est une réduction de $J$, alors $\hat{I} = \hat{J}$.
%\end{moncorollaire}
%\begin{proof}
%	D'une part $I$ est une réduction de $J$ et $J$ est une réduction de $\hat{J}$ alors, $I$ est une réduction de $\hat{J}$ par conséquent $\hat{J} \subset \hat{I}$, par le Théorème précédent. D'autre part $I \subset J \subset \hat{J} \subset \hat{I}$ et comme $I$ est une réduction de $\hat{I}$, alors il existe $n \in \mathbb{N^*}$ tel que $\hat{I}^{n+1} = I \hat{I}^n$.\\
%	$I \subset J \Rightarrow I \hat{I}^n \subset J \hat{I}^n$ or $\hat{I}^{n+1} = I \hat{I}^n$ donc $\hat{I}^{n+1} \subset J \hat{I}^n$. De plus $J \subset \hat{I} \Rightarrow  J \hat{I}^n \subset \hat{I}^{n+1}$, ainsi $\hat{I}^{n+1} = J \hat{I}^n$ par conséquent $J$ devient également une réduction de $\hat{I}$. Il s'ensuit que $\hat{I} \subset \hat{J}$, on obtient donc le résultat souhaité.
%\end{proof}


\subsection{Réduction minimale d'un idéal}
La notion d'idéal basique a été introduite et étudiée par Northcott et Rees \cite{No}.
\begin{madefinition}
	Un idéal $I$ de l'anneau local noethérien $(A,m)$ est basique si la seule réduction de $I$ est $I$ lui-m\^{e}me. 
	Northcott et Rees ont aussi défini la notion de réduction minimale d'un idéal $J$:
	
	Un idéal $I$ est une réduction minimale de $J$ si $I$ est une réduction de $J$ et si I est $minimal$ au sens de l'inclusion parmi l'ensemble des réductions de J. 
\end{madefinition}
\begin{maremarque}
	Bien qu'on puisse parler de réduction minimale pour un idéal quelconque sur un anneau local, cela n'est pas possible pour toutes les filtrations. On montre ainsi dans cet exemple que toutes les filtrations n'admettent pas de réduction minimale.
\end{maremarque}
\begin{monexemple}
	Soient l'anneau $A = k\left[ X\right]$ où $k$ est un corps et $I = (X)$ un idéal de $A$.\\
	Considérons la filtration $I_{2n} = I_{2n-1} = I^n, \text{ pour tout } n \in \mathbb{N}$.\\
	Montrons que $f$ est une filtration noethérienne.\\
	$\bullet$La filtration $f$ est fortement AP car.\\En prenant $k = 2$ on a: $I_{2n} = I^n = I_2^n$.\\
	L’anneau $A$ étant noethérien alors la filtration $f$ est noethérienne.\\
	$\bullet$ Supposons que la filtration $f$ est fortement noethérienne alors, il existe $k \geq 1$ tel que pour tous $m, n \geq k$ on a: $I_{m+n} = I_m I_n$.\\
	Posons $m = 2k+1$ et $n = 2p+1$ où $p \geq k$.\\
	$I_{m+n} = I_{2k+1+2p+1} = I_{2(k+p+1)} = I^{k+p+1}$.\\
	$I_m I_n =  I_{2k+1}  I_{2p+1} =  I_{2(k+1)-1}  I_{2(p+1)-1} = I^{k+1} I^{p+1} = I^{k+p+2}.$
	\begin{align*}
		I_{m+n} = I_m I_n &\text{ alors } I^{k+p+1} = I^{k+p+2}.\\
		I_{m+n} = I_m I_n & \text{ ainsi } (X)^{k+p+2} = (X)^{k+p+1}.\\
		I_{m+n} = I_m I_n &\text{ d'où } QX^{k+p+2} = X^{k+p+1}.\\
		I_{m+n} = I_m I_n &\text{ par suite } XQ = 1_A.
	\end{align*}
	Ainsi on a $X$ inversible ce qui est absurde, par suite la filtration $f$ n'est pas fortement  noethérienne. Comme la filtration $f$ n'est pas fortement noethérienne alors elle n'est pas $I$-bonne.\\ De plus il n'existe pas d'entier $r \geq 1$ tel que $I_r$ soit idempotent, par conséquent notre filtration $f$ n'admet pas de réduction minimale.
\end{monexemple}
\begin{maremarque}
	On prouve dans \cite{Di2} que la réduction minimale des filtrations $I$-bonnes existe toujours dans un anneau local noethérien. Ce qui n'est pas le cas en générale pour une filtration quelconque.
\end{maremarque}
%Ces notions s'étendent sans difficulté aux filtrations noethériennes d'un anneau noethérien $A$, non nécessairement local \cite{Di2}. 
%
%Les questions naturelles qui se posent sont alors les suivantes :
%\begin{enumerate}
%	\item[(a)] Quelles sont les filtrations basiques de $A$ si elles existent ?
%	\item[(b)] Toute filtration de $A$ admet-elle une réduction minimale?
%\end{enumerate}
%
%Nous avons obtenu une caractérisation des filtrations basiques et avons montré que la réponse à la question (b) est négative.
%
%Pour (a) nous avons obtenu le résultat suivant:
%\begin{maproposition}
%	Une filtration noethérienne $f $ de l'anneau noethérien A est basique si et seulement si $f$ est $I-adique$ avec $I$ idempotent ($I^{2}=I$).
%\end{maproposition}
%\begin{proof}
%	Pour le voir, considérons une filtration $f=(I_{n})$ noethérienne et basique. 
%	
%	Il existe alors un entier $k\geq 1$ tel que pour tout entier $n\geq k,$ on
%	ait $I_{n+k}=I_{k}I_{n}.$ Considérons la filtration $h=(H_{n})$ de $A$ définie par:
%	
%	$H_{n}=\left\{ 
%	\begin{array}{c}
%		I_{k-1}\text{ si }1\leq n\leq k-1 \\ 
%		I_{n}\text{ pour }n\geq k
%	\end{array}
%	\right. $ 
%	
%	On a alors pour tout $n\geq k$, $I_{n+k}=I_{k}H_{n},$ et par conséquent $h$ est une réduction de $f.$
%	
%	Comme $f$ est basique, on en déduit que $I_{n}=I_{k}$ pour tout $n.$
%	
%	D'autre part, $I_{2k}=I_{k}^{2}$ donc $f=f_{I}$ avec $I=I_{k}$ et $I^{2}=I.$
%	
%	Réciproquement, si $f$ est $I-adique$ avec $I=I^{2}$ et si $h=(H_{n})$ est une réduction de $f,$ on a pour tout $n\geq 1,$ $I=I^{n}$ et $
%	H_{n}\subset I.$
%	
%	De plus $h$ étant une réduction de $f,$ il existe un entier $k\geq 1$
%	tel que $I=I^{k+n}=I^{k}H_{n}\subset H_{n}\subset I$ pour tout $n\geq k.$
%	
%	D'où $\text{ pour tout } n\geq k$ ,$I=H_{n}.$ Or pour tout entier $n$ tel que $1\leq n\leq k-1,$ on a $H_{k}=I\subset H_{n}\subset I.$
%	
%	D'où $H_{n}=I,$ $f_{I}=h$ et $f_{I}$ est basique.
%\end{proof}
%
%Concernant la question (b), nous avons obtenu le résultat suivant:
%
%\begin{maproposition}
%	Une filtration noethérienne $f=(I_{n})$ de l'anneau noethérien $A$
%	admet une réduction minimale si et seulement s'il existe un entier $r\geq 1$ tel que $I_{r}$ soit un idéal idempotent.
%\end{maproposition}
%\begin{proof}
%	Il est facile de voir que la condition est nécessaire. Supposons en effet que $h=(H_{n})$ soit une réduction minimale de $f$. Alors $f $ est noethérienne et basique et d'après le résultat précédent énonce plus haut, il existe un idéal idempotent $I$ tel que $h=f_{I}$. 
%	
%	Or l'une des caractérisations de réduction d'une filtration sur une
%	autre assure l'existence d'un entier $N\geq 1$ tel que $I_{n}^{2}=I_{n}H_{n}=I_{n}I$ pour tout $n\geq N$. 
%	
%	De plus, $f$ étant noethérienne, elle est fortement $A.P.$ et il
%	existe un entier $k\geq N$ tel que $I_{nk}=I_{k}^{n}$ pour tout entier $n$. 
%	
%	Ainsi $I_{2k}=I_{k}^{2}=I_{k}I$ et \ $I_{2k}^{2}=I_{2k}I=I_{k}I^{2}=I_{k}I=I_{2k}$ donc $I_{2k}$ est idempotent.
%	
%	La réciproque est plus technique et nous renvoyons aux références citées plus haut. La preuve de la réciproque demande l'utilisation de critères assurant que l'anneau de Rees d'une filtration $g$ soit entier (respectivement soit une algèbre de type fini) sur l'anneau de Rees d'une filtration $f.$ Or, suivant les références, les auteurs ont utilisé soit l'anneau de Rees classique soit l'anneau de Rees généralisé associé à une filtration pour définir et caractériser la dépendance intégrale ou la dépendance intégrale forte d'une filtration sur une autre. 
%	
%	Pour montrer que les propriétés utilisées, portant tantôt sur l'un ou l'autre des anneaux de
%	Rees sont équivalentes, nous avons montré les résultats suivants:
%	
%	Étant données deux filtrations $f,g$ de A vérifiant $f\leq g$, on a:
%	\begin{enumerate}
%		\item[(i)]  $\mathcal{R}(A,g)$ est entier sur $\mathcal{R}(A,f)\text{ si et seulement si } $ R(A,g) est entier	sur R(A,f);
%		\item[(ii)] $\mathcal{R}(A,g)$ est entier sur $\mathcal{R}(A,f)-$ algèbre de type fini $\text{ si et seulement si } $ R(A,g) est entier sur $R(A,f)-algèbre$ de type fini.
%	\end{enumerate}
%	Concernant la noethérianité des anneaux de Rees, une preuve astucieuse de l'équivalence entre la noethérianité de l'anneau $\mathcal{R}(A,f)$ et celle de l'anneau $R(A,f)$ pour toute filtration $f$ de l'anneau noethérien $A$ a été donnée dans \cite{Ra2}. Nous l'exposons brièvement ici. Il est clair que si l'anneau de Rees de $f$ est noethérien, il en est de m\^{e}me de son anneau de Rees généralisé. La réciproque est donc triviale. Par conséquent, l'anneau de Rees $R(A,f)$ de $f$ est noethérien.
%	
%\end{proof}


