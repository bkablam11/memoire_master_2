\newpage
\addcontentsline{toc}{chapter}{\large{Résumé}}
\begin{center}
	\LARGE{\textbf{Résumé}}
\end{center}

Dans le cadre de ce mémoire, nous nous penchons sur une étude approfondie de la dépendance intégrale et de la réduction dans le contexte des filtrations I-adiques, en reprenant les travaux déjà publiés par divers chercheurs dans le domaine de la théorie des filtrations.

Notre approche consiste à généraliser ces concepts en examinant la dépendance et la réduction au sein de filtrations bonnes, afin d'explorer les subtilités conceptuelles qui y sont associées. En reprenant des travaux antérieurs, notre but principal est d'enrichir les connaissances dans ce domaine en mettant en évidence des structures et des relations qui n'ont pas encore été pleinement explorées, tout en établissant des connexions conceptuelles entre des idées en apparence distinctes.

En somme, notre démarche intellectuelle vise à éclairer les aspects complexes de la dépendance intégrale et de la réduction dans le contexte des filtrations I-adiques, dans le but de mieux comprendre l'ampleur de ces phénomènes et d'apporter une contribution significative à la compréhension globale des filtrations en mathématiques.
\\
\\
\\
\\
\\
\textbf{Mots-clés: Dépendance intégrale, Réduction, Filtration bonne.} 



\newpage
\addcontentsline{toc}{chapter}{\large{Abstract}}
\begin{center}
	\LARGE{\textbf{Abstract}}
\end{center}

In this dissertation, we focus on an in-depth study of integral dependence and reduction in the context of I-adic filtrations, taking up work already published by various researchers in the field of filtrations theory.

Our approach is to generalize these concepts by examining dependence and reduction within good filtrations, in order to explore the conceptual subtleties associated with them. Building on previous work, our main aim is to enrich knowledge in this field by highlighting structures and relationships that have not yet been fully explored, while establishing conceptual connections between seemingly distinct ideas.

In sum, our intellectual approach aims to shed light on the complex aspects of integral dependence and reduction in the context of I-adic filtrations, with the aim of better understanding the scope of these phenomena and making a significant contribution to the overall understanding of filtrations in mathematics.
\\
\\
\\
\\
\\
\textbf{Keywords: Integral dependence, Reduction, Good filtration.} 

