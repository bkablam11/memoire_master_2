\documentclass[11pt,a4paper]{beamer}
\usepackage[utf8]{inputenc}
\usepackage[francais]{babel}
\usepackage{stmaryrd}
\usepackage[T1]{fontenc}
\usepackage{fancyhdr}
\usepackage{tikz}
\usetheme{Boadilla}
\author{\textit{\textbf{KABLAM Edjabrou Ulrich Blanchard}}}
\title{\textbf{SOUTENANCE DE MÉMOIRE DE MASTER \\ OPTION: ALGÈBRE COMMUTATIVE ET CRYPTOGRAPHIE\\ SPÉCIALITÉ: THÉORIE DES FILTRATIONS}}
\institute{\textcolor{red}{\textbf{Université NANGUI ABROGOUA \\ UFR Sciences Fondamentales Appliquées}}}
\usepackage{graphicx}
\usepackage{wrapfig}
\usepackage{mwe}
\logo{\includegraphics[width=0.7cm]{./img/UNA.png}}
\date{10 Juillet 2024}

\begin{document}
\begin{frame}
\maketitle
\begin{block}{\begin{center}
\emph{THÈME:} \textbf{DÉPENDANCE INTÉGRALE, RÉDUCTION ET FILTRATIONS BONNES }
\end{center}}
\begin{center}
Directeur de Mémoire: Mr. ASSAN Abdoulaye \\
Encadrant scientifique: Mr. BROU Kouadjo Pierre
\end{center}
\end{block}
\end{frame}

\begin{frame}{
PLAN DE PRÉSENTATION}
\begin{enumerate}
\item \textcolor{blue}{INTRODUCTION}\\
\item \textcolor{blue}{DÉPENDANCE INTÉGRALE, RÉDUCTION ET FILTRATION BONNES }\\
\item \textcolor{blue}{CONCLUSION}\\
\end{enumerate}
\end{frame}
\setbeamercovered{transparent}

\begin{frame}{INTRODUCTION}
\framesubtitle{FILTRATIONS}
\begin{block}{}
	\begin{enumerate}
		\item[(i)] Une filtration de l'anneau $A$ est une suite $f=(I_n)_{n \in \mathbb{Z}}$ d'idéaux de $A$, décroissante pour l'inclusion et vérifiant $I_0 = A$ et $I_n I_m \subseteq I_{n+m}$.
		\item[(ii)] Une filtration $f=(I_n)_{n \in \mathbb{Z}}$ est dite $I-bonne$ si pour tout $n \in \mathbb{N}, \quad II_n \subseteq I_{n+1}$ et s'il existe $k$ un entier tel que pour tout $n \geqslant k$, $II_n = I_{n+1}$.	
	\end{enumerate}
\end{block}
\end{frame}

\begin{frame}{INTRODUCTION}
	\framesubtitle{PROPRIÉTÉ DE LA FILTRATION I-ADIQUE}
	\begin{block}{}
\begin{eqnarray*}
	f \text{ I-adique } \Longrightarrow  f \text{ I-bonne } \Longrightarrow  f \text{ fortement A.P } \Longrightarrow f \text{ A.P.} \\
\end{eqnarray*}
	\end{block}
\end{frame}

\begin{frame}{INTRODUCTION}
	\framesubtitle{ÉLÉMENT ENTIER ET RÉDUCTION}
	\begin{block}{}
		\begin{enumerate}
			\item[(i)] Un élément $x$ de $A$ est dit entier sur $f$ s'il existe un entier $m \in \mathbb{N}$ tel que : $x^m + a_1 x^{m-1} + \cdots + a_m = x^m + \sum_{i=1}^{m} a_i x^{m-i} = 0,$\\$ m \in \mathbb{N^*} \ \text{où} \ a_i \in I_i,\, \forall i=1, \cdots ,m.$
			\item[(ii)] $f$ est une $\beta$-réduction de $g$ si : \\
			\begin{enumerate}
				\item[a)] $f \leq g$
				\item[b)]  $\exists \, k \geq 1$ tel que $J_{n+k} = I_n J_k , \forall n \geq k$.
			\end{enumerate}
		\end{enumerate}
	\end{block}
\end{frame}

\begin{frame}{INTRODUCTION}
	\framesubtitle{PROBLÉMATIQUE ET ANNONCE DU PLAN}
	\begin{block}{}
		\begin{enumerate}
			\item[(i)] Comment étendre de manière rigoureuse les résultats obtenus dans le contexte restreint de la filtration I-adique à des filtrations bonnes 
			\item[(i)] Comment ces notions interagissent-elles dans des environnements mathématiques variés ?
		\end{enumerate}
	\end{block}
\end{frame}

\begin{frame}
\begin{enumerate}
\item<0> \textcolor{blue}{INTRODUCTION}\\
\item<1> \textcolor{blue}{DÉPENDANCE INTÉGRALE, RÉDUCTION ET FILTRATIONS BONNES }\\
\item<0> \textcolor{blue}{CONCLUSION}\\
\end{enumerate}
\end{frame}

\begin{frame}{DÉPENDANCE INTÉGRALE, RÉDUCTION ET FILTRATION BONNE}
	\framesubtitle{ÉNONCE}
	\begin{block}{Théorème Principal (1/4)}
	Soient $f=(I_{n})_{_{n\in \mathbb{N}}}\leq $ $g=(J_{n})_{_{n\in \mathbb{N}}}$ des filtrations sur l'anneau $A.$ Nous considérons les assertions suivantes:
		\begin{enumerate}
			\item[(i)] $f$ est une réduction de $g.$
			\item[(ii)] $J_{n}^{2}=I_{n}J_{n}$ pour tout $n$ assez grand.
			\item[(iii)] $I_{n}$ est une réduction de $J_{n}$ pour tout $n$ assez grand.
			\item[(iv)] Il existe un entier $s\geq 1$ tel que pour tout $n\geq s,$ $J_{s+n}=J_{s}J_{n},$
			$I_{s+n}=I_{s}I_{n},$ $J_{s}^{2}=I_{s}J_{s},$ $J_{s+p}I_{s}=I_{s+p}J_{s}$ pour tout $p=1,2,...,s-1$
			\item[(v)] Il existe un entier $k\geq 1$ tel que $g^{(k)}$ est $I_{k}-bonne$
		\end{enumerate}
	\end{block}
\end{frame}

\begin{frame}{DÉPENDANCE INTÉGRALE, RÉDUCTION ET FILTRATION BONNE}
	\framesubtitle{ÉNONCE}
	\begin{block}{Théorème Principal (2/4)}
		\begin{enumerate}
	\item[(vi)] Il existe un entier $r\geq 1$ tel que $f^{(r)}$ est une réduction de $g^{(r)}.$
			\item[(vii)] Pour tout entier $m\geq 1$ tel que $f^{(m)}$ est une réduction de $g^{(m)}.$
			\item[(viii)] $g$ est $enti\grave{e}re$ sur $f.$
			\item[(ix)] $g$ est $\ fortement$ $enti\grave{e}re$ sur $f.$
			\item[(x)] $g$ est $f-fine.$
		\end{enumerate}
	\end{block}
\end{frame}

\begin{frame}{DÉPENDANCE INTÉGRALE, RÉDUCTION ET FILTRATION BONNE}
	\framesubtitle{ÉNONCE}
	\begin{block}{Théorème Principal (3/4)}
		\begin{enumerate}
			\item[(xi)] $g$ est $f-bonne.$
			\item[(xii)] $g$ est $faiblement$ $f-bonne.$
			\item[(xiii)] Il existe un entier $N\geq 1$ tel que $t_{N}g\leq f\leq g$
			\item[(xiv)] Il existe un entier $N\geq 1$ tel que $t_{N}g^{\prime }\leq
			t_{N}f^{\prime \text{ }}$ o\`{u} $f^{\prime }$ est la clôture intégrale de $f.$
			\item[(xv)] $P(f)=P(g)$, o\`{u} $P(f)$ est la clôture prüférienne de $f.$
		\end{enumerate}
	\end{block}
\end{frame}

\begin{frame}{DÉPENDANCE INTÉGRALE, RÉDUCTION ET FILTRATION BONNE}
	\framesubtitle{RÉSULTAT}
	\begin{block}{Théorème Principal (4/4)}
		On a les résultats suivants:\\
		(1)
		\begin{enumerate}
			\item[(a)] $f$ est une réduction de $g$ si et seulement si pour tout entier $m\geq 1$ tel que $f^{(m)}$ est une réduction de $g^{(m)}.$ 
			\item[(b)] Il existe un entier $k\geq 1$ tel que $g^{(k)}$ est $I_{k}-bonne$ si et seulement s'il existe un entier $r\geq 1$ tel que $f^{(r)}$ est une réduction de $g^{(r)}.$
			\item[(c)] $g$ est $enti\grave{e}re$ sur $f$ si et seulement si $P(f)=P(g)$, o\`{u} $P(f)$ est la clôture prüférienne de $f.$
		\end{enumerate}
	\end{block}
\end{frame}

\begin{frame}{DÉPENDANCE INTÉGRALE, RÉDUCTION ET FILTRATION BONNE}
	\framesubtitle{RÉSULTAT}
	\begin{block}{Théorème Principal (5/4)}
		On a les résultats suivants:\\
		(1)
		\begin{enumerate}
			\item[(d)]Si $J_{n}^{2}=I_{n}J_{n}$ pour tout $n$ assez grand alors $I_{n}$ est une réduction de $J_{n}$ pour tout $n$ assez grand. 
			\item[(e)]S'il existe un entier $s\geq 1$ tel que pour tout $n\geq s,$ $J_{s+n}=J_{s}J_{n},$ $I_{s+n}=I_{s}I_{n},$ $J_{s}^{2}=I_{s}J_{s},$ $J_{s+p}I_{s}=I_{s+p}J_{s}$ pour tout $p=1,2,...,s-1$ alors $f$ est une réduction de $g.$
			\item[(f)]Si $f$ est une réduction de $g$ alors il existe un entier $k\geq 1$ tel que $g^{(k)}$ est $I_{k}-bonne$
		\end{enumerate}
	\end{block}
\end{frame}

\begin{frame}{DÉPENDANCE INTÉGRALE, RÉDUCTION ET FILTRATION BONNE}
	\framesubtitle{RÉSULTAT}
	\begin{block}{Théorème Principal (6/4)}
		On a les résultats suivants:\\
		(1)
		\begin{enumerate}
			\item[(g)] Si $g$ est fortement entière sur $f$ alors:
			\begin{enumerate}
				\item[-] Pour tout entier $m\geq 1$ tel que $f^{(m)}$ est une réduction de $g^{(m)}.$
				\item[-] $g$ est $faiblement$ $f-bonne.$
				\item[-] Il existe un entier $N\geq 1$ tel que $t_{N}g\leq f\leq g$
			\end{enumerate}
		\end{enumerate}
	\end{block}
\end{frame}

\begin{frame}
\bigskip
{\textcolor{blue}{\Large \textit{\textbf{\begin{center}
MERCI POUR VOTRE AIMABLE ATTENTION
\end{center}}}}}
\end{frame}
\end{document}