\documentclass[11pt,a4paper]{beamer}
\usepackage[utf8]{inputenc}
\usepackage[francais]{babel}
\usepackage{stmaryrd}
\usepackage[T1]{fontenc}
\usepackage{fancyhdr}
\usepackage{tikz}
\usetheme{Boadilla}
\author{\textit{\textbf{KABLAM Edjabrou Ulrich Blanchard}}}
\title{\textbf{SOUTENANCE DE MÉMOIRE DE MASTER \\ OPTION: ALGÈBRE COMMUTATIVE ET CRYPTOGRAPHIE\\ SPÉCIALITÉ: THÉORIE DES FILTRATIONS}}
\institute{\textcolor{red}{\textbf{Université NANGUI ABROGOUA \\ UFR Sciences Fondamentales Appliquées}}}
\usepackage{graphicx}
\usepackage{wrapfig}
\usepackage{mwe}
\logo{\includegraphics[width=0.7cm]{./img/UNA.png}}
\date{10 Juillet 2024}

\begin{document}
\begin{frame}
\maketitle
\begin{block}{\begin{center}
\emph{THÈME:} \textbf{DÉPENDANCE INTÉGRALE, RÉDUCTION ET FILTRATIONS BONNES }
\end{center}}
\begin{center}
Directeur de Mémoire: Mr. ASSAN Abdoulaye \\
Encadrant scientifique: Mr. BROU Kouadjo Pierre
\end{center}
\end{block}
\end{frame}

\begin{frame}{
PLAN DE PRÉSENTATION}
\begin{enumerate}
\item \textcolor{blue}{INTRODUCTION}\\
\item \textcolor{blue}{DÉPENDANCE INTÉGRALE, RÉDUCTION ET FILTRATION BONNES }\\
\item \textcolor{blue}{CONCLUSION}\\
\end{enumerate}
\end{frame}
\setbeamercovered{transparent}

\begin{frame}{INTRODUCTION}
\framesubtitle{FILTRATIONS}
\begin{block}{}
	\begin{enumerate}
		\item[(i)] Une filtration de l'anneau $A$ est une suite $f=(I_n)_{n \in \mathbb{Z}}$ d'idéaux de $A$, décroissante pour l'inclusion et vérifiant $I_0 = A$ et $I_n I_m \subseteq I_{n+m}$.
		\item[(ii)] Une filtration $f=(I_n)_{n \in \mathbb{Z}}$ est dite $I-bonne$ si pour tout $n \in \mathbb{N}, \quad II_n \subseteq I_{n+1}$ et s'il existe $k$ un entier tel que pour tout $n \geqslant k$, $II_n = I_{n+1}$.	
	\end{enumerate}
\end{block}
\end{frame}

\begin{frame}{INTRODUCTION}
	\framesubtitle{PROPRIÉTÉ DE LA FILTRATION I-ADIQUE}
	\begin{block}{}
\begin{eqnarray*}
	f \text{ I-adique } \Longrightarrow  f \text{ I-bonne } \Longrightarrow  f \text{ fortement A.P. } \Longrightarrow f \text{ A.P.} \\
\end{eqnarray*}
	\end{block}
\end{frame}

\begin{frame}{INTRODUCTION}
	\framesubtitle{ÉLÉMENT ENTIER ET RÉDUCTION}
	\begin{block}{}
		\begin{enumerate}
			\item[(i)] Un élément $x$ de $A$ est dit entier sur $f$ s'il existe un entier $m \in \mathbb{N}$ tel que : $x^m + a_1 x^{m-1} + \cdots + a_m = x^m + \sum_{i=1}^{m} a_i x^{m-i} = 0,$\\$ m \in \mathbb{N^*} \ \text{où} \ a_i \in I_i,\, \forall i=1, \cdots ,m.$
			\item[(ii)] $f$ est une $\beta$-réduction de $g$ si : \\
			\begin{enumerate}
				\item[a)] $f \leq g$
				\item[b)]  $\exists \, k \geq 1$ tel que $J_{n+k} = I_n J_k , \forall n \geq k$.
			\end{enumerate}
		\end{enumerate}
	\end{block}
\end{frame}

\begin{frame}{INTRODUCTION}
	\framesubtitle{PROBLÉMATIQUE ET ANNONCE DU PLAN}
	\begin{block}{}
		\begin{enumerate}
			\item[(i)] Comment étendre de manière rigoureuse les résultats obtenus dans le contexte restreint de la filtration I-adique à des filtrations bonnes 
			\item[(i)] Comment ces notions interagissent-elles dans des environnements mathématiques variés ?
		\end{enumerate}
	\end{block}
\end{frame}

\begin{frame}
\begin{enumerate}
\item<0> \textcolor{blue}{INTRODUCTION}\\
\item<1> \textcolor{blue}{DÉPENDANCE INTÉGRALE, RÉDUCTION ET FILTRATIONS BONNES }\\
\item<0> \textcolor{blue}{CONCLUSION}\\
\end{enumerate}
\end{frame}

\begin{frame}{DÉPENDANCE INTÉGRALE, RÉDUCTION ET FILTRATION BONNE}
	\framesubtitle{ÉNONCE}
	\begin{block}{Théorème Principal (1/11)}
	Soient $f=(I_{n})_{_{n\in \mathbb{N}}}\leq $ $g=(J_{n})_{_{n\in \mathbb{N}}}$ des filtrations sur l'anneau $A.$ Nous considérons les assertions suivantes:
		\begin{enumerate}
			\item[(i)] $f$ est une réduction de $g.$
			\item[(ii)] $J_{n}^{2}=I_{n}J_{n}$ pour tout $n$ assez grand.
			\item[(iii)] $I_{n}$ est une réduction de $J_{n}$ pour tout $n$ assez grand.
			\item[(iv)] Il existe un entier $s\geq 1$ tel que pour tout $n\geq s,$ $J_{s+n}=J_{s}J_{n},$
			$I_{s+n}=I_{s}I_{n},$ $J_{s}^{2}=I_{s}J_{s},$ $J_{s+p}I_{s}=I_{s+p}J_{s}$ pour tout $p=1,2,...,s-1$
			\item[(v)] Il existe un entier $k\geq 1$ tel que $g^{(k)}$ est $I_{k}-bonne$
		\end{enumerate}
	\end{block}
\end{frame}

\begin{frame}{DÉPENDANCE INTÉGRALE, RÉDUCTION ET FILTRATION BONNE}
	\framesubtitle{ÉNONCE}
	\begin{block}{Théorème Principal (2/11)}
		\begin{enumerate}
	\item[(vi)] Il existe un entier $r\geq 1$ tel que $f^{(r)}$ est une réduction de $g^{(r)}.$
			\item[(vii)] Pour tout entier $m\geq 1$ tel que $f^{(m)}$ est une réduction de $g^{(m)}.$
			\item[(viii)] $g$ est $enti\grave{e}re$ sur $f.$
			\item[(ix)] $g$ est $\ fortement$ $enti\grave{e}re$ sur $f.$
			\item[(x)] $g$ est $f-fine.$
		\end{enumerate}
	\end{block}
\end{frame}

\begin{frame}{DÉPENDANCE INTÉGRALE, RÉDUCTION ET FILTRATION BONNE}
	\framesubtitle{ÉNONCE}
	\begin{block}{Théorème Principal (3/11)}
		\begin{enumerate}
			\item[(xi)] $g$ est $f-bonne.$
			\item[(xii)] $g$ est $faiblement$ $f-bonne.$
			\item[(xiii)] Il existe un entier $N\geq 1$ tel que $t_{N}g\leq f\leq g$
			\item[(xiv)] Il existe un entier $N\geq 1$ tel que $t_{N}g^{\prime }\leq
			t_{N}f^{\prime \text{ }}$ où $f^{\prime }$ est la clôture intégrale de $f.$
			\item[(xv)] $P(f)=P(g)$, où $P(f)$ est la clôture prüférienne de $f.$
		\end{enumerate}
	\end{block}
\end{frame}

\begin{frame}{DÉPENDANCE INTÉGRALE, RÉDUCTION ET FILTRATION BONNE}
	\framesubtitle{RÉSULTATS}
	\begin{block}{Théorème Principal (4/11)}
		On a les résultats suivants:\\
		(1)
		\begin{enumerate}
			\item[(a)] $f$ est une réduction de $g$ si et seulement si pour tout entier $m\geq 1$ tel que $f^{(m)}$ est une réduction de $g^{(m)}.$ 
			\item[(b)] Il existe un entier $k\geq 1$ tel que $g^{(k)}$ est $I_{k}-bonne$ si et seulement s'il existe un entier $r\geq 1$ tel que $f^{(r)}$ est une réduction de $g^{(r)}.$
			\item[(c)] $g$ est $enti\grave{e}re$ sur $f$ si et seulement si $P(f)=P(g)$
		\end{enumerate}
	\end{block}
\end{frame}

\begin{frame}{DÉPENDANCE INTÉGRALE, RÉDUCTION ET FILTRATION BONNE}
	\framesubtitle{RÉSULTATS}
	\begin{block}{Théorème Principal (5/11)}
		On a les résultats suivants:\\
		(1)
		\begin{enumerate}
			\item[(d)]Si $J_{n}^{2}=I_{n}J_{n}$ pour tout $n$ assez grand alors $I_{n}$ est une réduction de $J_{n}$ pour tout $n$ assez grand. 
			\item[(e)]S'il existe un entier $s\geq 1$ tel que pour tout $n\geq s,$ $J_{s+n}=J_{s}J_{n},$ $I_{s+n}=I_{s}I_{n},$ $J_{s}^{2}=I_{s}J_{s},$ $J_{s+p}I_{s}=I_{s+p}J_{s}$ pour tout $p=1,2,...,s-1$ alors $f$ est une réduction de $g.$
			\item[(f)]Si $f$ est une réduction de $g$ alors il existe un entier $k\geq 1$ tel que $g^{(k)}$ est $I_{k}-bonne$
		\end{enumerate}
	\end{block}
\end{frame}

\begin{frame}{DÉPENDANCE INTÉGRALE, RÉDUCTION ET FILTRATION BONNE}
	\framesubtitle{RÉSULTATS}
	\begin{block}{Théorème Principal (6/11)}
		On a les résultats suivants:\\
		(1)
		\begin{enumerate}
			\item[(g)] Si $g$ est fortement entière sur $f$ alors:
			\begin{enumerate}
				\item[-] Pour tout entier $m\geq 1$ tel que $f^{(m)}$ est une réduction de $g^{(m)}.$
				\item[-] $g$ est $faiblement$ $f-bonne.$
				\item[-] Il existe un entier $N\geq 1$ tel que $t_{N}g\leq f\leq g$
			\end{enumerate}
		\end{enumerate}
	\end{block}
\end{frame}

\begin{frame}{DÉPENDANCE INTÉGRALE, RÉDUCTION ET FILTRATION BONNE}
	\framesubtitle{RÉSULTATS}
	\begin{block}{Théorème Principal (7/11)}
		On a les résultats suivants:\\
		(1)
		\begin{enumerate}
			\item[(h)] $f$ est une réduction de $g$ $\implies$ $g$ est $f-fine$ $\implies$ $g$ est $f-bonne$
			\item[(i)] $g$ est $f-bonne$ $\implies$  $g$ est $faiblement$ $f-bonne$ $\implies$ Il existe un entier $N\geq 1$ tel que $t_{N}g\leq f\leq g$
		\end{enumerate}
	\end{block}
\end{frame}

\begin{frame}{DÉPENDANCE INTÉGRALE, RÉDUCTION ET FILTRATION BONNE}
	\framesubtitle{RÉSULTATS}
	\begin{block}{Théorème Principal (8/11)}
		On a les résultats suivants:\\
		(2) Si de plus on suppose $A$ noethérien, alors:
		\begin{enumerate}
			\item[(j)] Il existe un entier $s\geq 1$ tel que pour tout $n\geq s,$ $J_{s+n}=J_{s}J_{n},$ si et seulement s'il existe un entier $N\geq 1$ tel que $t_{N}g^{\prime }\leq
			t_{N}f^{\prime \text{ }}$ où $f^{\prime }$ est la clôture intégrale de $f.$
			
			\item[(k)] $f$ est une réduction de $g$ si $J_{n}^{2}=I_{n}J_{n}$ pour tout $n$ assez grand.
			
			\item[(l)] $f$ est une réduction de $g$ $\implies$ $g$ est $\ fortement$ $enti\grave{e}re$ sur $f$ $\Longleftrightarrow$ $g$ est $faiblement$ $f-bonne.$
		\end{enumerate}
	\end{block}
\end{frame}

\begin{frame}{DÉPENDANCE INTÉGRALE, RÉDUCTION ET FILTRATION BONNE}
	\framesubtitle{RÉSULTATS}
	\begin{block}{Théorème Principal (9/11)}
		On a les résultats suivants:\\
		(3) Par ailleurs, si $f$ est noethérienne, alors $A$ est noethérien et les assertions suivantes sont équivalentes:
		\begin{enumerate}
			\item[(m)] $I_{n}$ est une réduction de $J_{n}$ pour tout $n$ assez grand $\Longleftrightarrow$ $g$ est $f-fine$ $\Longleftrightarrow$ $g$ est $f-bonne$ $\Longleftrightarrow$ $g$ est $faiblement$ $f-bonne$ $\Longleftrightarrow$ Il existe un entier $N\geq 1$ tel que $t_{N}g\leq f\leq g$
		\end{enumerate}
	\end{block}
\end{frame}

\begin{frame}{DÉPENDANCE INTÉGRALE, RÉDUCTION ET FILTRATION BONNE}
	\framesubtitle{RÉSULTATS}
	\begin{block}{Théorème Principal (10/11)}
		On a les résultats suivants:\\
		(4) Si $f$ et $g$ sont noethériennes alors nous avons:
		\begin{enumerate}
			\item[(n)] $I_{n}$ est une réduction de $J_{n}$ pour tout $n$ assez grand $\implies$ $g$ est entière sur $f$ $\Longleftrightarrow$ $g$ est fortement entière sur $f$
			\item[(o)] Il existe un entier $r\geq 1$ tel que $f^{(r)}$ est une réduction de $g^{(r)}$ $\implies$ $g$ est fortement entière sur $f$
		\end{enumerate}
	\end{block}
\end{frame}

\begin{frame}{DÉPENDANCE INTÉGRALE, RÉDUCTION ET FILTRATION BONNE}
	\framesubtitle{RÉSULTATS}
	\begin{block}{Théorème Principal (11/11)}
		On a les résultats suivants:\\
		(5) Si $f$ est fortement noethérienne et $g$ est noethérienne alors les quinze (15) assertions sont équivalentes et dans ce cas $g$ est fortement noethérienne.
	\end{block}
\end{frame}

\begin{frame}{DÉPENDANCE INTÉGRALE, RÉDUCTION ET FILTRATION BONNE}
	\framesubtitle{RÉSULTATS}
	\begin{block}{Démonstration}
			1)
		$(i)\Longleftrightarrow (vii).$
		Supposons $f$ est une réduction de $g$ et choisissons un entier $k$, $k \geqslant 1$ tel que $J_{k+n}  = J_{k}I_n$ pour tout $n \geqslant k$. Pour un tel entier $k$ et pour tout $m \geqslant 1$, $J_{mk}=J_{k}^{p}J_{(m-p)k}
		$ pour tout $p=1,2,\cdots,m$ alors pour tout
		entiers $m\geq 1$ et $n\geq k,$ $J_{m(k+n)}=J_{mk}I_{mn}$, ce qui entraîne que pour tout entier $m\geq 1$ tel que $f^{(m)}$ est une réduction de $g^{(m)}.$
		La réciproque est évidente.
	\end{block}
\end{frame}

\begin{frame}{DÉPENDANCE INTÉGRALE, RÉDUCTION ET FILTRATION BONNE}
	\framesubtitle{RÉSULTATS}
	\begin{block}{Démonstration}
		$(v)\Longrightarrow (vi).$
		Supposons qu'il existe un entier $k\geq 1$ tel que $g^{(k)}$ est $I_{k}-bonne$.
		Posons $f^{(k)}=(H_{n});$ $g^{(k)}=(K_{n});$ $H_{n}=I_{nk};$ $K_{n}=J_{nk};$ 
		$H_{1}=I_{k};$
		
		Par hypothèse, $H_{1}K_{n}\subseteq K_{n+1}$ pour tout entier $n$ et il
		existe un entier $n_{0}\geq 1$ tel que $H_{1}K_{n}=K_{n+1}$ pour tout $n\geq
		n_{0}.$
		
		Pour tout entier $m\geq 0,$ $K_{n_{0}+m}=H_{1}^{m}K_{n_{0}}\subseteq
		H_{m}K_{n_{0}}\subseteq K_{n_{0}+m}.$
		
		Donc $K_{n_{0}+m}=K_{n_{0}}H_{m}$ pour tout entier $m.$ Donc $f^{(k)}$
		est une réduction de $g^{(k)}.$
		
	\end{block}
\end{frame}

\begin{frame}{DÉPENDANCE INTÉGRALE, RÉDUCTION ET FILTRATION BONNE}
	\framesubtitle{RÉSULTATS}
	\begin{block}{Démonstration}
		$(vi)\Longrightarrow (v).$\\ Supposons qu'il existe un entier $r\geq 1$ tel que $f^{(r)}$ est une réduction de $g^{(r)}.$
		Il suffit de montrer que si $f$ est une réduction de $g$ alors il existe 
		$k\geq 1$ tel que $g^{(k)}$ est $I_{k}-bonne.$
		
		Posons un entier $k$, $k \geqslant 1$ tel que $J_{k+n}  = J_{k}I_n$ pour tout $n \geqslant k$. Pour un tel entier $k$ et pour tout $m \geqslant 1$, $J_{mk}=J_{k}^{p}J_{(m-p)k}
		$ pour tout $p=1,2,\cdots,m$, alors pour tout entiers $m\geq 1$ et  $J_{k(m+1)}=J_{mk}I_{k}$, donc $g^{(k)}$ est $I_{k}-bonne.$
	\end{block}
\end{frame}

\begin{frame}{DÉPENDANCE INTÉGRALE, RÉDUCTION ET FILTRATION BONNE}
	\framesubtitle{RÉSULTATS}
	\begin{block}{Démonstration}
		$(viii)\Longleftrightarrow (xv).$
		
		Si $g$ est entière sur $f$ alors $f\leq g\leq P(f),$ ainsi $P(f)\leq P(g)\leq P(P(f))=P(f),$ donc $P(g)=P(f).$
		
		Réciproquement si $P(f)=P(g)$ alors $g\leq P(g)=P(f)$ et donc $g$ est entière sur $f.$
		
		$(ii)\Longrightarrow (iii).$
		Évident.
	\end{block}
\end{frame}

\begin{frame}{DÉPENDANCE INTÉGRALE, RÉDUCTION ET FILTRATION BONNE}
	\framesubtitle{RÉSULTATS}
	\begin{block}{Démonstration}
		$(iv)\Longrightarrow (i).$\\
		Supposons qu'il existe un entier $s\geq 1$ tel que pour tout $n\geq s,$ $J_{s+n}=J_{s}J_{n},$ $I_{s+n}=I_{s}I_{n},$ $J_{s}^{2}=I_{s}J_{s},$ $J_{s+p}I_{s}=I_{s+p}J_{s}$ pour tout $p=1,2,...,s-1$.\\
		Posons $n\geq 2s$ et $n=qs+p$ avec $0\leq p<s.$
		
		Alors $J_{s+n}=J_{(q-2)s+2s+(s+p)}=J_{s}^{q-2}J_{2s+(s+p)}=J_{s}^{q-2}J_{s}^{2}J_{s+p}=J_{s}^{q-1}I_{s}J_{s+p}=J_{s}^{q-1}J_{s}I_{s+p}=J_{s}^{q}I_{s+p}=J_{s}I_{s}^{q-1}I_{s+p}\subseteq J_{s}I_{n}\subseteq J_{s+n}.
		$
		
		Par suite $J_{s+n}=J_{s}I_{n}$ pour tout $n\geq 2s.$ Donc $
		J_{2s+n}=J_{2s}I_{n}$ pour tout $n\geq 2s.$ D'où $f$ est une réduction de $g.$
		
		
		
	\end{block}
\end{frame}

\begin{frame}{DÉPENDANCE INTÉGRALE, RÉDUCTION ET FILTRATION BONNE}
	\framesubtitle{RÉSULTATS}
	\begin{block}{Démonstration}
		$(i)\Longrightarrow (v)$
		
		Évident car $(vi)\Longrightarrow (v).$
		
		$(ix)\Longrightarrow (viii)$
		
		Évident
		
		$(ix)\Longrightarrow (xii)\Longrightarrow (xiii)$ en utilisant la proposition 3.6 (5)
		
	\end{block}
\end{frame}

\begin{frame}{DÉPENDANCE INTÉGRALE, RÉDUCTION ET FILTRATION BONNE}
	\framesubtitle{RÉSULTATS}
	\begin{block}{Démonstration}
		$(i)\Longrightarrow (x).$ \\
		Supposons que $f$ est une réduction de $g.$\\
		Pour tout entier $n\geq N=2k-1,$ posons $n=qk+r,$ avec $0\leq r<k$ où $k$
		est $k \geqslant 1$ tel que $J_{k+n}  = J_{k}I_n$ pour tout $n \geqslant k$. Pour un tel entier $k$ et pour tout $m \geqslant 1$, $J_{mk}=J_{k}^{p}J_{(m-p)k}
		$ pour tout $p=1,2,\cdots,m$
		
		Alors $J_{n}=J_{k(q-1)}I_{k+r}.$
		
		Ainsi $1\leq k+r<2k-1,$ $J_{n}\subseteq
		\sum\limits_{p=1}^{N}I_{p}J_{n-p}\subseteq J_{n}$, d'où $J_{n}=\sum\limits_{p=1}^{N}I_{p}J_{n-p}$ pour tout $n\geq N=2k-1$.
		
		Ce qui prouve que $g$ est $f-fine.$
		
		$(x)\Longrightarrow (xi)$ car toute filtration f-bonne est f-fine
		
		$(xi)\Longrightarrow (xii)$ en utilisant la proposition 3.6 (1)
	\end{block}
\end{frame}

\begin{frame}
	\begin{enumerate}
		\item<0> \textcolor{blue}{INTRODUCTION}\\
		\item<0> \textcolor{blue}{DÉPENDANCE INTÉGRALE, RÉDUCTION ET FILTRATIONS BONNES }\\
		\item<1> \textcolor{blue}{CONCLUSION}\\
	\end{enumerate}
\end{frame}

\begin{frame}{CONCLUSION}
	\framesubtitle{PERSPECTIVES}
	\begin{block}{}	
Comme perspectives, nous projetons d'effectuer:\\
		\begin{enumerate}
			\item Une étude du nombre de réduction sur les filtrations bonnes i.e le nombre de Samuel.
			\item Une étude de la largeur analytique qui représente le lien entre la dépendance intégrale et la réduction des filtrations bonnes.
		\end{enumerate}
	\end{block}
\end{frame}

\begin{frame}
\bigskip
{\textcolor{blue}{\Large \textit{\textbf{\begin{center}
MERCI POUR VOTRE AIMABLE ATTENTION
\end{center}}}}}
\end{frame}
\end{document}