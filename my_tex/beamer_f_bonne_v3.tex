\documentclass[11pt,a4paper]{beamer}
\usepackage[utf8]{inputenc}
\usepackage[francais]{babel}
\usepackage{stmaryrd}
\usepackage[T1]{fontenc}
\usepackage{fancyhdr}
\usepackage{tikz}
\usetheme{Boadilla}
\author{\textit{\textbf{KABLAM Edjabrou Ulrich Blanchard}}}
\title{\textbf{SOUTENANCE DE MÉMOIRE DE MASTER \\ OPTION: ALGÈBRE COMMUTATIVE ET CRYPTOGRAPHIE\\ SPÉCIALITÉ: THÉORIE DES FILTRATIONS}}
\institute{\textcolor{red}{\textbf{Université NANGUI ABROGOUA \\ UFR Sciences Fondamentales Appliquées}}}
\usepackage{graphicx}
\usepackage{wrapfig}
\usepackage{mwe}
\logo{\includegraphics[width=0.7cm]{./img/UNA.png}}
\date{10 Juillet 2024}

\begin{document}
	\begin{frame}
		\maketitle
		\begin{block}{\begin{center}
					\emph{THÈME:} \textbf{DÉPENDANCE INTÉGRALE, RÉDUCTION ET FILTRATIONS BONNES }
			\end{center}}
			\begin{center}
				Directeur de Mémoire: Mr. ASSAN Abdoulaye, M.C. \\
				Encadrant scientifique: Mr. BROU Kouadjo Pierre, M.A.
			\end{center}
		\end{block}
	\end{frame}
	
	\begin{frame}{
			PLAN DE PRÉSENTATION}
		\begin{enumerate}
			\item \textcolor{blue}{PRÉLIMINAIRES}\\
			\item \textcolor{blue}{DÉPENDANCE INTÉGRALE, RÉDUCTION ET FILTRATIONS BONNES }\\
			\item \textcolor{blue}{CONCLUSION}\\
		\end{enumerate}
	\end{frame}
	\setbeamercovered{transparent}
	
	\begin{frame}{PRÉLIMINAIRES}
		\framesubtitle{FILTRATIONS}
		\begin{block}{}
				$f=(I_n)_{n \in \mathbb{Z}} \in \mathbb{F}(A)$ si: 
				\begin{enumerate}[(i)]
					\item $I_0 = A$;
					\item $I_{n+1} \subset I_n, \forall n \in \mathbb{Z}$;
					\item $I_{p}I_{q} \subset I_{p+q}, \forall p,q \in \mathbb{Z}$.
				\end{enumerate}
		\end{block}
	\end{frame}
	
		\begin{frame}{PRÉLIMINAIRES}
		\framesubtitle{FILTRATIONS}
		\begin{alertblock}{Remarque}
		On peut remarquer que pour tout $n\leq 0, I_n = A$.\\ En effet, en utilisant la décroissance des idéaux (ii) et que $I_0 = A$ (i), il vient $I_n = A, n \leq 0$ car pour tout $ n \in \mathbb{Z}$, les $I_n$ sont des idéaux de $A$.\\
		Ainsi au lieu d'étudier la famille $f = (I_n)_{n \in \mathbb{Z}}$ nous pouvons nous ramener à étudier la famille $f = (I_n)_{n \in \mathbb{N}}$.
		\end{alertblock}
	\end{frame}
		\begin{frame}{PRÉLIMINAIRES}
		\framesubtitle{CLASSES DES FILTRATIONS}
		\begin{block}{}
		\begin{center}
			\begin{tabular}{|l|c|}
				\hline
				f $I-adique$ &$I_n=I^n,\forall n \in \mathbb{N}^*$\\
				\hline
				f $I-bonne$ &$\exists \, n_0 \in \mathbb{N}$ tel que $II_n = I_{n+1}, \forall n \geqslant n_0.$\\
				\hline
				f $A.P.$ &$\exists \, (k_n)_{n\in \mathbb{N}} $ tel que $\forall$ n,m $\in \mathbb{N}$, $I_{mk_n} \subset I_n^{m}$ et $\underset{n\longrightarrow +\infty }{\lim }\dfrac{k_{n}}{n}=1$\\
				\hline
				f f.$A.P.$ &$\exists k\geqslant 1, \forall \, n \in \mathbb{N}, \ I_{nk} = I_k^n$\\
				\hline
				f noeth. & son anneau de Rees ${R}(A,f)$ est noethérien.\\
				\hline
				f f. noeth. & $\exists k\geqslant 1, \forall \, m, n \in \mathbb{Z}, \ m, n \geqslant k, I_m I_n = I_{m+n}$\\
				\hline
				f E.P & $\exists N\geqslant 1, \forall \, n \geqslant N, \ I_n =\sum\limits_{p=1}^{N} I_{n-p}I_p. $\\
				\hline
			\end{tabular}
		\end{center}
		\end{block}
	\end{frame}	
	
	\begin{frame}{PRÉLIMINAIRES}
		\framesubtitle{PROPRIÉTÉ DES FILTRATIONS I-BONNES}
		\begin{block}{}
			\begin{center}
				\begin{tikzpicture}
					% Création des nœuds
					\node (A) at (-1,0) {f I-adique};
					\node (B) at (2,0) {f I-bonne};
					\node (C) at (2,-2) {f fortement noethérienne};
					\node (D) at (6,-2) {f noethérienne};
					\node (E) at (9,0) {f A.P};
					\node (F) at (6,0) {f fortement A.P};
					\node (G) at (2,2) {f E.P};
					
					% Dessin des flèches avec des modifications pour les rendre plus visibles
					\draw[->, ultra thick, >=stealth] (A) -- (B);
					\draw[->, ultra thick, >=stealth] (B) -- (C);
					\draw[->, ultra thick, >=stealth] (B) -- (F);
					\draw[->, ultra thick, >=stealth] (C) -- (D);
					\draw[->, ultra thick, >=stealth] (F) -- (E);
					\draw[->, ultra thick, >=stealth] (F) -- (D);
					\draw[->, ultra thick, >=stealth] (B) -- (G);
				\end{tikzpicture}
			\end{center}
		\end{block}
	\end{frame}
	
%		\begin{frame}{PRÉLIMINAIRES}
%		\framesubtitle{DÉMONSTRATION}
%		\begin{block}{}
%			\begin{enumerate}[(i)]
%				\item Supposons que $f$ est $I-adique$ alors peu importe $n_0 \in \mathbb{N}$ choisi, $ II^n=I^{n+1},$ pour tout $ n \in \mathbb{N}.$ Donc $f$ est $I-bonne$.
%				\item De proche en proche, on a $I^{n}I_{n_0} = I_{n_{0}+n},$ pour tout $n \geqslant 1$.\\
%				En effet, $II_{n_0} = I_{n_{0}+1}$, en multipliant par $I$.\\ On a: $I^{2}I_{n_0} = II_{n_{0}+1}$ et $I^1I_{n_0+1} = I_{n_{0}+2}$. \\
%				
%			\end{enumerate}
%		\end{block}
%	\end{frame}
%	
%		\begin{frame}{PRÉLIMINAIRES}
%		\framesubtitle{DÉMONSTRATION}
%		\begin{block}{}
%			\begin{enumerate}[(iii)]
%				\item Supposons que $f$ est $I-bonne$ alors il existe $n_0 \in \mathbb{N}$ tel que pour tout $m\geqslant 1 $, $I^mI_n = I_{n+m}, \forall n \geqslant n_0.$\\
%				Posons $k=n_0+1,$ soient $m,n \in \mathbb{N}$ alors:\\
%				\begin{center}
%					$I_{m+n}=I^mI_n \subset I_1^mI_n \subset I_mI_n \subset I_{m+n}$
%				\end{center}
%				Donc $ \forall \, m, n \in \mathbb{Z}, \ m, n \geqslant k, I_m I_n = I_{m+n}$.\\
%				Par suite f est fortement noethérienne.
%			\end{enumerate}
%		\end{block}
%	\end{frame}
%			\begin{frame}{PRÉLIMINAIRES}
%		\framesubtitle{DÉMONSTRATION}
%		\begin{block}{}
%			\begin{enumerate}[(iv)]
%				\item Supposons que $f$ est $I-bonne$ alors il existe $n_0 \in \mathbb{N}$ tel que pour tout $m\geqslant 1 $, $I^mI_n = I_{n+m}, \forall n \geqslant n_0.$\\
%				Posons $k=N=n_0+1$.
%				\begin{center}
%					$\sum\limits_{p=1}^{N} I_{n-p}I_p= I_{n-1}I_1 + \sum\limits_{p=2}^{N} I_{n-p}I_p$
%				\end{center}
%				Prenons $n \geqslant N=n_0+1$ alors $n-1 \geqslant n_0$.\\
%				Alors $I_{n-1}I_1=I_n$.\\
%				D'où $I_n \subset \sum\limits_{p=1}^{N} I_{n-p}I_p \subset I_n$.\\
%				Par suite f est $E.P$
%			\end{enumerate}
%		\end{block}
%	\end{frame}
%	
%	\begin{frame}{PRÉLIMINAIRES}
%		\framesubtitle{DÉMONSTRATION}
%		\begin{block}{}
%			\begin{enumerate}[(v)]
%				\item Supposons que $f$ est $I-bonne$ alors il existe $n_0 \in \mathbb{N}$ tel que pour tout $m\geqslant 1 $, $I^mI_n = I_{n+m}, \forall n \geqslant n_0.$\\
%				Par récurrence sur $n \in \mathbb{N}$, montrons que $I_{nk} = I_k^n$.\\
%				Posons $k= n_0+1 \geqslant 1$\\
%				Initialisation: n=0, n=1, évident.\\
%				Prenons n= 2, $I_{2k} \subset I_kI_k=I_k^2 \subset I_{2k}$, donc $I_{2k} = I_k^2$.\\
%				Hérédité: Soit $n \geqslant 2$. Supposons que $I_{nk} = I_k^n$.\\
%				On a: $I_{(n+1)k}= I^kI_k^n\subset I_kI_k^n \subset I_k^{n+1}$, donc $I_{(n+1)k} = I_k^{n+1}$.\\
%				Par suite f fortement $A.P.$
%			\end{enumerate}
%		\end{block}
%	\end{frame}
	\begin{frame}{PRÉLIMINAIRES}
		\framesubtitle{ÉLÉMENT ENTIER ET RÉDUCTION}
		\begin{block}{}
			\begin{enumerate}
				\item[(i)] Un élément $x$ de $A$ est dit entier sur $f$ s'il existe un entier $m \in \mathbb{N}$ tel que : $x^m + a_1 x^{m-1} + \cdots + a_m = x^m + \sum_{i=1}^{m} a_i x^{m-i} = 0,$\\$ m \in \mathbb{N^*} \ \text{où} \ a_i \in I_i,\, \forall i=1, \cdots ,m.$
				\item[(ii)] $f$ est une $\beta$-réduction de $g$ si : \\
				\begin{enumerate}
					\item[a)] $f \leq g$
					\item[b)]  $\exists \, k \geq 1$ tel que $J_{n+k} = I_n J_k , \forall n \geq k$.
				\end{enumerate}
			\end{enumerate}
		\end{block}
	\end{frame}
	
	\begin{frame}{PRÉLIMINAIRES}
		\framesubtitle{FILTRATIONS f-BONNES}
		\begin{block}{}
			Soient $\varphi=(M_n)_{n \in \mathbb{Z}}$ $\in$ $\mathbb{F}(M)$, $f-compatible$, avec $f \in \mathbb{F}(A)$.
			\begin{enumerate}[(a)]
				\item $\varphi$ est \textbf{$f-$ bonne} s'il existe un entier naturel N $\geqslant 1$ tel que:
				\[\forall n > N, M_{n}=\sum_{p=1}^{N}I_{n-p}M_{p} \]
				\item Une filtration $f=(I_n)_{n \in \mathbb{Z}}$ est dite $I-bonne$ si: 
				\begin{enumerate}[(i)]
					\item $\forall n \in \mathbb{N}, \quad II_n \subseteq I_{n+1}$;
					\item $\exists k \in \mathbb{N}$, $II_n = I_{n+1}, n\geqslant k$.
				\end{enumerate}
			\end{enumerate}
		\end{block}
	\end{frame}
	
		\begin{frame}{PRÉLIMINAIRES}
		\framesubtitle{RELATION ENTRE ÉLÉMENT ENTIER, RÉDUCTION ET FILTRATION I-ADIQUE}
		\begin{block}{Proposition}
				Soient $A$ un anneau, $I$ un idéal de $A$ et $x \in A$.
				\begin{center}
					$x$ est entier sur $I$ si et seulement si $I$ est une réduction de $I + (x) = I +xA $.
				\end{center}
		\end{block}
	\end{frame}
			\begin{frame}{Démonstration}
		\begin{block}{}
	$(i)$ Supposons que $x$ est entier sur $I$. Alors il existe $n \in \mathbb{N^*}$ tel que $x^n = \displaystyle \sum_{i=1}^{n}{a_i x^{n-i}}$, avec $a_i \in I^i, i=1, \cdots ,n$.\\
	Montrons que $I$ est une réduction de $I + (x)$.\\
	$(I+(x))^n = (I+(x))(I+(x))^{n-1}= I(I+(x))^{n-1} + (x)(I+(x))^{n-1}$\\
	En prouvant que $(x)(I+(x))^{n-1} \subset I(I+(x))^{n-1}$ on aura:
		\end{block}
	\end{frame}
	
			\begin{frame}{Démonstration}
		\begin{block}{}
			\begin{center}
				$I(I+(x))^{n-1} + (x)(I+(x))^{n-1} = I(I+(x))^{n-1}$.
			\end{center}
			\begin{align*}
				(x)(I+(x))^{n-1} &= (x)\displaystyle \sum_{i=0}^{n-1}{I^i (x)^{n-1-i}}\\
				&= \displaystyle \sum_{i=0}^{n-1}{I^i (x)^{n-i}}\\
				&= (x)^n + \displaystyle \sum_{i=1}^{n-1}{I^i (x)^{n-i}}\\
				&= (x)^n + I\displaystyle \sum_{i=1}^{n-1}{I^{i-1} (x)^{n-i}}\\
				&= (x)^n + I\displaystyle \sum_{i=0}^{n-2}{I^i (x)^{n-1-i}}
			\end{align*}
		\end{block}
	\end{frame}
				\begin{frame}{Démonstration}
		\begin{block}{}
			Donc $(x)(I+(x))^{n-1} = (x)^n + \displaystyle \sum_{i=0}^{n-2}{I^i (x)^{n-1-i}} \subset (x)^n + \displaystyle \sum_{i=0}^{n-1}{I^i (x)^{n-1-i}}$\\
			d'où $(x)(I+(x))^{n-1} \subset (x)^n + I(I+(x))^{n-1}$\\ et comme $x^n = \displaystyle \sum_{i=1}^{n}{a_i x^{n-i}} \in \displaystyle \sum_{i=1}^{n}{I^i x^{n-i}} \Rightarrow x^n \in I\displaystyle \sum_{i=1}^{n}{I^{i-1} x^{n-i}} = I\displaystyle \sum_{i=0}^{n}{I^i x^{n-1-i}}$\\
			alors $(x)^n \in I(I+(x))^{n-1} \Rightarrow (x)^n + I(I+(x))^{n-1} = I(I+(x))^{n-1}$.\\
			En somme $(x)(I+(x))^{n-1} \subset I(I+(x))^{n-1} \Rightarrow (I+(x))^{n} = I(I+(x))^{n-1}$.\\
			Par conséquent $I$ est une réduction de $I + (x)$.
		\end{block}
	\end{frame}
	
					\begin{frame}{Démonstration}
		\begin{block}{}
				$(ii)$ Supposons que $I$ est une réduction de $I + (x)$.\\
			Alors $\exists \, n \in \mathbb{N^*}$ tel que $(I + (x))^{n+1} = I(I + (x))^{n}$
			$x^{n+1} \in (I + (x))^{n+1} = I(I + (x))^{n} \Rightarrow x^{n+1} \in I\displaystyle \sum_{i=0}^{n}{I^i (x)^{n-i}} = \displaystyle \sum_{i=0}^{n}{I^{i+1} (x)^{n-i}}$.\\
			D'où $x^{n+1} \in \displaystyle \sum_{i=1}^{n+1}{I^i (x)^{n+1-i}} \Rightarrow x^{n+1} =  \displaystyle \sum_{i=1}^{n+1}{a_i x^{n+1-i}}$, avec $a_i \in I^i$.\\ Ainsi $x$ est donc entier sur $I$.
		\end{block}
	\end{frame}
	
	
	
%	\begin{frame}{PRÉLIMINAIRES}
%		\framesubtitle{PROBLÉMATIQUE ET ANNONCE DU PLAN}
%		\begin{block}{}
%			\begin{enumerate}
%				\item[(i)] Comment étendre les résultats des filtrations I-adiques aux filtrations bonnes?
%				\item[(ii)] Comment la dépendance intégrale et la réduction interagissent-elles avec les filtrations bonnes ?
%			\end{enumerate}
%		\end{block}
%	\end{frame}
	
	\begin{frame}
		\begin{enumerate}
			\item<0> \textcolor{blue}{PRÉLIMINAIRES}\\
			\item<1> \textcolor{blue}{DÉPENDANCE INTÉGRALE, RÉDUCTION ET FILTRATIONS BONNES }\\
			\item<0> \textcolor{blue}{CONCLUSION}\\
		\end{enumerate}
	\end{frame}
	
	\begin{frame}{DÉPENDANCE INTÉGRALE, RÉDUCTION ET FILTRATION BONNE}
		\begin{block}{Théorème Principal}
			Soient $A$ noethérien, $f=(I_{n})_{_{n\in \mathbb{N}}}\leq $ $g=(J_{n})_{_{n\in \mathbb{N}}}$ $ \in \mathbb{F}(A).$ \\ Si $f$ est fortement noethérienne et $g$ est noethérienne alors les assertions sont équivalentes et dans ce cas $g$ est fortement noethérienne:
			\begin{enumerate}[(i)]
				\item $f$ est une réduction de $g.$
				\item $I_{n}$ est une réduction de $J_{n}$ pour tout $n$ assez grand.
				\item Il existe un entier $k\geq 1$ tel que $g^{(k)}$ est $I_{k}-bonne$
			\end{enumerate}
		\end{block}
	\end{frame}
	
	\begin{frame}{DÉPENDANCE INTÉGRALE, RÉDUCTION ET FILTRATION BONNE}
		\begin{block}{Théorème Principal}
			\begin{enumerate}
				\item [(iv)]$g$ est $enti\grave{e}re$ sur $f.$
				\item [(v)]$g$ est $\ fortement$ $enti\grave{e}re$ sur $f.$
				\item [(vi)]$g$ est $f-fine.$
				\item [(vii)]$g$ est $f-bonne.$
				\item [(viii)]$g$ est $faiblement$ $f-bonne.$
				\item [(ix)]$P(f)=P(g)$
			\end{enumerate}
		\end{block}
	\end{frame}
	
	\begin{frame}
		\begin{enumerate}
			\item<0> \textcolor{blue}{PRÉLIMINAIRES}\\
			\item<0> \textcolor{blue}{DÉPENDANCE INTÉGRALE, RÉDUCTION ET FILTRATIONS BONNES }\\
			\item<1> \textcolor{blue}{CONCLUSION}\\
		\end{enumerate}
	\end{frame}
	
	\begin{frame}{CONCLUSION}
		\framesubtitle{BILAN ET PERSPECTIVES}
		\begin{block}{}	
			\begin{enumerate}
				\item Propriétés des $f_I$ et réduction minimale des filtrations bonnes
				\item Étendre ces résultats aux autres classes de filtration.
			\end{enumerate}
		\end{block}
	\end{frame}
	
	\begin{frame}
		\bigskip
		{\textcolor{blue}{\Large \textit{\textbf{\begin{center}
							MERCI POUR VOTRE AIMABLE ATTENTION
		\end{center}}}}}
	\end{frame}
\end{document}