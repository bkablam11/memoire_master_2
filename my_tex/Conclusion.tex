\addcontentsline{toc}{chapter}{\large{Conclusion}}
\begin{center}
	\chapter*{Conclusion}
\end{center}

$ \quad $ L'objectif de ce travail était l'étude de la dépendance intégrale et de la réduction par rapport aux idéaux à travers la filtration I-adique qui est une filtration I-bonne. \\
Ensuite, nous avons montré la dépendance intégrale et la réduction des filtrations bonnes en générale. Nous pouvons donc retenir que toutes les filtrations bonnes admettent une réduction minimale. De plus, nous pouvons si certaines conditions sont vérifiées établir des propositions équivalentes entre les notions de réduction, dépendance intégrale et filtrations bonne.

Comme perspective, nous projetons d'étudier sous quelles hypothèses nous pourrons étendre ces résultats aux autres classes de filtrations notamment les filtrations noethériennes.


