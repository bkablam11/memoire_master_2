\documentclass[12pt, a4paper, oneside]{article}
\usepackage[utf8]{inputenc}
\usepackage[francais]{babel}
\usepackage[T1]{fontenc}
\usepackage{amsmath}
\usepackage{amsfonts}
\usepackage{arial}
\usepackage{setspace}
\usepackage{amsthm}
\usepackage{hyperref}
\usepackage{mathrsfs}
\usepackage{amssymb}
\usepackage{makeidx}
\usepackage{graphicx}
\usepackage{lmodern}
\usepackage[left=2.8cm,right=2.8cm,top=3.5cm,bottom=3.5cm]{geometry}
\usepackage{tocloft}
\usepackage{stmaryrd}
\hypersetup{colorlinks=true, urlcolor=blue,linkcolor=blue, breaklinks=true}
\usepackage{tikz}
\newtheorem{maremarque}{Remarque}
\newtheorem{montheoreme}{Théorème}
\newtheorem{madefinition}{Définition}
\newtheorem{maproposition}{Proposition}
\newtheorem{moncorollaire}{Corollaire}
\newtheorem{monlemme}{Lemme}
\newtheorem{monexemple}{Exemple}
\newtheorem{maconsequence}{Consequence}
\begin{document}
	\section{A RETENIR}
	
	\underline{\textbf{Introduction}}\\\\
	\underline{\textbf{Évaluateur}}: Excellence Monsieur le président du Jury, Honorable membres du Jury $\Longrightarrow $ évaluation du travail. \\\\
	\underline{\textbf{Collaborateur}} : Mr Assan A. (MC), Mr Brou K.P. (M.A) $\Longrightarrow $ l'accompagnement dans l'élaboration de ce mémoire. \\\\
	\underline{\textbf{Famille}} : Parents, amis, connaissance $\Longrightarrow $ déplacement et soutien. \\\\
	\underline{\textbf{Thème}} : Dépendance intégrale, réduction et filtrations bonnes.\\\\
	\underline{\textbf{Utilité}} : Structure des Anneaux et Modules, Étude des équation, Suite d'éléments algébriques.\\\\
	\underline{\textbf{Plan}} : Cas particulier et lien entre ces 3 notions et cas général des filtrations bonnes et liens.\\\\
	\underline{\textbf{Référence}} :
	\begin{enumerate}
		\item (Filtration) Dichi H., \textit{Travaux de recherches, en vue de l'habilitation à diriger une recherche}
		\item (Entier) Prüfer H.,\textit{ Untersuchungen über Teilbarkeitseigenschaften in Körpern} - 1930.
		\item ($\beta-reduction$) Dichi H., Sangare D., et Soumare M.,\textit{ Filtrations, integral dependence, reduction, f-good filtrations}.
		\item ($\alpha-reduction$) Okon J. S., et Ratliff L.,\textit{ Reductions of filtrations}
	\end{enumerate}
	\underline{\textbf{Relation}} : Ordre (R.A.T.) et Équivalence (R.S.T.) $\Longrightarrow $ L'ordre est dit total si deux éléments sont toujours comparables.\\\\
	\underline{\textbf{Outro}}\\\\
	\underline{\textbf{Bilan}}: Étude sur les filtrations I-adiques puis proposition.   $\Longrightarrow $ Filtrations bonnes et liens.\\\\
	\underline{\textbf{Perspectives}}: Dans un anneau local noethérien, la réduction minimale des filtrations I-bonnes existe toujours. $\Longrightarrow $ Est ce que toute les filtrations admettent une réduction minimal ? $\Longrightarrow $ Une filtration noethérienne admet une réduction minimale si et seulement s'il existe $r \geqslant 1 , I_r$ soit idempotent. \\\\
	\underline{\textbf{Remerciement}}: Excellence Monsieur le président du Jury, Honorable membres du Jury. $\Longrightarrow $  Nullement avoir cerné tous les contours $\Longrightarrow $  remarques et suggestions.\\\\
	
%	\begin{madefinition} (Filtration tronqué d'ordre k de $f$) \\
%		Soient $A$ un anneau et $I$ un idéal de A. $f = (I_n)_{n\in \mathbb{Z}}$ une filtration de l'anneau A.\\
%		Soit $k \in \mathbb{N}^{*}$, on pose $f^{(k)} = (I_{nk})_{n\in \mathbb{N}}$ et $t_{k}f=(K_n)$ avec $K_n = I_{n+k}$ si $n \geqslant 1 $ et $K_n = A$ si $n \leqslant 0$.\\
%		Ainsi $t_{k}f$ est une filtration de $A$ appelé \textbf{filtration tronqué d'ordre k de $f$}.
%	\end{madefinition}
%	\begin{madefinition}(Filtrations I-bonnes)\\
%		Soient $A$ un anneau et $I$ un idéal de l'anneau $A$.\\
%		$f = (I_n)_{n \in \mathbb{Z}}$ de $A$. $f$ est dite $I$-bonne si :
%		\begin{enumerate}
%			\item[i)]$ II_n \subseteq I_{n+1} \, \forall n \in \mathbb{Z}$.
%			\item[ii)]$\exists \, n_0 \in \mathbb{N}$ tel que $II_n = I_{n+1}, \forall n \geqslant n_0$
%		\end{enumerate}
%	\end{madefinition}
%	\begin{madefinition} (Filtrations A.P.) \\
%		La filtration $f = (I_n)_{n \in \mathbb{Z}}$ de l'anneau $A$ est dite \textbf{Approximable par Puissances d'idéaux} (en abrégé \textbf{A.P.}) s'il existe un entier $(k_{n})_{n \in \mathbb{N}}$ une suite d'entiers naturels telle que :
%		\begin{enumerate}
%			\item[(i)] $\forall$ n,m $\in \mathbb{N}$, $I_{mk_n} \subset I_n^{m}$
%			\item[(ii)] $\underset{n\longrightarrow +\infty }{\lim }\dfrac{k_{n}}{n}=1$
%		\end{enumerate}
%	\end{madefinition}
%	\begin{madefinition}(Filtrations fortement A.P.)\\
%		La filtration $f = (I_n)_{n \in \mathbb{Z}}$ de l'anneau $A$ est dite \textbf{fortement Approximable par Puissances d'idéaux} (en abrégé \textbf{fortement A.P.}) s'il existe un entier $k \geqslant 1$ tel que :
%		\[ \forall \, n \in \mathbb{N}, \ I_{nk} = I_k^n \]
%	\end{madefinition}
%	\begin{madefinition}(Filtrations E.P.)\\
%		La filtration $f = (I_n)_{n \in \mathbb{Z}}$ de l'anneau $A$ est dite \textbf{Essentiellement par Puissances d'idéaux} (en abrégé \textbf{E.P}) s'il existe un entier $N \geqslant 1$ tel que :
%		\[ \forall \, n \geqslant N, \ I_n =\sum_{p=1}^{N} I_{n-p}I_p. \]
%	\end{madefinition}
%	\begin{madefinition}(Filtrations noethériennes) \\
%		La filtration $f = (I_n)_{n \in \mathbb{Z}}$ de l'anneau $A$ est dite \textbf{noethérienne} si son anneau de Rees ${R}(A,f)$ est noethérien.
%	\end{madefinition}
%	\begin{madefinition}(Filtrations fortement noethériennes) \\
%		La filtration $f = (I_n)_{n \in \mathbb{Z}}$ de l'anneau $A$ est dite \textbf{fortement noethérienne} s'il existe un entier $k \geqslant 1$ tel que:
%		\[ \forall \, m, n \in \mathbb{Z}, \ m, n \geqslant k, I_m I_n = I_{m+n} \]
%	\end{madefinition}
%	\begin{maproposition}(Caractérisation des filtrations noethériennes)\\
%		Soit $f=(I_{n})_{n \in \mathbb{Z}}\in \mathbb{F}(A),$ si $A$ est noethérien, alors les assertions suivantes sont équivalentes:
%		\begin{enumerate}
%			\item[(i)] $\mathcal{R}(A,f)$ est noethérien
%			\item[(ii)] $R(A,f)$ est noethérien
%			\item[(iii)] $f$ est $E.P.$
%			\item[(iv)] $\exists n \geqslant 1,\forall n \geqslant k, I_{n+k} = I_nI_k$
%		\end{enumerate}
%	\end{maproposition}
%	\begin{moncorollaire}(Clôture prüférienne)\\
%		Soit $f=(I_n)_{n \in \mathbb{N}} \in \mathbb{F}(A)$. Alors:\\
%		$\forall k \in \mathbb{N}, \text{ on pose: } P_k(f)=\left\{x \in A, x \text{ entier sur } f^{(k)}\right\}$ est un idéal de $A$ et la famille \\ $P(f)=(P_k(f))_{k \in \mathbb{N}}$ est une filtration de $A$ appelé \textbf{clôture prüférienne} de $f$.
%	\end{moncorollaire}
%	\begin{maremarque} (Clôture intégrale et clôture prüférienne) \\
%		La clôture intégrale d'un idéal $I$ de $A$ est : $I'=P_1(f_I)$
%	\end{maremarque}
%	\begin{madefinition}(filtration entière et fortement entière)\\
%		Soit $f=(I_n)_{n \in \mathbb{N}} , g = (J_n)_{n \in \mathbb{N}}\in \mathbb{F}(A)$ et soit $k\geq 1,$ on a: $f^{(k)}=(I_{nk})_{n\in \mathbb{Z}}$ et $g^{(k)}=(J_{nk})_{n\in \mathbb{Z}}$ .  Alors:\\
%		\begin{enumerate}
%			\item[(a)]$g$ est \textbf{entière sur} $f$ si $g \leqslant P(f)$ (où $P(f)$ est la clôture prüférienne de $f$). C'est à dire:
%			\[\forall n \geqslant 1, J_n \subseteq P_{n}(f) \]
%			\item[(b)]$g$ est \textbf{fortement entière sur} $f$ si $f \leqslant g$ et si $R(A,g)$ est un $R(A,f)-module$ de type fini.
%		\end{enumerate}
%	\end{madefinition}
%	\begin{madefinition} (réduction basique et réduction minimale) \\
%		Un idéal $I$ de l'anneau local noethérien $(A,m)$ est basique si la seule réduction de $I$ est $I$ lui-m\^{e}me. 
%		Northcott et Rees ont aussi défini la notion de réduction minimale
%		d'un idéal $J$:
%		
%		Un idéal $I$ est une réduction minimale de $J$ si $I$ est une réduction de $J$ et si I est $minimal$ au sens de l'inclusion ( $\subseteq$ ) parmi l'ensemble des réductions de J. 
%	\end{madefinition}
%	\begin{maremarque}
%		On prouve dans \cite{Di2} que la réduction minimale des filtrations $I$-bonne existe toujours dans un anneau local noethérien. Ce qui n'est pas le cas en générale pour une filtration quelconque.
%	\end{maremarque}
%	\begin{madefinition}($\alpha$-réduction ou réduction au sens de Okon-Ratliff)\\
%		Soient $f=(I_n)_{n \in \mathbb{N}}$, $g=(J_n)_{n \in \mathbb{N}}$ deux filtrations de $A$.\\
%		$f$ est une $\alpha$-réduction de $g$ si : \\
%		\begin{enumerate}
%			\item[i)] $f \leq g$
%			\item[ii)] $\exists \, N \geq 1$ tel que $\forall n \geq N ; J_n = \displaystyle \sum_{p=0}^{N}{I_{n-p} J_p}$.
%		\end{enumerate}
%	\end{madefinition}
%	\begin{madefinition}($\beta$-réduction ou réduction au sens de Dichi-Sangaré)\\
%		Soient $f = (I_n)_{n \in \mathbb{N}}$, $g = (J_n)_{n \in \mathbb{N}}$ deux filtrations de $A$.\\
%		$f$ est une $\beta$-réduction de $g$ si : \\
%		\begin{enumerate}
%			\item[i)] $f \leq g$
%			\item[ii)]  $\exists \, k \geq 1$ tel que $J_{n+k} = I_n J_k , \forall n \geq k$.
%		\end{enumerate}
%	\end{madefinition}
%	\begin{maremarque}($\alpha$-réduction et $\beta$-réduction)\\
%		Si $f$ est une $\beta$-réduction de $g$ alors $f$ est une $\alpha$-réduction de $g$.
%	\end{maremarque}
%	\begin{madefinition} (faiblement f-bonne, f-bonne et f-fine) \\
%		\label{maprop11}
%		Soient $A$ un anneau et $M$ un $A-module$.\\
%		On suppose que $\varphi=(M_n)_{n \in \mathbb{Z}}$ est $f-compatible$, avec $f \in \mathbb{F}(A)$. Alors:
%		\begin{itemize}
%			\item[(a)] $\varphi$ est \textbf{faiblement $f-$ bonne} s'il existe un entier naturel N $\geqslant 1$ tel que:
%			\[\forall n > N, M_{n}=\sum_{p=0}^{N}I_{n-p}M_{p} \]
%			\item[(b)] $\varphi$ est \textbf{$f-$ bonne} s'il existe un entier naturel N $\geqslant 1$ tel que:
%			\[\forall n > N, M_{n}=\sum_{p=1}^{N}I_{n-p}M_{p} \]
%			\item[(c)] $\varphi$ est \textbf{$f-$ fine} s'il existe un entier naturel N $\geqslant 1$ tel que:
%			\[\forall n > N, M_{n}=\sum_{p=1}^{N}I_{p}M_{n-p} \]
%		\end{itemize} 
%	\end{madefinition}
	\begin{montheoreme}
		Soient $f=(I_{n})_{_{n\in \mathbb{N}}}\leq $ $g=(J_{n})_{_{n\in \mathbb{N}}}$ des filtrations sur l'anneau $A.$ Nous
		considérons les assertions suivantes:
		
		$i)$ $f$ est une réduction de $g.$
		
		$ii)$ $J_{n}^{2}=I_{n}J_{n}$ pour tout $n$ assez grand.
		
		$iii)$ $I_{n}$ est une réduction de $J_{n}$ pour tout $n$ assez grand.
		
		$iv)$ Il existe un entier $s\geq 1$ tel que pour tout $n\geq s,$ $J_{s+n}=J_{s}J_{n},$
		
		$I_{s+n}=I_{s}I_{n},$ $J_{s}^{2}=I_{s}J_{s},$ $J_{s+p}I_{s}=I_{s+p}J_{s}$ pour tout $p=1,2,...,s-1$
		
		$v)$ Il existe un entier $k\geq 1$ tel que $g^{(k)}$ est $I_{k}-bonne$
		
		$vi)$ Il existe un entier $r\geq 1$ tel que $f^{(r)}$ est une réduction de $g^{(r)}.$
		
		$vii)$ Pour tout entier $m\geq 1$ tel que $f^{(m)}$ est une réduction de $g^{(m)}.$
		
		$viii)$ $g$ est entière sur $f.$
		
		$ix)$ $g$ est fortement entière sur $f.$
		
		$x)$ $g$ est $f-fine.$
		
		$xi)$ $g$ est $f-bonne.$
		
		$xii)$ $g$ est $faiblement$ $f-bonne.$
		
		$xiii)$ Il existe un entier $N\geq 1$ tel que $t_{N}g\leq f\leq g$
		
		$xiv)$ Il existe un entier $N\geq 1$ tel que $t_{N}g^{\prime }\leq
		t_{N}f^{\prime \text{ }}$ où $f^{\prime }$ est la clôture intégrale de $f.$
		
		$xv)$ $P(f)=P(g)$, où $P(f)$ est la clôture prüférienne de $f.$
		
		1) On a:
		
		$(i)$ $\Longleftrightarrow (vii)$ ; $(v)$ $\Longleftrightarrow (vi)$ ; $(viii)$ $\Longleftrightarrow (xv)$ ; $(ii)$ $\Longrightarrow (iii)$ ; $(iv)$ 
		$\Longrightarrow (i)\Longrightarrow (v)$ ; $(ix)\Longrightarrow (vii),(xii)$
		et $(xiii)$ ;
		
		$(i)\Longrightarrow (x)\Longrightarrow (xi)\Longrightarrow
		(xii)\Longrightarrow (xiii)$
		
		2) Si de plus on suppose $A$ noethérien, alors:
		
		$(i)\Longleftrightarrow (xiv)$ ; $(i)\Longrightarrow (ix)\Longleftrightarrow
		(xii)$ ; $(i)\Longrightarrow (ii)$
		
		3) Par ailleurs, si $f$ est $noeth\acute{e}rienne,$ alors $A$ est noethérien et les assertions suivantes sont équivalentes:
		
		$(ix)\Longleftrightarrow (x)\Longleftrightarrow (xi)\Longleftrightarrow
		(xii)\Longleftrightarrow (xiii)$
		
		4) Si $f$ et $g$ sont noethériennes alors nous avons:
		
		$(iii)\Longrightarrow (viii)\Longleftrightarrow (ix)$ ; $(vi)\Longrightarrow (ix)$
		
		5) Si $f$ est fortement noethérienne et $g$ est noethérienne alors les quinze (15) assertions sont équivalentes et dans ce cas $g$ est fortement noethérienne
		
		$(i)\Longleftrightarrow (ii)\Longleftrightarrow (iii)\Longleftrightarrow
		(iv)\Longleftrightarrow (v)\Longleftrightarrow (vi)\Longleftrightarrow
		(vii)\Longleftrightarrow (viii)\Longleftrightarrow (ix)\Longleftrightarrow
		(x)\Longleftrightarrow (xi)\Longleftrightarrow (xii)\Longleftrightarrow
		(xiii)\Longleftrightarrow (xiv)\Longleftrightarrow (xv).$
	\end{montheoreme}
\end{document}
