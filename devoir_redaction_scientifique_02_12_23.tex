\documentclass[12pt, a4paper, oneside]{book}
\usepackage[utf8]{inputenc}
\usepackage[francais]{babel}
\usepackage[T1]{fontenc}
\usepackage{amsmath}
\usepackage{amsfonts}
\usepackage{arial}
\usepackage{setspace}
\usepackage{amsthm}
\usepackage{hyperref}
\usepackage{mathrsfs}
\usepackage{amssymb}

\usepackage[left=2.8cm,right=2.8cm,top=3.5cm,bottom=3.5cm]{geometry}


\newtheorem{madefinition}{Définition}[chapter]


\title{DEVOIR DE RÉDACTION SCIENTIFIQUE}
\author{KABLAM EDJABROU ULRICH BLANCHARD (CI0216045024)}
\begin{document}
	\maketitle
	\newpage
	\section*{I) Questions de cours}
	\subsection*{1) Donner les différentes composantes du corps d'un mémoire de Master. Explicitez ces composantes en quelques lignes.}

	Le corps d'un mémoire de master se concentre sur le fond qui est perçus à travers les composantes suivantes:\\

	
	a. La page de Dédicace:\\
	C'est un espace où l'étudiant peut exprimer sa gratitude ou son dévouement envers des personnes spécifiques \\
	
	b. La page de Remerciements:\\
	$-$ Expression de gratitude envers les personnes ou les institutions qui ont contribué à la réalisation du mémoire et qui ont participé de près ou de loin à la formation et à l'épanouissement de l'étudiant (enseignant, administration, étudiant, encadrant, etc...)\\
	
	c. La page de Résumé :\\
	$-$ Un résumé concis du mémoire, généralement pas plus de 250 mots.\\
	$-$ Mots-clés (environ 3 à 5).\\
	
	d. La page Abstract:\\
	Résumé traduit en anglais.\\
	
	e. La page de Notation :\\
	La page de notations mathématiques clarifie la signification des symboles et des expressions utilisés dans le mémoire, facilitant ainsi la compréhension pour le lecteur.\\
	
	f. Liste des tableaux et figures :\\
$-$ Si votre mémoire contient des tableaux ou des figures, une liste numérotée de ceux-ci avec leur titre et leur numéro de page.\\
	
	g. Table des matières :\\
	$-$ Liste détaillée des sections et sous-sections du mémoire avec les numéros de page correspondants.\\
	\newpage
	h. Introduction :\\
	$-$ Généralité et Présentation du sujet\\
	$-$ Analyse critique des travaux antérieurs liés au sujet (rappeler ce qui à déjà été fait)\\
	$-$ Formulation de la problématique\\
	$-$ Objectifs de l'étude\\
	$-$ Justification de la recherche et hypothèses de travail\\
	$-$ Annonce du mémoire\\
	
	i. Les chapitres :\\
	Chapitre 1 : Généralités\\
	Chapitre 2 et Chapitre 3 \\
	Et éventuellement un chapitre 4\\
	
	j. Conclusion et perspectives :\\
	$-$ Résumé des principales idées développées dans le mémoire\\
	$-$ Perspectives\\
	
	k. Bibliographie :\\
	$-$ Liste de toutes les sources citées dans le mémoire, formatées selon un style de citation spécifique et uniforme (par exemple lister par ordre alphabétique des noms)\\
	

	\subsection*{2) Donner un exemple de bibliographie ayant six (6) références. NB: Appliquer les règles de la rédaction scientifique}

	\begin{thebibliography}{99}
		
		\bibitem{1} Dichi H,\textit{ Integral dependence over a filtration}, Journal of Pure and Applied Algebra, Vol. 58, No. 1, 1989, pp. 7-18. 
		
		\bibitem{2} Dichi H., \textit{Travaux de recherches, en vue de l'habilitation à diriger une recherche}, 1999, pp. 14-15.	
		
		\bibitem{3} Dichi H., Sangare D., and Soumare M,\textit{ Filtrations, integral dependence, reduction, f-good filtrations}, Communications in Algebra, Vol. 20, No. 8, 1992, pp. 2393-2418.
		
		\bibitem{4} Northcott D. G., and Rees D.,\textit{ Reductions of ideals in local rings}, in Mathematical Proceedings of the Cambridge Philosophical Society, Vol. 50, No. 2, 1954, pp. 145-158.
		
		\bibitem{5} Okon J. S., and Ratliff L,\textit{ Reductions of filtrations}, Pacific Journal of Mathematics, Vol. 144 No. 1, 1990, pp. 137-154.
		
		\bibitem{6} Ratliff Jr L. J.,\textit{ Notes on essentially powers filtrations}, Michigan Mathematical Journal, Vol.26 No. 3, 1979, pp. 313-324.
		
	\end{thebibliography}
	
	
	\subsection*{3) La rubrique "notation" dans le corps d'un mémoire est-elle fondamentalement nécessaire dans la réalisation d'un mémoire? Justifier votre réponse.}
	
	Oui, la notation est extrêmement importante lors de la rédaction d'un mémoire. La notation mathématique est un langage précis qui permet de représenter de manière concise et formelle des concepts mathématiques complexes. Elle est importante car elle permet la précision et l'universalité des expressions mathématiques et facilite ainsi la compréhension des symboles utilisés dans le corps du mémoire.
	
	\section*{II) Réalisation d'un mémoire.}
	
	\subsection*{Écrivez à votre goût une page d'un chapitre d'un mémoire ayant en son sein trois citations de référence.\\ NB: la note qui vous sera attribuée, sera fonction des éléments importants de la rédaction scientifique.}
	
	\chapter{GÉNÉRALITÉS}
	Ce chapitre revêt la forme d'une rétrospective, visant à revisiter et consolider diverses définitions dans \cite{5} et propositions évoquées au sein des Unités d'Enseignement (U.E.) consacrées à l'Algèbre commutative et à la Théorie des filtrations. L'ensemble des résultats présentés dans ce chapitre s'enracine dans \cite{6}. Leur maîtrise s'avère incontournable pour une appréhension éclairée des sections à venir. Cependant, il convient de préciser que les démonstrations des propositions énoncées dans ce chapitre ne seront exposées que lorsqu'elles s'avéreront nécessaires. Il est donc impératif de mettre en exergue le fait que la plupart des résultats dans \cite{3} resteront sans démonstration. \\\\ Dans l'ensemble de ce mémoire, à sauf mention du contraire, on présumera que tous les anneaux sont de nature \textbf{commutative et unitaire}.
	\section{Groupe et sous-groupe}
	\subsection{Groupe}
	\begin{madefinition}
		\label{madef1}
		Soit G un ensemble non vide.\\
		On dit que $(G,\star)$ est un groupe si $\star$ est une loi qui a tout élément de $G$ associe un élément dans $G$ vérifiant:
		\begin{enumerate}
			\item[i] ) La loi $\star$ est associative 
			\[ a\star(b\star c) = (a\star b) \star a, \forall (a,b,c) \in G^3,\]
			\item[ii] ) G possède un élément neutre
			\[ \exists!e \in G, e\star a = e =  a\star e, \forall a \in G,\]
			\item[iii] ) Tout élément $a$ de $G$ admet un symétrique
			\[ a\star b = e = b \star a, \forall (a,b) \in G^2.\]
			Si de plus la loi $\star$ est commutative, 
			\[ a\star b = b \star a, \forall (a,b) \in G^2\]
			On dit que le groupe $(G,\star)$ est abélien ou commutatif.
		\end{enumerate}
	\end{madefinition}
\end{document}