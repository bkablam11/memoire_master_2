\begin{center}
	\LARGE{\textbf{Dédicace}}
\end{center}

\begin{center}
	Je dédie ce travail\\
	\uppercase{à} la mémoire de mon père, KABLAM Oi Kablam Simon, \\ en témoignage de ma reconnaissance pour tout ce qu’il a fait pour moi et en signe de mon attachement indéfectible à ses principes et règles. \\
	\uppercase{à} ma mère, KASSI Ettien Cécile, \\ qui m'a inculqué des valeurs de persévérance, d'honnêteté et de bonne moralité. \\
	\uppercase{à} mes frères et sœurs, pour leur soutien tout au long de mon parcours.\\
	\uppercase{à} mes amis, pour leurs encouragements et leurs conseils.\\
	\uppercase{à} tous ceux que j'aime.\\
	\uppercase{à} tous ceux qui m'aiment.
\end{center} 

\newpage
\addcontentsline{toc}{chapter}{\large{Remerciements}}
\begin{center}
	\LARGE{\textbf{Remerciements}}
\end{center}

C’est avec un immense plaisir que je m’apprête à soutenir mon mémoire de \\ MASTER. Je me livre à l’exercice délicat des remerciements.\\\\
Je rends gloire et louange à Dieu pour la grâce qu’il m’a accordée, me permettant de maintenir le moral haut tout au long de mon parcours malgré les nombreuses \\ péripéties.\\\\
Je tiens à remercier chaleureusement Monsieur DIAGANA Youssouf, Directeur du Laboratoire de Mathématiques et Informatique de l’Unité de Formation et de\\ Recherche Sciences Fondamentales et Appliquées (UFR-SFA) et Professeur Titulaire de l’Université NANGUI ABROGOUA, d’avoir accepté de présider ce jury de mémoire de MASTER. Il a été une source d’inspiration pour nous et nous a transmis un véritable amour pour les Mathématiques.\\\\
J’adresse une note spéciale d’appréciation à mon Directeur de mémoire, Monsieur ASSANE Abdoulaye, Maître de Conférences à l’Université NANGUI ABROGOUA. Ses conseils éclairés et ses orientations ont été d’une grande valeur tout au long de la rédaction de ce mémoire.\\\\
Mes sincères remerciements vont également à mon encadrant scientifique, Monsieur BROU Kouadjo Pierre, Maître-Assistant à l’Université NANGUI ABROGOUA. Ses encouragements, ses conseils, sa disponibilité et ses critiques objectives ont été d’une aide précieuse pendant la réalisation de ce travail.\\\\
Je souhaite également exprimer ma gratitude envers les membres du jury, notamment Monsieur KOUAKOU Kouassi Vincent, Maître-Assistant à l’Université NANGUI \\ ABROGOUA, pour l’honneur qu’il nous fait en faisant partie de notre jury en tant qu’examinateur.\\\\
Mes remerciements s’adressent au Directeur de l’Unité de Formation et de Recherche Sciences Fondamentales et Appliquées (UFR-SFA), Monsieur KRE N’Guessan \\ Raymond, Professeur Titulaire à l’Université NANGUI ABROGOUA, ainsi qu’au Vice Doyen de l’UFR-SFA, Monsieur BENIÉ Anoubilé, Professeur Titulaire à l’Université NANGUI ABROGOUA et à Madame KOUASSI Bienvenue, Secrétaire Principale de l’UFR-SFA, qui, ensemble œuvrent inlassablement pour le bien-être des étudiant(e)s de l’UFR-SFA.\\\\
Enfin, je tiens à exprimer ma gratitude envers mes amis et condisciples de parcours depuis mes débuts à l’Université NANGUI ABROGOUA.
