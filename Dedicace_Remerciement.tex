\begin{center}
	\LARGE{\textbf{Dédicace}}
\end{center}

\begin{center}
	Je dédie ce travail\\
	\uppercase{à} la mémoire de mon père KABLAM Oi Kablam Simon, \\ que ce travail soit un témoignage de ma reconnaissance pour tout ce qu'il a fait pour moi et de mon attachement indéfectible à ses principes et règles. \\
	\uppercase{à} ma mère KASSI Ettien Cécile, \\ qui m'a inculqué des valeurs de persévérance, d'honnêteté et de bonne moralité. \\
	\uppercase{à} mes frères et sœurs pour leur soutien tout au long de mon parcours.\\
	\uppercase{à} mes amis pour leurs encouragements et leurs conseils.\\
	\uppercase{à} tous ceux que j'aime.\\
	\uppercase{à} tous ceux qui m'aiment.
	
	
\end{center} 

\newpage
\addcontentsline{toc}{chapter}{\large{Remerciements}}
\begin{center}
	\LARGE{\textbf{Remerciements}}
\end{center}

C'est avec un immense plaisir que je m'apprête à soutenir mon mémoire de MASTER. Je me donne, à cet effet, au délicat exercice des remerciements.\\\\
Je rends gloire et louange à Dieu, pour la grâce qu'il m'a accordée à pouvoir garder le moral haut le long de mon parcours en dépit de nombreuses péripéties.\\\\
Je souhaite adresser une note spéciale d'appréciation à mon Directeur de mémoire, Monsieur ASSANE Abdoulaye, Maître de Conférences à l'Université NANGUI ABROGOUA. Il n'a ménagé aucun effort pour donner forme à ce travail. Vos conseils éclairés et vos orientations ont été d'une grande valeur tout au long de la rédaction de ce mémoire.\\\\
J'exprime une reconnaissance infinie et mes sincères remerciements à mon encadrant scientifique, Monsieur BROU Kouadjo Pierre, Maître Assistant à l'Université NANGUI ABROGOUA. Ses encouragements, ses conseils, sa disponibilité et ses critiques objectives ont été d'une aide précieuse durant de la réalisation de ce travail.\\\\
Je souhaite également remercier le Professeur DIAGANA Youssouf, Directeur du Laboratoire de Mathématiques et Informatique de l'Unité de Formation et de Recherche Sciences Fondamentales et Appliquées (UFR-SFA) de l'Université NANGUI ABROGOUA. Il a été une source d'inspiration pour nous et nous a transmis un véritable amour pour les Mathématiques.\\\\
Je tiens à exprimer ma gratitude envers Monsieur BOMISSO Jean-Marc, Maître Assistant à l'Université NANGUI ABROGOUA, pour ses conseils et ses critiques lors de l'élaboration de ce travail, ainsi que pour son soutien et son assistance envers nous, étudiants de Master, tout au long de notre parcours.\\\\
Je souhaite également remercier les membres du jury, notamment Monsieur KPATA Akon Berenger, Maître de Conférences à l'Université NANGUI ABROGOUA, pour l'honneur qu'il nous fait en faisant partie de notre jury, ainsi que le professeur DIAGANA Youssouf pour avoir accepté de présider le jury de ce mémoire.\\\\
Mes remerciements s'adressent également au Directeur de l'Unité de Formation et de Recherche Sciences Fondamentales et Appliquées (UFR-SFA), Monsieur KRE N'Guessan Raymond, Professeur Titulaire, au Vice Doyen de l'UFR-SFA, le Professeur BENIE Anoubilé, ainsi qu'à la secrétaire principale, Madame KOUASSI Bienvenue, qui œuvrent inlassablement pour le bien-être des étudiants de l'UFR-SFA.\\\\
Enfin, je souhaite exprimer ma gratitude envers mes amis et condisciples de parcours depuis mes débuts à l'Université NANGUI ABROGOUA.
