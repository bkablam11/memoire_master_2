\documentclass[11pt,a4paper]{beamer}
\usepackage[utf8]{inputenc}
\usepackage[francais]{babel}
\usepackage{stmaryrd}
\usepackage[T1]{fontenc}
\usepackage{fancyhdr}
\usepackage{tikz}
\usetheme{Boadilla}
\author{\textit{\textbf{KABLAM Edjabrou Ulrich Blanchard}}}
\title{\textbf{SOUTENANCE DE MÉMOIRE DE MASTER \\ OPTION: ALGÈBRE COMMUTATIVE ET CRYPTOGRAPHIE\\ SPÉCIALITÉ: THÉORIE DES FILTRATIONS}}
\institute{\textcolor{red}{\textbf{Université NANGUI ABROGOUA \\ UFR Sciences Fondamentales Appliquées}}}
\usepackage{graphicx}
\usepackage{wrapfig}
\usepackage{mwe}
\logo{\includegraphics[width=0.7cm]{./img/UNA.png}}
\date{10 Juillet 2024}

\begin{document}
	\begin{frame}
		\maketitle
		\begin{block}{\begin{center}
					\emph{THÈME:} \textbf{DÉPENDANCE INTÉGRALE, RÉDUCTION ET FILTRATIONS BONNES }
			\end{center}}
			\begin{center}
				Directeur de Mémoire: Mr. ASSAN Abdoulaye, M.C. \\
				Encadrant scientifique: Mr. BROU Kouadjo Pierre, M.A.
			\end{center}
		\end{block}
	\end{frame}
	
	\begin{frame}{
			PLAN DE PRÉSENTATION}
		\begin{enumerate}
			\item \textcolor{blue}{INTRODUCTION}\\
			\item \textcolor{blue}{DÉPENDANCE INTÉGRALE, RÉDUCTION ET FILTRATION BONNES }\\
			\item \textcolor{blue}{CONCLUSION}\\
		\end{enumerate}
	\end{frame}
	\setbeamercovered{transparent}
	
	\begin{frame}{INTRODUCTION}
		\framesubtitle{FILTRATIONS}
		\begin{block}{}
			\begin{enumerate}[(a)]
				\item $f=(I_n)_{n \in \mathbb{Z}} \in \mathbb{F}(A)$ si: 
				\begin{enumerate}[(i)]
					\item $I_0 = A$
					\item $I_{n+1} \subset I_n, \forall n \in \mathbb{Z}$
					\item $I_{p}I_{q} \subset I_{p+q}, \forall p,q \in \mathbb{Z}$
				\end{enumerate}
			\end{enumerate}
		\end{block}
	\end{frame}
	
		\begin{frame}{INTRODUCTION}
		\framesubtitle{FILTRATIONS}
		\begin{alertblock}{Remarque}
		On peut remarquer que pour tout $n\leq 0, I_n = A$.\\ En effet, en utilisant la décroissance des idéaux (ii) et que $I_0 = A$ (i), il vient $I_n = A, n \leq 0$ car $\forall n \in \mathbb{Z}$, les $I_n$ sont des idéaux de $A$.\\		
		Ainsi au lieu d'étudier la famille $f = (I_n)_{n \in \mathbb{Z}}$ nous pouvons nous ramener à étudier la famille $f = (I_n)_{n \in \mathbb{N}}$.
		\end{alertblock}
	\end{frame}
	
		\begin{frame}{INTRODUCTION}
		\framesubtitle{CLASSES DES FILTRATIONS}
		\begin{block}{}
		\begin{center}
			\begin{tabular}{|l|c|}
				\hline
				f $I-adique$ &$I_n=I^n,\forall n \in \mathbb{N}^*$\\
				\hline
				f $I-bonne$ &$\exists \, n_0 \in \mathbb{N}$ tel que $II_n = I_{n+1}, \forall n \geqslant n_0.$\\
				\hline
				f $A.P.$ &$\exists \, (k_n)_{n\in \mathbb{N}} $ tel que $\forall$ n,m $\in \mathbb{N}$, $I_{mk_n} \subset I_n^{m}$ et $\underset{n\longrightarrow +\infty }{\lim }\dfrac{k_{n}}{n}=1$\\
				\hline
				f f.$A.P.$ &$\exists k\geqslant 1, \forall \, n \in \mathbb{N}, \ I_{nk} = I_k^n$\\
				\hline
				f noeth. & son anneau de Rees ${R}(A,f)$ est noethérien.\\
				\hline
				f f. noeth. & $\exists k\geqslant 1, \forall \, m, n \in \mathbb{Z}, \ m, n \geqslant k, I_m I_n = I_{m+n}$\\
				\hline
				f E.P & $\exists N\geqslant 1, \forall \, n \geqslant N, \ I_n =\sum\limits_{p=1}^{N} I_{n-p}I_p. $\\
				\hline
			\end{tabular}
		\end{center}
		\end{block}
	\end{frame}
	
	
	
	\begin{frame}{INTRODUCTION}
		\framesubtitle{PROPRIÉTÉ DES FILTRATIONS I-ADIQUES}
		\begin{block}{}
			\begin{center}
				\begin{tikzpicture}
					% Création des nœuds
					\node (A) at (-1,0) {f I-adique};
					\node (B) at (2,0) {f I-bonne};
					\node (C) at (2,-2) {f fortement noethérienne};
					\node (D) at (6,-2) {f noethérienne};
					\node (E) at (9,0) {f A.P};
					\node (F) at (6,0) {f fortement A.P};
					\node (G) at (2,-4) {f E.P};
					
					% Dessin des flèches avec des modifications pour les rendre plus visibles
					\draw[->, ultra thick, >=stealth] (A) -- (B);
					\draw[->, ultra thick, >=stealth] (B) -- (C);
					\draw[->, ultra thick, >=stealth] (B) -- (F);
					\draw[->, ultra thick, >=stealth] (C) -- (D);
					\draw[->, ultra thick, >=stealth] (F) -- (E);
					\draw[<->, ultra thick, >=stealth] (F) -- (D);
					\draw[->, ultra thick, >=stealth] (C) -- (G);
				\end{tikzpicture}
			\end{center}
		\end{block}
	\end{frame}
	
	
	\begin{frame}{INTRODUCTION}
		\framesubtitle{ÉLÉMENT ENTIER ET RÉDUCTION}
		\begin{block}{}
			\begin{enumerate}
				\item[(i)] Un élément $x$ de $A$ est dit entier sur $f$ s'il existe un entier $m \in \mathbb{N}$ tel que : $x^m + a_1 x^{m-1} + \cdots + a_m = x^m + \sum_{i=1}^{m} a_i x^{m-i} = 0,$\\$ m \in \mathbb{N^*} \ \text{où} \ a_i \in I_i,\, \forall i=1, \cdots ,m.$
				\item[(ii)] $f$ est une $\beta$-réduction de $g$ si : \\
				\begin{enumerate}
					\item[a)] $f \leq g$
					\item[b)]  $\exists \, k \geq 1$ tel que $J_{n+k} = I_n J_k , \forall n \geq k$.
				\end{enumerate}
			\end{enumerate}
		\end{block}
	\end{frame}
	
	\begin{frame}{INTRODUCTION}
		\framesubtitle{FILTRATIONS f-BONNES}
		\begin{block}{}
			Soient $\varphi=(M_n)_{n \in \mathbb{Z}}$ $\in$ $\mathbb{F}(M)$, $f-compatible$, avec $f \in \mathbb{F}(A)$.
			\begin{enumerate}[(a)]
				\item $\varphi$ est \textbf{$f-$ bonne} s'il existe un entier naturel N $\geqslant 1$ tel que:
				\[\forall n > N, M_{n}=\sum_{p=1}^{N}I_{n-p}M_{p} \]
				\item Une filtration $f=(I_n)_{n \in \mathbb{Z}}$ est dite $I-bonne$ si: 
				\begin{enumerate}[(i)]
					\item $\forall n \in \mathbb{N}, \quad II_n \subseteq I_{n+1}$;
					\item $\exists k \in \mathbb{N}$, $II_n = I_{n+1}, n\geqslant k$.
				\end{enumerate}
			\end{enumerate}
		\end{block}
	\end{frame}
	
	\begin{frame}{INTRODUCTION}
		\framesubtitle{PROBLÉMATIQUE ET ANNONCE DU PLAN}
		\begin{block}{}
			\begin{enumerate}
				\item[(i)] Comment étendre les résultats des filtrations I-adiques aux filtrations bonnes?
				\item[(ii)] Comment la dépendance intégrale et la réduction interagissent-elles avec les filtrations bonnes ?
			\end{enumerate}
		\end{block}
	\end{frame}
	
	\begin{frame}
		\begin{enumerate}
			\item<0> \textcolor{blue}{INTRODUCTION}\\
			\item<1> \textcolor{blue}{DÉPENDANCE INTÉGRALE, RÉDUCTION ET FILTRATIONS BONNES }\\
			\item<0> \textcolor{blue}{CONCLUSION}\\
		\end{enumerate}
	\end{frame}
	
	\begin{frame}{DÉPENDANCE INTÉGRALE, RÉDUCTION ET FILTRATION BONNE}
		\framesubtitle{ÉNONCE}
		\begin{block}{Théorème Principal}
			Soient $A$ noethérien, $f=(I_{n})_{_{n\in \mathbb{N}}}\leq $ $g=(J_{n})_{_{n\in \mathbb{N}}}$ $ \in \mathbb{F}(A).$ \\ Si $f$ est fortement noethérienne et $g$ est noethérienne alors les assertions sont équivalentes et dans ce cas $g$ est fortement noethérienne:
			\begin{enumerate}[(i)]
				\item $f$ est une réduction de $g.$
				\item $I_{n}$ est une réduction de $J_{n}$ pour tout $n$ assez grand.
				\item Il existe un entier $k\geq 1$ tel que $g^{(k)}$ est $I_{k}-bonne$
			\end{enumerate}
		\end{block}
	\end{frame}
	
	\begin{frame}{DÉPENDANCE INTÉGRALE, RÉDUCTION ET FILTRATION BONNE}
		\framesubtitle{ÉNONCE}
		\begin{block}{Théorème Principal}
			\begin{enumerate}[(iv)]
				\item $g$ est $enti\grave{e}re$ sur $f.$
				\item $g$ est $\ fortement$ $enti\grave{e}re$ sur $f.$
				\item $g$ est $f-fine.$
				\item $g$ est $f-bonne.$
				\item $g$ est $faiblement$ $f-bonne.$
				\item $P(f)=P(g)$
			\end{enumerate}
		\end{block}
	\end{frame}
	
	\begin{frame}
		\begin{enumerate}
			\item<0> \textcolor{blue}{INTRODUCTION}\\
			\item<0> \textcolor{blue}{DÉPENDANCE INTÉGRALE, RÉDUCTION ET FILTRATIONS BONNES }\\
			\item<1> \textcolor{blue}{CONCLUSION}\\
		\end{enumerate}
	\end{frame}
	
	\begin{frame}{CONCLUSION}
		\framesubtitle{BILAN ET PERSPECTIVES}
		\begin{block}{}	
			\begin{enumerate}
				\item Propriétés des $f_I$ et réduction minimale des filtrations bonnes
				\item Étendre ces résultats aux autres classes de filtration.
			\end{enumerate}
		\end{block}
	\end{frame}
	
	\begin{frame}
		\bigskip
		{\textcolor{blue}{\Large \textit{\textbf{\begin{center}
							MERCI POUR VOTRE AIMABLE ATTENTION
		\end{center}}}}}
	\end{frame}
\end{document}