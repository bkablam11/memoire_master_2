% packages to load
% customize layout
% sets bookmark
% for inline and display quotations
% to customize citations
% sets the bibliography style / agsm: Harvard style output
% to import external graphics
% to customize captions
% to customize the header and footer
% to build tables (refer to the previous video):
% to customize lenght titles tables
% to customize table with multicolumn
% to produce long table	
% functionality of the threeparttable package to tables created using the longtable package
% to use toprule, bottomrule ... in tables
% packages for math
% to use \eqref to refer equations
% package for programming code	
%% Verbatim
%% listings
%%% define color
%\input{tcilatex}
%\input{tcilatex}
%\input{tcilatex}
%\input{tcilatex}
%\input{tcilatex}
%\input{tcilatex}
%\input{tcilatex}
%\input{tcilatex}
%\input{tcilatex}
%\input{tcilatex}
%\input{tcilatex}
%\input{tcilatex}


\documentclass[11pt, twoside]{article}
%%%%%%%%%%%%%%%%%%%%%%%%%%%%%%%%%%%%%%%%%%%%%%%%%%%%%%%%%%%%%%%%%%%%%%%%%%%%%%%%%%%%%%%%%%%%%%%%%%%%%%%%%%%%%%%%%%%%%%%%%%%%%%%%%%%%%%%%%%%%%%%%%%%%%%%%%%%%%%%%%%%%%%%%%%%%%%%%%%%%%%%%%%%%%%%%%%%%%%%%%%%%%%%%%%%%%%%%%%%%%%%%%%%%%%%%%%%%%%%%%%%%%%%%%%%%
\usepackage{amsfonts}
\usepackage[a4paper, width=150mm, top=25mm, bottom=25mm,headheight=15pt,bindingoffset=6mm]{geometry}
\usepackage[open, openlevel=1]{bookmark}
\usepackage{csquotes}
\usepackage{natbib}
\usepackage{graphicx}
\usepackage{caption}
\usepackage{fancyhdr}
\usepackage{array}
\usepackage{multirow}
\usepackage{longtable}
\usepackage{threeparttablex}
\usepackage{booktabs}
\usepackage{amsmath}
\usepackage{verbatim}
\usepackage{fancyvrb}
\usepackage{listings}
\usepackage{xcolor}
\usepackage{tikz}

\setcounter{MaxMatrixCols}{10}
%TCIDATA{OutputFilter=Latex.dll}
%TCIDATA{Version=5.50.0.2890}
%TCIDATA{<META NAME="SaveForMode" CONTENT="1">}
%TCIDATA{BibliographyScheme=Manual}
%TCIDATA{LastRevised=Monday, June 24, 2024 11:35:01}
%TCIDATA{<META NAME="GraphicsSave" CONTENT="32">}

\bibliographystyle{agsm} 
\definecolor{codegreen} {rgb} {0,0.6,0}
\definecolor{codegray} {rgb} {0.5,0.5,0.5}
\definecolor{codepurple} {rgb} {0.58,0,0.82}
\definecolor {backcolour} {rgb} {0.95,0.95,0.92}
\input{tcilatex}
\begin{document}


UFR SFA

SFA 2

Examen l\`{e}re session de M\'{e}thodolgie

\bigskip 

Dur\'{e}e 2 H 

\bigskip 

2023-2024

Exercice 1

a) Montrer que $\surd 5$ et $\surd 10$  sont des nombres irrationnels.

b) Montrer que $\frac{\ln 3}{\ln 2}$ est un nombre irrationnel.

Exercice 2 (3 pts)

Les fonctions suivantes sont-elles injectives? surjectives? bijectives?
Justifier.

$f_{1}:Z\longrightarrow Z,$ $n\longmapsto 3n,$ $f_{2}:R\longrightarrow R,$ $%
x\longmapsto x^{3}$

$f_{3}:R\longrightarrow R_{+},$ $x\longmapsto x^{2},$ $f_{4}:C%
\longrightarrow C,$ $z\longmapsto z^{3}.$

Exercice 3

Soit $(u_{n})$ une suite num\'{e}rique telle que:

\bigskip $\qquad \qquad \qquad \qquad \left\{ 
\begin{array}{c}
u_{2}-u_{1}=2 \\ 
u_{n+1}=4u_{n}-3u_{n-1}%
\end{array}%
\right. $

I) Soit $(w_{n})_{n\in N}$ la suite num\'{e}rique telle que $%
w_{n}=u_{n+1}-u_{n}$.

a) Montrer que $(w_{n})_{n\in N}$  est g\'{e}om\'{e}trique et d\'{e}terminer
sa raison et son terme g\'{e}n\'{e}ral.

b) Montrer que $(u_{2n+1})_{n\in N}$ est croissante.

c) Montrer que $(u_{2n})_{n\in N}$ est croissante.

\bigskip 

II) On suppose que $u_{1}=1$.

D\'{e}terminer alors le terme g\'{e}n\'{e}ral de $(u_{n})$ et v\'{e}rifier
pour $(w_{n}).$

\end{document}
