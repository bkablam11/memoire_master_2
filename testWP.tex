% packages to load
% customize layout
% sets bookmark
% for inline and display quotations
% to customize citations
% sets the bibliography style / agsm: Harvard style output
% to import external graphics
% to customize captions
% to customize the header and footer
% to build tables (refer to the previous video):
% to customize lenght titles tables
% to customize table with multicolumn
% to produce long table	
% functionality of the threeparttable package to tables created using the longtable package
% to use toprule, bottomrule ... in tables
% packages for math
% to use \eqref to refer equations
% package for programming code	
%% Verbatim
%% listings
%%% define color
%\input{tcilatex}
%\input{tcilatex}


\documentclass[11pt, twoside]{article}
%%%%%%%%%%%%%%%%%%%%%%%%%%%%%%%%%%%%%%%%%%%%%%%%%%%%%%%%%%%%%%%%%%%%%%%%%%%%%%%%%%%%%%%%%%%%%%%%%%%%%%%%%%%%%%%%%%%%%%%%%%%%%%%%%%%%%%%%%%%%%%%%%%%%%%%%%%%%%%%%%%%%%%%%%%%%%%%%%%%%%%%%%%%%%%%%%%%%%%%%%%%%%%%%%%%%%%%%%%%%%%%%%%%%%%%%%%%%%%%%%%%%%%%%%%%%
\usepackage{amsfonts}
\usepackage[a4paper, width=150mm, top=25mm, bottom=25mm,headheight=15pt,bindingoffset=6mm]{geometry}
\usepackage[open, openlevel=1]{bookmark}
\usepackage{csquotes}
\usepackage{natbib}
\usepackage{graphicx}
\usepackage{caption}
\usepackage{fancyhdr}
\usepackage{array}
\usepackage{multirow}
\usepackage{longtable}
\usepackage{threeparttablex}
\usepackage{booktabs}
\usepackage{amsmath}
\usepackage{verbatim}
\usepackage{fancyvrb}
\usepackage{listings}
\usepackage{xcolor}
\usepackage{tikz}

\setcounter{MaxMatrixCols}{10}
%TCIDATA{OutputFilter=Latex.dll}
%TCIDATA{Version=5.50.0.2890}
%TCIDATA{<META NAME="SaveForMode" CONTENT="1">}
%TCIDATA{BibliographyScheme=Manual}
%TCIDATA{LastRevised=Friday, May 17, 2024 13:12:44}
%TCIDATA{<META NAME="GraphicsSave" CONTENT="32">}

\bibliographystyle{agsm} 
\definecolor{codegreen} {rgb} {0,0.6,0}
\definecolor{codegray} {rgb} {0.5,0.5,0.5}
\definecolor{codepurple} {rgb} {0.58,0,0.82}
\definecolor {backcolour} {rgb} {0.95,0.95,0.92}

%\input{tcilatex}

\begin{document}




1.1.4 Réduction minimale. Réduction basique.

La notion d'idéal basique à été introduite et étudiée par Northcott et Rees: 

Un idéal I de l'anneau local noethérien $(A,m)$ est basique si la
seule réduction de I est $I$ lui-m\^{e}me. 

Northcott et Rees ont aussi défini la notion de réduction minimale
d'un idéal $J$:

Un idéal $I$ est une réduction minimale de $J$ si $I$ est une réduction de $J$ et si I est $minimal$ au sens de l'inclusion parmi l'ensemble
des réductions de J. 

Il a été montré dans [4] que tout idéal $J$ de $(A,m)$ admet
une réduction minimale.

Ces notions s'étendent sans difficulté aux filtrations noethériennes d'un anneau noethérien $A$, non nécessairement local [2]. 

Les questions naturelles qui se posent sont alors les suivantes :

(a) Quelles sont les filtrations basiques de $A$ si elles existent ?

(b) Toute filtration de $A$ admet-elle une réduction minimale?

Nous avons obtenu une caractérisation des filtrations basiques et avons
montré que la réponse à la question (b) est négative.

Pour (a) nous avons obtenu le résultat suivant [2]:

Une filtration noethérienne $f $ de l'anneau noethérien A est
basique si et seulement si $f$ est $I-adique$ avec $I$ idempotent ($I^{2}=I$).

Pour le voir, considérons une filtration $f=(I_{n})$ noethérienne et
basique. 

Il existe alors un entier $k\geq 1$ tel que pour tout entier $n\geq k,$ on
ait $I_{n+k}=I_{k}I_{n}.$ Considérons la filtration $h=(H_{n})$ de $A$ définie par:

$H_{n}=\left\{ 
\begin{array}{c}
I_{k-1}\text{ si }1\leq n\leq k-1 \\ 
I_{n}\text{ pour }n\geq k%
\end{array}%
\right. $ 

On a alors pour tout $n\geq k$, $I_{n+k}=I_{k}H_{n},$ et par conséquent $h$ est une réduction de $f.$

Comme $f$ est basique, on en déduit que $I_{n}=I_{k}$ pour tout $n.$

D'autre part, $I_{2k}=I_{k}^{2}$ donc $f=f_{I}$ avec $I=I_{k}$ et $I^{2}=I.$

Réciproquement, si $f$ est $I-adique$ avec $I=I^{2}$ et si $h=(H_{n})$
est une réduction de $f,$ on a pour tout $n\geq 1,$ $I=I^{n}$ et $%
H_{n}\subseteq I.$

De plus $h$ étant une réduction de $f,$ il existe un entier $k\geq 1$
tel que $I=I^{k+n}=I^{k}H_{n}\subseteq H_{n}\subseteq I$ pour tout $n\geq k.$

D'où $\forall n\geq k$ ,$I=H_{n}.$ Or pour tout entier $n$ tel que $1\leq n\leq k-1,$ on a $H_{k}=I\subseteq H_{n}\subseteq I.$

D'où $H_{n}=I,$ $f_{I}=h$ et $f_{I}$ est basique.

Concernant la question (b), nous avons obtenu le résultat suivant [2]:

Une filtration noethérienne $f=(I_{n})$ de l'anneau noethérien $A$
admet une réduction minimale si et seulement s'il existe un entier $r\geq 1$ tel que $I_{r}$ soit un idéal idempotent.

Il est facile de voir que la condition est nécessaire. Supposons en
effet que $h=(H_{n})$ soit une réduction minimale de $f$. Alors $f $
est noethérienne et basique et d'après le résultat précédent énonce plus haut, il existe un idéal idempotent $I$ tel que $h=f_{I}$. 

Or l'une des caractérisations de réduction d'une filtration sur une
autre assure l'existence d'un entier $N\geq 1$ tel que $I_{n}^{2}=I_{n}H_{n}=I_{n}I$ pour tout $n\geq N$. 

De plus, $f$ étant noethérienne, elle est fortement $A.P.$ et il
existe un entier $k\geq N$ tel que $I_{nk}=I_{k}^{n}$ pour tout entier $n$. 

Ainsi $I_{2k}=I_{k}^{2}=I_{k}I$ et \ $I_{2k}^{2}=I_{2k}I=I_{k}I^{2}=I_{k}I=I_{2k}$ donc $I_{2k}$ est idempotent.

La réciproque est plus technique et nous renvoyons aux références citées plus haut. La preuve de la réciproque demande
l'utilisation de critères assurant que l'anneau de Rees d'une filtration 
$g$ soit entier (respectivement soit une algèbre de type fini) sur l'anneau de
Rees d'une filtration $f.$ Or, suivant les références, les auteurs
ont utilisé soit l'anneau de Rees classique soit l'anneau de Rees généralisé associé à une filtration pour définir et caractériser la dépendance intégrale ou la dépendance intégrale forte d'une filtration sur une autre. Pour montrer que les propriétés utilisées, portant tant\^{o}t sur l'un ou l'autre des anneaux de
Rees sont équivalentes, nous avons montré les résultats suivants
[2] :

Étant données deux filtrations $f,g$ de A vérifiant $f\leq g$, on a:

(i) $R(A,g)$ est entier sur $R(A,f)\Longleftrightarrow $ R(A,g) est entier
sur R(A,f)

(ii) $R(A,g)$ est entier sur $R(A,f)-$ algèbre de type fini $\Longleftrightarrow $ R(A,g) est entier sur R(A,f)-algèbre de type fini

Concernant la noethérianité des anneaux de Rees, une preuve
astucieuse de l'équivalence entre la noethérianité de l'anneau $R(A,f)$ et celle de l'anneau $R(A,f)$ pour toute filtration $f$ de l'anneau
noethérien $A$ a été donnée par K.Kurano. Nous l'exposons brièvement ici. Il est clair que si l'anneau de Rees de $f$ est noethérien, il en est de m\^{e}me de son anneau de Rees généralisé. La réciproque est donc triviale. Par conséquent, l'anneau de Rees $R(A,f)$ de $f$ est noethérien.

\end{document}
