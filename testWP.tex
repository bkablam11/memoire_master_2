% packages to load
% customize layout
% sets bookmark
% for inline and display quotations
% to customize citations
% sets the bibliography style / agsm: Harvard style output
% to import external graphics
% to customize captions
% to customize the header and footer
% to build tables (refer to the previous video):
% to customize lenght titles tables
% to customize table with multicolumn
% to produce long table	
% functionality of the threeparttable package to tables created using the longtable package
% to use toprule, bottomrule ... in tables
% packages for math
% to use \eqref to refer equations
% package for programming code	
%% Verbatim
%% listings
%%% define color
%\input{tcilatex}
%\input{tcilatex}


\documentclass[11pt, twoside]{article}
\usepackage{eurosym}
%%%%%%%%%%%%%%%%%%%%%%%%%%%%%%%%%%%%%%%%%%%%%%%%%%%%%%%%%%%%%%%%%%%%%%%%%%%%%%%%%%%%%%%%%%%%%%%%%%%%%%%%%%%%%%%%%%%%%%%%%%%%%%%%%%%%%%%%%%%%%%%%%%%%%%%%%%%%%%%%%%%%%%%%%%%%%%%%%%%%%%%%%%%%%%%%%%%%%%%%%%%%%%%%%%%%%%%%%%%%%%%%%%%%%%
\usepackage{amsfonts}
\usepackage[a4paper, width=150mm, top=25mm, bottom=25mm,headheight=15pt,bindingoffset=6mm]{geometry}
\usepackage[open, openlevel=1]{bookmark}
\usepackage{csquotes}
\usepackage{natbib}
\usepackage{graphicx}
\usepackage{caption}
\usepackage{fancyhdr}
\usepackage{array}
\usepackage{multirow}
\usepackage{longtable}
\usepackage{threeparttablex}
\usepackage{booktabs}
\usepackage{amsmath}
\usepackage{verbatim}
\usepackage{fancyvrb}
\usepackage{listings}
\usepackage{xcolor}
\usepackage{tikz}

\setcounter{MaxMatrixCols}{10}
%TCIDATA{OutputFilter=Latex.dll}
%TCIDATA{Version=5.50.0.2890}
%TCIDATA{<META NAME="SaveForMode" CONTENT="1">}
%TCIDATA{BibliographyScheme=Manual}
%TCIDATA{LastRevised=Thursday, May 16, 2024 20:45:42}
%TCIDATA{<META NAME="GraphicsSave" CONTENT="32">}

\bibliographystyle{agsm} 
\definecolor{codegreen} {rgb} {0,0.6,0}
\definecolor{codegray} {rgb} {0.5,0.5,0.5}
\definecolor{codepurple} {rgb} {0.58,0,0.82}
\definecolor {backcolour} {rgb} {0.95,0.95,0.92}

\input{tcilatex}

\begin{document}


Cependant, le fait que 9 soit fortement enti\`{e}re sur \U{192} n'implique
pas n\'{e}cessairement que \U{192} soit une r\'{e}duction de g, m\^{e}me si 
\U{192} et g sont noeth\'{e}riennes. Or peut le voir sur l'exemple suivant :
Soit A= k[X] l'anneau des polynomes \`{a} une ind\'{e}termin\'{e}e sur le
corps k. Soit I = XA. On consid\`{e}re les filtrations f = (In) et g = (Jn) d%
\'{e}finies par :

In

J1 =

Isi n est pair

Isi n est impair

I si n est pair

Isi n est impair

On v\'{e}rifie que g est noeth\'{e}rienne et que f\U{2264}g. De plus, la
filtration g est enti\`{e}re sur f. En effet pour tout \'{e}l\'{e}ment b E
Jn, b2 \euro\ J2n = 12n et on a (bY")2=62y2n R(f,A). L'anneau R(g, A) est
Jonc entier sur R(f.A). De plus, comme g est noeth\'{e}rienne, il r\'{e}%
sulte de [3] que g est fortement

13

1.1.4 R

La not

neau

\`{a} ce

minin

est

3iem

ana (N-

Ce

n\v{e}

(a

N

\bigskip 

Rees filtration et une

R(f.A).

= J,

ndant infini

est nce

Ce

a

1

enti\`{e}re sur / et toujours comme cons\'{e}quence de [3] que / est noeth%
\'{e}rienne. N\'{e}anmoins, f n'est pas une r\'{e}duction de 9 puisqu'on n'a
pas Jap+1= 12p+1/2p+1, pour p suffisamment grand, condition n\'{e}cessaire
pour qu'une filtration / soit une r\'{e}duction de g quand l'anneau A est
noeth\'{e}rjen. On montre dans [5] que lorsque l'anneau A est principal avec
get g noeth\'{e}rienne, est une r\'{e}duction de g si et seulement si 3m 
\U{2265} 1 tel que tmftm9.

Lorsque fest une filtration fortement noeth\'{e}rienne et g est une
filtration noeth\'{e}rienne de l'anneau noeth\'{e}rien A v\'{e}rifiant f =
(In) \U{2264} 9 = (J), on montre que les assertions suivantes sont \'{e}%
quiva- lentes [5]:

(i) fest une r\'{e}duction de g

(iii) L'id\'{e}al In est une r\'{e}duction de l'id\'{e}al Jn pour tout n 
\TEXTsymbol{>}\TEXTsymbol{>} 0

(ii) J = In Jn, Vn \TEXTsymbol{>}\TEXTsymbol{>} 0

(iv) Il existe un entier k\U{2265} 1 tel que g(*) soit I-bonne (v) Vm 
\TEXTsymbol{>} 1, f(m) est une r\'{e}duction de g(m)

(vi) 3m \TEXTsymbol{>} 1 tel que f(m) soit une r\'{e}duction de g(m) (vii) g
est enti\`{e}re sur f

(viii) g est fortement enti\`{e}re sur f

(ix) g est f-fine

(xi) g est faiblement f-bonne

(x) 9 est f-bonne

(xii) 3m

1 tel que tmf f\U{2264}g

45

(iii) f* = 9*

En particulier, il r\'{e}sulte des \'{e}quivalences ci-dessus que si \U{192}
est une filtration I-adique de l'anneau noeth\'{e}rien A et si g est une
filtration noeth\'{e}rienne domin\'{e}e par g, les notions suivantes sont 
\'{e}quiv- alentes :

(1) fr est une r\'{e}duction de g

(2) g est enti\`{e}re sur fr.

(3) g est fortement enti\`{e}re sur fr

(4) g est I-bonne

1.1.4 R\'{e}duction minimale. R\'{e}duction basique.

La notion d'id\'{e}al basique \`{a} \'{e}t\'{e} introduite et \'{e}tudi\'{e}%
e par Northcott et Rees: un id\'{e}al I de l'an- neau local noeth\'{e}rien
(A,M) est basique si la seule r\'{e}duction de I est. I lui-m\^{e}me. On
peut voir \`{a} ce propos le travail de Hays (H 12]. Northcott et Rees ont
aussi d\'{e}fini la notion de r\'{e}duction

une r\'{e}duction de J et si I

1 et ce de tion effet Cout

La

est

(xii) m 2 1 tel que tmfS/Sg (iii) f = g

En particulier, il r\'{e}sulte des \'{e}quivalences ci-dessus que si est une
filtration I-adique de l'anneau noeth\'{e}rien A et si g est une filtration
noeth\'{e}rienne domin\'{e}e par g, les notions suivantes sont \'{e}quiv-
alentes

(1) fr est une r\'{e}duction de g

(2) g est enti\`{e}re sur f1.

(3) g est fortement enti\`{e}re sur fr (4) g est I-bonne

1.1.4 R\'{e}duction minimale. R\'{e}duction basique.

La notion d'id\'{e}al basique a \'{e}t\'{e} introduite et \'{e}tudi\'{e}e
par Northcott et Rees: un id\'{e}al I de l' an- neau local noeth\'{e}rien
(A,M) est basique si la seule r\'{e}duction de I est I lui-m\^{e}me. On peut
voir \`{a} ce propos le travail de Hays [H2]. Northcott et Rees ont aussi,d%
\'{e}fini la notion de r\'{e}duction minimale d'un id\'{e}al J: un id\'{e}al
I est une r\'{e}duction minimale de J si I est une r\'{e}duction de J et si
l est minimal au sens de l'inclusion parmi l'ensemble des r\'{e}ductions de
J. Nous reviendrons dans la 3ieme partie de cet expos\'{e} sur cette derni%
\`{e}re notion qui est \'{e}troitement li\'{e}e \`{a} la notion d'\'{e}l\'{e}%
ments analytiquement ind\'{e}pendants, donc li\'{e}e \`{a} la largeur
analytique d'un id\'{e}al. Il a \'{e}t\'{e} montr\'{e} dans [N-R] que tout id%
\'{e}al J de (A,M) admet une r\'{e}duction minimale.

Ces notions s'\'{e}tendent sans difficult\'{e} aux filtrations noeth\'{e}%
riennes d'un anneau noeth\'{e}rien A, non n\'{e}cessairement local [5]. Les
questions naturelles qui se posent sont alors les suivantes :

(a) Quelles sont les filtrations basiques de A si elles existent ?

(b) Toute filtration de A admet-elle une r\'{e}duction minimale?

Nous avons obtenu une caract\'{e}risation des filtrations basiques et avons
montr\'{e} que la r\'{e}ponse \`{a} la question (b) est r\'{e}gative

Pour (a) nous avons obtenu le r\'{e}sultat suivant [5]:

Une filtration noeth\'{e}rienne \U{192} de l'anneau noeth\'{e}rien A est
basique si et seulement si \U{192} est I-adique avec I idempotent.

Pour le voir, consid\'{e}rons une filtration f = (In) noeth\'{e}rienne et
basique. Il existe alors un entier

14

\bigskip 

en d\'{e}duit que II pour tout n.

de A d\'{e}finie par H = I si 1 \U{2264} n \U{2264}k-1 et H= In pour tout
nk. On a alors p k\TEXTsymbol{>} 1 tel que pour tout entier n \TEXTsymbol{>}%
k, on ait In+k = [x!n. Consid\'{e}rons la filtration h tout nk, Ink = Ik Hn,
et par cons\'{e}quent h est une r\'{e}duction de f. Comme f est basique, R%
\'{e}ciproquement, si f cst I-adique avec I 12 et si h (Hn) est une r\'{e}%
duction de f, on a p D'autre part 12k 1 donc f fr avec I = Ik et 121 que 1 =
1k+n IkH, CH, CI pour tout n\TEXTsymbol{>} k. D'o\`{u} Vnk, I H. Or pour
tout entier n t tout n 1, II" et H, CI. De plus, h \'{e}tant une r\'{e}%
duction de f, il existe un entier k

que 1 \U{2264} n \U{2264}k-1, on a HICH, CI d'o\`{u} Hn = 1. fi = h et fi
est basique. Concernant la question (b), nous avons obtenti le r\'{e}sultat
suivant [5]:

seulement si il existe un entier 1 tel que 1, soit un id\'{e}al idempotent.

Une question naturelle

de A non n\'{e}cessairem sir est entier sur / et des \'{e}l\'{e}ments a par r%
\'{e}currence sur avec din E Inti n+

vy(ain) filtration quelo Soit A l'ar

de A d\'{e}finie

Une filtration noeth\'{e}rienne f= (In) de l'anneau noeth\'{e}rien A admet
une r\'{e}duction minimale si e Il est facile de voir que la condition est n%
\'{e}cessaire. Supposons en effet que h= (Hn) soit une re duction minimale
de f. Alors \U{192} est noeth\'{e}rienne et basique et d'apr\`{e}s le r\'{e}%
sultat pr\'{e}c\'{e}dent \'{e}nonce plus haut, il existe un id\'{e}al
idempotent I tel que h = fi. Or l'une des caract\'{e}risations de r\'{e}%
duction d'une filtration sur une autre assure l'existence d'un entier N%
\U{2265} 1 tel que I2 = InHn = In! pout tout n \TEXTsymbol{>} N. De plus, f 
\'{e}tant noeth\'{e}rienne, elle est fortement AP et il existe un entier k 
\TEXTsymbol{>} N tel que Ink = In pour tout entier n. Ainsi 2x = 12 = II et
12 = 12k I = II2 = II = 12k, donc l est idempotent.

Pour toute

tels que u que pour

a pour t

la pseu

r\'{e}ciproque est et nous renvoyons aus r\'{e}f\'{e}rences cit\'{e}es plus
haut. La de la r\'{e}ciproque demande l'utilisation de crit\`{e}res assurant
que l'anneau de Rees d'une filtration; soit entier (resp. soit une alg\`{e}%
bre finie ou de type fini) sur l'anneau de Rees d'une filtration f Or,
suivant les r\'{e}f\'{e}rences, les auteurs ont utilis\'{e} soit l'anneau de
Rees classique soit l'anneau de Rees g\'{e}n\'{e}ralis\'{e} associ\'{e} \`{a}
une filtration pour d\'{e}finir et caract\'{e}riser la d\'{e}pendance int%
\'{e}grale ou la d\'{e}pendance int\'{e}grale forte d'une filtration sur une
autre. Pour montrer que les propri\'{e}t\'{e}s utilis\'{e}es, portant tant%
\^{o}t sur l'un ou l'autre des anneaux de Rees sont \'{e}quivalentes, nous
avons montr\'{e} les r\'{e}sultats suivants [5] :

\'{E}tant donn\'{e}es deux filtrations f, g de A v\'{e}rifiant fg, on a

(i) R(g, A) est entier sur R(f, A) R(g,A) est entier sur R(f,A) (ii) R(g,.1)
est une R(f, A)-alg\`{e}bre de type fini

La

sav

2

des talents 2007

44

Accider

acolaires

D

e-Vallesse

S 30

pour tout entier n.

est idempotent.

La r\'{e}ciproque est plur technique et nous renvoyons aus r\'{e}f\'{e}re

tees a'une trats.

Rees g\'{e}n\'{e}ralis\'{e} associ\'{e} \`{a} une filtration pour d\'{e}%
finir et caract\'{e}riser la d\'{e}pendance int\'{e}grale os) soit entier
(resp. soit une algebre finie ou de type fini) sur l'anneau de Rees d'une
filtration de la r\'{e}ciproque demande l'utilisation de crit\`{e}res
assurant que l'anneau de Or, suivant les r\'{e}f\`{e}rences, les auteurs ont
utilis\'{e} soit l'anneau de Rees classique soit l'anness portant tant\^{o}t
sur l'un ou l'autre des anneaux de Rees sont \'{e}quivalentes, nous avons
montis d\'{e}pendance int\'{e}grale forte d'une filtration sur une autre.
Pour montrer que les propri\'{e}t\'{e}s utilis

r\'{e}sultats suivants [5]:

Etant donn\'{e}es deux filtrations f, g de A v\'{e}rifiant fg, on a

(1) R(g, A) est entier sur R(A) R(g,A) est entier sur R(/,A)

Tanneau noeth\'{e}rien symptotique de /coin /est une filtration noeth 116 Op%
\'{e}rations semi-prem F(A) a \'{e}t\'{e} \'{e}tudi\'{e}e par Petro

(u) R(g..1) est une RJ, A)-alg\`{e}bre de type fini \TEXTsymbol{<}%
\TEXTsymbol{>} R(g,A) est une R(f,A)-alg\`{e}bre de type fin Concernant la
noeth\'{e}rianit\'{e} des anneaux de Rees, une preuve astucieuse de l'\'{e}%
quivalence entre l (iii) R(g, A) est une R(f, A)-algebre finie R(g,A) est
une R(f,A)-alg\`{e}bre finie norta\'{e}rien A a \'{e}t\'{e} donn\'{e}e par
K.Kurano [Kul. Nous l'exposons bri\`{e}vement ici. Il est clair que si
noetherianit\'{e} de l'anneau R(A) et celle de l'anneau R(J.A) pour toute
filtration / de l'annea l'anneau de Rees de f est noeth\'{e}rien, il en est
de m\^{e}me de son anneau de Rees g\'{e}n\'{e}ralis\'{e}. Inverse- ment, si
on suppose l'anneau R, 4) noeth\'{e}rien, et si Y est une ind\'{e}termin\'{e}%
e, on a

CAX, u, Y] avec u

= lim

v(x")

X

R(f, AY

On d\'{e}finit alors une une structure d'anneau gradu\'{e} de type 7 sur

[510

(07-7

Cette op\'{e}ration v\'{e}rifie de

(v)Vghe (A) teller Une op\'{e}ration d\'{e}finie s dite semi-premi\`{e}re. O
est noeth\'{e}rien (3). O //v\'{e}rifie la 10-Ra 1). Or au vu de 10-Ra 11
est un Nous avons mont l'op\'{e}ration f portant sur bonne illustra Prenons
ideal enger

de l'anneau gradu\'{e} A[X, u, Y] et R(f, A) est isomorphe \`{a} la
composante homog\`{e}ne de degr\'{e} z\'{e}ro AX, u, en posant : degr\'{e} X
= 1 et degr\'{e} Y =-1. R(f, A)[Y] est alors un sous anneau gradue de
l'anneau nocth\'{e}rien R(/, A)[Y], Par cons\'{e}quent, l'anneau de Rees
R(f,A) de \U{192} est noeth\'{e}rien. 1.15 Cl\^{o}ture asymptotique d'une
filtration. Soit A un anneau noeth\'{e}rien et soient I J deux id\'{e}aux de
A. On sait associer \`{a} I sa pseudo-valuation homog\`{e}ne u d\'{e}finie
par (2) pour tout A. On sait que I est une r\'{e}duction de J si et
seulement l'id\'{e}al J est entier sur la filtration I-adique fr. Or D.Rees
dans le cas local noeth\'{e}rien puis M. Adam [Mac 1 sans l'hypoth\`{e}se
local ont montr\'{e} que si A est un anneau noeth\'{e}rien, I est une r\'{e}%
duction de J si et seulement si ur(J) \TEXTsymbol{>} 1. Ainsi si A est un
anneau noeth\'{e}rien, si I et. J sont deux id\'{e}aux de A v\'{e}rifiant
ICJ et si fest la filtration I-adique de A, on a :

81700

15

\end{document}
