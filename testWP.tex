% packages to load
% customize layout
% sets bookmark
% for inline and display quotations
% to customize citations
% sets the bibliography style / agsm: Harvard style output
% to import external graphics
% to customize captions
% to customize the header and footer
% to build tables (refer to the previous video):
% to customize lenght titles tables
% to customize table with multicolumn
% to produce long table	
% functionality of the threeparttable package to tables created using the longtable package
% to use toprule, bottomrule ... in tables
% packages for math
% to use \eqref to refer equations
% package for programming code	
%% Verbatim
%% listings
%%% define color
%\input{tcilatex}
%\input{tcilatex}
%\input{tcilatex}
%\input{tcilatex}
%\input{tcilatex}
%\input{tcilatex}


\documentclass[11pt, twoside]{article}
%%%%%%%%%%%%%%%%%%%%%%%%%%%%%%%%%%%%%%%%%%%%%%%%%%%%%%%%%%%%%%%%%%%%%%%%%%%%%%%%%%%%%%%%%%%%%%%%%%%%%%%%%%%%%%%%%%%%%%%%%%%%%%%%%%%%%%%%%%%%%%%%%%%%%%%%%%%%%%%%%%%%%%%%%%%%%%%%%%%%%%%%%%%%%%%%%%%%%%%%%%%%%%%%%%%%%%%%%%%%%%%%%%%%%%%%%%%%%%%%%%%%%%%%%%%%
\usepackage{amsfonts}
\usepackage[a4paper, width=150mm, top=25mm, bottom=25mm,headheight=15pt,bindingoffset=6mm]{geometry}
\usepackage[open, openlevel=1]{bookmark}
\usepackage{csquotes}
\usepackage{natbib}
\usepackage{graphicx}
\usepackage{caption}
\usepackage{fancyhdr}
\usepackage{array}
\usepackage{multirow}
\usepackage{longtable}
\usepackage{threeparttablex}
\usepackage{booktabs}
\usepackage{amsmath}
\usepackage{verbatim}
\usepackage{fancyvrb}
\usepackage{listings}
\usepackage{xcolor}
\usepackage{tikz}

\setcounter{MaxMatrixCols}{10}
%TCIDATA{OutputFilter=Latex.dll}
%TCIDATA{Version=5.50.0.2890}
%TCIDATA{<META NAME="SaveForMode" CONTENT="1">}
%TCIDATA{BibliographyScheme=Manual}
%TCIDATA{LastRevised=Sunday, May 19, 2024 11:41:59}
%TCIDATA{<META NAME="GraphicsSave" CONTENT="32">}

\bibliographystyle{agsm} 
\definecolor{codegreen} {rgb} {0,0.6,0}
\definecolor{codegray} {rgb} {0.5,0.5,0.5}
\definecolor{codepurple} {rgb} {0.58,0,0.82}
\definecolor {backcolour} {rgb} {0.95,0.95,0.92}

\input{tcilatex}

\begin{document}


3.3.

Supposons que $\varphi =(M_{n})$ est une filtration de $M$ qui est $f-fine$,
o\~{A}%
%TCIMACRO{\U{b9} }%
%BeginExpansion
${{}^1}$
%EndExpansion
$f=(I_{n})$ une filtration de $A.$

Alors il existe $N\geq 1$ tel que pour tout $n>N,M_{n}=$ $%
\sum\limits_{p=1}^{N}I_{p}M_{n-p}.$

Comme $n>N,$ posons $n=N+1$, ainsi

$M_{N+1}=$ $\sum\limits_{p=1}^{N}I_{p}M_{N+1-p}=$ $\sum%
\limits_{q=1}^{N}I_{N+1-q}M_{q},$ avec $q=N+1-p.$

Ainsi, il vient de proche en proche que $M_{N+j}=$ $\sum%
\limits_{p=1}^{N}I_{N+j-p}M_{p}\,\ ,$ pour tout $j$ avec $1\leq j\leq m.$

Alors $M_{N+m}=$ $\sum\limits_{p=1}^{N}I_{p}M_{N+m-p}\,=\sum%
\limits_{q=m}^{N+m-1}I_{N+m-q}M_{q}\,=\sum\limits_{q=m}^{N}I_{N+m-q}M_{q}\,+%
\sum\limits_{q=N+1}^{N+m-1}I_{N+m-q}M_{q}=\sum%
\limits_{q=m}^{N}I_{N+m-q}M_{q}\,+\sum\limits_{q=N+1}^{N+m-1}I_{N+m-q}(\sum%
\limits_{p=1}^{N}I_{q-p}M_{p}).$

Or $\sum\limits_{q=m}^{N}I_{N+m-q}M_{q}\,\subseteq
\sum\limits_{p=1}^{N}I_{N+m-p}M_{p}\,$\ et $\sum%
\limits_{q=N+1}^{N+m-1}I_{N+m-q}(\sum\limits_{p=1}^{N}I_{q-p}M_{p})=\sum%
\limits_{p=1}^{N}(\sum\limits_{q=N+1}^{N+m-1}I_{N+m-p})M_{p}=\sum%
\limits_{p=1}^{N}I_{N+m-p}M_{p}\subseteq M_{N+m}$

Par suite $\varphi $ est $f-bonne,$ l'inclusion inverse \~{A}\copyright tant 
\~{A}\copyright vidente.

\bigskip

3.5

$M_{n}=$ $\sum\limits_{p=0}^{N}I_{n-p}M_{p}$ et pour tout $n\geq
1,I_{n}=\sum\limits_{p=1}^{N^{\prime }}I_{n-p}I_{p}.$ Alors pour $%
n>N^{\prime \prime }=N+N^{\prime},$

$M_{n}=$ $\sum\limits_{p=0}^{N}I_{n-p}M_{p}=\sum\limits_{p=0}^{N}(\sum%
\limits_{q=1}^{N^{\prime }}I_{n-p-q}I_{p})M_{p}=\sum\limits_{q=1}^{N^{\prime
}}I_{q}(\sum\limits_{p=0}^{N}I_{n-p-q}M_{p})=\sum\limits_{q=1}^{N^{\prime
}}I_{q}M_{n-q}\subseteq \sum\limits_{q=1}^{N^{^{\prime \prime
}}}I_{q}M_{n-q} $

Donc $M_{n}=\sum\limits_{q=1}^{N^{^{\prime \prime }}}I_{q}M_{n-q}$ ,
l'inclusion inverse \~{A}\copyright tant triviale.

\bigskip

3.7

Il existe un entier $N\geq 1$ tel que pour tout $n>N,J_{n}=\sum%
\limits_{p=1}^{N}I_{n-p}J_{p}\subseteq
\sum\limits_{p=1}^{N}J_{n-p}J_{p}\subseteq J_{n}$

Donc $J_{n}=\sum\limits_{p=1}^{N}J_{n-p}J_{p}$ pour tout $n>N.$

Cette \~{A}\copyright galit\~{A}\copyright\ est valable si $1\leq n\leq N.$

Comme $g$ est $E.P$ et $A$ noeth\~{A}\copyright rien alors $g$ est fortement
enti\~{A}%
%TCIMACRO{\U{a8}}%
%BeginExpansion
\"{}%
%EndExpansion
re sur $f.$

Par suite $g$ est noeth\~{A}\copyright rien et d'apr\~{A}%
%TCIMACRO{\U{a8}}%
%BeginExpansion
\"{}%
%EndExpansion
s le th\~{A}\copyright or\~{A}%
%TCIMACRO{\U{a8}}%
%BeginExpansion
\"{}%
%EndExpansion
me de Eakin $f$ est noeth\~{A}\copyright rien.

3.8

Soient $f=(I_{n})_{_{n\in \mathbb{N}}},g=(J_{n})_{_{n\in\mathbb{N}}}\in F(A).
$

Alors il existe un entier $N\geq 1$ tel que $I_{n}\subseteq J_{n}\subseteq
I_{n-N}\subseteq J_{n-N}$ pour tout $N\geq 1.$

Si $f$ est $A.P.$ alors il existe une suite d'entiers $(k_{n})_{n\in \mathbb{%
N}}$ telle que $\underset{n\longrightarrow \infty }{\lim }\frac{k_{n}}{n}=1$
et $I_{k_{n}m}\subseteq I_{n}^{m}$

pour tout $m,n\in \mathbb{N}.$

Par suite, $J_{(k_{n}+N)m}\subseteq J_{k_{n}m+Nm}\subseteq
J_{k_{n}m+N}\subseteq I_{k_{n}m+N}\subseteq I_{k_{n}m}\subseteq
I_{n}^{m}\subseteq J_{n}^{m}.$

D'o\~{A}%
%TCIMACRO{\U{b9} }%
%BeginExpansion
${{}^1}$
%EndExpansion
$\underset{n\longrightarrow \infty }{\lim }\frac{k_{n}+N}{n}=1,$ $g$ est $%
A.P.$

R\~{A}\copyright ciproquement si $g$ est $A.P.$ alors il existe une suite
d'entiers $(k_{n}^{^{\prime }})_{n\in \mathbb{N}}$ associ\~{A}\copyright e 
\~{A}~ $g.$

Alors $I_{k_{n}^{\prime }+N.m}\subseteq J_{k_{n}^{\prime }+N.m}\subseteq
J_{n+N}^{m}\subseteq I_{n}^{m}$ pour tout $m,n\in \mathbb{N}.$

Et $\underset{n\longrightarrow \infty }{\lim }\frac{k_{n}^{\prime }+N}{n}=1,f
$ est $A.P.$

\bigskip 

3.14

Si $g$ est fortement $A.P.$ alors $f$ est fortement $A.P.$

Alors Supposons que seulement $f$ est fortement $A.P.$ de rang $r.$

Il existe un entier $N\geq 0$ tel que $t_{N}g\leq f\leq g.$

Posons $f=(I_{n}),g=(J_{n}),f_{1}=f^{(rm)},g_{1}=g^{(rm)}.$

Nous avons $J_{N+n}\subseteq I_{n}$ pour tout $n\in \mathbb{N}.$

De même $J_{rm(N+n)}\subseteq J_{N+rmn}$ $\subseteq I_{rmn}$  pour tout $n\in \mathbb{N}.$

D'où $t_{N}g_{1}\leq f_{1}\leq g_{1}.$

Or par hypothèse, $f^{(r)}=f_{I_{r}\text{ }}$alors $f^{(rm)}=f_{I_{rm}}$
qui est noethérienne.

Donc $g_{1}$ est fortement entière sur $f_{1}.$

\end{document}
