\addcontentsline{toc}{chapter}{\large{Conclusion}}
\begin{center}
	\chapter*{Conclusion}
\end{center}

$ \quad $ L'objectif de ce travail était l'étude de la dépendance intégrale et de la réduction par rapport aux idéaux à travers la filtration I-adique qui est une filtration I-bonne. \\
$ \quad $ Ensuite, nous avons montré la dépendance intégrale et la réduction des filtrations bonnes en générale\\

Comme perspectives, nous projetons d'effectuer:\\
\begin{itemize}
	\item Une étude du nombre de réduction sur les filtrations bonnes i.e le nombre de Samuel.
	\item Une étude de la largeur analytique qui représente le lien entre la dépendance intégrale et la réduction des filtrations bonnes.\\
\end{itemize}


