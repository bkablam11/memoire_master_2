\addcontentsline{toc}{chapter}{\large{Conclusion et perspectives}}
\begin{center}
	\chapter*{Conclusion et perspectives}
\end{center}

$ \quad $ L’objectif de ce travail était l’étude de la dépendance intégrale et de la réduction par rapport aux idéaux à travers la filtration I-adique, qui est une filtration I-bonne. \\

Ensuite, nous avons examiné la dépendance intégrale et la réduction dans le contexte des filtrations bonnes en général. Nous pouvons donc retenir que dans un anneau local noethérien, toutes les filtrations bonnes admettent une réduction minimale. De plus, sous certaines conditions vérifiées, nous pouvons établir des propositions équivalentes entre les notions de réduction, de dépendance intégrale et de filtrations bonnes. \\

En tant que perspectives, nous envisageons d’étudier sous quelles hypothèses nous pourrons étendre ces résultats aux autres classes de filtrations, notamment les filtrations noethériennes et les filtrations de modules. De plus, nous nous interrogeons sur la possibilité d’étendre ces résultats à des objets algébriques qui ne sont pas nécessairement décroissants.


