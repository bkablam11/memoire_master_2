\chapter{DÉPENDANCE, RÉDUCTION ET FILTRATIONS BONNES}
\chaptermark{DÉPENDANCE, RÉDUCTION ET FILTRATIONS BONNES}

\section{Dépendance intégrale de filtration}
\begin{madefinition}
	Soient $A$ un anneau commutatif unitaire, $I$ un idéal de $A$\\
	et $f=(I_n)_{n \in N} \in \mathbb{F}(A)$. Un élément $x$ de $A$ est dit entier sur $f$ s'il existe un entier $m \in \mathbb{N}$ tel que : 
	\[ 	x^m + a_1 x^{m-1} + \cdots + a_m = x^m + \sum_{i=1}^{m} a_i x^{m-i} = 0, \; m \in \mathbb{N^*} \ où \ a_i \in I_i,\, \forall i=1, \cdots ,m. \]	
\end{madefinition}
\begin{maproposition}
	Soit $f=(I_n)_{n \in \mathbb{N}} \in \mathbb{F}(A), x \in A $ et $n \in \mathbb{N}$.\\
	Les assertions suivantes sont équivalentes: \\
	\begin{enumerate}
		\item[i)] $x$ est entier sur $f^{(n)}=((I_{nk})_{k \in \mathbb{N}})$.
		\item[ii)] $xX^n(\in A[X])$ est entier sur $R(A,f)$.
		\item[iii)] $xX^n(\in A[X])$ est entier sur $ \mathcal{R}(A,f)$.
	\end{enumerate}
	\begin{proof}
		i)$\Rightarrow$ ii)
		
		Supposons $x$ entier sur $f^{(n)}$.
		
		Alors il existe $r\geq 1,$ $x^{r}+\sum\limits_{i=1}^{r}a_{i}x^{r-i}=0,$ $%
		a_{i}\in I_{n_{i}}$.
		
		Alors il existe $r\geq 1,$ $(xX^{n})^{r}+\sum%
		\limits_{i=1}^{r}a_{i}X^{n_{i}}(xX^{n})^{r-i}=0,$ $a_{i}\in
		X^{n_{i}}I_{n_{i}}\subset R(A,f)$.
		
		Donc $xX^{n}$ est entier sur $R(A,f).$ \\
		
		ii)$\Rightarrow$ iii) Évident car $R(A,f)\subset \mathcal{R}(A,f).$ \\
		iii)$\Rightarrow$ i) Supposons $xX^{n}$ est entier sur $\mathcal{R}(A,f).$
		
		Alors il existe $m\geq 1,$ $(xX^{n})^{m}+\sum\limits_{i=1}^{m}a_{i}(xX^{n})^{m-i}=0,$ $a_{i}\in \mathcal{R}(A,f)$.
		
		D'où $\deg ((xX^{n})^{m})=nm$ alors $\deg (a_{i})=n_{i}$
		
		Ainsi $a_{i}\in I_{n_{i}}X^{n_{i}},$ alors $a_{i}=\alpha _{i}X^{n_{i}}$ ,$\alpha _{i}\in I_{n_{i}}$
		
		Donc il existe $m\geq 1,$ $(xX^{n})^{m}+\sum\limits_{i=1}^{m}(\alpha
		_{i}X^{n_{i}})(xX^{n})^{m-i}=0,$ $\alpha _{i}\in I_{n_{i}}$.
		
		Alors, il existe $m\geq 1,$ $X^{nm}[x^{m}+\sum\limits_{i=1}^{m}\alpha
		_{i}x^{m-i}]=0,$ $\alpha _{i}\in I_{n_{i}}$.
		
		Par identification des polynômes, il vient, il existe $m\geq 1,$ $%
		x^{m}+\sum\limits_{i=1}^{m}\alpha _{i}x^{m-i}=0,\alpha _{i}\in I_{n_{i}}$.
		
		Par suite, $x$ est entier sur $f^{(n)}$.
	\end{proof}
\end{maproposition}
\section{Clôture intégrale d'une filtration}
\begin{maproposition}
	Soit $f=(I_n)_{n \in N} \in \mathbb{F}(A).$ Alors:\\
	$f'=(I'_n)_{n \in N} \in \mathbb{F}(A)$ appelée \textbf{clôture intégrale de f}
\end{maproposition}
\begin{proof}
	Supposons que $f=(I_{n})_{n\in \mathbb{N}}\in F(A).$
	
	i) $I_{0}^{\prime }=\{x\in A,x$ entier sur $I_{0}\}=A$
	
	Car $I_{0}=A$ $.$
	
	ii) Soit $n\in \mathbb{N}.$
	Comme $I_{n+1}\subset I_{n}$ alors par croissance, $I_{n+1}^{\prime }\subset I_{n}^{\prime }.$
	
	iii) Soient $p,q\in \mathbb{N}.$
	$I_{q}^{\prime }I_{p}^{\prime }\subset (I_{q}I_{p})^{\prime }\subset
	(I_{p+q})^{\prime }=I_{p+q}^{\prime }$ car $I^{\prime }J^{\prime }\subset
	(IJ)^{\prime }$ pour tout idéaux de $A.$
	
	Par suite $f^{\prime }=(I_{n}^{\prime })_{n\in \mathbb{N}}\in F(A).$
\end{proof}
\begin{moncorollaire}
	Soit $f=(I_n)_{n \in \mathbb{N}} \in \mathbb{F}(A)$. Alors:\\
	$\forall k \in \mathbb{N}, \text{ on pose: } P_k(f)=\left\{x \in A, x \text{ entier sur } f^{(k)}\right\}$ est un idéal de $A$ et la famille \\ $P(f)=(P_k(f))_{k \in \mathbb{N}}$ est une filtration de $A$ appelé \textbf{clôture prüfériennes} de $f$.
\end{moncorollaire}
\begin{proof}
	Soit $k\in \mathbb{N}.$ \\
	A) Montons que $P_{k}(f)$ est un idéal de $A$.
	
	i) Par définition, $P_{k}(f)\subset A.$
	
	ii) $0_{A}$ est entier sur $A,$ donc $0_{A}\in P_{k}(f)$
	
	iii) Soient $x,y\in P_{k}(f).$
	
	Comme $x,y\in P_{k}(f)$ alors $xX^{k},yX^{k}$ sont entiers sur $R(A,f)$
	
	D'où $xX^{k}+yX^{k}$ est entier sur $R(A,f)$ car $R(A,f)^{\prime }$ est
	un anneau
	
	Donc $x+y$ est entier sur $f^{(k)}$
	
	Par suite $x+y\in P_{k}(f).$
	
	iv) Soit $a\in A,$ soit $x\in P_{k}(f).$
	
	$x\in P_{k}(f)$ alors $xX^{k}$ est entier sur $R(A,f).$
	
	$(ax)X^{k}$ est entier sur $R(A,f)$
	
	Donc $ax$ est entier sur $f^{(k)}$
	
	D'où $ax\in P_{k}(f)$
	
	Par suite $P_{k}(f)$ est un idéal de $A.$ \\
	B) Montrons que $P(f)=(P_{k}(f))_{k\in \mathbb{N}}\in F(A)$
	
	i) $P_{0}(f)=\{x\in A,$ $x$ entier sur $f^{(0)}=(A,...,A)=A\}$
	
	D'où $P_{0}(f)=A$
	
	ii) Soit $x\in P_{k+1}(f).$
	
	Ainsi $x$ est entier sur $f^{(k+1)}.$
	
	Alors il existe $n\geq 1,$ $x^{n}+\sum\limits_{i=1}^{n}a_{i}x^{n-i}=0,$ $%
	a_{i}\in I_{(k+1)i}\subset I_{ki}$.
	
	Donc $x$ est entier sur $f^{(k)}$
	
	Par suite $x\in P_{k}(f).$
	
	D'où $P_{k+1}(f)\subset P_{k}(f).$
	
	iii) Soient $x\in P_{k_{1}}(f)$ et $y\in P_{k_{2}}(f)$.
	
	$xX^{k_{1}}\in R(A,f)^{\prime }$et $yX^{k_{2}}\in R(A,f)^{\prime }.$
	
	D'où $xyX^{k_{1}+k_{2}}\in R(A,f)^{\prime }$ car $R(A,f)^{\prime }$ est
	un anneau.
	
	Donc $xy\in P_{k_{1}+k_{2}}(f)$
	
	Par suite, $P_{k_{1}}(f)$ $P_{k_{2}}(f)\subset P_{k_{1}+k_{2}}(f).$
	
	On en déduit que $P(f)=(P_{k}(f))_{k\in \mathbb{N}}\in F(A).$
\end{proof}
\begin{maremarque}
	La clôture intégrale d'un idéal $I$ de $A$ est : $I'=P_1(f_I)$
\end{maremarque}
\begin{maproposition}
	Soit $f=(I_n)_{n \in \mathbb{N}} \in \mathbb{F}(A)$. Alors:\\
	\begin{enumerate}
		\item[(i)] $ f \leqslant f' \leqslant P(f)$,
		\item[(ii)] $ P(P(f)) = P(f)$,
		\item[(iii)] $P(f) = P(f')$, avec $f'=(I'_n)_{n \in \mathbb{N}}$ est la clôture intégrale de $f$.
	\end{enumerate}
\end{maproposition}
\begin{proof}
	i) Il s'agit de montrer que pour tout $n\in \mathbb{N},$ $I_{n}\subset I_{n}^{\prime }\subset P_{n}(f)$
	
	Soit $n\in \mathbb{N}.$ \\
	a) Soit $x\in I_{n}.$
	
	Posons $r=1.~$Ainsi $x+a_{1}=x+(-x)=0,$ $a_{1}=-x$
	
	Alors $x\in I_{n}^{\prime }.$ D'où $I_{n}\subset I_{n}^{\prime }.$
	
	b) Soit $x\in I_{n}^{\prime }.$
	
	Soit $n\in \mathbb{N}.$
	$x\in I_{n}^{\prime }$ alors il existe $r\geq 1,$ $x^{r}+\sum%
	\limits_{i=1}^{r}a_{i}x^{r-i}=0,\alpha _{i}\in I_{n}^{i}$.
	
	Or $I_{n}^{i}\subset $ $I_{ni}$ , d'où il existe $m=r\geq 1,$ $%
	x^{m}+\sum\limits_{i=1}^{m}\alpha _{i}x^{m-i}=0,\alpha _{i}=a_{i}\in I_{ni}$%
	.
	
	D'où $x\in P_{n}(f)$
	
	Donc $I_{n}^{\prime }\subset P_{n}(f)$
	
	Par conséquent, pour tout $n\in \mathbb{N},$ $I_{n}\subset I_{n}^{\prime }\subset P_{n}(f).$
	
	c'est-\`{a}-dire que $f\leq f^{\prime }\leq P(f)$
	
	ii) Montrons que $P(P(f))=P(f).$
	
	A) D'après i), $f\leq P(f)\Rightarrow P(f)\subset P(P(f))$ d'où $%
	P(f)\subset P(P(f))$
	
	B) Posons $g=P(f)=(J_{n})_{n\in \mathbb{N}}$ avec $J_{n}=P_{n}(f)$
	
	Donc $P(P(f))=P(g)$
	
	Soient $n\in N,x\in P_{n}(g).$
	
	$x\in P_{n}(g)$ alors $x$ est entier sur $g^{(n)}$
	
	$x\in P_{n}(g)$ alors il existe $s\geq 1,$ $x^{s}+\sum\limits_{i=1}^{s}a_{i}x^{s-i}=0,a_{i}\in J_{ni}$.
	
	Or $J_{ni}=P_{ni}(f)$ , ainsi $a_{i}\in P_{ni}(f)$ pour tout $i\in \llbracket 1, s \rrbracket.$
	
	Ainsi les $a_{i}$ sont entiers sur $f^{(ni)}.$ D'où $a_{i}X^{ni}\in R(A,f)^{\prime }$
	
	Alors $(xX^{n})^{s}+\sum\limits_{i=1}^{s}a_{i}X^{ni}(xX^{n})^{s-i}=0,$ avec 
	$a_{i}X^{ni}\in R(A,f)^{\prime }$.
	
	Donc $xX^{n}$ est entier sur $R(A,f)^{\prime }$.
	
	D'où $xX^{n}\in \lbrack R(A,f)^{\prime }]^{\prime }=R(A,f)^{\prime }$
	
	Par suite $xX^{n}$ est entier sur $R(A,f)$.
	
	Donc $x$ est entier sur $f^{(n)}$
	
	Par suite $x\in P_{n}(f).$
	
	D'où $P_{n}(g)\subset P_{n}(f).$
	
	c'est-\`{a}-dire $P(P(f))\subset P_{n}(f).$Donc $P(P(f))=P_{n}(f).$
	
	
	
	iii) Montrons que $P(f)=P(f^{\prime })$
	
	D'après i) on a $f\leq f^{\prime }\Rightarrow P(f)\leq P(f^{\prime })$
	
	Réciproquement, $f^{\prime }\leq P(f)\Rightarrow P(f^{\prime })\leq
	P(P(f))=P(f)$
	
	D'où $P(f)\leq P(f^{\prime })\leq P(f)$
	
	Par suite $P(f)=P(f^{\prime })$
\end{proof}
\begin{madefinition}
	Soit $f=(I_n)_{n \in \mathbb{N}} \in \mathbb{F}(A)$. Alors:\\
	On dit que $f$ est \textbf{intégralement fermée} si $f'=f$; ce qui signifie aussi que $I_n$ est intégralement fermé pour tout $n$.
\end{madefinition}
\begin{maproposition}
	Soit $f=(I_n)_{n \in \mathbb{N}} \in \mathbb{F}(A)$. Alors:\\
	\begin{enumerate}
		\item[(i)] $x \in P_k(f)$ si et seulement si $xX^k$ est entier sur $R(A,f)$,
		\item[(ii)] $ I_k \subseteq P_k(f) \subseteq \sqrt[]{f}$ et en particulier $\ \sqrt[]{(P(f))} = \sqrt[]{f}$.
	\end{enumerate}
\end{maproposition}
\begin{maproposition}
	Soient $A \subset B$ deux anneaux.\\
	$A' =\left\{x \in B, x\text{ entier sur } A\right\}$.
	\begin{enumerate}
		\item[i)] $A \subset A' \subset B $,
		\item[ii)] $A'' = A'$.
	\end{enumerate}
\end{maproposition}
\begin{proof}
	i) Montrons que $A\subset A^{\prime }\subset B$
	
	a) Soit $x\in A.$
	
	On a: $x^{1}+a_{1}=x^{1}+(-x)=0\Rightarrow x\in A^{\prime }.$
	
	Donc $A\subset A^{\prime }.$
	
	b) Soit $x^{\prime }\in A^{\prime }.$ 
	
	Par construction $x\in B.$
	
	D'où $A^{\prime }\subset B.$
	
	Par suite $A \subset A^{\prime }\subset B$
	
	
	
	ii) D'après i) $A\subset A^{\prime }\Rightarrow A^{\prime }\subset
	A^{\prime \prime }$ (Par croissance)
	
	Réciproquement soit $x\in A^{\prime \prime }\Rightarrow x$ entier sur $%
	A^{\prime }$
	
	Ainsi il existe $n\geq 1,$ tel que  $x^{n}+\sum%
	\limits_{i=1}^{n}a_{i}x^{n-i}=0,a_{i}\in A^{\prime }.$
	
	Alors $a_{i}$ entier sur $A,$ pour tout $i\in \llbracket 1, n \rrbracket.$
	
	D'où $A[a_{i}]$ est un $A$-module de type fini, ainsi $%
	A[a_{1},a_{2},...,a_{n}]$ est un A-module de type fini.
	
	En effet, si
	
	$\left\{ 
	\begin{array}{c}
		A[a_{1}]=A(1_{A},a_{1},a_{1}^{2},...a_{1}^{n-1}) \\ 
		A[a_{2}]=A(1_{A},a_{2},a_{2}^{2},...a_{2}^{m-1})%
	\end{array}%
	\right. $
	
	des A-modules de type fini, alors 
	
	$A[a_{1},a_{2}]=A(a_{1}^{i}a_{2}^{j})_{\substack{ 1\leq i\leq n-1 \\ 1\leq
			j\leq m-1}}$
	
	En procédant de proche en proche, il vient $A[a_{1},a_{2},...,a_{n}]$
	est un A-module de type fini.
	
	Par suite, $K=A[a_{1}x^{n-1},a_{2}x^{n-2},...,a_{n}]$ est un A-module de
	type fini.
	
	Soient $(y_{1},...,y_{n})$ les générateurs de $K.$
	
	Donc $K=A(y_{1},...,y_{n},x^{n-1},x^{n-2},...,x^{2},x,1)$
	
	D'où $A[x]\subset H.$
	
	Par suite $x$ est entier sur $A$
	
	Donc $x\in A^{\prime }$
	
	Par suite $A^{\prime \prime }\subset A$
	
	Donc  $A^{\prime \prime }=A$
	
	iii) De la m\^{e}me manière, on montre que $I^{\prime \prime }=I^{\prime
	}.$
\end{proof}
\begin{madefinition}
	Soit $f=(I_n)_{n \in \mathbb{N}} , g = (J_n)_{n \in \mathbb{N}}\in \mathbb{F}(A)$.  Alors:\\
	\begin{enumerate}
		\item[(a)]$g$ est \textbf{entière sur} $f$ si $g \leqslant P(f)$ (où $P(f)$ est la clôture prüférienne de $f$). C'est à dire:
		\[\forall n \geqslant 1, J_n \subseteq P_{n}(f) \]
		\item[(b)]$g$ est \textbf{fortement entière sur} $f$ si $f \leqslant g$ et si $R(A,g)$ est un $R(A,f)-module$ de type fini.
	\end{enumerate}
\end{madefinition}
\section{Réduction d'une filtration}

\subsection{Réduction au sens de Okon-Ratliff}
\begin{madefinition}
	($\alpha$-réduction ou réduction au sens de Okon-Ratliff)\\
	Soient $f=(I_n)_{n \in \mathbb{N}}$, $g=(J_n)_{n \in \mathbb{N}}$ deux filtrations de $A$.\\
	$f$ est une $\alpha$-réduction de $g$ si : \\
	\begin{enumerate}
		\item[i)] $f \leq g$
		\item[ii)] $\exists \, N \geq 1$ tel que $\forall n \geq N ; J_n = \displaystyle \sum_{p=0}^{N}{I_{n-p} J_p}$.
	\end{enumerate}
\end{madefinition}
\begin{maremarque}
	Soit $\Re$ la relation $f \Re g \Leftrightarrow$ $f$ est une $\alpha$-réduction de $g$. Notons que cette relation est une relation d'ordre sur l'ensemble des filtrations.	
\end{maremarque}
\begin{proof}
	(i) \textbf{Réflexivité} \\
	Posons $N=1$. \\
	$\displaystyle \sum_{p=0}^{1}I_{n-p}I_{p}=I_{n}I_{0}+I_{n-1}I_{1}=I_{n-1}I_{1}=I_{n}$ \
	(car $I_{0}\subset I_{1}$).
	
	D'où $f \Re f$ \\
	(ii) \textbf{Transitivité} \\
	Soient $f=(I_{n})_{n\in \mathbb{N}}$ $,g=(J_{n})_{n\in \mathbb{N}}$ $,h=(H_{n})_{n\in \mathbb{N}}$ \ des filtrations de $A.$ \\
	Supposons que $f \Re g$
	alors $f\leq g$ et il existe $N_{1}\geq 1,\forall m\geq N_{1},$ $J_{m}=\displaystyle  \sum_{p=0}^{N_{1}}I_{m-p}J_{p}$.
	Supposons que $g \Re h$ alors $g\leq h$ et il existe $N_{2}\geq 1,\forall k\geq N_{2},$ $H_{k}=\displaystyle  \sum_{p=0}^{N_{2}}J_{k-p}H_{p}$.
	D'où $f\leq g\leq h.$ Donc $f\leq h$. \\
	De plus posons $N=N_{1}+N_{2}$ \\
	$H_{k}=\displaystyle  \sum_{p=0}^{N_{2}}J_{k-p}H_{p}+\displaystyle  \sum_{p=N_{2}+1}^{N}J_{k-p}H_{p}=\displaystyle  \sum_{p=0}^{N_{2}}J_{k-p}H_{p}+\displaystyle  \sum_{p=N_{2}+1}^{N_{1}+N_{2}}J_{k-p}H_{p}$ pour tout $k\geq N_{1}+N_{2}$ \\
	Comme $k\geq N_{1}+N_{2}$ alors $k-N_{2}\geq N_{1}$ et $p\leq N_{2}$ alors $k-p\geq N_{1}$ \\
	D'où $J_{k-p}=\displaystyle  \sum_{i=0}^{N_{1}}I_{k-p-i}J_{i}$ \\
	Ainsi $H_{k}=\displaystyle  \sum_{p=0}^{N_{2}}(\displaystyle  \sum_{i=0}^{N_{1}}I_{k-p-i}J_{i})H_{p}+\displaystyle  \sum_{p=N_{2}+1}^{N_{1}+N_{2}}J_{k-p}H_{p}$ \\
	Donc $H_{k}\subset \displaystyle  \sum_{p=0}^{N_{2}}\displaystyle  \sum_{i=0}^{N_{1}}I_{k-p-i}J_{i}H_{p+i}+\displaystyle  \sum_{p=N_{2}+1}^{N_{1}+N_{2}}J_{k-p}H_{p}$  car $g\leq h$ d'où $ J_{i}\subset H_{i}\subset H_{i}H_{p}\subset H_{i+p}$ \\
	Posons $l=p+i,$ ainsi $0\leq l\leq N_{1}+N_{2}$ car $0\leq p\leq N_{2}$ et $ 0\leq i\leq N_{1}$ \\
	D'où $H_{k}\subset \displaystyle  \sum_{l=0}^{N_{1}+N_{2}}I_{k-l}H_{l}+\displaystyle  \sum_{p=N_{2}+1}^{N_{1}+N_{2}}J_{k-p}H_{p}$ \\
	Posons $K=\displaystyle  \sum_{p=N_{2}+1}^{N_{1}+N_{2}}J_{k-p}H_{p}$ \\
	Comme $p\geq N_{2}$ alors $H_{p}=\displaystyle  \sum_{p=0}^{N_{2}}J_{p-i}H_{i}$ \\
	D'où $K=\displaystyle  \sum_{p=N_{2}+1}^{N_{1}+N_{2}}J_{k-p}(\displaystyle  \sum_{p=0}^{N_{2}}J_{p-i}H_{i})=\displaystyle  \sum_{p=N_{2}+1}^{N_{1}+N_{2}}\displaystyle  \sum_{p=0}^{N_{2}}J_{k-p}J_{p-i}H_{i}$ \\
	Donc $K\subset\displaystyle  \sum_{p=N_{2}+1}^{N_{1}+N_{2}}\displaystyle  \sum_{p=0}^{N_{2}}J_{k-i}H_{i}=\displaystyle  \sum_{p=0}^{N_{2}}J_{k-i}H_{i}$
	$K\subset \displaystyle  \sum_{p=0}^{N_{2}}J_{k-i}H_{i}$
	or $0\leq i\leq N_{2}$ et $N_{1}+N_{2}\leq k$ \\
	d'où $k-i\geq N_{1}$ \\
	Ainsi $J_{k-i}=\displaystyle  \sum_{l=0}^{N_{1}}I_{k-i-l}J_{l}$ \\
	D'où $K\subset\displaystyle  \sum_{p=0}^{N_{2}}(\displaystyle  \sum_{l=0}^{N_{1}}I_{k-i-l}J_{l})H_{i}\subset\displaystyle  \sum_{p=0}^{N_{2}}\displaystyle  \sum_{l=0}^{N_{1}}I_{k-i-l}J_{l}H_{i}\subset $
	Posons $p=i+l$ \\
	D'où $K\subset \displaystyle  \sum_{p=0}^{N_{1}+N_{2}}I_{k-p}H_{p}$
	or $H_{k}\subset
	\displaystyle  \sum_{l=0}^{N_{1}+N_{2}}I_{k-l}H_{l}+\displaystyle  \sum_{p=N_{2}+1}^{N_{1}+N_{2}}J_{k-p}H_{p}=\displaystyle  \sum_{l=0}^{N_{1}+N_{2}}I_{k-l}H_{l}+K\subset \displaystyle  \displaystyle  \sum_{l=0}^{N_{1}+N_{2}}I_{k-l}H_{l}$ car $K\subset \displaystyle  \displaystyle  \sum_{l=0}^{N_{1}+N_{2}}I_{k-l}H_{l}$ \\
	Donc $H_{k}\subset H_{k-p}H_{p}\subset H_{k}$ \\
	Finalement $H_{k}=\displaystyle  \displaystyle  \sum_{l=0}^{N_{1}+N_{2}}I_{k-l}H_{l}$
	Par suite $f \Re h$. \\
	
	(iii) \textbf{Anti-symétrie} \\
	Soient $f=(I_{n})_{n\in \mathbb{N}}$ $,g=(J_{n})_{n\in \mathbb{N} }$\ des filtrations de $A$ telles que: \\
	$f \Re g$ alors $f\leq g$ et $g \Re f$ alors $g\leq f.$ \\
	D'où $f=g$ \\
	On en déduit que $\Re$ est une relation d'ordre. 
\end{proof}
\begin{maproposition}
	Soient $f=(I_n)_{n \in \mathbb{N}}$ et $g=(J_n)_{n \in \mathbb{N}}$ deux filtrations de $A$. $f \geq g$ et chaque $J_n$ est de type fini, alors $f$ est une $\alpha$-réduction de $g$ si et seulement si $R(A,g)$ est un $R(A,f)$-module de type fini.
\end{maproposition}
\begin{proof}
	Supposons que $f$ est une $\alpha$-réduction de $g$.\\
	Alors $f \leq g$ et $\exists \, N \geq 1$ tel que $\forall n \geq N ; J_n = \displaystyle \sum_{p=0}^{N}{I_{n-p} J_p}$.\\
	Prouvons que $R(A,g) = R(A,f)(J_0, J_1X_1, \cdots , J_NX_N) = M$.\\
	$J_nX_n \subseteq R(A,g), \forall n \in \mathbb{N} \Rightarrow R(A,f) \subseteq R(A,g)$ car $f \leq g$,
	Donc $M \subseteq R(A,g)$.\\
	Montrons par récurrence que $\forall n \in \mathbb{N}, J_nX^n \subseteq M$
	Si $n \leq N, J_nX^n \subseteq M$, par construction.\\
	Soit $n \geq N$, supposons la propriété vraie et montrons que $J_{n+1}X^{n+1} \subseteq M$.\\
	$J_{n+1} = \displaystyle \sum_{p=0}^{N}{I_{n+1-p} J_p} \Rightarrow J_{n+1}X^{n+1} = \displaystyle \sum_{p=0}^{N}{I_{n+1-p} J_pX^{n+1}}$, \\
	D’où $ J_{n+1}X^{n+1} = \displaystyle \sum_{p=0}^{N}{I_{n+1-p}X^{n+1-p} J_pX^{p}} \subseteq M$.\\
	Ainsi $R(A,g)$ est un $R(A,f)$-module de type fini.\\
	Supposons que $R(A,g)$ est un $R(A,f)$-module de type fini.\\
	Montrons que $f$ est une $\alpha$-réduction de $g$.\\
	Par hypothèse on a : $f \leq g$.\\
	Trouvons $N \in \mathbb{N}$ tel que $\forall n \geq N ; J_n = \displaystyle \sum_{p=0}^{N}{I_{n-p} J_p}$.\\
	$R(A,g)$ étant un $R(A,f)$-module de type fini alors \\
	$R(A,g) = R(A,f)(1, Z_1, \cdots , Z_r), Z_i \in R(A,g)$ et les $Z_i$ sont homogènes de degré $i$.\\
	Posons $N=r$\\
	Soit $z \in J_n$.\\
	$z \in J_n \Rightarrow zX^n \in J_nX^n \subseteq R(A,g)$, d'où $zX^n = \displaystyle \sum_{p=0}^{r}{h_p Z_p}$ , $h_p \in R(A,f)$ et $Z_0 = 1$.\\
	$h_p$ homogène de degré $n-p \Rightarrow h_p \in I_{n-p}X^{n-p}$ , on déduit de cela que: $h_p = a_{n-p}X^{n-p} \in I_{n-p}X^{n-p}$.\\
	Aussi $Z_p$ est homogène de degré $p \Rightarrow Z_p = b_{p}X^{p} \in J_pX^p$.\\
	Ainsi $zX^n = \displaystyle \sum_{p=0}^{r}{a_{n-p}X^{n-p} b_{p}X^{p}} \Rightarrow zX^n = \displaystyle \sum_{p=0}^{r}{a_{n-p}X^{n} b_{p}}$, ce qui implique que $z = \displaystyle \sum_{p=0}^{r}{a_{n-p} b_{p}} \in \displaystyle \sum_{p=0}^{r}{I_{n-p} J_{p}}$, par conséquent $J_n \subseteq \displaystyle \sum_{p=0}^{r}{I_{n-p} J_{p}} \subseteq J_n$.\\
	En sommes $J_n = \displaystyle \sum_{p=0}^{r}{I_{n-p} J_{p}}$, donc $f$ est une $\alpha$-réduction de $g$.
\end{proof}
\subsection{Réduction au sens de Dichi-Sangaré}
\begin{madefinition}
	($\beta$-réduction ou réduction au sens de Dichi-Sangaré)\\
	Soient $f = (I_n)_{n \in \mathbb{N}}$, $g = (J_n)_{n \in \mathbb{N}}$ deux filtrations de $A$.\\
	$f$ est une $\beta$-réduction de $g$ si : \\
	\begin{enumerate}
		\item[i)] $f \leq g$
		\item[ii)]  $\exists \, k \geq 1$ tel que $J_{n+k} = I_n J_k , \forall n \geq k$.
	\end{enumerate}
\end{madefinition}
\begin{maremarque}
	Si $f$ est une $\beta$-réduction de $g$ alors $f$ est une $\alpha$-réduction de $g$.\\	
\end{maremarque}
\begin{proof}
	Supposons que $f$ est une $\beta$-réduction de $g$.\\
	Alors $f \leq g$ et $\exists \, k \geq 1$ tel que $I_{n+k} = I_n J_k , \forall n \geq k$.\\
	Posons $N = 2k$\\
	Soit $n \geq N= 2k$,\\
	$\displaystyle \sum_{p=0}^{2k}{I_{n-p} J_{p}} = \displaystyle \sum_{p=0}^{k-1}{I_{n-p} J_{p}} + I_{n-k} J_k + \displaystyle \sum_{p=0}^{k+1}{I_{n-p} J_{p}}$, or $n \geq  2k \Rightarrow n-k \geq k$ et comme $f$ est une $\beta$-réduction de $g$ alors, $I_{n-k} J_k = J_n$.\\
	Donc $\displaystyle \sum_{p=0}^{2k}{I_{n-p} J_{p}} = \displaystyle \sum_{p=0}^{k-1}{I_{n-p} J_{p}} + J_n + \displaystyle \sum_{p=0}^{k+1}{I_{n-p} J_{p}} \Rightarrow J_n \subseteq \displaystyle \sum_{p=0}^{2k}{I_{n-p} J_{p}}$.\\
	De plus on a : $\displaystyle \sum_{p=0}^{2k}{I_{n-p} J_{p}} \subseteq J_n$, 
	Par conséquent $J_n = \displaystyle \sum_{p=0}^{2k}{I_{n-p} J_{p}}$.\\
	On déduit donc de tout ce qui précède que $f$ est une $\alpha$-réduction de $g$.
\end{proof}
\begin{maproposition}
	Soient $I$ et $J$ deux idéaux de $A$.\\
	Alors les assertions suivantes sont équivalentes.\\
	$i)$ $I$ est une réduction de $J$.\\
	$ii)$ $f_I$ est une $\alpha$-réduction de $f_J$.\\
	$iii)$ $f_I$ est une $\beta$-réduction de $f_J$.
\end{maproposition}
\begin{proof}
	$i) \Rightarrow ii)$\\
	Supposons que $I$ est une réduction de $J$.\\
	Alors $\exists N \in \mathbb{N^*}$ tel que $J^{N+1} = IJ^N$,\\ $I \subseteq J \Rightarrow I^n \subseteq J^n , \forall n \in \mathbb{N}$, d'où $f_I \leq f_J$.\\
	Posons $N_0 = N+1$ , \\
	Soit $n \geq N_0$ , \\
	$\displaystyle \sum_{p=0}^{N+1}{I^{n-p} J^{p}} = \displaystyle \sum_{p=0}^{N}{I^{n-p} J^{p}} + I^{n-N-1} J^{N+1}$, comme $I$ est une réduction de $J$ alors, $I^{n-N-1} J^{N+1} = J^{n-N-1+N+1} = J^n$. Donc $\displaystyle \sum_{p=0}^{N+1}{I^{n-p} J^{p}} = \displaystyle \sum_{p=0}^{N}{I^{n-p} J^{p}} + J^{n}$,\\ ainsi $J^n \subseteq \displaystyle \sum_{p=0}^{N+1}{I^{n-p} J^{p}} \subseteq J^n$.\\
	$f_I$ est donc une $\alpha$-réduction de $f_J$.\\
	$ii) \Rightarrow iii)$\\
	Supposons que $f_I$ est une $\alpha$-réduction de $f_J$.\\
	Alors $f_I \leq f_J$ et $\exists \, N_0 \in \mathbb{N^*} , \forall \, n \geq N_0 \, J^n = \displaystyle \sum_{p=0}^{N_0}{I^{n-p} J^{p}}$.\\
	Posons $N = N_0$\\
	Soit $n \geq N$, \\
	$I^n J^{N_0} = \displaystyle \sum_{p=0}^{N_0}{I^n I^{N_0-p} J^{p}} = \displaystyle \sum_{p=0}^{N_0}{I^{n+N_0-p} J^{p}} = J^{N_0+n} \Rightarrow I^n J^{N_0} = J^{N_0+n}$,\\d'où $f_I$ est une $\beta$-réduction de $f_J$.\\
	$iii) \Rightarrow i)$
	Supposons que $f_I$ est une $\beta$-réduction de $f_J$.\\
	$f_I \leq f_J$ et $\exists N_0 \in \mathbb{N^*}$ tel que $\forall n \geq N_0 , J^{n+N_0} = I^n J^{N_0}$\\
	$f_I \leq f_J \Rightarrow I \subseteq J$.\\
	Posons $N = 2N_0$\\
	$J^{N+1} = J^{2N_0+1} = J^{N_0+N_0+1} = I^{N_0+1} J^N_0 = I I^{N_0} J^{N_0} = IJ^{2N_0}$.\\ Donc $J^{N+1}= IJ^{N}$ ce qui fait que $I$ est une réduction de $J$.
\end{proof}
\begin{maproposition}
	Soient $A$ un anneau commutatif unitaire, $I$ un idéal de $A$ et $g = (J_n)_{n \in \mathbb{N}}$ une filtration de $A$.\\
	$f_I$ est une $\beta$-réduction de $g$ si et seulement si $g$ est $I$-bonne.
\end{maproposition}
\begin{proof}
	Supposons que $f_I$ est une $\beta$-réduction de $g$.\\
	Cela implique que $f_I \leq g$ et $\exists \, N_0 \in \mathbb{N^*} , \forall n \geq N_0 \, J_{n+N_0} = I^n J_{N_0}$.\\
	$f_I \leq g \Rightarrow \forall n \in \mathbb{N} , IJ_n \subseteq J_{n+1}$.\\
	Posons $N = 2N_0$\\
	Soit $n \geq N$, 
	\begin{align*}
		J_{n+1} &= J_{n-N_0-N_0+1}\\
		&= J_{n-N_0+1N_0}\\
		&= I^{n-N_0+1} J_{N_0}\\
		&= II^{n-N_0} J_{N_0}\\
		&= IJ_n
	\end{align*}
	D'où $\forall n \geq 2N_0$, $IJ_n = J_{n+1}$, $g$ est donc $I$-bonne. 
\end{proof}
\begin{maremarque}
	En effet bien qu'on puisse parler de réduction minimale pour un idéal quelconque sur un anneau local, cela n'est pas possible pour toutes les filtrations. On montre ainsi dans cet exemple que toutes les filtrations n'admettent pas de réduction minimale.\\	
\end{maremarque}
\begin{monexemple}
	Soient l'anneau $A = k\left[ X\right]$ où $k$ est un corps et $I = (X)$ un idéal de $A$.\\
	Considérons la filtration $f = (A, I, I, I^2, I^2, I^3, I^3, \cdots)$\\
	$I_{2n} = I_{2n-1} = I^n, \forall n \in \mathbb{N}$\\
	Montrons que $f$ est une filtration noethérienne mais pas une filtration $I$-bonne.\\
	$\bullet$La filtration $f$ est fortement AP car, \\en prenant $k = 2$ on a: $I_{2n} = I^n = I_2^n$.\\
	L’anneau $A$ étant noethérien alors la filtration $f$ est noethérienne.\\
	$\bullet$ Supposons que la filtration $f$ est fortement noethérienne alors, il existe $k \geq 1$ tel que pour tous $m, n \geq k$ on a: $I_{m+n} = I_m I_n$.\\
	Posons $m = 2k+1$ et $n = 2p+1$ où $p \geq k$\\
	$I_{m+n} = I_{2k+1+2p+1} = I_{2(k+p+1)} = I^{k+p+1}$.\\
	$I_m I_n =  I_{2k+1}  I_{2p+1} =  I_{2(k+1)-1}  I_{2(p+1)-1} = I^{k+1} I^{p+1} = I^{k+p+2}$
	\begin{align*}
		I_{m+n} = I_m I_n &\Rightarrow I^{k+p+1} = I^{k+p+2}\\
		& \Rightarrow (X)^{k+p+2} = (X)^{k+p+1}\\
		&\Rightarrow X^{k+p+1} = QX^{k+p+2}\\
		&\Rightarrow 1 = XQ
	\end{align*}
	Ainsi on a donc $X$ inversible ce qui est absurde, par suite la filtration $f$ n'est pas noethérienne. Comme la filtration $f$ n'est pas noethérienne alors elle n'est pas $I$-bonne.\\ De plus il existe pas d'entier $r \geq 1$ tel que $I_r$ soit idempotent, par conséquent notre filtration $f$ n'admet pas de réduction minimale.
\end{monexemple}
\begin{maremarque}
	\begin{enumerate}
		\item[(1)] Soient $I,J$ des idéaux de $A$ tels que $I \subseteq J$. Alors $f_I$ est une réduction de $f_J$ $\Longleftrightarrow$ $I$ est une réduction de $J$.
		\item[(2)] Soit $I$ un idéal de $A$.\\ Alors $f_I$ est une réduction de la filtration $g$ de $A$ $\Longleftrightarrow$ $g$ est $I-bonne$.
	\end{enumerate}
\end{maremarque}

\begin{maproposition}
	Soit $f,g \in \mathbb{F}(A),$ telles que $f \leqslant g$.
	\begin{enumerate}
		\item[(i)] $f$ est un réduction de $g$ si et seulement s'il existe un entier naturel $k \geqslant 1$ tel que $J_{k+n}  = J_{k}I_n$ pour tout $n \geqslant k$
		\item[(ii)] Si $f$ est une réduction de $g$ et que $g$ est une réduction de $h \in \mathbb{F}(A)$, alors $f$ est une réduction de $h$. 
		\item[(iii)] Si $f$ est une réduction de $g$ et si $h$ est une filtration $A$ telle que $f \leqslant h \leqslant g$ alors $h$ est une réduction de $g$
	\end{enumerate}
\end{maproposition}
\begin{proof}
	i) Supposons que $f$ soit une réduction de $g$. Alors:
	\begin{enumerate}
		\item[(a)] $f \leqslant g$
		\item[(b)] $\exists r \geqslant 1,n_o \geqslant 0 ,\quad \forall n \geqslant n_0,\quad J_{r+n}= J_r I_n $
	\end{enumerate}
	Soit $m_{0}\in \mathbb{N},$ tel que $m_{0}r\geq n_{0}$
	
	Posons $k=m_{0}r$
	
	Alors $J_{k+n}=J_{m_{0}r+n}=J_{m_{0}r}I_{n}=J_{k}I_{n}$ car $k\geq n_{0}.$
	
	La réciproque est évidente.
	
	\bigskip 
	
	ii) Supposons que $f$ est une réduction de $g$ et $g$ une réduction
	de $h.$
	
	* $f\leq g\leq h\Rightarrow f\leq h$
	
	* Comme $\ g$ est une réduction de $h$ alors, il existe $k^{\prime }\geq
	1,$ $H_{k^{\prime }+n}=H_{k^{\prime }}J_{n},$ pour tout $n\geq k^{\prime }.$
	
	Posons $k^{\prime \prime }=k^{\prime }(k^{\prime }+1)$ comme dans (i)
	
	Ainsi en utilisant (i) car $f$ est une réduction de $g$, il vient  $H_{k^{^{\prime \prime }}+n}=H_{k^{^{\prime \prime }}}I_{n},$ pour tout $n\geq k^{^{\prime \prime }}.$
	
	Par suite $f$ est une réduction de $h$. \\
	
	iii) Supposons que $f$ réduction de $g$ et que $f\leq h\leq g.$
	
	Soit $k$ comme dans (i).
	
	Comme $h\leq g$ alors pour tout $n\geq k,$ $J_{k}H_{n}\subseteq
	J_{k}J_{n}=J_{k+n}\subseteq J_{k}H_{n}$ car $f\leq h.$
	
	Donc $J_{k+n}=J_{k}H_{n}$ $,$ pour tout $n\geq k.$
	
	Par suite $h$ est réduction de $g.$
\end{proof}
\begin{maproposition}
	Si $f=(I_n)$ est une réduction de $g=(J_n)$ alors:
	\begin{enumerate}
		\item[(i)] $f$ est $A.P.$ et $g$ est fortement $A.P$
		\item[(ii)] $g$ est $E.P$ et $f-bonne$
		\item[(iii)] En plus, si $A$ est noethérien alors $f$ et $g$ sont noethériennes et g est fortement entière sur $f$.
	\end{enumerate}
\end{maproposition}
\begin{maremarque}
	Cependant, le fait que $g$ soit fortement entière sur $f$ n'implique pas
	nécessairement que $f$ soit une réduction de $g$, m\^{e}me si $f$ et 
	$g$ sont noethériennes. Or peut le voir sur l'exemple suivant : 
	
	Soit $A=k[X]$ l'anneau des polynômes à une indéterminée sur le
	corps $k$. Soit $I=XA$. 
	
	On considère les filtrations $f=(I_{n})$ et $g=(J_{n})$ définies par
	:
	
	$I_{n}=\left\{ 
	\begin{array}{c}
		I^{\frac{3n}{2}}\text{ si }n\text{ pair} \\ 
		I^{\frac{3n+3}{2}}\text{ si }n\text{ impair}%
	\end{array}%
	\right. $
	
	$J_{n}=\left\{ 
	\begin{array}{c}
		I^{\frac{3n}{2}}\text{ si }n\text{ pair} \\ 
		I^{\frac{3n+1}{2}}\text{ si }n\text{ impair}
	\end{array}
	\right. $
	
	On vérifie que $g$ est noethérienne et que $f\leq g$. De plus, la
	filtration $g$ est entière sur $f$. 
	
	En effet pour tout élément $b\in J_{n}$, $b^{2}\in J_{2n}$ et on a $(bY^{n})^{2}=b^{2}Y^{2n}\in R(A,f)$. 
	
	L'anneau $R(A,g)$ est donc entier sur $R(A,f)$. De plus, comme $g$ est noethérienne, il résulte que $g$ est fortement
	
	entière sur $f$ et que $f$ est noethérienne. 
	
	Néanmoins, $f$ n'est pas une réduction de $g$ puisqu'on n'a pas $J_{2p+1}^{2}=I_{2p+1}J_{2p+1}$, pour $p$ suffisamment grand, condition nécessaire pour qu'une filtration $f$ soit une réduction de $g$ quand
	l'anneau A est noethérien. 
\end{maremarque}
\begin{maremarque}
	Soit $f \in \mathbb{F}(A)$. On suppose que A est noethérien. Alors $f$ est une réduction de $f$ $\Longleftrightarrow$ f est noethérienne.
\end{maremarque}
\section{Filtrations f-bonnes}
\begin{madefinition}
	Soient $A$ un anneau et $M$ un $A-module$.\\
	On suppose que $\varphi=(M_n)_{n \in \mathbb{Z}}$ est $f-compatible$, avec $f \in \mathbb{F}(A)$. Alors:
	\begin{itemize}
		\item[(a)] $\varphi$ est \textbf{faiblement $f-$ bonne} s'il existe un entier naturel N $\geqslant 1$ tel que:
		\[\forall n > N, M_{n}=\sum_{p=0}^{N}I_{n-p}M_{p} \]
		\item[(b)] $\varphi$ est \textbf{$f-$ bonne} s'il existe un entier naturel N $\geqslant 1$ tel que:
		\[\forall n > N, M_{n}=\sum_{p=1}^{N}I_{n-p}M_{p} \]
		\item[(c)] $\varphi$ est \textbf{$f-$ fine} s'il existe un entier naturel N $\geqslant 1$ tel que:
		\[\forall n > N, M_{n}=\sum_{p=1}^{N}I_{p}M_{n-p} \]
	\end{itemize} 
\end{madefinition}
\begin{maremarque}
	\begin{enumerate}
		\item[(1)] Toute filtration $f-$bonne est faiblement $f-$ bonne
		\item[(2)] Soit $f \in \mathbb{F}(A)$. Alors f est faiblement $f-$bonne.
		\item[(3)] Soit $f \in \mathbb{F}(A)$.Alors f est $f-$bonne si et seulement si f est $E.P.$
		\item[(4)] Soient $g=(J_n)$ et $  f = (I_n) \in \mathbb{F}(A)$ telles que $f \leqslant g$ alors:
		\begin{itemize}
			\item g est fortement entière sur f $\implies$ g est faiblement $f-bonne$
			\item  En d'autres termes, si $g$ est faiblement $f-bonne$ alors il existe un entier naturel $N \geqslant 1$ tel que $t_{N}g \leqslant f \leqslant g$
		\end{itemize}
		\item[(5)] Soient $\varphi \in \mathbb{F}(M)$ et $I$ un idéal de $A$. Alors $\varphi$ est $I-bonne$ $\Longleftrightarrow$ $\varphi$ est $f_{I}-bonne$, où $f_{I}$ est la filtration $I-adique$.
	\end{enumerate}
\end{maremarque}
\begin{maproposition}
	Toute filtration $\varphi$ de $M$ $f-fine$ est $f-bonne$
\end{maproposition}
\begin{proof}
	Supposons que $\varphi =(M_{n})$ est une filtration de $M$ qui est $f-fine$,
	où $f=(I_{n})$ une filtration de $A.$
	
	Alors il existe $N\geq 1$ tel que pour tout $n>N,M_{n}=$ $%
	\sum\limits_{p=1}^{N}I_{p}M_{n-p}.$
	
	Comme $n>N,$ posons $n=N+1$, ainsi
	
	$M_{N+1}=$ $\sum\limits_{p=1}^{N}I_{p}M_{N+1-p}=$ $\sum\limits_{q=1}^{N}I_{N+1-q}M_{q},$ avec $q=N+1-p.$
	
	Ainsi, il vient de proche en proche que $M_{N+j}=$  $\sum\limits_{p=1}^{N}I_{N+j-p}M_{p}\,\ ,$ pour tout $j$ avec $1\leq j\leq m.$
	
	Alors $M_{N+m}=$ $\sum\limits_{p=1}^{N}I_{p}M_{N+m-p}\,=\sum\limits_{q=m}^{N+m-1}I_{N+m-q}M_{q}\,=\sum\limits_{q=m}^{N}I_{N+m-q}M_{q}\,+\sum\limits_{q=N+1}^{N+m-1}I_{N+m-q}M_{q}=\sum\limits_{q=m}^{N}I_{N+m-q}M_{q}\,+\sum\limits_{q=N+1}^{N+m-1}I_{N+m-q}(\sum\limits_{p=1}^{N}I_{q-p}M_{p}).$
	
	Or $\sum\limits_{q=m}^{N}I_{N+m-q}M_{q}\,\subseteq
	\sum\limits_{p=1}^{N}I_{N+m-p}M_{p}\,$\ et $\sum\limits_{q=N+1}^{N+m-1}I_{N+m-q}(\sum\limits_{p=1}^{N}I_{q-p}M_{p})=\sum\limits_{p=1}^{N}(\sum\limits_{q=N+1}^{N+m-1}I_{N+m-p})M_{p}=\sum\limits_{p=1}^{N}I_{N+m-p}M_{p}\subseteq M_{N+m}$
	
	Par suite $\varphi $ est $f-bonne,$ l'inclusion inverse étant évidente.
\end{proof}
\begin{moncorollaire}
	Soient $f,g \in \mathbb{F}(A)$ avec $f \leqslant g$. Si $A$ est noethérien alors:\\ 
	$g$ faiblement $f-bonne$ $\Longleftrightarrow$ $g$ est fortement entière sur $f$
\end{moncorollaire}
\begin{maproposition}
	Soient $f=(I_n)$ une filtration $E.P.$ de $A$ et $\varphi=(M_n) \in \mathbb{F}(M)$. Nous avons les assertions suivantes:
	\[ \varphi \text{ est } f-fine \Longleftrightarrow \varphi \text{ est } f-bonne \Longleftrightarrow \varphi \text{ est faiblement } f-bonne   \]
\end{maproposition}
\begin{proof}
	Il suffit de montrer que $\varphi$ est faiblement $f-bonne$ si $\varphi$ est $f-fine$.\\ Supposons que $\varphi$ est faiblement $f-bonne$.\\
	Soient $N, N' \geqslant 1$ des entiers tels que pour tout $n \geqslant N,$
	$M_{n}=$ $\sum\limits_{p=0}^{N}I_{n-p}M_{p}$ et pour tout $n\geq
	1,I_{n}=\sum\limits_{p=1}^{N^{\prime }}I_{n-p}I_{p}.$ Alors pour $n>N^{\prime \prime }=N+N^{\prime},$
	
	$M_{n}=$ $\sum\limits_{p=0}^{N}I_{n-p}M_{p}=\sum\limits_{p=0}^{N}(%
	\sum\limits_{q=1}^{N^{\prime
	}}I_{n-p-q}I_{p})M_{p}=\sum\limits_{q=1}^{N^{\prime
	}}I_{q}(\sum\limits_{p=0}^{N}I_{n-p-q}M_{p})=\sum\limits_{q=1}^{N^{\prime
	}}I_{q}M_{n-q}\subseteq \sum\limits_{q=1}^{N^{^{\prime \prime }}}I_{q}M_{n-q}
	$
	
	Donc $M_{n}=\sum\limits_{q=1}^{N^{^{\prime \prime }}}I_{q}M_{n-q}$ ,
	l'inclusion inverse étant triviale. 
\end{proof}
\begin{moncorollaire}
	Soient $f,g \in \mathbb{F}(A)$. Si $A$ est noethérien, $f \leqslant g$ et $f$ noethérien. Alors nous avons les assertions suivantes:
	\[ g \text{ est } f-fine \Longleftrightarrow  g \text{ est } f-bonne \Longleftrightarrow  g \text{ est faiblement } f-bonne \Longleftrightarrow  g \text{ est fortement entière sur } f \]
\end{moncorollaire}
\begin{maproposition}
	Soient $f=(I_n), g=(J_n) \in \mathbb{F}(A)$, tel que $f \leqslant g$.\\ Si $g$ est $f-bonne$, $E.P.$ et $A$ est noethérien alors $f$ et $g$ sont noethériennes.
\end{maproposition}
\begin{proof}
	Il existe un entier $N\geq 1$ tel que pour tout $n>N,J_{n}=\sum%
	\limits_{p=1}^{N}I_{n-p}J_{p}\subseteq
	\sum\limits_{p=1}^{N}J_{n-p}J_{p}\subseteq J_{n}$
	
	Donc $J_{n}=\sum\limits_{p=1}^{N}J_{n-p}J_{p}$ pour tout $n>N.$
	
	Cette égalité est valable si $1\leq n\leq N.$
	
	Comme $g$ est $E.P$ et $A$ noethérien alors $g$ est fortement entière sur $f.$ 
	
	Par suite $g$ est noethérien et d'après le théorème de
	Eakin, $f$ est noethérien. 
\end{proof}
\begin{maproposition}
	Soient $f=(I_n), g=(J_n) \in \mathbb{F}(A)$, tel que $f \leqslant g$.\\Si $g$ est faiblement $f-bonne$ alors:\\ $f$ est $A.P$ $\Longleftrightarrow$ $g$ est $A.P$.
\end{maproposition}
\begin{proof}
	Soient $f=(I_{n})_{_{n\in \mathbb{N}}},g=(J_{n})_{_{n\in\mathbb{N}}}\in F(A).$
	
	Alors il existe un entier $N\geq 1$ tel que $I_{n}\subseteq J_{n}\subseteq
	I_{n-N}\subseteq J_{n-N}$ pour tout $N\geq 1.$
	
	Si $f$ est $A.P.$ alors il existe une suite d'entiers $(k_{n})_{n\in \mathbb{N}}$ telle que $\underset{n\longrightarrow \infty }{\lim }\frac{k_{n}}{n}=1$
	et $I_{k_{n}m}\subseteq I_{n}^{m}$
	
	pour tout $m,n\in \mathbb{N}.$
	
	Par suite, $J_{(k_{n}+N)m}\subseteq J_{k_{n}m+Nm}\subseteq
	J_{k_{n}m+N}\subseteq I_{k_{n}m+N}\subseteq I_{k_{n}m}\subseteq
	I_{n}^{m}\subseteq J_{n}^{m}.$
	
	D'où $\underset{n\longrightarrow \infty }{\lim }\frac{k_{n}+N}{n%
	}=1,$ $g$ est $A.P.$
	
	Réciproquement si $g$ est $A.P.$ alors il existe une suite d'entiers $(k_{n}^{^{\prime }})_{n\in \mathbb{N}}$ associée à $g.$
	
	Alors $I_{k_{n}^{\prime }+N.m}\subseteq J_{k_{n}^{\prime }+N.m}\subseteq
	J_{n+N}^{m}\subseteq I_{n}^{m}$ pour tout $m,n\in \mathbb{N}.$
	
	Et $\underset{n\longrightarrow \infty }{\lim }\frac{k_{n}^{\prime }+N}{n}=1,f
	$ est $A.P.$
\end{proof}
\begin{moncorollaire}
	Soient $f=(I_n), g=(J_n) \in \mathbb{F}(A)$, tel que $f \leqslant g$ et $A$ est noethérien.\\ Si $f$ ou $g$ est noethérien alors $g$ est $f-bonne$ $\Longleftrightarrow$ $g$ est fortement entière sur $f$.
\end{moncorollaire}
\begin{maproposition}
	Soit $A$ un anneau noethérien. Soient $f,g \in \mathbb{F}(A)$.\\ Si $f$ est noethérienne alors $g$ est fortement entière sur $f$ $\Longleftrightarrow$ il existe un entier naturel $N \geqslant 1$ tel que $t_{N}g \leqslant f \leqslant g$.
\end{maproposition}
\begin{maproposition}
	Soient $f,g \in \mathbb{F}(A)$. Alors:
	\[ g \text{ est entière sur } f \Longleftrightarrow \forall k \in \mathbb{N}^{*}, g^{(k)} \text{est entière sur } f^{(k)} \Longleftrightarrow \exists k \in \mathbb{N}^{*}, g^{(k)} \text{est entière sur } f^{(k)} \]
\end{maproposition}
\begin{maproposition}
	Soit A un anneau noethérien et $g$ une filtration de $A$ fortement entière sur $f$ ou $g$ est fortement $A.P.$ de rang $r$, alors pour tout entier naturel $m \geqslant 1$, $g^{(rm)}$ est fortement entière sur $f^{(rm)}$.  
\end{maproposition}
\begin{proof}
	Si $g$ est fortement $A.P.$ alors $f$ est fortement $A.P.$
	Alors Supposons que seulement $f$ est fortement $A.P.$ de rang $r.$
	
	Il existe un entier $N\geq 0$ tel que $t_{N}g\leq f\leq g.$\\
	Posons $f=(I_{n}),g=(J_{n}),f_{1}=f^{(rm)},g_{1}=g^{(rm)}.$
	
	Nous avons $J_{N+n}\subseteq I_{n}$ pour tout $n\in \mathbb{N}.$
	
	De même $J_{rm(N+n)}\subseteq J_{N+rmn}$ $\subseteq I_{rmn}$  pour tout $n\in \mathbb{N}.$
	
	D'où $t_{N}g_{1}\leq f_{1}\leq g_{1}.$
	
	Or par hypothèse, $f^{(r)}=f_{I_{r}\text{ }}$alors $f^{(rm)}=f_{I_{rm}}$ qui est noethérienne.
	
	Donc $g_{1}$ est fortement entière sur $f_{1}.$
	
\end{proof}
\begin{moncorollaire}
	Soient $f,g \in \mathbb{F}(A)$, tel que $f \leqslant g$. Si $A$ est noethérien et $g$ est fortement entière sur $f$. Alors $f$ est fortement $A.P$ $\Longleftrightarrow$ $g$ est fortement $A.P.$
\end{moncorollaire}
\begin{maproposition}
	Si $f$ est une filtration I-adique alors nous avons les implications suivantes:\\
	$f$ est fortement A.P $\Longleftrightarrow $ $f$ est A.P $\Longleftrightarrow $ $f$ est fortement noethérienne $\Longleftrightarrow $ $f$ est noethérienne $\Longleftrightarrow $ $f$ est E.P
\end{maproposition}

\begin{moncorollaire}
	Soient $f=(I_n)_{n \in \mathbb{N}}$ et $g=(J_n)_{n \in \mathbb{N}}$ deux filtrations de $A$,\\ tel que $f \leqslant  g$.
	Si $f$ ou $g$ est noethérienne. Alors: \\
	$g$ est $f-bonne$ $\Longleftrightarrow $ $g$ est fortement intégral sur $f$
\end{moncorollaire}
\begin{maproposition}
	Soient $f=(I_n)_{n \in \mathbb{N}}$ et $g=(J_n)_{n \in \mathbb{N}}$ deux filtrations de $A$ tel que $f$ est une réduction de $g$ alors:\\
	$g$ est E.P et $g$ est $f-bonne$.
\end{maproposition}

\begin{maproposition}
	Lorsque $f$ est une filtration fortement noethérienne et $g$ est une
	filtration noethérienne de l'anneau noethérien A vérifiant $
	f=(I_{n})\leq g=(J_{n})$, on montre que les assertions suivantes sont équivalentes:
	
	(i) $f$ est une réduction de $g$
	
	(ii) $J_{n}^{2}=I_{n}J_{n},\forall n>>0$
	
	(iii) L'idéal $I_{n}$ est une réduction de l'idéal $J_{n}$ pour
	tout $n>>0$
	
	(iv) Il existe un entier $k\geq 1$ tel que $g^{k}$ soit $I_{k}-bonne$
	
	(v) $\forall m\geq 1$, $f^{(m)}$ est une réduction de $g^{(m)}$
	
	(vi) $\exists m\geq 1,$ tel que $f^{(m)}$ soit une réduction de $g^{(m)}$
	
	(vii) $g$ est entière sur $f$
	
	(viii) $g$ est fortement entière sur $f$
	
	(ix) $g$ est $f-fine$
	
	(xi) $g$ est faiblement $f-bonne$
	
	(x) $g$ est $f-bonne$
	
	(xii) $\exists m\geq 1,$ tel que $t_{m}f\leq f\leq g$
	
	(iii) $(P_{k}(f))=(P_{k}(g))$ pour tout $k\in\mathbb{N}$
	
	En particulier, il résulte des équivalences ci-dessus que si $f$ est une filtration $I-adique$ de l'anneau noethérien $A$ et si $g$ est
	une filtration noethérienne dominée par $g$, les notions suivantes
	sont équivalentes :
	
	(1) $f_{I}$ est une réduction de $g$
	
	(2) $g$ est entière sur $f_{I}$.
	
	(3) $g$ est fortement entière sur $f_{I}$
	
	(4) $g$ est $I-bonne$
	
\end{maproposition}

\begin{maproposition}
	\label{th1}
	Soient $f=(I_n)_{n \in \mathbb{N}}$ et $g=(J_n)_{n \in \mathbb{N}}$ deux filtrations de $A$,\\ tel que $f \leqslant  g$. Nous considérons les assertions suivantes:\\
	\begin{enumerate}
		\item[(i)] $g$ est $f-bonne$
		
		\item[(ii)] $g$ est fortement entière sur $f$
		
		\item[(iii)] $g$ est $f-fine$
		
		\item[(iv)] $g$ est faiblement $f-bonne$
		
		\item[(v)]  $\exists N \geqslant 1$ un entier tel que $t_Ng \leqslant f \leqslant g$.
	\end{enumerate}
\end{maproposition}
\begin{maproposition}
	Sous les hypothèses du théorème \eqref{th1} en admettant que $A$ est noethérien
	et $f$ est noethérienne alors:
	(i) $\Longleftrightarrow $ (ii) $\Longleftrightarrow $ (iii) $\Longleftrightarrow $ (vi) $\Longleftrightarrow $ (v). 
\end{maproposition}
