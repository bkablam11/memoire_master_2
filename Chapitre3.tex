\chapter{DÉPENDANCE, RÉDUCTION ET FILTRATIONS BONNES}
\chaptermark{DÉPENDANCE, RÉDUCTION ET FILTRATIONS BONNES}

\section{Dépendance intégrale de filtration}
\begin{madefinition}
	Soient $A$ un anneau commutatif unitaire, $I$ un idéal de $A$\\
	et $f=(I_n)_{n \in N} \in \mathbb{F}(A)$. Un élément $x$ de $A$ est dit entier sur $f$ s'il existe un entier $m \in \mathbb{N}$ tel que : 
	\[ 	x^m + a_1 x^{m-1} + \cdots + a_m = x^m + \sum_{i=1}^{m} a_i x^{m-i} = 0, \; m \in \mathbb{N^*} \ où \ a_i \in I_i,\, \forall i=1, \cdots ,m. \]	
\end{madefinition}
\begin{maproposition}
	\label{maprop1}
	Soit $f=(I_n)_{n \in \mathbb{N}} \in \mathbb{F}(A), x \in A $ et $n \in \mathbb{N}$.\\
	Les assertions suivantes sont équivalentes: \\
	\begin{enumerate}
		\item[i)] $x$ est entier sur $f^{(n)}=((I_{nk})_{k \in \mathbb{N}})$.
		\item[ii)] $xX^n(\in A[X])$ est entier sur $R(A,f)$.
		\item[iii)] $xX^n(\in A[X])$ est entier sur $ \mathcal{R}(A,f)$.
	\end{enumerate}
	\begin{proof}
		i)$\Rightarrow$ ii)
		
		Supposons $x$ entier sur $f^{(n)}$.
		
		Alors il existe $r\geq 1,$ $x^{r}+\sum\limits_{i=1}^{r}a_{i}x^{r-i}=0,$ $
		a_{i}\in I_{n_{i}}$.
		
		Alors il existe $r\geq 1,$ $(xX^{n})^{r}+\sum
		\limits_{i=1}^{r}a_{i}X^{n_{i}}(xX^{n})^{r-i}=0,$ $a_{i}\in
		X^{n_{i}}I_{n_{i}}\subseteq R(A,f)$.
		
		Donc $xX^{n}$ est entier sur $R(A,f).$ \\
		
		ii)$\Rightarrow$ iii) Évident car $R(A,f)\subseteq \mathcal{R}(A,f).$ \\
		iii)$\Rightarrow$ i) Supposons $xX^{n}$ est entier sur $\mathcal{R}(A,f).$
		
		Alors il existe $m\geq 1,$ $(xX^{n})^{m}+\sum\limits_{i=1}^{m}a_{i}(xX^{n})^{m-i}=0,$ $a_{i}\in \mathcal{R}(A,f)$.
		
		D'où $\deg ((xX^{n})^{m})=nm$ alors $\deg (a_{i})=n_{i}$
		
		Ainsi $a_{i}\in I_{n_{i}}X^{n_{i}},$ alors $a_{i}=\alpha _{i}X^{n_{i}}$ ,$\alpha _{i}\in I_{n_{i}}$
		
		Donc il existe $m\geq 1,$ $(xX^{n})^{m}+\sum\limits_{i=1}^{m}(\alpha
		_{i}X^{n_{i}})(xX^{n})^{m-i}=0,$ $\alpha _{i}\in I_{n_{i}}$.
		
		Alors, il existe $m\geq 1,$ $X^{nm}[x^{m}+\sum\limits_{i=1}^{m}\alpha
		_{i}x^{m-i}]=0,$ $\alpha _{i}\in I_{n_{i}}$.
		
		Par identification des polynômes, il vient, il existe $m\geq 1,$ $
		x^{m}+\sum\limits_{i=1}^{m}\alpha _{i}x^{m-i}=0,\alpha _{i}\in I_{n_{i}}$.
		
		Par suite, $x$ est entier sur $f^{(n)}$.
	\end{proof}
\end{maproposition}
\section{Clôture intégrale d'une filtration}
\begin{maproposition}
	Soit $f=(I_n)_{n \in N} \in \mathbb{F}(A).$ Alors:\\
	$f'=(I'_n)_{n \in N} \in \mathbb{F}(A)$ appelée \textbf{clôture intégrale de f}
\end{maproposition}
\begin{proof}
	Supposons que $f=(I_{n})_{n\in \mathbb{N}}\in F(A).$
	
	i) $I_{0}^{\prime }=\{x\in A,x$ entier sur $I_{0}\}=A$
	
	Car $I_{0}=A$ $.$
	
	ii) Soit $n\in \mathbb{N}.$
	Comme $I_{n+1}\subseteq I_{n}$ alors par croissance, $I_{n+1}^{\prime }\subseteq I_{n}^{\prime }.$
	
	iii) Soient $p,q\in \mathbb{N}.$
	$I_{q}^{\prime }I_{p}^{\prime }\subseteq (I_{q}I_{p})^{\prime }\subseteq
	(I_{p+q})^{\prime }=I_{p+q}^{\prime }$ car $I^{\prime }J^{\prime }\subseteq
	(IJ)^{\prime }$ pour tout idéaux de $A.$
	
	Par suite $f^{\prime }=(I_{n}^{\prime })_{n\in \mathbb{N}}\in F(A).$
\end{proof}
\begin{moncorollaire}
	Soit $f=(I_n)_{n \in \mathbb{N}} \in \mathbb{F}(A)$. Alors:\\
	$\forall k \in \mathbb{N}, \text{ on pose: } P_k(f)=\left\{x \in A, x \text{ entier sur } f^{(k)}\right\}$ est un idéal de $A$ et la famille \\ $P(f)=(P_k(f))_{k \in \mathbb{N}}$ est une filtration de $A$ appelé \textbf{clôture prüférienne} de $f$.
\end{moncorollaire}
\begin{proof}
	Soit $k\in \mathbb{N}.$ \\
	A) Montons que $P_{k}(f)$ est un idéal de $A$.
	
	i) Par définition, $P_{k}(f)\subseteq A.$
	
	ii) $0_{A}$ est entier sur $A,$ donc $0_{A}\in P_{k}(f)$
	
	iii) Soient $x,y\in P_{k}(f).$
	
	Comme $x,y\in P_{k}(f)$ alors $xX^{k},yX^{k}$ sont entiers sur $R(A,f)$
	
	D'où $xX^{k}+yX^{k}$ est entier sur $R(A,f)$ car $R(A,f)^{\prime }$ est
	un anneau
	
	Donc $x+y$ est entier sur $f^{(k)}$
	
	Par suite $x+y\in P_{k}(f).$
	
	iv) Soit $a\in A,$ soit $x\in P_{k}(f).$
	
	$x\in P_{k}(f)$ alors $xX^{k}$ est entier sur $R(A,f).$
	
	$(ax)X^{k}$ est entier sur $R(A,f)$
	
	Donc $ax$ est entier sur $f^{(k)}$
	
	D'où $ax\in P_{k}(f)$
	
	Par suite $P_{k}(f)$ est un idéal de $A.$ \\
	B) Montrons que $P(f)=(P_{k}(f))_{k\in \mathbb{N}}\in F(A)$
	
	i) $P_{0}(f)=\{x\in A,$ $x$ entier sur $f^{(0)}=(A,...,A)=A\}$
	
	D'où $P_{0}(f)=A$
	
	ii) Soit $x\in P_{k+1}(f).$
	
	Ainsi $x$ est entier sur $f^{(k+1)}.$
	
	Alors il existe $n\geq 1,$ $x^{n}+\sum\limits_{i=1}^{n}a_{i}x^{n-i}=0,$ $
	a_{i}\in I_{(k+1)i}\subseteq I_{ki}$.
	
	Donc $x$ est entier sur $f^{(k)}$
	
	Par suite $x\in P_{k}(f).$
	
	D'où $P_{k+1}(f)\subseteq P_{k}(f).$
	
	iii) Soient $x\in P_{k_{1}}(f)$ et $y\in P_{k_{2}}(f)$.
	
	$xX^{k_{1}}\in R(A,f)^{\prime }$et $yX^{k_{2}}\in R(A,f)^{\prime }.$
	
	D'où $xyX^{k_{1}+k_{2}}\in R(A,f)^{\prime }$ car $R(A,f)^{\prime }$ est
	un anneau.
	
	Donc $xy\in P_{k_{1}+k_{2}}(f)$
	
	Par suite, $P_{k_{1}}(f)$ $P_{k_{2}}(f)\subseteq P_{k_{1}+k_{2}}(f).$
	
	On en déduit que $P(f)=(P_{k}(f))_{k\in \mathbb{N}}\in F(A).$
\end{proof}
\begin{maremarque}
	La clôture intégrale d'un idéal $I$ de $A$ est : $I'=P_1(f_I)$
\end{maremarque}
\begin{maproposition}
	Soit $f=(I_n)_{n \in \mathbb{N}} \in \mathbb{F}(A)$. Alors:\\
	\begin{enumerate}
		\item[(i)] $ f \leqslant f' \leqslant P(f)$,
		\item[(ii)] $ P(P(f)) = P(f)$,
		\item[(iii)] $P(f) = P(f')$, avec $f'=(I'_n)_{n \in \mathbb{N}}$ est la clôture intégrale de $f$.
	\end{enumerate}
\end{maproposition}
\begin{proof}
	i) Il s'agit de montrer que pour tout $n\in \mathbb{N},$ $I_{n}\subseteq I_{n}^{\prime }\subseteq P_{n}(f)$
	
	Soit $n\in \mathbb{N}.$ \\
	a) Soit $x\in I_{n}.$
	
	Posons $r=1.~$Ainsi $x+a_{1}=x+(-x)=0,$ $a_{1}=-x$
	
	Alors $x\in I_{n}^{\prime }.$ D'où $I_{n}\subseteq I_{n}^{\prime }.$
	
	b) Soit $x\in I_{n}^{\prime }.$
	
	Soit $n\in \mathbb{N}.$
	$x\in I_{n}^{\prime }$ alors il existe $r\geq 1,$ $x^{r}+\sum\limits_{i=1}^{r}a_{i}x^{r-i}=0,\alpha _{i}\in I_{n}^{i}$.
	
	Or $I_{n}^{i}\subseteq $ $I_{ni}$ , d'où il existe $m=r\geq 1,$ $x^{m}+\sum\limits_{i=1}^{m}\alpha _{i}x^{m-i}=0,\alpha _{i}=a_{i}\in I_{ni}$
	.
	
	D'où $x\in P_{n}(f)$
	
	Donc $I_{n}^{\prime }\subseteq P_{n}(f)$
	
	Par conséquent, pour tout $n\in \mathbb{N},$ $I_{n}\subseteq I_{n}^{\prime }\subseteq P_{n}(f).$
	
	c'est-\`{a}-dire que $f\leq f^{\prime }\leq P(f)$
	
	ii) Montrons que $P(P(f))=P(f).$
	
	A) D'après i), $f\leq P(f)\Rightarrow P(f)\subseteq P(P(f))$ d'où $P(f)\subseteq P(P(f))$
	
	B) Posons $g=P(f)=(J_{n})_{n\in \mathbb{N}}$ avec $J_{n}=P_{n}(f)$
	
	Donc $P(P(f))=P(g)$
	
	Soient $n\in N,x\in P_{n}(g).$
	
	$x\in P_{n}(g)$ alors $x$ est entier sur $g^{(n)}$
	
	$x\in P_{n}(g)$ alors il existe $s\geq 1,$ $x^{s}+\sum\limits_{i=1}^{s}a_{i}x^{s-i}=0,a_{i}\in J_{ni}$.
	
	Or $J_{ni}=P_{ni}(f)$ , ainsi $a_{i}\in P_{ni}(f)$ pour tout $i\in \llbracket 1, s \rrbracket.$
	
	Ainsi les $a_{i}$ sont entiers sur $f^{(ni)}.$ D'où $a_{i}X^{ni}\in R(A,f)^{\prime }$
	
	Alors $(xX^{n})^{s}+\sum\limits_{i=1}^{s}a_{i}X^{ni}(xX^{n})^{s-i}=0,$ avec 
	$a_{i}X^{ni}\in R(A,f)^{\prime }$.
	
	Donc $xX^{n}$ est entier sur $R(A,f)^{\prime }$.
	
	D'où $xX^{n}\in \lbrack R(A,f)^{\prime }]^{\prime }=R(A,f)^{\prime }$
	
	Par suite $xX^{n}$ est entier sur $R(A,f)$.
	
	Donc $x$ est entier sur $f^{(n)}$
	
	Par suite $x\in P_{n}(f).$
	
	D'où $P_{n}(g)\subseteq P_{n}(f).$
	
	c'est-\`{a}-dire $P(P(f))\subseteq P_{n}(f).$Donc $P(P(f))=P_{n}(f).$
	
	
	
	iii) Montrons que $P(f)=P(f^{\prime })$
	
	D'après i) on a $f\leq f^{\prime }\Rightarrow P(f)\leq P(f^{\prime })$
	
	Réciproquement, $f^{\prime }\leq P(f)\Rightarrow P(f^{\prime })\leq
	P(P(f))=P(f)$
	
	D'où $P(f)\leq P(f^{\prime })\leq P(f)$
	
	Par suite $P(f)=P(f^{\prime })$
\end{proof}

\begin{maproposition}
	Soit $f=(I_n)_{n \in \mathbb{N}} \in \mathbb{F}(A)$. Alors:\\
	\begin{enumerate}
		\item[(i)] $x \in P_k(f)$ si et seulement si $xX^k$ est entier sur $R(A,f)$,
		\item[(ii)] $ I_k \subseteq P_k(f) \subseteq \sqrt[]{f}$ et en particulier $\ \sqrt[]{(P(f))} = \sqrt[]{f}$.
	\end{enumerate}
\end{maproposition}
\begin{maproposition}
	Soient $A \subseteq B$ deux anneaux.\\
	$A' =\left\{x \in B, x\text{ entier sur } A\right\}$.
	\begin{enumerate}
		\item[i)] $A \subseteq A' \subseteq B $,
		\item[ii)] $A'' = A'$.
	\end{enumerate}
\end{maproposition}
\begin{proof}
	i) Montrons que $A\subseteq A^{\prime }\subseteq B$
	
	a) Soit $x\in A.$
	
	On a: $x^{1}+a_{1}=x^{1}+(-x)=0\Rightarrow x\in A^{\prime }.$
	
	Donc $A\subseteq A^{\prime }.$
	
	b) Soit $x^{\prime }\in A^{\prime }.$ 
	
	Par construction $x\in B.$
	
	D'où $A^{\prime }\subseteq B.$
	
	Par suite $A \subseteq A^{\prime }\subseteq B$
	
	
	
	ii) D'après i) $A\subseteq A^{\prime }\Rightarrow A^{\prime }\subseteq
	A^{\prime \prime }$ (Par croissance)
	
	Réciproquement soit $x\in A^{\prime \prime }\Rightarrow x$ entier sur $
	A^{\prime }$
	
	Ainsi il existe $n\geq 1,$ tel que  $x^{n}+\sum
	\limits_{i=1}^{n}a_{i}x^{n-i}=0,a_{i}\in A^{\prime }.$
	
	Alors $a_{i}$ entier sur $A,$ pour tout $i\in \llbracket 1, n \rrbracket.$
	
	D'où $A[a_{i}]$ est un $A$-module de type fini, ainsi $
	A[a_{1},a_{2},...,a_{n}]$ est un A-module de type fini.
	
	En effet, si
	
	$\left\{ 
	\begin{array}{c}
		A[a_{1}]=A(1_{A},a_{1},a_{1}^{2},...a_{1}^{n-1}) \\ 
		A[a_{2}]=A(1_{A},a_{2},a_{2}^{2},...a_{2}^{m-1})
	\end{array}
	\right. $
	
	des A-modules de type fini, alors 
	
	$A[a_{1},a_{2}]=A(a_{1}^{i}a_{2}^{j})_{\substack{ 1\leq i\leq n-1 \\ 1\leq
			j\leq m-1}}$
	
	En procédant de proche en proche, il vient $A[a_{1},a_{2},...,a_{n}]$
	est un A-module de type fini.
	
	Par suite, $K=A[a_{1}x^{n-1},a_{2}x^{n-2},...,a_{n}]$ est un A-module de
	type fini.
	
	Soient $(y_{1},...,y_{n})$ les générateurs de $K.$
	
	Donc $K=A(y_{1},...,y_{n},x^{n-1},x^{n-2},...,x^{2},x,1)$
	
	D'où $A[x]\subseteq H.$
	
	Par suite $x$ est entier sur $A$
	
	Donc $x\in A^{\prime }$
	
	Par suite $A^{\prime \prime }\subseteq A$
	
	Donc  $A^{\prime \prime }=A$
	
	iii) De la même manière, on montre que $I^{\prime \prime }=I^{\prime
	}.$
\end{proof}
\begin{madefinition}
	Soit $I$ un idéal de l'anneau $A$ et $f \in \mathbb{F}(A)$. \\
	On dit que $I$ est entier sur $f$ si tout élément de $I$ est entier sur $f$. \\
	Ce qui signifie que I est entier sur $f$ si $I \subseteq P_1(f)$
\end{madefinition}
\begin{maconsequence}
	Soit $I$ un idéal de l'anneau $A$ et $f \in \mathbb{F}(A)$. \\
	$I$ est entier sur $f$ si et seulement si $f_I \leqslant P(f)$
\end{maconsequence}
\begin{proof}
	$(ii)\Longrightarrow (i)$
	
	Supposons que $f_{I}\leq P(f).$ Alors $I^{n}\subseteq P_{n}(f),$ pour tout $n\geq 1.$
	
	En particulier pour $n=1,$ on a $I\subseteq P_{1}(f).$ Donc $I$ est entier sur $f.$
	
	$(i)\Longrightarrow (ii)$
	
	Supposons que $I$ est entier sur $f.$ Alors $I\subseteq P_{1}(f).$ Montrons que $I^{n}\subseteq P_{n}(f),$ pour tout $n\geq 1.$
	
	\textbf{Initialisation:}
	
	Comme $I\subseteq P_{1}(f)$ alors la propriété est vrai à l'ordre 1.
	
	\textbf{Hérédité:}
	
	Soit $n\geq 1.$ Supposons que la propriété est vraie jusqu'à l'ordre $n$, c'est à dire $I^{n}\subseteq P_{n}(f).$
	
	Montrons que la propriété est vraie jusqu'à l'ordre $n+1$, c'est à dire $I^{n+1}\subseteq P_{n+1}(f).$
	
	On a: $I^{n}\subseteq P_{n}(f)\Longrightarrow I^{n+1}\subseteq
	IP_{n}(f)\subseteq P_{1}(f)P_{n}(f)\subseteq P_{n+1}(f)$ (car $
	(P_{n}(f))_{n\in \mathbb{Z}}\in \mathbb{F}(A)$)
	
	D'où $I^{n+1}\subseteq P_{n+1}(f).$
	
	Donc la propriété est vraie jusqu'à l'ordre $n+1.$
	
	Par suite $I^{n}\subseteq P_{n}(f),$ pour tout $n\geq 1.$ Par conséquent $f_{I}\leq P(f)$
\end{proof}

\begin{madefinition}
	Soit $f=(I_n)_{n \in \mathbb{N}} , g = (J_n)_{n \in \mathbb{N}}\in \mathbb{F}(A)$.  Alors:\\
	\begin{enumerate}
		\item[(a)]$g$ est \textbf{entière sur} $f$ si $g \leqslant P(f)$ (où $P(f)$ est la clôture prüférienne de $f$). C'est à dire:
		\[\forall n \geqslant 1, J_n \subseteq P_{n}(f) \]
		\item[(b)]$g$ est \textbf{fortement entière sur} $f$ si $f \leqslant g$ et si $R(A,g)$ est un $R(A,f)-module$ de type fini.
	\end{enumerate}
\end{madefinition}
\begin{maproposition}
	\label{maprop2}
	Soient $f$ et $g$ deux filtrations de l'anneau $A$ telles que $f \geqslant g$.\\ Alors $R(A,g)$ est entier sur $R(A,f)$ si et seulement si $\mathcal{R}(A,g)$ est entier sur $\mathcal{R}(A,f)$. 
\end{maproposition}
\begin{proof}
	Soient $f=(I_n)_{n \in \mathbb{Z}},g=(J_n)_{n \in \mathbb{Z}}$ deux filtrations de $A$ telles que $f \geqslant g$. \\ Supposons que $\mathcal{R}(A,g)$ est entier sur $\mathcal{R}(A,f)$. \\
	Soit $b \in J_n$ alors $bX^n \in \mathcal{R}(A,g)$ est entier sur $\mathcal{R}(A,f)$ d'après \ref{maprop1}, on a $bX^n$ entier sur $R(A,f)$. Par conséquent $R(A,g)$ est entier sur $R(A,f)$.
	Réciproquement, il suffit de remarquer que $\mathcal{R}(A,f)=R(A,f)[u]$ qui est une algèbre de type fini sur $R(A,f)$.
\end{proof}
\begin{maproposition}
	Soient $f,g$ deux filtrations de $A$ telles que $f \geqslant g$. Alors les assertions suivantes sont équivalentes:
	\begin{enumerate}
		\item[(i)] $g$ est entière sur $f$
		\item[(ii)] $R(A,g)$ est entier sur $R(A,f)$
		\item[(iii)] $\mathcal{R}(A,g)$ est entier sur $\mathcal{R}(A,f)$
	\end{enumerate}
\end{maproposition}
\begin{proof}
	(i) $\Longleftrightarrow$ (ii), immédiat en utilisant la proposition \ref{maprop1} \\
	(ii) $\Longleftrightarrow$ (iii), immédiat en utilisant la proposition \ref{maprop2}
\end{proof}
\begin{maproposition} \cite{Di2} \\
	Soient $f,g$ deux filtrations de $A$. Alors les assertions suivantes sont équivalentes:
	\begin{enumerate}
		\item[(i)] $g$ est entière sur $f$
		\item[(ii)] $\forall k \geqslant 1$, $g^{k}$ est entière sur $f^{(k)}$
		\item[(iii)] $\exists k \geqslant 1$, $g^{k}$ est entière sur $f^{(k)}$
	\end{enumerate}
\end{maproposition}
\begin{proof}
	Soient $f=(I_n)_{n \in \mathbb{Z}},g=(J_n)_{n \in \mathbb{Z}}$ deux filtrations de $A$. \\
	$(i)\Longrightarrow (ii)$
	
	Supposons que $g$ est entière sur $f.$ Montrons que pour tout $k\geq 1,$ $g^{(k)}$ est entière sur $f^{(k)}.$
	
	Soit $k\geq 1,$ on a: $f^{(k)}=(I_{nk})_{n\in \mathbb{Z}}$ et $g^{(k)}=(J_{nk})_{n\in \mathbb{Z}}$ .
	
	Comme $g$ est entière sur $f$ alors l'idéal $J_{nk\text{ }}$est entier sur $f^{(nk)}.$
	
	Or $f^{(nk)}=(f^{(k)})^{(n)}$ donc $J_{nk}$est entier sur $(f^{(k)})^{(n)}.$
	
	Par conséquent $g^{(k)}$ est entière sur $f^{(k)}$ pour tout $k\geq 1.$
	
	$(ii)\Longrightarrow (iii)$ immédiat
	
	$(iii)\Longrightarrow (i)$
	
	Supposons qu'il existe $k\geq 1$ tel que $g^{(k)}$ est entière sur $f^{(k)}$. Montrons que $g$ est entière sur $f.$
	
	Pour cela il suffit de montrer que pour tout $p\geq 1,$ l'idéal $J_{p}$ est entier sur $f^{(p)}.$
	
	Soit $x\in $ $J_{p}.$ Alors $x^{k}$ $\in $ $J_{p}^{k}\subseteq J_{pk}.$
	
	Or $J_{pk}$ est entier sur $f^{(pk)}=(f^{(k)})^{(p)}$. D'où il existe $n\in \mathbb{N}^{\ast }$ tel que:
	
	$(x^{k})^{n}+a_{1}(x^{k})^{n-1}+\cdots +a_{j}(x^{k})^{n-j}+\cdots +a_{n}=0,$ avec $a_{j}\in I_{pkj}.$
	
	On a pour tout $j=1,\cdots ,n$ , $a_{j}(x^{k})^{n-j}=a_{j}x^{kn-kj},$ ainsi $x^{kn}+a_{1}x^{kn-k}+\cdots +a_{j}x^{kn-kj}+\cdots +a_{n}=0,$ avec $a_{j}\in I_{pkj}.$
	
	Posons $m=kn.$ On obtient $x^{m}+a_{1}x^{m-k}+\cdots +a_{j}x^{m-kj}+\cdots +a_{n}=0,$ avec $a_{j}\in I_{p(kj)}.$
	
	Par suite $x$ est entier sur $f^{(p)}$pour tout $p\geq 1.$ Par conséquent pour tout $p\geq 1$ $J_{p}$ est entier sur $f^{(p)}.$
	
	Donc $g$ est entière sur $f.$
	
\end{proof}

\section{Réduction d'une filtration}
\subsection{Réduction au sens de Okon-Ratliff}
\begin{madefinition}
	($\alpha$-réduction ou réduction au sens de Okon-Ratliff)\\
	Soient $f=(I_n)_{n \in \mathbb{N}}$, $g=(J_n)_{n \in \mathbb{N}}$ deux filtrations de $A$.\\
	$f$ est une $\alpha$-réduction de $g$ si : \\
	\begin{enumerate}
		\item[i)] $f \leq g$
		\item[ii)] $\exists \, N \geq 1$ tel que $\forall n \geq N ; J_n = \displaystyle \sum_{p=0}^{N}{I_{n-p} J_p}$.
	\end{enumerate}
\end{madefinition}
\begin{maremarque}
	Soit $\Re$ la relation $f \Re g \Leftrightarrow$ $f$ est une $\alpha$-réduction de $g$. Notons que cette relation est une relation d'ordre sur l'ensemble des filtrations.	
\end{maremarque}
\begin{proof}
	(i) \textbf{Réflexivité} \\
	Posons $N=1$. \\
	$\displaystyle \sum_{p=0}^{1}I_{n-p}I_{p}=I_{n}I_{0}+I_{n-1}I_{1}=I_{n-1}I_{1}=I_{n}$ \
	(car $I_{0}\subseteq I_{1}$).
	
	D'où $f \Re f$ \\
	(ii) \textbf{Transitivité} \\
	Soient $f=(I_{n})_{n\in \mathbb{N}}$ $,g=(J_{n})_{n\in \mathbb{N}}$ $,h=(H_{n})_{n\in \mathbb{N}}$ \ des filtrations de $A.$ \\
	Supposons que $f \Re g$
	alors $f\leq g$ et il existe $N_{1}\geq 1,\forall m\geq N_{1},$ $J_{m}=\displaystyle  \sum_{p=0}^{N_{1}}I_{m-p}J_{p}$.
	Supposons que $g \Re h$ alors $g\leq h$ et il existe $N_{2}\geq 1,\forall k\geq N_{2},$ $H_{k}=\displaystyle  \sum_{p=0}^{N_{2}}J_{k-p}H_{p}$.
	D'où $f\leq g\leq h.$ Donc $f\leq h$. \\
	De plus posons $N=N_{1}+N_{2}$ \\
	$H_{k}=\displaystyle  \sum_{p=0}^{N_{2}}J_{k-p}H_{p}+\displaystyle  \sum_{p=N_{2}+1}^{N}J_{k-p}H_{p}=\displaystyle  \sum_{p=0}^{N_{2}}J_{k-p}H_{p}+\displaystyle  \sum_{p=N_{2}+1}^{N_{1}+N_{2}}J_{k-p}H_{p}$ pour tout $k\geq N_{1}+N_{2}$ \\
	Comme $k\geq N_{1}+N_{2}$ alors $k-N_{2}\geq N_{1}$ et $p\leq N_{2}$ alors $k-p\geq N_{1}$ \\
	D'où $J_{k-p}=\displaystyle  \sum_{i=0}^{N_{1}}I_{k-p-i}J_{i}$ \\
	Ainsi $H_{k}=\displaystyle  \sum_{p=0}^{N_{2}}(\displaystyle  \sum_{i=0}^{N_{1}}I_{k-p-i}J_{i})H_{p}+\displaystyle  \sum_{p=N_{2}+1}^{N_{1}+N_{2}}J_{k-p}H_{p}$ \\
	Donc $H_{k}\subseteq \displaystyle  \sum_{p=0}^{N_{2}}\displaystyle  \sum_{i=0}^{N_{1}}I_{k-p-i}J_{i}H_{p+i}+\displaystyle  \sum_{p=N_{2}+1}^{N_{1}+N_{2}}J_{k-p}H_{p}$  car $g\leq h$ d'où $ J_{i}\subseteq H_{i}\subseteq H_{i}H_{p}\subseteq H_{i+p}$ \\
	Posons $l=p+i,$ ainsi $0\leq l\leq N_{1}+N_{2}$ car $0\leq p\leq N_{2}$ et $ 0\leq i\leq N_{1}$ \\
	D'où $H_{k}\subseteq \displaystyle  \sum_{l=0}^{N_{1}+N_{2}}I_{k-l}H_{l}+\displaystyle  \sum_{p=N_{2}+1}^{N_{1}+N_{2}}J_{k-p}H_{p}$ \\
	Posons $K=\displaystyle  \sum_{p=N_{2}+1}^{N_{1}+N_{2}}J_{k-p}H_{p}$ \\
	Comme $p\geq N_{2}$ alors $H_{p}=\displaystyle  \sum_{p=0}^{N_{2}}J_{p-i}H_{i}$ \\
	D'où $K=\displaystyle  \sum_{p=N_{2}+1}^{N_{1}+N_{2}}J_{k-p}(\displaystyle  \sum_{p=0}^{N_{2}}J_{p-i}H_{i})=\displaystyle  \sum_{p=N_{2}+1}^{N_{1}+N_{2}}\displaystyle  \sum_{p=0}^{N_{2}}J_{k-p}J_{p-i}H_{i}$ \\
	Donc $K\subseteq\displaystyle  \sum_{p=N_{2}+1}^{N_{1}+N_{2}}\displaystyle  \sum_{p=0}^{N_{2}}J_{k-i}H_{i}=\displaystyle  \sum_{p=0}^{N_{2}}J_{k-i}H_{i}$
	$K\subseteq \displaystyle  \sum_{p=0}^{N_{2}}J_{k-i}H_{i}$
	or $0\leq i\leq N_{2}$ et $N_{1}+N_{2}\leq k$ \\
	d'où $k-i\geq N_{1}$ \\
	Ainsi $J_{k-i}=\displaystyle  \sum_{l=0}^{N_{1}}I_{k-i-l}J_{l}$ \\
	D'où $K\subseteq\displaystyle  \sum_{p=0}^{N_{2}}(\displaystyle  \sum_{l=0}^{N_{1}}I_{k-i-l}J_{l})H_{i}\subseteq\displaystyle  \sum_{p=0}^{N_{2}}\displaystyle  \sum_{l=0}^{N_{1}}I_{k-i-l}J_{l}H_{i}\subseteq $
	Posons $p=i+l$ \\
	D'où $K\subseteq \displaystyle  \sum_{p=0}^{N_{1}+N_{2}}I_{k-p}H_{p}$
	or $H_{k}\subseteq
	\displaystyle  \sum_{l=0}^{N_{1}+N_{2}}I_{k-l}H_{l}+\displaystyle  \sum_{p=N_{2}+1}^{N_{1}+N_{2}}J_{k-p}H_{p}=\displaystyle  \sum_{l=0}^{N_{1}+N_{2}}I_{k-l}H_{l}+K\subseteq \displaystyle  \displaystyle  \sum_{l=0}^{N_{1}+N_{2}}I_{k-l}H_{l}$ car $K\subseteq \displaystyle  \displaystyle  \sum_{l=0}^{N_{1}+N_{2}}I_{k-l}H_{l}$ \\
	Donc $H_{k}\subseteq H_{k-p}H_{p}\subseteq H_{k}$ \\
	Finalement $H_{k}=\displaystyle  \displaystyle  \sum_{l=0}^{N_{1}+N_{2}}I_{k-l}H_{l}$
	Par suite $f \Re h$. \\
	
	(iii) \textbf{Anti-symétrie} \\
	Soient $f=(I_{n})_{n\in \mathbb{N}}$ $,g=(J_{n})_{n\in \mathbb{N} }$\ des filtrations de $A$ telles que: \\
	$f \Re g$ alors $f\leq g$ et $g \Re f$ alors $g\leq f.$ \\
	D'où $f=g$ \\
	On en déduit que $\Re$ est une relation d'ordre. 
\end{proof}
\begin{maproposition}
	Soient $f=(I_n)_{n \in \mathbb{N}}$ et $g=(J_n)_{n \in \mathbb{N}}$ deux filtrations de $A$. $f \geq g$ et chaque $J_n$ est de type fini, alors $f$ est une $\alpha$-réduction de $g$ si et seulement si $R(A,g)$ est un $R(A,f)$-module de type fini.
\end{maproposition}
\begin{proof}
	Supposons que $f$ est une $\alpha$-réduction de $g$.\\
	Alors $f \leq g$ et $\exists \, N \geq 1$ tel que $\forall n \geq N ; J_n = \displaystyle \sum_{p=0}^{N}{I_{n-p} J_p}$.\\
	Prouvons que $R(A,g) = R(A,f)(J_0, J_1X_1, \cdots , J_NX_N) = M$.\\
	$J_nX_n \subseteq R(A,g), \forall n \in \mathbb{N} \Rightarrow R(A,f) \subseteq R(A,g)$ car $f \leq g$,
	Donc $M \subseteq R(A,g)$.\\
	Montrons par récurrence que $\forall n \in \mathbb{N}, J_nX^n \subseteq M$
	Si $n \leq N, J_nX^n \subseteq M$, par construction.\\
	Soit $n \geq N$, supposons la propriété vraie et montrons que $J_{n+1}X^{n+1} \subseteq M$.\\
	$J_{n+1} = \displaystyle \sum_{p=0}^{N}{I_{n+1-p} J_p} \Rightarrow J_{n+1}X^{n+1} = \displaystyle \sum_{p=0}^{N}{I_{n+1-p} J_pX^{n+1}}$, \\
	D’où $ J_{n+1}X^{n+1} = \displaystyle \sum_{p=0}^{N}{I_{n+1-p}X^{n+1-p} J_pX^{p}} \subseteq M$.\\
	Ainsi $R(A,g)$ est un $R(A,f)$-module de type fini.\\
	Supposons que $R(A,g)$ est un $R(A,f)$-module de type fini.\\
	Montrons que $f$ est une $\alpha$-réduction de $g$.\\
	Par hypothèse on a : $f \leq g$.\\
	Trouvons $N \in \mathbb{N}$ tel que $\forall n \geq N ; J_n = \displaystyle \sum_{p=0}^{N}{I_{n-p} J_p}$.\\
	$R(A,g)$ étant un $R(A,f)$-module de type fini alors \\
	$R(A,g) = R(A,f)(1, Z_1, \cdots , Z_r), Z_i \in R(A,g)$ et les $Z_i$ sont homogènes de degré $i$.\\
	Posons $N=r$\\
	Soit $z \in J_n$.\\
	$z \in J_n \Rightarrow zX^n \in J_nX^n \subseteq R(A,g)$, d'où $zX^n = \displaystyle \sum_{p=0}^{r}{h_p Z_p}$ , $h_p \in R(A,f)$ et $Z_0 = 1$.\\
	$h_p$ homogène de degré $n-p \Rightarrow h_p \in I_{n-p}X^{n-p}$ , on déduit de cela que: $h_p = a_{n-p}X^{n-p} \in I_{n-p}X^{n-p}$.\\
	Aussi $Z_p$ est homogène de degré $p \Rightarrow Z_p = b_{p}X^{p} \in J_pX^p$.\\
	Ainsi $zX^n = \displaystyle \sum_{p=0}^{r}{a_{n-p}X^{n-p} b_{p}X^{p}} \Rightarrow zX^n = \displaystyle \sum_{p=0}^{r}{a_{n-p}X^{n} b_{p}}$, ce qui implique que $z = \displaystyle \sum_{p=0}^{r}{a_{n-p} b_{p}} \in \displaystyle \sum_{p=0}^{r}{I_{n-p} J_{p}}$, par conséquent $J_n \subseteq \displaystyle \sum_{p=0}^{r}{I_{n-p} J_{p}} \subseteq J_n$.\\
	En sommes $J_n = \displaystyle \sum_{p=0}^{r}{I_{n-p} J_{p}}$, donc $f$ est une $\alpha$-réduction de $g$.
\end{proof}
\subsection{Réduction au sens de Dichi-Sangaré}
\begin{madefinition}
	($\beta$-réduction ou réduction au sens de Dichi-Sangaré)\\
	Soient $f = (I_n)_{n \in \mathbb{N}}$, $g = (J_n)_{n \in \mathbb{N}}$ deux filtrations de $A$.\\
	$f$ est une $\beta$-réduction de $g$ si : \\
	\begin{enumerate}
		\item[i)] $f \leq g$
		\item[ii)]  $\exists \, k \geq 1$ tel que $J_{n+k} = I_n J_k , \forall n \geq k$.
	\end{enumerate}
\end{madefinition}
\begin{maremarque}
	Si $f$ est une $\beta$-réduction de $g$ alors $f$ est une $\alpha$-réduction de $g$.\\	
\end{maremarque}
\begin{proof}
	Supposons que $f$ est une $\beta$-réduction de $g$.\\
	Alors $f \leq g$ et $\exists \, k \geq 1$ tel que $I_{n+k} = I_n J_k , \forall n \geq k$.\\
	Posons $N = 2k$\\
	Soit $n \geq N= 2k$,\\
	$\displaystyle \sum_{p=0}^{2k}{I_{n-p} J_{p}} = \displaystyle \sum_{p=0}^{k-1}{I_{n-p} J_{p}} + I_{n-k} J_k + \displaystyle \sum_{p=0}^{k+1}{I_{n-p} J_{p}}$, or $n \geq  2k \Rightarrow n-k \geq k$ et comme $f$ est une $\beta$-réduction de $g$ alors, $I_{n-k} J_k = J_n$.\\
	Donc $\displaystyle \sum_{p=0}^{2k}{I_{n-p} J_{p}} = \displaystyle \sum_{p=0}^{k-1}{I_{n-p} J_{p}} + J_n + \displaystyle \sum_{p=0}^{k+1}{I_{n-p} J_{p}} \Rightarrow J_n \subseteq \displaystyle \sum_{p=0}^{2k}{I_{n-p} J_{p}}$.\\
	De plus on a : $\displaystyle \sum_{p=0}^{2k}{I_{n-p} J_{p}} \subseteq J_n$, 
	Par conséquent $J_n = \displaystyle \sum_{p=0}^{2k}{I_{n-p} J_{p}}$.\\
	On déduit donc de tout ce qui précède que $f$ est une $\alpha$-réduction de $g$.
\end{proof}
\begin{maproposition}
	Soient $A$ un anneau et $I$ , $J$ deux idéaux de $A$.\\
	Alors les assertions suivantes sont équivalentes.\\
	$i)$ $I$ est une réduction de $J$.\\
	$ii)$ $f_I$ est une $\alpha$-réduction de $f_J$.\\
	$iii)$ $f_I$ est une $\beta$-réduction de $f_J$.
\end{maproposition}
\begin{proof}
	$i) \Rightarrow ii)$\\
	Supposons que $I$ est une réduction de $J$.\\
	Alors $\exists N \in \mathbb{N^*}$ tel que $J^{N+1} = IJ^N$,\\ $I \subseteq J \Rightarrow I^n \subseteq J^n , \forall n \in \mathbb{N}$, d'où $f_I \leq f_J$.\\
	Posons $N_0 = N+1$ , \\
	Soit $n \geq N_0$ , \\
	$\displaystyle \sum_{p=0}^{N+1}{I^{n-p} J^{p}} = \displaystyle \sum_{p=0}^{N}{I^{n-p} J^{p}} + I^{n-N-1} J^{N+1}$, comme $I$ est une réduction de $J$ alors, $I^{n-N-1} J^{N+1} = J^{n-N-1+N+1} = J^n$. Donc $\displaystyle \sum_{p=0}^{N+1}{I^{n-p} J^{p}} = \displaystyle \sum_{p=0}^{N}{I^{n-p} J^{p}} + J^{n}$,\\ ainsi $J^n \subseteq \displaystyle \sum_{p=0}^{N+1}{I^{n-p} J^{p}} \subseteq J^n$.\\
	$f_I$ est donc une $\alpha$-réduction de $f_J$.\\
	$ii) \Rightarrow iii)$\\
	Supposons que $f_I$ est une $\alpha$-réduction de $f_J$.\\
	Alors $f_I \leq f_J$ et $\exists \, N_0 \in \mathbb{N^*} , \forall \, n \geq N_0 \, J^n = \displaystyle \sum_{p=0}^{N_0}{I^{n-p} J^{p}}$.\\
	Posons $N = N_0$\\
	Soit $n \geq N$, \\
	$I^n J^{N_0} = \displaystyle \sum_{p=0}^{N_0}{I^n I^{N_0-p} J^{p}} = \displaystyle \sum_{p=0}^{N_0}{I^{n+N_0-p} J^{p}} = J^{N_0+n} \Rightarrow I^n J^{N_0} = J^{N_0+n}$,\\d'où $f_I$ est une $\beta$-réduction de $f_J$.\\
	$iii) \Rightarrow i)$
	Supposons que $f_I$ est une $\beta$-réduction de $f_J$.\\
	$f_I \leq f_J$ et $\exists N_0 \in \mathbb{N^*}$ tel que $\forall n \geq N_0 , J^{n+N_0} = I^n J^{N_0}$\\
	$f_I \leq f_J \Rightarrow I \subseteq J$.\\
	Posons $N = 2N_0$\\
	$J^{N+1} = J^{2N_0+1} = J^{N_0+N_0+1} = I^{N_0+1} J^N_0 = I I^{N_0} J^{N_0} = IJ^{2N_0}$.\\ Donc $J^{N+1}= IJ^{N}$ ce qui fait que $I$ est une réduction de $J$.
\end{proof}
\begin{maproposition}
	Soient $A$ un anneau commutatif unitaire, $I$ un idéal de $A$ et $g = (J_n)_{n \in \mathbb{N}}$ une filtration de $A$.\\
	$f_I$ est une $\beta$-réduction de $g$ si et seulement si $g$ est $I$-bonne.
\end{maproposition}
\begin{proof}
	Supposons que $f_I$ est une $\beta$-réduction de $g$.\\
	Cela implique que $f_I \leq g$ et $\exists \, N_0 \in \mathbb{N^*} , \forall n \geq N_0 \, J_{n+N_0} = I^n J_{N_0}$.\\
	$f_I \leq g \Rightarrow \forall n \in \mathbb{N} , IJ_n \subseteq J_{n+1}$.\\
	Posons $N = 2N_0$\\
	Soit $n \geq N$, 
	\begin{align*}
		J_{n+1} &= J_{n-N_0-N_0+1}\\
		&= J_{n-N_0+1N_0}\\
		&= I^{n-N_0+1} J_{N_0}\\
		&= II^{n-N_0} J_{N_0}\\
		&= IJ_n
	\end{align*}
	D'où $\forall n \geq 2N_0$, $IJ_n = J_{n+1}$, $g$ est donc $I$-bonne. 
\end{proof}
\begin{maremarque}
	En effet bien qu'on puisse parler de réduction minimale pour un idéal quelconque sur un anneau local, cela n'est pas possible pour toutes les filtrations. On montre ainsi dans cet exemple que toutes les filtrations n'admettent pas de réduction minimale.
\end{maremarque}
\begin{monexemple}
	Soient l'anneau $A = k\left[ X\right]$ où $k$ est un corps et $I = (X)$ un idéal de $A$.\\
	Considérons la filtration $f = (A, I, I, I^2, I^2, I^3, I^3, \cdots)$\\
	$I_{2n} = I_{2n-1} = I^n, \forall n \in \mathbb{N}$\\
	Montrons que $f$ est une filtration noethérienne mais pas une filtration $I$-bonne.\\
	$\bullet$La filtration $f$ est fortement AP car, \\en prenant $k = 2$ on a: $I_{2n} = I^n = I_2^n$.\\
	L’anneau $A$ étant noethérien alors la filtration $f$ est noethérienne.\\
	$\bullet$ Supposons que la filtration $f$ est fortement noethérienne alors, il existe $k \geq 1$ tel que pour tous $m, n \geq k$ on a: $I_{m+n} = I_m I_n$.\\
	Posons $m = 2k+1$ et $n = 2p+1$ où $p \geq k$\\
	$I_{m+n} = I_{2k+1+2p+1} = I_{2(k+p+1)} = I^{k+p+1}$.\\
	$I_m I_n =  I_{2k+1}  I_{2p+1} =  I_{2(k+1)-1}  I_{2(p+1)-1} = I^{k+1} I^{p+1} = I^{k+p+2}$
	\begin{align*}
		I_{m+n} = I_m I_n &\Rightarrow I^{k+p+1} = I^{k+p+2}\\
		& \Rightarrow (X)^{k+p+2} = (X)^{k+p+1}\\
		&\Rightarrow X^{k+p+1} = QX^{k+p+2}\\
		&\Rightarrow 1_A = XQ
	\end{align*}
	Ainsi on a donc $X$ inversible ce qui est absurde, par suite la filtration $f$ n'est pas noethérienne. Comme la filtration $f$ n'est pas noethérienne alors elle n'est pas $I$-bonne.\\ De plus il existe pas d'entier $r \geq 1$ tel que $I_r$ soit idempotent, par conséquent notre filtration $f$ n'admet pas de réduction minimale.
\end{monexemple}

\begin{maproposition}
	Soit $f,g \in \mathbb{F}(A),$ telles que $f \leqslant g$.
	\begin{enumerate}
		\label{maprop4}
		\item[(i)] $f$ est un réduction de $g$ si et seulement s'il existe un entier naturel $k \geqslant 1$ tel que $J_{k+n}  = J_{k}I_n$ pour tout $n \geqslant k$. Pour un tel entier $k$ et pour tout $m \geqslant 1$, $J_{mk}=J_{mk+pk-pk}=J_{pk+(m-p)k}=J_{pk}I_{(m-p)k}=J_{pk}J_{(m-p)k}=J_{k}^{p}J_{(m-p)k}=J_{k}^{p}I_{(m-p)k}=J_{k}^{p}J_{(m-p)k}
		$ pour tout $p=1,2,\cdots,m$
		\item[(ii)] Si $f$ est une réduction de $g$ et que $g$ est une réduction de $h \in \mathbb{F}(A)$, alors $f$ est une réduction de $h$. 
		\item[(iii)] Si $f$ est une réduction de $g$ et si $h$ est une filtration $A$ telle que $f \leqslant h \leqslant g$ alors $h$ est une réduction de $g$
	\end{enumerate}
\end{maproposition}
\begin{proof}
	i) Supposons que $f$ soit une réduction de $g$. Alors:
	\begin{enumerate}
		\item[(a)] $f \leqslant g$
		\item[(b)] $\exists r \geqslant 1,n_o \geqslant 0 ,\quad \forall n \geqslant n_0,\quad J_{r+n}= J_r I_n $
	\end{enumerate}
	Soit $m_{0}\in \mathbb{N},$ tel que $m_{0}r\geq n_{0}$
	
	Posons $k=m_{0}r$
	
	Alors $J_{k+n}=J_{m_{0}r+n}=J_{m_{0}r}I_{n}=J_{k}I_{n}$ car $k\geq n_{0}.$
	
	La réciproque est évidente.
	
	ii) Supposons que $f$ est une réduction de $g$ et $g$ une réduction
	de $h.$
	
	* $f\leq g\leq h\Rightarrow f\leq h$
	
	* Comme $\ g$ est une réduction de $h$ alors, il existe $k^{\prime }\geq
	1,$ $H_{k^{\prime }+n}=H_{k^{\prime }}J_{n},$ pour tout $n\geq k^{\prime }.$
	
	Posons $k^{\prime \prime }=k^{\prime }(k^{\prime }+1)$ comme dans (i)
	
	Ainsi en utilisant (i) car $f$ est une réduction de $g$, il vient  $H_{k^{^{\prime \prime }}+n}=H_{k^{^{\prime \prime }}}I_{n},$ pour tout $n\geq k^{^{\prime \prime }}.$
	
	Par suite $f$ est une réduction de $h$. \\
	
	iii) Supposons que $f$ réduction de $g$ et que $f\leq h\leq g.$
	
	Soit $k$ comme dans (i).
	
	Comme $h\leq g$ alors pour tout $n\geq k,$ $J_{k}H_{n}\subseteq
	J_{k}J_{n}=J_{k+n}\subseteq J_{k}H_{n}$ car $f\leq h.$
	
	Donc $J_{k+n}=J_{k}H_{n}$ $,$ pour tout $n\geq k.$
	
	Par suite $h$ est réduction de $g.$
\end{proof}

\begin{maremarque}
	Cependant, le fait que $g$ soit fortement entière sur $f$ n'implique pas
	nécessairement que $f$ soit une réduction de $g$, même si $f$ et 
	$g$ sont noethériennes. Or peut le voir sur l'exemple suivant : 
	
	Soit $A=k[X]$ l'anneau des polynômes à une indéterminée sur le
	corps $k$. Soit $I=XA$. 
	
	On considère les filtrations $f=(I_{n})_{_{n\in \mathbb{N}}}$ et $g=(J_{n})_{_{n\in \mathbb{N}}}$ définies par
	:
	
	$I_{n}=\left\{ 
	\begin{array}{c}
		I^{\frac{3n}{2}}\text{ si }n\text{ pair} \\ 
		I^{\frac{3n+3}{2}}\text{ si }n\text{ impair}
	\end{array}
	\right. $
	
	$J_{n}=\left\{ 
	\begin{array}{c}
		I^{\frac{3n}{2}}\text{ si }n\text{ pair} \\ 
		I^{\frac{3n+1}{2}}\text{ si }n\text{ impair}
	\end{array}
	\right. $
	
	On vérifie que $g$ est noethérienne et que $f\leq g$. De plus, la
	filtration $g$ est entière sur $f$. 
	
	En effet pour tout élément $b\in J_{n}$, $b^{2}\in J_{2n}$ et on a $(bY^{n})^{2}=b^{2}Y^{2n}\in R(A,f)$. 
	
	L'anneau $R(A,g)$ est donc entier sur $R(A,f)$. De plus, comme $g$ est noethérienne, il résulte que $g$ est fortement
	
	entière sur $f$ et que $f$ est noethérienne. 
	
	Néanmoins, $f$ n'est pas une réduction de $g$ puisqu'on n'a pas $J_{2p+1}^{2}=I_{2p+1}J_{2p+1}$, pour $p$ suffisamment grand, condition nécessaire pour qu'une filtration $f$ soit une réduction de $g$ quand
	l'anneau A est noethérien. 
\end{maremarque}
%\begin{maremarque}
%	Soit $f \in \mathbb{F}(A)$. On suppose que A est noethérien. Alors $f$ est une réduction de $f$ $\Longleftrightarrow$ f est noethérienne.
%\end{maremarque}
\section{Filtrations f-bonnes}
\begin{madefinition}
	\label{maprop11}
	Soient $A$ un anneau et $M$ un $A-module$.\\
	On suppose que $\varphi=(M_n)_{n \in \mathbb{Z}}$ est $f-compatible$, avec $f \in \mathbb{F}(A)$. Alors:
	\begin{itemize}
		\item[(a)] $\varphi$ est \textbf{faiblement $f-$ bonne} s'il existe un entier naturel N $\geqslant 1$ tel que:
		\[\forall n > N, M_{n}=\sum_{p=0}^{N}I_{n-p}M_{p} \]
		\item[(b)] $\varphi$ est \textbf{$f-$ bonne} s'il existe un entier naturel N $\geqslant 1$ tel que:
		\[\forall n > N, M_{n}=\sum_{p=1}^{N}I_{n-p}M_{p} \]
		\item[(c)] $\varphi$ est \textbf{$f-$ fine} s'il existe un entier naturel N $\geqslant 1$ tel que:
		\[\forall n > N, M_{n}=\sum_{p=1}^{N}I_{p}M_{n-p} \]
	\end{itemize} 
\end{madefinition}
\begin{maremarque}
	\label{maprop6}
	\begin{enumerate}
		\item[(1)] Toute filtration $f-$bonne est faiblement $f-$ bonne
		\item[(2)] Soit $f \in \mathbb{F}(A)$. Alors f est faiblement $f-$bonne.
		\item[(3)] Soit $f \in \mathbb{F}(A)$. Alors f est $f-$bonne si et seulement si f est $E.P.$
		\item[(4)] Soient $\varphi \in \mathbb{F}(M)$ et $I$ un idéal de $A$. Alors $\varphi$ est $I-bonne$ $\Longleftrightarrow$ $\varphi$ est $f_{I}-bonne$, où $f_{I}$ est la filtration $I-adique$.
		\item[(5)] Soit $g \in \mathbb{F}(A)$ telle que $f \leqslant g$. Alors si $g$ est fortement entière alors g est faiblement $f-bonne$
	\end{enumerate}
\end{maremarque}
\begin{proof}
	(1) Soit $f=(I_n)$ une filtration $f-bonne$. Alors il existe $N \geqslant 1$ tel que :
	\[ \forall n > N, I_n = \sum\limits_{p=1}^{N} I_{n-p}I_p. \]
	Ainsi pour $n> N$, $\sum\limits_{p=0}^{N} I_{n-p}I_p =  I_n + \sum\limits_{p=1}^{N} I_{n-p}I_p = I_n + I_n = I_n$. \\
	Donc $f$ est $faiblement$ $f-bonne$ \\
	(2) Soit $f=(I_n) \in \mathbb{F}(A)$. Alors:\\
	$I_n \subseteq \sum\limits_{p=0}^{N} I_{n-p}I_p$ et $ I_{n-p}I_p \subseteq I_n$
	Donc $\sum\limits_{p=0}^{N}I_{n-p}I_p \subseteq I_n$. \\ Par suite, $I_n = \sum\limits_{p=0}^{N} I_{n-p}I_p.$ Et donc $f$ est $faiblement$ $f-bonne$ \\
	(3) Par définition toute filtration $f-bonne$ est $E.P$. \\
	(4) Supposons que $\varphi \in \mathbb{F}(M)$ est $I-bonne$. Alors: \\
	Pour tout $n \in \mathbb{N}$, $IM_n \subseteq M_{n+1}$ et il existe $n_0 \in \mathbb{N}^{*}$ tel que pour tout $n \geqslant n_0$, $IM_n = M_{n+1}$.
	On a: $IM_n=M{n+1}$ alors $I^{2}M_n=M_{n+2}$, d'où $I^{n-p}M_p=M_n$ pour tout $n \geqslant n_0$\\ Par suite $\sum\limits_{p=0}^{n_0}I^{n-p}M_p=M_n$ et donc $\varphi$ est $f_{I}-bonne$. \\
	Réciproquement supposons que $\varphi$ est $f_{I}-bonne$ alors il existe $N \geqslant 1$ tel que pour tout $n> N$, $M_{n+1} = \sum\limits_{p=0}^{N}I^{n+1-p}M_p =I(\sum\limits_{p=0}^{N}I^{n-p}M_p ) = IM_n$. Ainsi $\varphi$ est $I-bonne$.	
	
\end{proof}
\begin{maproposition}
	\label{maprop7}
	Si $f=(I_n)$ est une réduction de $g=(J_n)$ alors:
	\begin{enumerate}
		\item[(i)] $f$ est $A.P.$ et $g$ est fortement $A.P$
		\item[(ii)] $g$ est $E.P$ et $f-bonne$
		\item[(iii)] En plus, si $A$ est noethérien alors $f$ et $g$ sont noethériennes et g est fortement entière sur $f$.
	\end{enumerate}
\end{maproposition}
\begin{proof}
	Supposons que $f=(I_{n})_{_{n\in \mathbb{N}}}$ est une réduction de $g=(J_{n})_{_{n\in \mathbb{N}}}.$
	
	Alors $f\leq g$ et il existe $k\geq 1,$ tel que pour tout $n\geq k,J_{n+k}=I_{n}J_{k}=J_{k}J_{n}$
	
	i) Nous avons $J_{nk}=J_{k}^{n}$ pour tout $n.$ Donc $g$ est $fortement$ $A.P.$
	
	De plus la division euclidienne de $n$ par $k$ donne $n=kq_{n}+r_{n}$, avec $0\leq r_{n}<k.$
	
	Posons $k_{n}=k(q_{n}+1).$
	
	Alors $\underset{n\longrightarrow +\infty }{\lim }\frac{k_{n}}{n}=\underset{n\longrightarrow +\infty }{\lim }\frac{kq_{n}+r_{n}+k-r_{n}}{n}=\underset{n\longrightarrow +\infty }{\lim }1+\frac{k-r_{n}}{n}=1.$
	
	Par ailleurs, $J_{k_{n}m}=J_{k_{n}}^{m}=J_{k(q_{n}+1)}^{m}\subseteq J_{n}^{m}.$
	
	Posons $k_{n}^{\prime }=k_{2k+n}.$
	
	Alors $I_{k_{n}^{\prime }m}\subseteq J_{k_{n}^{\prime }m}\subseteq
	J_{(k_{2k+n})m}=J_{k_{2k+n}}^{m}=J_{k}^{m}I_{k+n}^{m}\subseteq I_{n}^{m}.$
	
	Par suite $\underset{n\longrightarrow +\infty }{\lim }\frac{k_{n}^{\prime }}{n}=1$
	
	Donc $f$ est $A.P.$
	
	
	ii) Posons $N=2k.$ Alors si $n\geq N,$ $n-k\geq k$ et $J_{n}=I_{n-k}J_{k}\subseteq \sum\limits_{p=1}^{2k}I_{n-p}J_{p}$
	
	Donc $J_{n}=\sum\limits_{p=1}^{2k}I_{n-p}J_{p}$ et $g$ est $f-bonne$ et donc $g$ est $E.P.$
	
	iii) Si $A$ est noethérien alors d'après ii) $g$ et $f$ sont noethérienne. Et donc $g$ est $fortement$ $entière$ sur $f.$
	
\end{proof}
\begin{maproposition}
	\label{maprop3}
	Toute filtration $\varphi$ de $M$ $f-fine$ est $f-bonne$
\end{maproposition}
\begin{proof}
	Supposons que $\varphi =(M_{n})$ est une filtration de $M$ qui est $f-fine$,
	où $f=(I_{n})_{_{n\in \mathbb{N}}}$ une filtration de $A.$
	
	Alors il existe $N\geq 1$ tel que pour tout $n>N,M_{n}=$ $
	\sum\limits_{p=1}^{N}I_{p}M_{n-p}.$
	
	Comme $n>N,$ posons $n=N+1$, ainsi
	
	$M_{N+1}=$ $\sum\limits_{p=1}^{N}I_{p}M_{N+1-p}=$ $\sum\limits_{q=1}^{N}I_{N+1-q}M_{q},$ avec $q=N+1-p.$
	
	Ainsi, il vient de proche en proche que $M_{N+j}=$  $\sum\limits_{p=1}^{N}I_{N+j-p}M_{p}\,\ ,$ pour tout $j$ avec $1\leq j\leq m.$
	
	Alors $M_{N+m}=$ $\sum\limits_{p=1}^{N}I_{p}M_{N+m-p}\,=\sum\limits_{q=m}^{N+m-1}I_{N+m-q}M_{q}\,=\sum\limits_{q=m}^{N}I_{N+m-q}M_{q}\,+\sum\limits_{q=N+1}^{N+m-1}I_{N+m-q}M_{q}=\sum\limits_{q=m}^{N}I_{N+m-q}M_{q}\,+\sum\limits_{q=N+1}^{N+m-1}I_{N+m-q}(\sum\limits_{p=1}^{N}I_{q-p}M_{p}).$
	
	Or $\sum\limits_{q=m}^{N}I_{N+m-q}M_{q}\,\subseteq
	\sum\limits_{p=1}^{N}I_{N+m-p}M_{p}\,$\ et $\sum\limits_{q=N+1}^{N+m-1}I_{N+m-q}(\sum\limits_{p=1}^{N}I_{q-p}M_{p})=\sum\limits_{p=1}^{N}(\sum\limits_{q=N+1}^{N+m-1}I_{N+m-p})M_{p}=\sum\limits_{p=1}^{N}I_{N+m-p}M_{p}\subseteq M_{N+m}$
	
	Par suite $\varphi $ est $f-bonne,$ l'inclusion inverse étant évidente.
\end{proof}
\begin{moncorollaire}
	\label{maprop8}
	Soient $f,g \in \mathbb{F}(A)$ avec $f \leqslant g$. Si $A$ est noethérien alors:\\ 
	$g$ faiblement $f-bonne$ $\Longleftrightarrow$ $g$ est fortement entière sur $f$
\end{moncorollaire}
\begin{maproposition}
	Soient $f=(I_n)$ une filtration $E.P.$ de $A$ et $\varphi=(M_n) \in \mathbb{F}(M)$. Nous avons les assertions suivantes:
	\[ \varphi \text{ est } f-fine \Longleftrightarrow \varphi \text{ est } f-bonne \Longleftrightarrow \varphi \text{ est faiblement } f-bonne   \]
\end{maproposition}
\begin{proof}
	Il suffit de montrer que $\varphi$ est faiblement $f-bonne$ si $\varphi$ est $f-fine$.\\ Supposons que $\varphi$ est faiblement $f-bonne$.\\
	Soient $N, N' \geqslant 1$ des entiers tels que pour tout $n \geqslant N,$
	$M_{n}=$ $\sum\limits_{p=0}^{N}I_{n-p}M_{p}$ et pour tout $n\geq
	1,I_{n}=\sum\limits_{p=1}^{N^{\prime }}I_{n-p}I_{p}.$ Alors pour $n>N^{\prime \prime }=N+N^{\prime},$
	
	$M_{n}=$ $\sum\limits_{p=0}^{N}I_{n-p}M_{p}=\sum\limits_{p=0}^{N}(
	\sum\limits_{q=1}^{N^{\prime
	}}I_{n-p-q}I_{p})M_{p}=\sum\limits_{q=1}^{N^{\prime
	}}I_{q}(\sum\limits_{p=0}^{N}I_{n-p-q}M_{p})=\sum\limits_{q=1}^{N^{\prime
	}}I_{q}M_{n-q}\subseteq \sum\limits_{q=1}^{N^{^{\prime \prime }}}I_{q}M_{n-q}
	$
	
	Donc $M_{n}=\sum\limits_{q=1}^{N^{^{\prime \prime }}}I_{q}M_{n-q}$ ,
	l'inclusion inverse étant triviale. 
\end{proof}
\begin{moncorollaire}
	\label{maprop9}
	Soient $f,g \in \mathbb{F}(A)$. Si $A$ est noethérien, $f \leqslant g$ et $f$ noethérien. Alors nous avons les assertions suivantes:
	\[ g \text{ est } f-fine \Longleftrightarrow  g \text{ est } f-bonne \Longleftrightarrow  g \text{ est faiblement } f-bonne \Longleftrightarrow  g \text{ est fortement entière sur } f \]
\end{moncorollaire}
\begin{maproposition}
	Soient $f=(I_n), g=(J_n) \in \mathbb{F}(A)$, tel que $f \leqslant g$.\\ Si $g$ est $f-bonne$, $E.P.$ et $A$ est noethérien alors $f$ et $g$ sont noethériennes.
\end{maproposition}
\begin{proof}
	Il existe un entier $N\geq 1$ tel que pour tout $n>N,J_{n}=\sum
	\limits_{p=1}^{N}I_{n-p}J_{p}\subseteq
	\sum\limits_{p=1}^{N}J_{n-p}J_{p}\subseteq J_{n}$
	
	Donc $J_{n}=\sum\limits_{p=1}^{N}J_{n-p}J_{p}$ pour tout $n>N.$
	
	Cette égalité est valable si $1\leq n\leq N.$
	
	Comme $g$ est $E.P$ et $A$ noethérien alors $g$ est fortement entière sur $f.$ 
	
	Par suite $g$ est noethérien et d'après le théorème de
	Eakin \cite{Eak}, $f$ est noethérien. 
\end{proof}
\begin{maproposition}
	Soient $f=(I_n), g=(J_n) \in \mathbb{F}(A)$, tel que $f \leqslant g$.\\Si $g$ est faiblement $f-bonne$ alors:\\ $f$ est $A.P$ $\Longleftrightarrow$ $g$ est $A.P$.
\end{maproposition}
\begin{proof}
	Soient $f=(I_{n})_{_{n\in \mathbb{N}}},g=(J_{n})_{_{n\in\mathbb{N}}}\in F(A).$
	
	Alors il existe un entier $N\geq 1$ tel que $I_{n}\subseteq J_{n}\subseteq
	I_{n-N}\subseteq J_{n-N}$ pour tout $N\geq 1.$
	
	Si $f$ est $A.P.$ alors il existe une suite d'entiers $(k_{n})_{n\in \mathbb{N}}$ telle que $\underset{n\longrightarrow \infty }{\lim }\frac{k_{n}}{n}=1$
	et $I_{k_{n}m}\subseteq I_{n}^{m}$
	
	pour tout $m,n\in \mathbb{N}.$
	
	Par suite, $J_{(k_{n}+N)m}\subseteq J_{k_{n}m+Nm}\subseteq
	J_{k_{n}m+N}\subseteq I_{k_{n}m+N}\subseteq I_{k_{n}m}\subseteq
	I_{n}^{m}\subseteq J_{n}^{m}.$
	
	D'où $\underset{n\longrightarrow \infty }{\lim }\frac{k_{n}+N}{n
	}=1,$ $g$ est $A.P.$
	
	Réciproquement si $g$ est $A.P.$ alors il existe une suite d'entiers $(k_{n}^{^{\prime }})_{n\in \mathbb{N}}$ associée à $g.$
	
	Alors $I_{k_{n}^{\prime }+N.m}\subseteq J_{k_{n}^{\prime }+N.m}\subseteq
	J_{n+N}^{m}\subseteq I_{n}^{m}$ pour tout $m,n\in \mathbb{N}.$
	
	Et $\underset{n\longrightarrow \infty }{\lim }\frac{k_{n}^{\prime }+N}{n}=1,f
	$ est $A.P.$
\end{proof}
\begin{maproposition}
	\label{maprop10}
	Soient $A$ un anneau noethérien, $f=(I_n) , g=(J_n)$ deux filtrations de $A$. Si $f$ est noethérienne alors les assertions suivantes sont équivalentes:
	\begin{enumerate}
		\item[(i)] $g$ est fortement entière sur $f$
		\item[(ii)] Il existe un entier $N \geqslant 1$, tel que $t_Ng \leqslant f \leqslant g$
	\end{enumerate}
\end{maproposition}
\begin{proof}
	D'après \ref{maprop6} (3), il suffit de montrer que $(ii) \implies (i)$. Ce qui est une conséquence de \ref{maprop8} et de (\cite{Ok},2.9)
\end{proof}
\begin{maremarque}
	\label{maprop12}
	Sous certaines conditions (voir \cite{Di1}, 3.1 (2), (b)) certaines propositions sur les filtrations entières sont équivalentes avec des filtrations fortement entières. 
\end{maremarque}
\begin{maproposition}
	Soit A un anneau noethérien et $g$ une filtration de $A$ fortement entière sur $f$ ou $g$ est fortement $A.P.$ de rang $r$, alors pour tout entier naturel $m \geqslant 1$, $g^{(rm)}$ est fortement entière sur $f^{(rm)}$.  
\end{maproposition}
\begin{proof}
	Si $g$ est fortement $A.P.$ alors $f$ est fortement $A.P.$
	Alors Supposons que seulement $f$ est fortement $A.P.$ de rang $r.$
	
	Il existe un entier $N\geq 0$ tel que $t_{N}g\leq f\leq g.$\\
	Posons $f=(I_{n})_{_{n\in \mathbb{N}}},g=(J_{n})_{_{n\in \mathbb{N}}},f_{1}=f^{(rm)},g_{1}=g^{(rm)}.$
	
	Nous avons $J_{N+n}\subseteq I_{n}$ pour tout $n\in \mathbb{N}.$
	
	De même $J_{rm(N+n)}\subseteq J_{N+rmn}$ $\subseteq I_{rmn}$  pour tout $n\in \mathbb{N}.$
	
	D'où $t_{N}g_{1}\leq f_{1}\leq g_{1}.$
	
	Or par hypothèse, $f^{(r)}=f_{I_{r}\text{ }}$alors $f^{(rm)}=f_{I_{rm}}$ qui est noethérienne.
	
	Donc $g_{1}$ est fortement entière sur $f_{1}.$
\end{proof}

\begin{montheoreme}
	Soient $f=(I_{n})_{_{n\in \mathbb{N}}}\leq $ $g=(J_{n})_{_{n\in \mathbb{N}}}$ des filtrations sur l'anneau $A.$ Nous
	considérons les assertions suivantes:
	
	$i)$ $f$ est une réduction de $g.$
	
	$ii)$ $J_{n}^{2}=I_{n}J_{n}$ pour tout $n$ assez grand.
	
	$iii)$ $I_{n}$ est une réduction de $J_{n}$ pour tout $n$ assez grand.
	
	$iv)$ Il existe un entier $s\geq 1$ tel que pour tout $n\geq s,$ $J_{s+n}=J_{s}J_{n},$
	
	$I_{s+n}=I_{s}I_{n},$ $J_{s}^{2}=I_{s}J_{s},$ $J_{s+p}I_{s}=I_{s+p}J_{s}$ pour tout $p=1,2,...,s-1$
	
	$v)$ Il existe un entier $k\geq 1$ tel que $g^{(k)}$ est $I_{k}-bonne$
	
	$vi)$ Il existe un entier $r\geq 1$ tel que $f^{(r)}$ est une réduction de $g^{(r)}.$
	
	$vii)$ Pour tout entier $m\geq 1$ tel que $f^{(m)}$ est une réduction de $g^{(m)}.$
	
	$viii)$ $g$ est entière sur $f.$
	
	$ix)$ $g$ est fortement entière sur $f.$
	
	$x)$ $g$ est $f-fine.$
	
	$xi)$ $g$ est $f-bonne.$
	
	$xii)$ $g$ est $faiblement$ $f-bonne.$
	
	$xiii)$ Il existe un entier $N\geq 1$ tel que $t_{N}g\leq f\leq g$
	
	$xiv)$ Il existe un entier $N\geq 1$ tel que $t_{N}g^{\prime }\leq
	t_{N}f^{\prime \text{ }}$ où $f^{\prime }$ est la clôture intégrale de $f.$
	
	$xv)$ $P(f)=P(g)$, où $P(f)$ est la clôture prüférienne de $f.$
	
	1) On a:
	
	$(i)$ $\Longleftrightarrow (vii)$ ; $(v)$ $\Longleftrightarrow (vi)$ ; $(viii)$ $\Longleftrightarrow (xv)$ ; $(ii)$ $\Longrightarrow (iii)$ ; $(iv)$ 
	$\Longrightarrow (i)\Longrightarrow (v)$ ; $(ix)\Longrightarrow (vii),(xii)$
	et $(xiii)$ ;
	
	$(i)\Longrightarrow (x)\Longrightarrow (xi)\Longrightarrow
	(xii)\Longrightarrow (xiii)$
	
	2) Si de plus on suppose $A$ noethérien, alors:
	
	$(i)\Longleftrightarrow (xiv)$ ; $(i)\Longrightarrow (ix)\Longleftrightarrow
	(xii)$ ; $(i)\Longrightarrow (ii)$
	
	3) Par ailleurs, si $f$ est $noeth\acute{e}rienne,$ alors $A$ est noethérien et les assertions suivantes sont équivalentes:
	
	$(ix)\Longleftrightarrow (x)\Longleftrightarrow (xi)\Longleftrightarrow
	(xii)\Longleftrightarrow (xiii)$
	
	4) Si $f$ et $g$ sont noethériennes alors nous avons:
	
	$(iii)\Longrightarrow (viii)\Longleftrightarrow (ix)$ ; $(vi)\Longrightarrow (ix)$
	
	5) Si $f$ est fortement noethérienne et $g$ est noethérienne alors les quinze (15) assertions sont équivalentes et dans ce cas $g$ est fortement noethérienne
	
	$(i)\Longleftrightarrow (ii)\Longleftrightarrow (iii)\Longleftrightarrow
	(iv)\Longleftrightarrow (v)\Longleftrightarrow (vi)\Longleftrightarrow
	(vii)\Longleftrightarrow (viii)\Longleftrightarrow (ix)\Longleftrightarrow
	(x)\Longleftrightarrow (xi)\Longleftrightarrow (xii)\Longleftrightarrow
	(xiii)\Longleftrightarrow (xiv)\Longleftrightarrow (xv).$
\end{montheoreme}
\begin{proof}
	
	1)
	
	$(i)\Longleftrightarrow (vii).$
	
	Supposons $(i)$ et choisissons $k$ comme dans \ref{maprop4} (i) alors pour tout
	entiers $m\geq 1$ et $n\geq k,$ $J_{m(k+n)}=J_{mk}I_{mn}$, ce qui entraîne $(vii).$
	
	La réciproque est évidente.
	
	$(v)\Longrightarrow (vi).$
	
	Posons $f^{(k)}=(H_{n});$ $g^{(k)}=(K_{n});$ $H_{n}=I_{nk};$ $K_{n}=J_{nk};$ 
	$H_{1}=I_{k};$
	
	Par hypothèse, $H_{1}K_{n}\subseteq K_{n+1}$ pour tout entier $n$ et il
	existe un entier $n_{0}\geq 1$ tel que $H_{1}K_{n}=K_{n+1}$ pour tout $n\geq
	n_{0}.$
	
	Pour tout entier $m\geq 0,$ $K_{n_{0}+m}=H_{1}^{m}K_{n_{0}}\subseteq
	H_{m}K_{n_{0}}\subseteq K_{n_{0}+m}.$
	
	Donc $K_{n_{0}+m}=K_{n_{0}}H_{m}$ pour tout entier $m.$ Et donc $f^{(k)}$
	est une réduction de $g^{(k)}.$
	
	$(vi)\Longrightarrow (v).$
	
	Il suffit de montrer que si $f$ est une réduction de $g$ alors il existe 
	$k\geq 1$ tel que $g^{(k)}$ est $I_{k}-bonne.$
	
	Posons $k$ comme dans \ref{maprop4} (i), alors pour tout entiers $m\geq 1$ et  $J_{k(m+1)}=J_{mk}I_{k}$, donc $g^{(k)}$ est $I_{k}-bonne.$
	
	Donc $(vi)\Longrightarrow (v).$
	
	$(viii)\Longleftrightarrow (xv).$
	
	Si $g$ est entière sur $f$ alors $f\leq g\leq P(f),$ ainsi $P(f)\leq
	P(g)\leq P(P(f))=P(f),$ donc $P(g)=P(f).$
	
	Réciproquement si $P(f)=P(g)$ alors $g\leq P(g)=P(f)$ et donc $g$ est entière sur $f.$
	
	$(ii)\Longrightarrow (iii).$
	
	Évident.
	
	$(iv)\Longrightarrow (i).$
	
	Posons $n\geq 2s$ et $n=qs+p$ avec $0\leq p<s.$
	
	Alors $J_{s+n}=J_{(q-2)s+2s+(s+p)}=J_{s}^{q-2}J_{2s+(s+p)}=J_{s}^{q-2}J_{s}^{2}J_{s+p}=J_{s}^{q-1}I_{s}J_{s+p}=J_{s}^{q-1}J_{s}I_{s+p}=J_{s}^{q}I_{s+p}=J_{s}I_{s}^{q-1}I_{s+p}\subseteq J_{s}I_{n}\subseteq J_{s+n}.
	$
	
	Par suite $J_{s+n}=J_{s}I_{n}$ pour tout $n\geq 2s.$ Donc $
	J_{2s+n}=J_{2s}I_{n}$ pour tout $n\geq 2s.$ D'où $(i).$
	
	
	$(i)\Longrightarrow (v)$
	
	Évident car $(vi)\Longrightarrow (v).$
	
	$(ix)\Longrightarrow (viii)$
	
	Évident
	
	$(ix)\Longrightarrow (xii)\Longrightarrow (xiii)$ en utilisant \ref{maprop6} (5)
	
	$(i)\Longrightarrow (x).$
	
	Pour tout entier $n\geq N=2k-1,$ posons $n=qk+r,$ avec $0\leq r<k$ où $k$
	est comme dans $(4.3)$ $(i).$
	
	Alors $J_{n}=J_{k(q-1)}I_{k+r}.$
	
	Ainsi $1\leq k+r<2k-1,$ $J_{n}\subseteq
	\sum\limits_{p=1}^{N}I_{p}J_{n-p}\subseteq J_{n}$, d'où $J_{n}=\sum\limits_{p=1}^{N}I_{p}J_{n-p}$ pour tout $n\geq N=2k-1$.
	
	Ce qui prouve que $g$ est $f-fine.$
	
	$(x)\Longrightarrow (xi)$ par \ref{maprop3}
	
	$(xi)\Longrightarrow (xii)$ par \ref{maprop6} (1).
	
	
	2) 
	
	On suppose maintenant que $A$ est noethérien.
	
	Alors $(i)\Longrightarrow (ix)$ en utilisant \ref{maprop7}.
	
	$(i)\Longrightarrow (ii)$
	
	$f$ est noethérienne par $\ref{maprop7}$ donc il existe un entier $k^{\prime }$ tel que $I_{n+k^{\prime }}=I_{n}I_{k^{\prime }},$ pour tout $n\geq k^{\prime }$.
	
	Choisissons $k$ comme dans \ref{maprop4} (i) nous pouvons supposons que $k=k^{\prime }$ et même prendre $kk^{\prime }$ \`{a} la place de $k$ ou $k^{\prime }$ si nécessaire.
	
	Pour tout $n\geq 3k,$ posons $n=qk+r,$ avec $0\leq r<k.$ Alors $q=E(\frac{n}{k})\geq 3.$
	
	$J_{n}=J_{k}I_{(q-1)k+r}=J_{k}I_{k}^{q-2}I_{k+r}.$
	
	$J_{n}^{2}=J_{k}^{2}I_{k}^{q-3}(I_{k}^{q-1}I_{k+r})I_{k+r}\subseteq J_{2k}I_{(q-3)k}I_{n}I_{k+r}.$
	
	D'où $J_{n}^{2}\subseteq J_{n}I_{n}$
	
	Donc $J_{n}^{2}=J_{n}I_{n}$ pour tout $n\geq 3k.$
	
	$(i)\Longleftrightarrow (iv).$ 
	
	D'après 1) il suffit de montrer que $(i)\Longrightarrow (iv).$
	
	Nous avons vu que $(i)\Longrightarrow (ii).$ Alors il existe un entier $
	k^{\prime }\geq 1$ tel que $J_{n}^{2}=I_{n}J_{n}$ pour tout $n\geq k^{\prime}.$
	
	Dans la preuve de la même implication, nous avons aussi montrer qu'il existe un entier $k\geq 1$ tel que $J_{k+n}=J_{k}I_{n}=J_{k}J_{n}$ et que $I_{k+n}=I_{k}I_{n}$ pour tout $n\geq k.$
	
	Posons $n\geq 2kk^{\prime }=s,$ $k"=kk^{\prime }$ et $n=qk"+r$ avec $0\leq r<k".$ Alors $q\geq 2$ et:
	
	$J_{s+n}=J_{(q+2)k"+r}=J_{k"}^{3}J_{(q-1)k"+r}=I_{k"}^{2}J_{k"}J_{(q-1)k"+r}=I_{s}J_{n}.$
	
	$J_{s+n}=J_{s}I_{n}=J_{s}J_{n}$
	
	$I_{s+n}=I_{s}I_{n}$
	
	$J_{s}^{2}=I_{s}J_{s}$
	
	D'où $(iv).$
	
	$(ix)\Longleftrightarrow (xii)$ d'après \ref{maprop8}
	
	$(iii)\Longleftrightarrow (xiv)$
	
	Nous savons que pour tout idéal $I\subseteq J$ d'un anneau noethérien, $I$ est une réduction de $J$ si et seulement si $I^{\prime }=J^{\prime },$ où $I^{\prime }$est la clôture intégrale de $I.$ D'où l'équivalence. 
	
	3) Supposons que $f$ est noethérienne.
	
	Alors d'après \ref{maprop10}, $(ix)\Longleftrightarrow (xiii)$ et d'après $\ref{maprop9},$ 
	
	$(ix)\Longleftrightarrow (x)\Longleftrightarrow (xi)\Longleftrightarrow (xii)\Longleftrightarrow (xiii)$
	
	4) Supposons que $f$ et $g$ sont noethériens. Alors $(viii)\Longleftrightarrow (ix)$ d'après (\cite{Di1}, \ref{maprop11},(b)).
	
	$(iii)\Longrightarrow (viii).$
	
	Supposons que $I_{n}$ est une réduction de $J_{n}$ pour tout $n\geq n_{0}.$
	
	$f$ et $g$ sont noethérien d'où fortement $A.P.$ \`{a} partir d'un rang
	commun $k.$
	
	L'idéal $J_{n_{0}k}$ est entière sur l'idéal $I_{n_{0}k}.$ D'où $g$ est entière sur $f$ d'après (\cite{Di1}, 4.5).
	
	$(vi)\Longrightarrow (ix).$
	
	Si $f^{(r)}$ est un réduction de $g^{(r)}$ alors $g^{(r)}$ est fortement entière sur $f^{(r)}$ d'après \ref{maprop7} (iii) et $g$ est fortement entière sur $f$ d'après \ref{maprop12}
	
	5) Supposons que $f$ est fortement noethérienne. D'après
	l'implication précédente il est facile de montrer que $(xii)\Longrightarrow (i).$
	
	Supposons que $(xii),$ alors il existe un entier $N\geq 1$ tel que pour tout $n>N,$ $J_{n}=\sum\limits_{p=0}^{N}I_{n-p}J_{p}.$
	
	$f$ étant fortement noethérienne, il existe un entier $N^{\prime}\geq 1$ tel que $I_{m}I_{n}=I_{m+n}$ pour tout $m,n\geq N^{\prime }.$
	
	Posons $n\geq k=N+N^{\prime }.$
	
	Si $0\leq p\leq N,$ alors $N^{\prime }=k-N\leq k-p\leq n-p,$ $J_{n+k}=\sum\limits_{p=0}^{N}I_{n+k-p}J_{p}=\sum\limits_{p=0}^{N}I_{n+}I_{k-p}J_{p}=I_{n}J_{k}$ , et $f$ est une réduction de $g.$
	
	Pour compléter la preuve, nous avons montrer par exemple que si $f$ est une réduction de $g$ et si $f$ est fortement noethérienne alors $g$ est fortement noethérien.
	
	Soient $k,$ $k^{\prime }$ des entiers $\geq 1$ tel que $J_{k+n}=J_{k}I_{n}$
	pour tout $n\geq k$ et $I_{m+n}=I_{m}I_{n}$ pour tout $m,n\geq k^{\prime }.$
	
	Posons $m,n\geq k^{\prime }.$ Alors $J_{m}J_{n}\subseteq
	J_{m+n}=J_{k}I_{m+(n-k)}=J_{k}I_{m}I_{n-k}\subseteq J_{m}J_{n},$ d'où $J_{m+n}=J_{m}J_{n}$ pour tout $m,n\geq k+k^{\prime }$ et $g$ est fortement noethérienne.
\end{proof}